\newpage
{\LARGE \textbf{Notation}}
\vspace{1.6cm}
\begin{center}
\begin{tabular}{ll}
  $\N$ & die Menge der natürlichen Zahlen \\
  $\Z$ & die Menge der ganzen Zahlen\\
  $\Q$ & die Menge der rationalen Zahlen\\
  $\R$ & die Menge der reellen Zahlen\\
  $\C$ & die Menge der komplexen Zahlen\\
  $K^*$& die Menge der Einheiten im Körper K\\
  $x \in A$ & x ist Element der Menge A\\
  $A\subseteq B (A \subset B)$& A ist eine (echte) Untermenge von B \\
  $A \cap B, A \cup B$ & Durchschnitt, Vereinigung der Mengen A, B\\
  $A \setminus B$ & die Menge aller Elemente aus A, die nicht in B liegen \\
  $\varnothing$ & die leere Menge \\
  $ a \leq b$ & a ist kleiner oder gleich b\\
  |a| & der absolute Betrag von a\\
  P, P(G) & der Positivbereich einer Gruppe G \\
  $W_G$ & Menge der wohlgeordneten Teilmengen einer angeordneten Menge G\\
  $a \ll b$ & a ist unendlich kleiner als b\\
  $a \sim b$ & a ist archimedisch äquivalent zu b\\
  $\langle z \rangle$ & die von z erzeugte Untergruppe\\
  $\Sigma$ & die Menge konvexer Untergruppen einer angeordneten abelschen Gruppe\\
  %$V \oplus W$ & die direkte Summe der Untervektorräume V, W \\
 % $A \hookrightarrow B$ & A eingebettet in B\\
 % $\langle u \rangle $ & die von u erzeugte konvexe Untergruppe\\
 \end{tabular}
\end{center}