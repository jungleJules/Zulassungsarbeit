\chapter{Schluss}
Die Grundsteine für die Theorie wurden von Hans Hahn 1907 gelegt. Seine Beweise, dass formale Potenzreihen auf einer angeordneten abelschen Gruppe über $\R$ einen Körper bilden, wurde im Lauf der Jahre immer weiter ausgebaut. Neumann verallgemeinerte Hahns Ergebnisse und zeigte, dass formale Potenzreihen auf einer multiplikativen Gruppe in der nicht-kommutativen Sichtweise einen Schiefkörper formen. Im Laufe der Jahre konnte die Theorie der formalen Potenzreihen, als Verallgemeinerung der Laurentreihen und Pusieuxreihen, immer weiter ausgebaut werden. Mithilfe von Faktorsystemen weiteten Neumann und Malcev' die Multiplikation im Körper aus. Angeführt von Wolfgang Krull, erreichte die Bewertungstheorie die Vereinfachung des Beweises, dass es sich um Potenzreihenkörper handelt.  