\documentclass[a4paper,11pt,twoside=false]{scrbook}

\usepackage[left=2.7cm,right=2.7cm,top=2cm,bottom=2cm,includeheadfoot]{geometry}
\usepackage[T1]{fontenc}        %Umlaute und Akzente
%\usepackage[latin1]{inputenc}
\usepackage[utf8x]{inputenc}    %UTF-8
\usepackage[ngerman]{babel}     %neue Rechtschreibung & richtige Silbentrennung
%\usepackage{nomencl}           %Abkürzungen definieren
%\usepackage{pxfonts} 			% griechisches Alphabet
\usepackage{amsmath,amsthm,amsfonts,amssymb} %mathematische Bibliotheken
%\usepackage{scrpage}
%\usepackage{csquotes}          % zusätzliche Optionen zum Zitieren
%\usepackage{url}            %URLs hinzufügen
\usepackage{color}
\usepackage[boxed]{algorithm}
\usepackage{algorithmic}
\usepackage[small]{caption}      %kleine Untertitel
\usepackage{setspace}  
\usepackage{relsize}          %Zeilenabstand ändern
%\usepackage{ulem}               %erweiterte Optionen zum Unterstreichen
\usepackage{multicol}           % Text in Spalten ermöglichen
\usepackage{enumerate}           %?
\usepackage[pdftex]{graphicx}    %Grafiken ermöglichen
\usepackage{tikz}
\usetikzlibrary{arrows,calc} 
\usepackage[pdftex]{hyperref}    %Hyperlinks ermöglichen
\definecolor{linkcolor}{rgb}{0, 0, 0.7}
\hypersetup{
  pdfstartview = FitH,
  breaklinks = true,
  colorlinks = true,
  bookmarksopen = true,
  linkcolor = linkcolor,
  anchorcolor = linkcolor,
  citecolor = linkcolor,
  filecolor = linkcolor,
  menucolor = linkcolor,
  urlcolor = linkcolor
}


\setlength{\parindent}{0em}     %Kein Einrücken von Absätzen

%\pagestyle{plain}
%\pagestyle{scrheadings}
%\renewcommand\sectionmark[1]{\markright{#1}}
%\renewcommand\chaptermark[1]{\markboth{#1}{}}
%%\ohead{\headmark}
%\headsep=1.5cm


\newtheoremstyle{dfn_style}           % name
  {10pt}            % Space above
  {10pt}            % Space below
  {}              % Body font
  {}              % Indent amount 1
  {\bfseries}            % Theorem head font
  {}              % Punctuation after theorem head
  {.5em}            % Space after theorem head 2
  {\thmname{#1}\thmnumber{ #2}.\thmnote{ #3}}    % Theorem head spec (can be
                                                  %left empty, meaning ‘normal’)


\newtheoremstyle{satz_style}        % name
  {10pt}        % Space above
  {10pt}        % Space below
  {\itshape}    % Body font
  {}            % Indent amount 1
  {\bfseries}   % Theorem head font
  {}            % Punctuation after theorem head
  {.5em}        % Space after theorem head (must not be empty!!! at least " ")
  {\thmname{#1}\thmnumber{ #2}.\thmnote{ #3}}    % Theorem head spec (can be left empty, meaning ‘normal’)



\usepackage{fancyhdr}
\usepackage[Lenny]{fncychap}
\pagestyle{fancy}

\lhead{}
\chead{}
\rhead{}
\lfoot{}
\cfoot{\thepage}
\rfoot{}
\renewcommand{\headrulewidth}{0pt}


\usepackage{mytheorem}
%Mengensymbole
\newcommand{\C}{\mathbb{C}}
\newcommand{\R}{\mathbb{R}}
\newcommand{\N}{\mathbb{N}}
\newcommand{\Z}{\mathbb{Z}}
\newcommand{\T}{\mathbb{T}}
\newcommand{\F}{\mathbb{F}}
\newcommand{\Q}{\mathbb{Q}}
\newcommand{\K}{\mathbb{K}}
\newcommand{\D}{\mathbb{D}}
\newcommand{\E}{\mathbb{E}}
\newcommand{\Hm}{\mathbb{H}}
\newcommand{\Prim}{\mathbb{P}}
\newcommand{\Cq}{\overline{\mathbb{C}}}

%Skript-Zeichen
\newcommand{\Os}{\mathcal{O}}
\newcommand{\Fs}{\mathcal{F}}
\newcommand{\Ms}{\mathcal{M}}
\newcommand{\Cs}{\mathcal{C}}
\newcommand{\Bs}{\mathcal{B}}
\newcommand{\As}{\mathcal{A}}
\newcommand{\Psk}{\mathcal{P}} 
\newcommand{\Gs}{\mathcal{G}}
\newcommand{\Ks}{\mathcal{K}}
\newcommand{\Nw}{\mathcal{N}}
\newcommand{\sep}{\text{sep}}
\newcommand{\Uw}{\mathcal{U}}
\newcommand{\Ps}{\mathfrak{P}} %Pontenzmenge
\newcommand{\Ft}{F_{\vartheta}}
\newcommand{\Wx}{W_{k}^{+}}

%Mathe-Operatoren
\DeclareMathOperator*{\argu}{arg}
\DeclareMathOperator*{\cont}{cont}
\DeclareMathOperator*{\Arg}{Arg}
\DeclareMathOperator*{\Ima}{im}
\DeclareMathOperator*{\Gal}{Gal}
\DeclareMathOperator{\Imag}{Im}
\DeclareMathOperator*{\Rea}{Re}
\DeclareMathOperator*{\grad}{grad}
\DeclareMathOperator*{\Log}{Log}
\DeclareMathOperator*{\Int}{Int}
\DeclareMathOperator*{\Ext}{Ext}
\DeclareMathOperator*{\ord}{ord}
\DeclareMathOperator*{\Res}{Res}
\DeclareMathOperator*{\Bi}{Bi}
\DeclareMathOperator*{\Erw}{E}
\DeclareMathOperator*{\Var}{Var}
\DeclareMathOperator*{\di}{d}
\DeclareMathOperator*{\id}{id}
\DeclareMathOperator*{\Mu}{Mu}
\DeclareMathOperator*{\Rg}{Rg}
\DeclareMathOperator*{\Cov}{Cov}
\DeclareMathOperator*{\Mat}{Mat}
\DeclareMathOperator*{\Abb}{Abb}
\DeclareMathOperator*{\Bij}{Bij}
\DeclareMathOperator*{\Aut}{Aut}
\DeclareMathOperator*{\Nor}{Nor}
\DeclareMathOperator*{\GL}{GL}
\DeclareMathOperator*{\On}{O}
\DeclareMathOperator*{\SO}{SO}
\DeclareMathOperator*{\Bild}{Bild}
\DeclareMathOperator*{\Matr}{Mat}
\DeclareMathOperator*{\SL}{SL}
\DeclareMathOperator*{\ggT}{ggT}
\DeclareMathOperator*{\kgV}{kgV}
\DeclareMathOperator*{\rang}{rang}
\DeclareMathOperator*{\Supp}{Supp}
\DeclareMathOperator*{\sign}{sign}
\DeclareMathOperator*{\Sym}{Sym}
\DeclareMathOperator*{\Stab}{Stab}
\DeclareMathOperator*{\Zen}{Zen}
\DeclareMathOperator*{\Bee}{B}
\DeclareMathOperator*{\dcup}{\dot{\cup}}
\newcommand{\ordn}{\ord\nolimits}


%Mathematische Rechensymbole
\newcommand{\bint}[2]{\mathlarger{\mathlarger{\int}}\limits_{#1}^{#2}}
\newcommand{\sumt}[2]{\sum\limits_{#1}^{#2}}
\newcommand{\funkt}[4]{#1: \left[#2,#3\right] \to #4}
\newcommand{\tilga}{\tilde{\gamma}}
\newcommand{\gammast}{\gamma^*}
\newcommand{\alphast}{\alpha^*}
\newcommand{\Li}{\text{Li}}
\newcommand{\sym}{\text{Sym}}
\newcommand{\LT}{\text{LT}}
\newcommand{\LC}{\text{LC}}
\newcommand{\lex}{\text{Lex}}
\newcommand{\Lex}{\text{Lex}}
\newcommand{\intk}[1]{\left[#1\right]}
\newcommand{\dsc}{D \subseteq \C}
\newcommand{\xfolge}{x_{1},\ldots,x_{n}}
\newcommand{\Xfolge}{X_{1},\ldots,X_{n}}
\newcommand{\Xfolgen}[1]{X_{1},\ldots,X_{#1}}
\newcommand{\Ufolge}{U_{1},\ldots,U_{n}}
\newcommand{\Yfolge}{Y_{1},\ldots,Y_{n}}
\newcommand{\Xfolgem}{X_{1},\ldots,X_{m}}
\newcommand{\matriz}[4]{\left(\begin{array}{cc} #1 & #2 \\ #3 & #4\end{array}\right)}
\newcommand{\folge}[2]{#1_{1},\ldots,#1_{#2}}
\newcommand{\brac}[1]{\langle#1\rangle}
\newcommand{\bbrac}[1]{\langle\langle#1\rangle\rangle}
\newcommand{\entspr}{\widehat{=}}
\newcommand{\chara}{\text{char}}
\newcommand{\gh}{Gruppenhomomorphismus}
\newcommand{\mov}[1]{\overline{\vphantom{b}#1}}
\newcommand{\zmod}{\Z/_{n\Z}}
\newcommand{\ezmod}{(\Z/_{n\Z})^{\times}}
\newcommand{\zmodw}[1]{\Z/_{#1\Z}}
\newcommand{\ezmodw}[1]{(\Z/_{#1\Z})^{\times}}
\newcommand{\cpfeil}{\xrightarrow{\sim}}
\newcommand{\nt}{\vartriangleleft}
\newcommand{\timesp}{\times_{\Phi}}
\newcommand{\norn}{\Nor\nolimits}
\newcommand{\mysim}{\xrightarrow{\hphantom{u}\sim\hphantom{u}}}
\newcommand{\zkzm}[4]{\left(\begin{array}[c]{cc}#1 & #2\\ #3 & #4\end{array}\right)}
\newcommand{\dkdm}[9]{\left(\begin{array}[c]{ccc}#1 & #2 & #3\\ #4 & #5 & #6 \\ #7 & #8 & #9\end{array}\right)}
\newcommand{\winkel}{\measuredangle}
\newcommand{\klr}[1]{\left(#1\right)}
\newcommand{\klg}[1]{\left\{#1\right\}}
\newcommand{\kle}[1]{\left[#1\right]}
\newcommand{\klv}[1]{\left|#1\right|}
\newcommand{\Bdot}{\Bee\limits^{\bullet}}
\newcommand{\zh}[1]{\parbox[0pt][#1][c]{0cm}{}}

\newcommand{\beweis}[1]{
\begin{proof}[Beweis:]
~\\[0,2cm]
#1
\end{proof}}

\begin{document}



\renewcommand{\labelenumi}{(\alph{enumi})}
%\clubpenalty = 10000
%\widowpenalty = 10000
%\displaywidowpenalty = 10000

\onehalfspacing
%\begin{center}
%\Large
%Universität Passau \\
%Fakultät für Informatik und Mathematik \\
% Prof. Dr. Tobias Kaiser\\
%\vspace{120pt}
%\end{center}
%%
%%
%\begin{center}
%\LARGE
%\textbf{Zulassungsarbeit}
%\vspace{50pt}
%\end{center}

\thispagestyle{empty}
%
\begin{center}
\includegraphics[scale=2]{logo.png}
\end{center}
%
\begin{center}
\Large
~\\[0.2cm]
Universität Passau \\
Fakultät für Informatik und Mathematik \\
Prof. Dr. Tobias Kaiser \\
\vspace{50pt}
\end{center}
%
{\huge \centering
\textbf{Zulassungsarbeit}
\vspace{40pt}\\}

%

{\Large\centering
%Ausarbeitung\\[0.4cm] 
\rule{\textwidth}{3pt}
~\\[0.01cm]
{\fontsize{30pt}{25pt}\selectfont\bfseries Potenzreihenkörper}
~\\[0.01cm]
\rule{\textwidth}{3pt}
~\\[0.4cm]
\Large Wintersemester 2014/15\\
}
%\vspace{120pt}
%}
\vfill
%
%
%
{\centering
Julia Kronawitter\\[0.3cm]
\today\\}
%\maketitle
%
\tableofcontents
%
\nocite{fuchs66}\nocite{priesscrampe83}\nocite{priesscrampe69}\nocite{rainer08}\nocite{hulek12}\nocite{carruth48}\nocite{fischer08}\nocite{priesscrampe69}\nocite{taraz12}\nocite{ribenboi}\nocite{malzew48}
\newpage
{\LARGE \textbf{Notation}}
\vspace{1.6cm}
\begin{center}
\begin{tabular}{ll}
  $\N$ & die Menge der natürlichen Zahlen \\
  $\Z$ & die Menge der ganzen Zahlen\\
  $\Q$ & die Menge der rationalen Zahlen\\
  $\R$ & die Menge der reellen Zahlen\\
  $\C$ & die Menge der komplexen Zahlen\\
  $K^*$& die Menge der Einheiten im Körper K\\
  $x \in A$ & x ist Element der Menge A\\
  $A\subseteq B (A \subset B)$& A ist eine (echte) Untermenge von B \\
  $A \cap B, A \cup B$ & Durchschnitt, Vereinigung der Mengen A, B\\
  $A \setminus B$ & die Menge aller Elemente aus A, die nicht in B liegen \\
  $\varnothing$ & die leere Menge \\
  $ a \leq b$ & a ist kleiner oder gleich b\\
  |a| & der absolute Betrag von a\\
  P, P(G) & der Positivbereich einer Gruppe G \\
  $W_G$ & Menge der wohlgeordneten Teilmengen einer angeordneten Menge G\\
  $a \ll b$ & a ist unendlich kleiner als b\\
  $a \sim b$ & a ist archimedisch äquivalent zu b\\
  $\langle z \rangle$ & die von z erzeugte Untergruppe\\
  $\Sigma$ & die Menge konvexer Untergruppen einer angeordneten abelschen Gruppe\\
  %$V \oplus W$ & die direkte Summe der Untervektorräume V, W \\
 % $A \hookrightarrow B$ & A eingebettet in B\\
 % $\langle u \rangle $ & die von u erzeugte konvexe Untergruppe\\
 \end{tabular}
\end{center}
\chapter{Einleitung}
%
In einem Artikel der 53. Ausgabe der \textsc{Mathematical Gazette} (1969) beschrieben Cole and Davie (vgl. \cite{cole69}) ein kleines Spiel für 2 Spieler, das auf dem Euklidischen Algorithmus basiert. Sie nannten es anlässlich dieses Zusammenhangs das \textsc{Spiel Euklid}.\\
Wie bei fast allen Spielen stellt sich die Frage, ob es Strategien gibt, die zu einem sicheren Sieg führen. Wir werden in dieser Arbeit zeigen, dass bei bestimmten Ausgangssituationen ein Spieler durch geschickte Spielzüge immer siegt.\\
Die Seminararbeit gliedert sich in 3 Kapitel: Nach der Einleitung werden in Kapitel \ref{chap2} die Fibonacci-Zahlen und der Goldene Schnitt wiederholt. Wir werden sehen, dass sie bei diesem Spiel eine tragende Rolle spielen. In Kapitel \ref{chap3} werden zunächst kurz die Spielregeln erläutert. Danach werden wir die Gewinnstrategie des Spieles näher betrachten und erkennen, dass das Spiel bereits durch die Anfangskonstellation entschieden ist, falls beide Spieler die Strategie kennen. Abschließend geben wir noch einen kurzen Ausblick auf das Spiel 3-Euklid, welches eine Erweiterung des ursprünglichen Spiels darstellt.

%
\chapter{Mathematische Grundlagen}
In diesem Kapitel werden die für die Ausarbeitung benötigten theoretischen Grundlagen zusammengestellt. Wir betrachten zunächst zentrale algebraische Strukturen und fokussieren uns im weiteren Verlauf auf Gruppen und Ordnungen die in ihnen definiert werden können.
%Erster Teil: Definition Gruppe, Ring, Körper (Algebraische Strukturen)
%Zweiter Teil: Bewertung (Einblick in die Bewertungstheorie) 
%Dritter Teil: Anordnung Wohlordnung Definitionen.(Ordnungsbegriff) viell. erst auf teilweise Ordnung -> Anordnung -> Wohlordnung
\section{Algebraische Strukturen}
Wir beginnen mit der Definition der elementaren algebraischen Strukturen Gruppe, Ring, Körper und Quotientenkörper. Die Vertrautheit mit grundlegenden Begriffen über Mengen und Abbildungen, sowie den wichtigen Zahlenmengen $\N, \Z, \Q, \R$ und $\C$ wird vorausgesetzt. Die folgenden Ausführungen sind orientiert an \cite{rainer08} und \cite{fischer08}.
%
\begin{defn}\label{Gruppe}
Eine nichtleere Menge $G$ mit der Verknüpfung $\circ \colon G \times G \rightarrow G, \left( a, b\right) \mapsto a \circ b$ heißt \textit{Gruppe}, wenn die folgenden Bedingungen erfüllt sind:
\begin{enumerate}
\item[G1: ] (Assoziativgesetz) $a\circ \left(b\circ c\right) = \left(a\circ b\right) \circ c$ für alle $a, b, c \in G$.
\item[G2: ] (Neutrales Element) Es gibt ein eindeutig bestimmtes Element $1_G \in G$ mit $1_G \circ a  = a \circ 1_G = a$ für alle $a \in G$.
\item[G3: ] (Inverses Element) Zu jedem $a \in G$ gibt es ein Element $a^{-1}$ in $G$ mit $a^{-1} \circ a = a \circ a^{-1} = 1_G$. \\
\item[] Die Gruppe heißt \textit{abelsch}, falls folgendes gilt: 
\item[G4: ] (Kommutativgesetz) $a \circ b = b \circ a$ für alle $a, b \in G$.  
\end{enumerate} 
\end{defn}
%
%
\begin{bem}
Wenn $\left(G, \circ\right)$ eine Gruppe ist, so wird das Inverse eines Elements $a\in G$ mit $a^{-1}$ bezeichnet. \\
Wenn nichts anderes gesagt ist, verwenden wir in abelschen Gruppen die Verknüpfung $+$ und nennen das neutrale Element $0_G$ und $-a$ das Inverse zu $a\in G$. 
\end{bem}
%
\begin{bsp}
$\left( \Z, +\right), \left(\Q, +\right)$ und $\left(\R, +\right)$ sind abelsche Gruppen. 
\end{bsp}
%
\begin{defn}\label{Untergruppe}
Sei $\left(G, \circ\right)$ eine Gruppe mit neutralem Element $1_G$. Die Teilmenge $U\subseteq G$ heißt \textit{Untergruppe}, wenn gilt:
\begin{enumerate}
\item[U1: ] $1_G \in U$.
\item[U2: ] $a, b \in U \Rightarrow a\circ b \in U$.
\item[U3: ] $a \in U \Rightarrow a^{-1} \in U$.
\end{enumerate} 
\end{defn}


\begin{defn}\label{Ring} %\cite{fischer08} nach Skript Funktionentheorie Kaiser
Sei $R$ eine nichtleere Menge und seien $+ : R \times R \to R \text{ und } \cdot: R \times R \to R $ zwei Verknüpfungen auf $R$. Das Tripel $\left(R, +, \cdot\right)$ bezeichnen wir als \textit{Ring mit Eins}, wenn gilt:
%
\begin{enumerate}
\item[R1: ] $(R, +)$ ist eine abelsche Gruppe (deren neutrales Element mit $0_R$ bezeichnet wird).
\item[R2: ] Die Multiplikation $\cdot$ ist assoziativ: Für $a, b,c \in R$ gilt $a\cdot \left(b \cdot c\right) = \left(a \cdot b \right) \cdot c$ (das neutrale Element wird mit $1_R$ bezeichnet).%Blöd weil ich Monoid brauche: $\left(R, \cdot\right)$ ist ein Monoid,
\item[R3: ] (Distributivgesetze) Für alle a,b,c $\in$ R gilt\\
\[a \cdot(b +c) = a \cdot b + a \cdot c \text{ und }
(a+b) \cdot c = a \cdot c + b \cdot c \] 
\end{enumerate}
\end{defn}
%Das neutrale Element bezüglich der Addition wird mit \textbf{$0_R$}, das neutrale Element bezüglich der Multiplikation mit \textbf{$1_R$} bezeichnet.

\begin{bem} %evtl doch keine extra Bemerkung
Ist die Multiplikation kommutativ, so heißt $\left(R, +, \cdot\right)$ \textit{kommutativer Ring mit Eins}. Anstelle von $\left(R, +, \cdot\right)$ sprechen wir vereinfachend von dem Ring $R$.
\end{bem}
%
%evtl Definition von Nullteiler, Einheit, Integritätsbereich? alles nach Erinnerung Kaiser
\begin{defn}
Sei $\left(R, +, \cdot\right)$ ein kommutativer Ring mit Eins und sei $a \in R$.
\begin{enumerate}
\item Ein Element $a \in R$ heißt \textit{Nullteiler}, falls $a \neq 0$ und ein $b \in R, b \neq 0$ existiert mit $ab =0$.
\item Ein Element $a \in R$ heißt \textit{Einheit}, falls ein $b \in R$ existiert mit $ab = 1$. 
\end{enumerate}
\end{defn}
%
%
%
\begin{bem}
Falls $a\in R$ eine Einheit ist, existiert das Inverse und es ist eindeutig bestimmt. Wir bezeichnen das Inverse mit $a^{-1}$.
\end{bem}

\begin{defn} \label{Integritätsbereich}
Ein kommutativer Ring $R$ mit Eins heißt \textit{Integritätsbereich}, falls es in $R$ keine Nullteiler gibt.
\end{defn}
%
%
%
%
%
%
% 
%
%
% 
\begin{defn}
Sei $K$ eine nichtleere Menge mit mindestens zwei Elementen und seien $+, \cdot$ zwei Verknüpfungen auf $K$. Genau dann ist $\left(K, ~+,~ \cdot\right)$ ein \textit{Körper}, wenn folgende Gesetze gelten:
\begin{enumerate}
\item[(a)] Addition
\begin{enumerate}
\item[(i)] (Assoziativgesetz) Für alle $a, b, c \in K$ ist $\left(a + b \right) + c = a + \left(b +c\right)$.
\item[(ii)] (Kommutativgesetz) Für alle $a, b \in K$ ist $a+b = b + a$.
\item[(iii)] (Nullelement) Es gibt genau ein Element $0_K \in K$ mit $0_K + a = a + 0_K = a$ für alle $a \in K$.
\item[(iv)] (Negatives Element) Zu jedem $a \in K$ gibt es (genau) ein $-a \in K$ mit $\left(-a\right)+ a = a + \left(-a\right)= 0$.
\end{enumerate} 
\item[(b)] Multiplikation:
\begin{enumerate}
\item[(i)] (Assoziativgesetz) Für alle $a, b, c \in K$ ist $\left(ab\right)c = a\left(bc)\right)$.
\item[(ii)] (Kommutativgesetz) Für alle $a, b \in K$ ist $ab = ba$.
\item[(iii)] (Einselement) Es gibt genau ein Element $1_K \in K$ mit $1_K a = a 1_K = a$ für alle $a \in K$.
\item[(iv)] (Inverses Element) Zu jedem $a \in K$ mit $a\neq 0 $ gibt es genau ein $a^{-1} \in K$ mit $a^{-1} a = a a^{-1} = 1$.
\end{enumerate}
\item[(c)] Distributivgesetz:
\begin{itemize}
\item Für alle $a, b, c \in K$ ist $a\left(b +c\right) = ab + bc$.
\end{itemize}
\end{enumerate}
\end{defn}


%Quotientenkörper Definition
%In seinem großen Werk $\glqq$Von Zahlen und Größen - Dritthalbtausend Jahre Praxis und Theorie$\grqq$ beschreibt Lüneburg die lange Entwicklungsgeschichte der Mathematik und ihre Ursprünge. Darin definiert er auch den Quotientenkörper \cite[S. 557]{Lueneburg08}.

\begin{satz}\label{quotkoerper} %\cite{Lueneburg08} statt RxR* mach ich RxR\0 weil R ist nicht definiert.
Es sei $R$ ein Integritätsbereich bestehend aus wenigstens zwei Elementen. Wir definieren auf $R\times R\setminus\lbrace 0\rbrace$ eine Äquivalenzrelation $\sim$ durch $\left(r,u\right) \sim \left(s, v\right)$ genau dann, wenn $rv = su$ ist. Man sieht sofort, dass die Relation reflexiv und symmetrisch ist. Sei $\left(r,u\right) \sim \left(s, v\right)$ und $ \left(s, v\right) \sim \left(t, w\right) $, also $rv = su$ und $sw = tv$. Wir können nun schreiben: \\
\[rwv = rvw = suw = swu = tvu = tuv.\]
Nach Voraussetzung ist $R$ ein Integritätsbereich. Da $v$ ungleich null ist, folgt $rw = tu$, so dass $\left(r,u \right) \sim \left(t, w\right)$ ist. Auf $R\times R\setminus \lbrace 0\rbrace$ definieren wir eine Addition und eine Multiplikation durch 
\[\left(r,u\right)+ \left(s,v\right) := \left(rv + su, uv\right),\]
\[\left(r,u\right)\left(s,v\right) := \left(rs, uv\right)\text{ definiert.}\] 
Wir setzen \textup{Quot}$(R) := R\times R\setminus\lbrace 0\rbrace \ \sim$ und bezeichnen die Äquivalenzklasse von $(r,u) $ mit $\frac{r}{u}$. In diesem Fall gilt: 
\[\frac{r}{u} + \frac{s}{v} = \frac{rv + su}{uv}\]
\[\frac{r}{u}\frac{s}{v} = \frac{rs}{uv}\]
\textup{Quot}$\left(R, +, \cdot\right)$ ist ein Körper, wir nennen ihn den \textit{Quotientenkörper} von $R$.
\end{satz}
\beweis{Wir prüfen zunächst die Wohldefiniertheit der Verknüpfungen. \\Seien $\left(r_1, u_1 \right), \left(r_2, u_2\right), \left(s, v\right) \in R\times R\setminus\lbrace 0\rbrace$ und sei $\left(r_1, u_1 \right) \sim \left(r_2, u_2\right)$. Nach Definition der Äquivalenzrelation ist $r_1u_2 = r_2u_1$. Daher erhalten wir
\begin{equation*}
\left(r_1v + su_1\right)u_2v \\= r_1u_2v^2 + su_1u_2v \\
=r_2u_1v^2 + su_1u_2v \\
=\left( r_2v + su_2\right)u_1v \\
\end{equation*}
also 
\[\left(r_1v + su_1, u_1v\right) \sim \left(r_2v + su_2, u_2v\right)\]
Die Wohldefiniertheit der Multiplikation folgt analog. Man sieht
\begin{equation*}
\left(r_2s, s_2v\right) \sim \left(r_1s, u_1v\right),
\end{equation*}
denn 
\begin{equation*}
\left(r_2s\right)\left(u_1v\right) = r_1u_2sv = \left(r_1s\right)\left(u_2v\right).
\end{equation*}
Es genügt, die Körperaxiome für (Quot(R), +, $\cdot$) nachzurechnen. 
%Der Quotientenkörper ist nach Definition der kleinste Körper in den ein Integritätsring \textit{R} eingebettet werden kann. %\textit{Quot(R)} enthält alle Elemente der Form $\frac{a}{b}, \text{ mit} a,b \in R, b \neq 0. $ Es gibt einen injektiven Ringhomomorphismus $\phi : R \rightarrow Quot(R), \phi(a) = frac{a}{1}$.
}
%
%
% 
%
%
\begin{bem}
Der Quotientenkörper ist bis auf Isomorphie der kleinste Körper in den $R$ als Unterring eingebettet werden kann.
\end{bem}
%
%
%
%
%
%
%
\begin{bsp}
Ist D $\subseteq \C$ ein Gebiet, so ist der Ring $ \mathcal{O} (D) $ der in $D$ holomorphen Funktionen ein Integritätsring. Man nennt \\
$ M \left(D\right) := $Quot$\left( \mathcal{O} \left( D \right)\right) = \lbrace \frac{f}{g}: f,g \in \mathcal{O} (D), ~g \neq 0\rbrace$ den Körper der meromorphen Funktionen. Der Nenner kann unendlich viele Nullstellen besitzen, diese liegen allerdings isoliert. 
\end{bsp}
%

%
%
\section{Ordnung in algebraischen Strukturen}
\subsection{Anordnung}
\begin{defn}\label{defgs} 
Eine Menge $A$ heißt \textit{teilweise geordnet}, wenn es eine Relation $``\leq " $ auf $A$ gibt die folgende Eigenschaften für alle $ a,b,c \in A$  erfüllt.
%
\begin{enumerate}
\item[T1:] \textit{Reflexivität: } $a \leq  a$,
\item[T2:] \textit{Antisymmetrie: } Aus $a \leq  b$ und $b~ \leq a$ folgt $a = b$,
\item[T3:] \textit{Transitivät: } Aus $a \leq b$ und $b \leq c$ folgt $a \leq c$.
\end{enumerate}
%
Die Relation $``\leq "$ bezeichnet eine teilweise Ordnung auf $A$.
\end{defn}
Die oben definierte Ordnungsrelation wird als Anordnung beziehungsweise totale Ordnung bezeichnet, wenn neben T1-T3 die anschließende Bedingung erfüllt ist:
%
\begin{enumerate}
\item[T4:] Für alle $a, b \in A$ besteht entweder $a < b$, oder $a = b$, oder $a > b$. Dabei gilt $a < b$ genau dann, wenn $a \leq b$ und $a\neq b$. 
\end{enumerate}
%
%
%
%
%
%
%
\begin{defn}\label{ordnungsisomorph}
Seien $A$ und $A'$ teilweise geordnete Mengen. Eine Abbildung $\phi \colon A \rightarrow A', a \mapsto a'$ wird \textit{Ordnungsisomorphismus} von $A$ nach $A'$ genannt, falls folgende Anforderungen erfüllt sind:
\begin{enumerate}
\item (Ordnungstreue) Wenn $a \leq b$ gilt, so folgt $\phi(a) \leq \phi(b)$ für alle $a, b  \in A$.
\item (Bijektivität) Für jedes $a' \in A'$ existiert genau ein $a \in A$, mit $a' = \phi(a)$.
\end{enumerate}
$A$ und $A'$ werden in diesem Fall \textit{ordnungsisomorph}, kurz \textit{o-isomorph} genannt.
%Ist $G$ mit dem Positivbereich $P$ eine angeordnete Gruppe,
%so ist $G$ auch mit dem Positivbereich $\left(−P\right)$ eine angeordnete Gruppe. Die Abbildung
%$a \mapsto −a$ ist ein \textit{ordnungserhaltender Isomorphismus} zwischen diesen
%beiden Gruppen. Man nennt einen solchen Isomorphismus einen \textit{Ordnungsisomorphismus}.
%Man nennt Gruppen \textit{ordnungsisomorph (o-isomorph)}, wenn es zwischen ihnen einen Ordnungsisomorphismus gibt.
\end{defn}
%
\begin{defn}\label{twgG} % \cite{fuchs66}
Eine \textit{teilweise geordnete Gruppe} bezeichnet eine Menge $G$ mit folgenden Eigenschaften: 
%
\begin{enumerate}
\item[G1:] $G$ ist eine Gruppe bezüglich der Multiplikation,
\item[G2:] eine teilweise geordnete Menge bezüglich einer Relation $``\leq " $, wie in \ref{defgs}, 
\item[G3:] das Monotoniegesetz ist erfüllt: Für $a, b \in  G$ gilt: Aus $a \leq b$ folgt $ca \leq  cb$ und \\ $ac \leq bc$ für alle $c \in G$.
\end{enumerate}
% 
\end{defn}
%
%
\begin{defn}\label{agG}
Eine Gruppe wird als \textit{angeordnete Gruppe} bezeichnet, wenn ihre Ordnung total ist.
\end{defn}
%
%
%
\begin{bsp}\label{beispielUntergruppeAngeordnet}
Eine Untergruppe $U$ einer angeordneten Gruppe $G$ ist bezüglich der selben Relation wie $G$ angeordnet.
\end{bsp}
%
\begin{bsp}\label{OrdnungNundZ}
Wir betrachten die natürliche Ordnung auf den natürlichen, ganzen und reellen Zahlen.
\begin{enumerate}
\item Die Menge der natürlichen Zahlen $\N$ ist total geordnet bezüglich der Relation $``\leq "$. Es gilt für $a, b \in \N$, dass
\[ a \leq b \text{ ist genau dann, wenn } b-a \in \N
\]
für alle $a,b \in \N$.
\item Die Menge der ganzen Zahlen $\Z$ ist total geordnet bezüglich der Relation $``\leq "$. Es gilt für $a, b \in \Z$, dass
\[ a\leq b \text{ ist, genau dann, wenn } b -a \in \N \text{ oder } a = b \text{ gilt.}
\]
\item Die Menge der reellen Zahlen $\R$ ist total geordnet bezüglich der Relation $``\leq "$. Es gilt für alle $a, b \in \R$, dass
\[a \leq b \text{ ist, genau dann, wenn } 0 < b -a \text{ oder } a = b \text{ gilt.} 
\]
\end{enumerate}
\end{bsp}
%
% 
%
%
%ab jetzt nach Priess crampe
\begin{satz} \label{angeordnetFolgtTorsionsfrei} %\cite{priesscrampe83}
Jede angeordnete Gruppe ist torsionsfrei. 
\end{satz}
%
\beweis{
Dies folgt unmittelbar aus obiger Definition einer angeordneten Gruppe. Denn angenommen die angeordnete Gruppe wäre nicht torsionsfrei, so würde sich für die Elemente der Torsionsgruppe ein Widerspruch mit dem Monotoniegesetz G3 ergeben. %(formal siehe kleiner Block)
}
%
\begin{bem}\label{afG} %nach Fuchs S. 25
%
Genügt eine Teilmenge $P := \lbrace x \in G | x \geq 0\rbrace$ einer Gruppe $G$ den Bedingungen P1- P3, so nennt man $\left(G,\circ\right)$ \textit{anordnungsfähig}. Wir nennen $P$ den \textit{Positivbereich} von $G$.
%
\begin{enumerate}
\item[P1:] $\lbrace0\rbrace \cup P\cup -P = G$, $P \cap -P = \varnothing$,
\item[P2:] $P \circ P \subseteq P$,
\item[P3:] $x+P+(-x) \subseteq P$ für jedes $x \in G$.
\end{enumerate}

\end{bem}
%
%
%
\begin{bsp}
Ist $G$ mit dem Positivbereich $P$ eine angeordnete Gruppe, so ist $G$ auch mit dem Positivbereich $(-P)$ eine angeordnete Gruppe. Die Abbildung $a \mapsto -a$ ist ein Ordnungsisomorphismus.
\end{bsp}
%

\subsection{Wohlordnung}
Nun beschäftigen wir uns mit dem Begriff der Wohlordnung. Diese wird später bei der näheren Betrachtung des Trägers einer Potenzreihe eine wichtige Rolle spielen. Wir definieren wohlgeordnete Mengen wie in \cite[S. 16]{fuchs66}.
\begin{defn} \label{wohlgeordn} %\cite{fuchs66} 
Eine angeordnete Menge $W$ nennt man \textit{wohlgeordnet}, wenn jede nichtleere Teilmenge $V$ von $W$ ein kleinstes Element enthält. Es existiert also ein Element $ u \in V, \text{ mit } u \le v $ für alle $ v \in V.$ 
\end{defn}
%
Der Wohlordnungssatz, ein von Ernst Zermelo bewiesenes Prinzip der Mengenlehre, besagt, dass auf jeder Menge eine Wohlordnung existiert. Dieses Theorem, so stellte sich nach erfolglosen Widerlegungsversuchen zahlreicher Mathematiker heraus, ist äquivalent zum Auswahlaxiom und dem Lemma von Zorn. \\
Beispielsweise ist die natürliche Anordnung der natürlichen Zahlen $\N$ eine Wohlordnung. Die Menge $\Z$ ist mit der natürlichen Anordnung $``\leq "$ total geordnet, jedoch nicht wohlgeordnet, da die negativen Elemente von $\Z$ nicht nach unten beschränkt sind und somit $\Z$ kein kleinstes Element enthält. Nach der Konstruktion der ganzen Zahlen auf Basis der natürlichen Zahlen mittels einer Äquivalenzrelation auf $\N \times \N$  überträgt sich das Wohlordnungsprinzip von $\N \text{ auf } \Z$.
\begin{bem} %\cite{rainer08}
Ist $W \subseteq \Z$ eine nach unten beschränkte Teilmenge, so hat $W$ ein eindeutig bestimmtes kleinstes Element. 
\end{bem} 
%Der Beweis wird nicht benötigt. \beweis{ Es gilt: $ \Z$ ist ein kommutativer nullteilerfreier Ring mit Einselement und somit ein Integritätsbereich. Die Rechenoperationen sind wohldefiniert, wie leicht zu zeigen ist. Sei $M \subseteq \Z \text{ eine nach unten beschränkte Teilmenge von } \Z. $ Da M nach unten beschränkt ist gibt es ein $ a \in M \text{ sodass für alle } m \in M: a \le m.$ Noch zu zeigen ist, dass a eindeutig bestimmt ist. Dies folgt da $\le$ eine totale Ordnung auf $\Z$ definiert. Angenommen es gibt ein Element a' $\in$ M mit $a' \neq a \text{ und } \forall m \in M: a' \le m.$ Dann folgt nach Voraussetzung $a\le a' \text{ und } a'\le a$, und nach Definition einer totalen Ordnung \ref{twgG} [T2] $a' = a$, damit Widerspruch zur Voraussetzung. \\
%}

%

%\begin{bsp}
%Betrachte auf $\Z$ die Ordnung: $ a\prec b \Leftrightarrow (|a| \le|b| \vee |a| = |b|, a > 0).$ \newline %\footnote{http://www.mathematik.tu-dortmund.de/lsviii/new/media/veranstaltungen/wise1011/mathinf1/SkriptRek.pdf}. 
%Daher gilt in $\Z$: $ 0 \prec 1 \prec -1 \prec 2 \prec -2 ...$. \newline
%Das kleinste Element von $\Z$ in dieser Ordnung entspricht dem Element mit dem kleinsten Index. 
%\end{bsp}
%
%
\begin{bem}\label{Teilmengewohlgeordnet}
Jede Teilmenge einer wohlgeordneten Menge ist wohlgeordnet.
\end{bem}
\begin{bsp}
Betrachte die Relation $``\preceq " $ auf $\Z$. Es gilt,  
\[a \preceq b \text{ genau dann, wenn } |a| \le |b| \text{ oder }  |a| = |b| \text{ und } a \leq  b\]
ist.
Die Relation $``\preceq "$ ist eine Wohlordnung auf $\Z$ und wir erhalten 0$  \preceq -1 \preceq 1 \preceq -2 \preceq 2 \preceq 3 \preceq -3 .... $ 
\end{bsp}
%

%
 
\chapter{Angeordnete abelsche Gruppen}\label{chap2} %TODO: Wichtig hier nur alles was abelsch ist rein!!!
%Teil 1: Definition angeordnete abelsche Gruppe und Beispiele (Natürlichen Zahlen, usw.)
%Teil 2: Zusammenhang angeordnete abelsche Gruppe mit Wohlordnung
%Teil 3: Satz von Hölder
In diesem Kapitel fassen wir jene Begriffe und Bezeichnungen zusammen, die zur Betrachtung des Potenzreihenkörpers benötigt werden. Nach einer Einführung in die Theorie angeordneter abelscher Gruppen beschäftigen wir uns mit deren Wohlordnung, eine Eigenschaft, die für die Konstruktion des Potenzreihenkörpers unabdingbar ist. Mithilfe der Archimedizität führen wir eine spezielle Art der Anordnung von Gruppen ein. Die Familie der konvexen Untergruppen führt uns zu Aussagen über die Anordnungsfähigkeit von Gruppen. Daran schließt die zentrale Aussage des Kapitels an: der Satz von Hölder, der besagt, dass archimedisch angeordnete Gruppen in die additive Gruppe des $\R$ eingebettet werden können. \\
%TODO: vielleicht zu ausufernd? Die Theorie der angeordneten Strukturen, in unserem Fall ausschließlich Gruppen, liefert wichtige Erkenntnisse zur späteren Konstruktion des Körpers von formalen Potenzreihen. Die Funktionen, die durch Potenzreihen dargestellt werden, sind nicht mehr nur auf den natürlichen Zahlen, sondern jeder angeordneten abelschen Gruppe definierbar, wobei auf die Wohlordnung nicht verzichtet werden kann. 
Die nachfolgenden Ausführungen sind angelehnt an \cite[S. 21 - 28]{fuchs66} und \cite[S. 1 -  4]{priesscrampe83}.
%
%

% Kaiser Antwort abwarten-> raus
%\begin{defn} %\cite{priesscrampe83}
%Eine abelsche Gruppe $\left(G, +\right))$ heißt \textit{teilbar}, wenn zu jedem $a\in G$ und $n \in \N$ ein Element $b \in$ $G$ mit $nb = a$ existiert.
%\end{defn}
%\begin{satz}\label{torsionsfreiHülle} %\cite{priesscrampe83}
%Eine torsionsfreie, abelsche Gruppe $\left(G, +\right)$ ist bis auf Isomorphie in genau einer minimalen teilbaren, abelschen Gruppe $\left((\overline{G}, +\right)$ enthalten. $\overline{G}$ heißt die teilbare Hülle von $G$. 
%\end{satz}
%\beweis{
%Wir betrachten $\overline{G} $, bestehend aus der Menge der Paare $\left(x, n\right)$ mit $x\in G$, $n\in \N$ wobei $\left(x, n\right) = \left(y, m\right)$ für $mx = ny$ gelte. Die Addition über $\overline{G}$ wird definiert durch $\left(x, n\right) + \left(y, m\right) = \left(mx + ny, mn\right)$. Mit dieser Verknüpfung ist $\overline{G}$ eine abelsche Gruppe, denn $\left(x, n\right) + \left(y, m\right) = \left(mx + ny, mn\right) = \left(ny + mx, nm\right)= \left(y, m\right) + \left(x, n\right)$, da $G$ nach Voraussetzung abelsch und $\N$ ein kommutativer Halbring ist. \\
%Die Gruppe $\overline{G}$ ist torsionsfrei, da jedes Element, bis auf das neutrale, unendliche Ordnung hat, nach Konstruktion von $\overline{G}$. Durch die Abbildung  $G \rightarrow \overline{G}, a \mapsto (a, 1)$ ist eine Einbettung von $G$ in $\overline{G}$ gegeben, die jedem Element aus $G$ ein Element in $\overline{G}$ zuordnet.\\
%Wir konstruieren eine minimal teilbare Gruppe einer teilbaren Obergruppe $G^*$ von $G$. Als minimal teilbare Untergruppe von $G^*$, die $G$ enthält nach Konstruktion, wählen wir $\Q G = \lbrace q x: q\in \Q, x\in G\rbrace$. Durch die Abbildung $(a, n) \mapsto \frac{1}{n} \cdot a$ ist ein Isomorphismus von $\overline{G}$ auf $\Q\cdot G$ definiert.}
%%

%
\begin{defn} %
Eine angeordnete abelsche Gruppe ist ein Tripel $\left(G, +, \leq\right)$, wobei $\left(G, +\right)$ eine additiv geschriebene, abelsche Gruppe ist, die bezüglich $``\leq "$ total geordnet ist.
\end{defn}
%
%
%
%
\begin{nota}
Wir verwenden im Hauptteil \ref{chap3} für eine angeordnete abelsche Gruppe $\left(G, +, \leq\right)$ meist vereinfachend die Bezeichnung $G$.
\end{nota}

\begin{bem}\label{angeordnetAbelsch} %\cite{Lueneburg08} 
Ist eine abelsche Gruppe $G$ mit dem Positivbereich $P$ angeordnet, so definieren wir $a \leq b$ genau dann, wenn $b - a \in P$ ist für $a, b \in G$.
%aus: http://wwwmath.uni-muenster.de/users/ischebeck/algebra.pdf 
\end{bem}
%
%
%
%
%
\begin{bsp}
\begin{itemize}
\item[]
\item Die Menge der ganzen Zahlen $\left(\Z, +, \leq\right)$ ist eine angeordnete abelsche Gruppe.
\item Die Menge der reellen Zahlen $\left(\R, +, \leq\right)$ ist eine angeordnet abelsche Gruppe.
\end{itemize} 
\end{bsp}
                                                                                                                     
%

\section{Wohlordnung in angeordneten abelschen Gruppen}
Die folgenden Aussagen orientieren sich an der Arbeit $\glqq$A residue theorem for Malcev--Neumann series$\grqq$ von Guoce Xin.
\begin{satz}\label{wohlgeordnetabnehmendeFolge} 
Sei $`` \le "$ eine totale Ordnung auf der Menge $W$. Dann ist $W$ genau dann wohlgeordnet, wenn es keine streng monoton fallende Folge in $W$ gibt.
\end{satz}
\beweis{ $``\Rightarrow "$ Sei W wohlgeordnet. Angenommen es gibt eine streng monoton fallende Folge von Elementen in W, nämlich $a_1 > a_2 > a_3 > ...$. Damit erhalten wir eine Teilmenge $\left( a_i\right)_{i\in \N}$, die kein kleinstes Element besitzt. Dies ist ein Widerspruch zur Wohlordnung von $W$.\\
$``\Leftarrow " $ Wir nehmen an, es gibt keine streng monoton fallende Folge in $W$ und betrachten den Fall $W$ ist nicht wohlgeordnet. Dann gibt es eine Teilmenge $A$ von $W$, die kein kleinstes Element enthält. Für ein beliebiges Element $a_1 \in A$ finden wir ein $a_2 \in A$ mit $a_2 < a_1$. Dieses Verfahren lässt sich endlos fortsetzen und wir erhalten eine streng monoton fallende Folge $a_1 > a_2 > a_3 > ...$, Widerspruch.}
%Quelle: http://arxiv.org/pdf/math/0405133v1.pdf
%
%
%
%
%
%
\begin{bsp}\label{TotalGeordnetEndlichIstWOhlgeordnet}
Total geordnete endliche Mengen sind wohlgeordnet.
\end{bsp}
%
%
%
%
%
%
%
%
Betrachte nun die Menge aller wohlgeordneten Teilmengen $W_A$ einer total geordneten Menge $A$, die nicht zwangsläufig wohlgeordnet ist. 
\begin{lemma}\label{wohlgeordnvereinigung} %\cite{xin04}
Sei $w_{n} \in W_A$ für alle $n\in \N$, dann gilt $\bigcap_{n \in \N} w_n \in W_A$ und für $w_1, w_2 \in W_A$ gilt $w_1 \cup w_2 \in W_A$.
\end{lemma}
\beweis{Die erste Aussage ist trivial. Zum Beweis der zweiten Behauptung führen wir einen Widerspruchsbeweis. Angenommen $w_1\cup w_2$ sei nicht wohlgeordnet, dann gibt es nach \ref{wohlgeordnetabnehmendeFolge} eine streng monoton fallende Folge $a_1 > a_2 > ...$ in $w_1\cup w_2$.\\
Betrachten wir alle Elemente der Teilmenge $w_1$. Wir können diese, da $w_1$ total geordnet ist, als Folge $a_{i_1} > a_{i_2}...$ schreiben. Aufgrund der Wohlordnung von $w_1$ ist die so erhaltene fallende Folge endlich. Die selbe Argumentation wählen wir für $w_2$ und erhalten die endliche fallende Folge $a_{j_1} > a_{j_2}...$. Aber jedes Element der streng monoton fallenden Folge $a_1 > a_2 > ...$ ist in einer der beiden endlichen Folgen enthalten.  Widerspruch!}
%
%
%
%
% 
%
%
%
%
Die Menge $W_A$ ist somit unter endlicher Vereinigung und unendlichem Schnitt abgeschlossen.
\begin{lemma}\label{unendlicheFolgeEigenschaften}
Wir betrachten eine total geordnete Menge $A$. Jede Folge $\left(a_n\right)_{n \in \N}$ in $A$ erfüllt mindestens eine der drei folgenden Eigenschaften:
\begin{enumerate}
\item[(1)] $a_1, a_2, ...$ enthält eine streng monoton wachsende Teilfolge.
\item[(2)] $a_1, a_2, ...$ enthält eine konstante Teilfolge.
\item[(3)] $a_1, a_2, ...$ enthält eine streng monoton fallende Teilfolge.
\end{enumerate}
\beweis{
Angenommen die Folge $\left(a_n\right)_{n \in \N}$ erfüllt weder die Bedingung (2), noch (3). Wir wollen zeigen, dass sie eine streng monoton wachsende Teilfolge enthält.\\
Da die Folge somit keine streng monoton fallende Teilfolge enthält, gibt es ein kleinstes Element $a_{i_1}$, denn andernfalls ließe sich eine streng monoton fallende Teilfolge konstruieren. Die Folge bleibt unendlich, wenn wir die ersten Folgenglieder $i_1$ aus $\lbrace a_n\rbrace_{n\ge 1 }$ entfernen, da es nur endlich viele Folgenelemente nach Voraussetzung gibt, die gleich $a_{i1}$ sind. In der daraus entstandenen Folge ist jedes Element größer als $a_{i1}$ und sie enthält wiederum keine streng monoton fallende oder konstante Teilfolge. In der so entstandenen Folge ist jedes enthaltene Element echt größer als $a_{i1}$. \\
Wir wiederholen das durchgeführte Verfahren und konstruieren so die streng monoton wachsende Teilfolge $a_{i1} < a_{i2} <...$.\\
}
%
\end{lemma}
%
Bernhard Hermann Neumann ein deutsch-englisch-australischer Mathematiker bewies in seinem Werk $\glqq$On ordered division rings$\grqq$ \cite[S. 206]{neumann49} die beiden folgenden wichtigen Lemmata, deren volle Bedeutung sich im Hauptteil \ref{eq: multPotenzreihenkoerper} erschließen wird. 
\begin{lemma}\label{wohlgeordnetwennkeineabfallendeFolge} %\cite{neumann49}
Die Menge $W$ ist genau dann wohlgeordnet, wenn jede Folge $\left(w_n\right)_{n\in\N}$ von Elementen aus W eine monoton steigende Teilfolge  $w_{\tau(1)} \le w_{\tau(2)} \le ...$ enthält.
\end{lemma}
\beweis{$``\Rightarrow "$ Sei die total geordnete Menge $W$ wohlgeordnet. Dann gilt nach Lemma \ref{unendlicheFolgeEigenschaften} und mit Satz \ref{wohlgeordnetabnehmendeFolge}, dass eine Folge $\left(w_n\right)_{n\in\N}$ von Elementen aus $W$ entweder eine streng monoton steigende oder konstante Teilfolge enthält. Jede Folge aus $W$ besitzt daher eine monoton steigende Teilfolge.\\
$``\Leftarrow "$ Jede Folge von Elementen aus $W$ enthält eine monoton steigende Teilfolge. Angenommen $W$ sei nicht wohlgeordnet. Nach Satz \ref{wohlgeordnetabnehmendeFolge} existiert somit eine streng monoton fallende Folge in $W$. Nach Voraussetzung muss jede Folge eine monoton steigende Teilfolge enthalten. Widerspruch, da eine streng monoton fallende Folge keine monoton steigende Teilfolge enthalten kann.}
Wir bezeichnen mit $W_G$ die Menge der wohlgeordneten Teilmengen einer angeordneten abelschen Gruppe $G$.
%
%

%Auskommentiert, weil gleiche Aussage wie von Neumann Lemma
%\begin{satz}\label{produktInWohlordnung}
%Sei $G$ eine angeordnete abelsche Gruppe. Seien $w_1, w_2 \in W_S$ dann gilt $w_1+w_2 \in W_G$. 
%\end{satz}
%\beweis{
%Angenommen $w_1+w_2$ liegt nicht in der Menge der wohlgeordneten Teilmengen $W_G$. Es gibt also eine streng monoton fallende Folge $a_1+b_1 > a_2+b_2 > ...$, wobei $a_i \in w_1, b_i \in w_2, \forall i \in \N$. Da $w_1$ wohlgeordnet ist, enthält die streng monoton fallende Folge $\left( a_n\right)_{n\ge1} $ nach Definition der Wohlordnung keine streng monoton fallende Folge. Nach \ref{unendlicheFolgeEigenschaften} gibt es eine streng monoton steigende oder konstante Teilfolge $a_{i1} \le a_{i2} \le ...$. Da aber gilt $a_{i1}+b_{i1} > a_{i2}+b_{i2} > ...$ erhalten wir eine  streng monoton fallende Folge $b_{i1} > b_{i2} > ...$ in $w_2$. Dies widerspricht der Tatsache, dass $w_2$ wohlgeordnet ist.}
%%die drei Lemmas aus: http://arxiv.org/pdf/math/0405133v1.pdf


\begin{lemma}[Lemma von B.H. Neumann] \label{LemmaNeumann} %\cite{neumann49}
Sei $G$ eine angeordnete Gruppe und $V, W \subseteq G$ wohlgeordnet, dann ist U = V + W ebenso wohlgeordnet. 
\end{lemma}
\beweis{
Sei \[u_1 = v_1 + w_1,~ u_2 = v_2+w_2 ~...\text{, mit } v_r \in V, w_r\in W \text{ und } r\in\N \] eine beliebige Folge von Elementen aus $U$. Es gibt eine Folge $v_1, v_2, ...$ mit $v_{\tau(1)}\le v_{\tau(2)} \le ...$ und zu der Folge $w_{\tau(1)}, w_{\tau(2)},...$ eine monoton steigende Teilfolge $w_{\tau(\sigma(1))}\le w_{\tau(\sigma(2))} \le ...$. Daraus folgt, es gibt zu der beliebigen Folge von Elementen aus $U$ ebenso eine monoton steigende Teilfolge $u_{\tau(\sigma(1))}\le u_{\tau(\sigma(2))} \le ...$, Nach Lemma \ref{wohlgeordnetwennkeineabfallendeFolge} ist $U$ damit wohlgeordnet.}
%
%
% 
%
%
%
%
\begin{folg}\label{FolgerungNeumann} %\cite{neumann49}
Seien V, W wohlgeordnete Teilmengen einer angeordneten Gruppe $G$, dann gibt es für ein $g \in G$ nur endlich viele Paare $\left(v, w\right) \in V\times W$ mit $v + w = g$.
\end{folg}
\beweis{Angenommen es gäbe unendlich viele $v_n \in V, w_n \in W$ wobei $g = v_n + w_n$, $n \in \N$, und die Folge besitzt paarweise verschiedene Glieder. Da $V$ und $W$ wohlgeordnet sind, besitzt weder $\left(v_n\right)_{n\in\N}$ noch $\left(w_n\right)_{n\in\N}$ eine streng monoton fallende Teilfolge. Nach Satz \ref{unendlicheFolgeEigenschaften} hat $\left(v_n + w_n\right)_{n \in \N}$ eine streng monoton steigende Teilfolge. Widerspruch, da $g$ konstant bleiben muss.
}
%
%
%
%
%
%
%
%
%
%
\section{Archimedisch angeordnete abelsche Gruppen}\label{Archimedisch angeordnete Gruppen}
%
Erst seit dem Ende des 19. Jahrhunderts kristallisierte sich die hohe Bedeutung geordneter Strukturen in der Mathematik heraus. Man erkannte, dass das archimedische Axiom unverzichtbar für die nähere Untersuchung dieses Bereichs war. Es spielte unter anderem eine wichtige Rolle bei der Entwicklung der reellen Zahlen mithilfe des Dedekindschen Schnittes (1872). Genau genommen ermöglicht die archimedische Eigenschaft die Herstellung von Kommutativität und Vollständigkeit. \\ %\cite{hahn07}\\
Wir orientieren uns an dem Kapitel $\glqq$Angeordnete Gruppen$\grqq$ in \cite[S. 73 - 93]{fuchs66}, sowie an Arbeiten von Prieß-Crampe \cite{priesscrampe69}, \cite{priesscrampe83}.

\begin{defn} \label{betrag}
Der \textit{absolute Betrag} $|a|$ eines Elements a $\in  G$, wobei $G$ eine angeordnete Gruppe sei, ist definiert als $|a| = \textup{max}\lbrace a, -a \rbrace$.
\end{defn}

Wenn die angeordnete Gruppe zusätzlich abelsch ist, gilt die \textit{Dreiecksungleichung} für alle $a, b \in G$:
\[|a+ b | \le |a| + |b|, \text{ für alle } a, b \in G.\]
Die Ungleichung gilt trivialerweise wenn beide Elemente das gleiche Vorzeichen haben. Sei also $a < 0$ und $b > 0$. Dann ist $a= -|a|$.\\
Falls $|a|\le |b|$ ist, erhalten wir
\[ |a + b |= |-|a|+b| = b - |a| \le b = |b|\le|a| + |b|. \]
Falls $|a| >b$ ist, erhalten wir 
\[|a+b| = |-|a| +b| = |a| - b \le |a|\le |a| + |b|.\]  
%
% 
%
%
%
%
%
\begin{defn}\label{archim}
Eine angeordnete abelsche Gruppe $\left(G,+\right)$ heißt \textit{archimedisch}, wenn es für alle $a, b \in G$  mit $0 < a < b$ ein $n \in \N $ gibt, mit $b < na$.
\end{defn}
%
%
\begin{defn}\label{uek}
Seien $a, b \in G$, wobei $G$ eine angeordnete Gruppe ist. Das Element $a$ wird als \textit{unendlich kleiner} als $b$ bezeichnet, wenn für alle $  n \in \N $ gilt: 
\[n|a| < |b|.\]
In Zeichen schreiben wir $a \ll b$.
\end{defn}
%
\begin{defn}\label{aae}
Sei $G$ eine angeordnete abelsche Gruppe, und $|a|$ der absolute Betrag eines Elements $a$ aus $G$. Zwei Elemente $a,b \in G$ werden als \textit{archimedisch äquivalent} bezeichnet, wenn natürliche Zahlen $m$ und $n$ existieren, so dass: 
\[|a| < m|b| \text{ und } |b| < n|a|.\]
In diesem Fall schreiben wir $ a \sim b $. 
\end{defn}
%
\begin{folg}
Für jedes Paar von Elementen $a, b \in G$ gilt genau eine der anschließenden Relationen: 
\begin{multicols}{3}
\begin{enumerate}
\item[(i)] $a \ll b$
\item[(ii)] $a \sim b$
\item[(iii)] $b \ll a$, 
\end{enumerate}
\end{multicols}
%
Des Weiteren schließen wir aus Definition \ref{uek} und \ref{aae}:
\begin{enumerate}
\item[(i)] Aus $a \ll b$ folgt $-g+a+g $ $\ll$ $-g+b+g$ für alle $g \in G$;
\item[(ii)] Aus $a \ll b$ und $a \sim c$ folgt $c \ll b$;
\item[(iii)] Aus $a \ll b$ und $b \sim d$, folgt $a \ll d$;
\item[(iv)] Aus $a \ll b$ und $b \ll c$ folgt $a \ll c$;
\item[(v)] Aus $a \sim b$ und $b \sim c$ folgt $a \sim c$.
\end{enumerate}
Sind alle Elemente einer Gruppe $G\setminus\lbrace 0 \rbrace$ archimedisch äquivalent, so ist die Gruppe \textit{archimedisch angeordnet}. \\ 
Durch die archimedische Äquivalenz werden die Elemente von $G$ in disjunkte Klassen unterteilt, die angeordnet werden können. Es bezeichne $[g]$ die \textit{archimedische Klasse} in der das Element $g \in G$ liegt, $[G]$ die Gesamtheit aller archimedischen Klassen von $G$. \label{archimedischeKlassen}\\
% evtl überflüssig: Sind zwei Elemente $a,~b \in G$ nicht archimedisch äquivalent, gilt entweder für alle
%$n \in \N$ ist $n|a|<|b|$, oder für alle $n \in \N$ ist $n|b| <|a|$.
\end{folg}
%
%\begin{satz}\label{agkku}
%Eine archimedische Gruppe G enthält keine konvexen Untergruppen außer sich selbst und der trivialen.
%\end{satz}
%
%
%\beweis{
%Angenommen es gibt eine Untergruppe $\left(U, \circ\right))$ $\subseteq$ G, mit $ U \neq \varnothing und U \neq G$. Nach den Untergruppenaxiomen gilt für $a, b \in U: a\circ b \in U$ Da G archimedisch geordnet ist und alle Elemente aus U natürlich auch in G liegen, muss für $a, b \in U \text{ gelten, wenn 0 < a < b dann gibt es ein n \in \N: } b < na.$ Da U echte Untergruppe von G, gibt es ein Element $x\in G, \text{ aber } x \notin U$. Da G archimedisch geordnet gibt es ein   }
%
%
%
% 
%
%
%
%\begin{satz}\label{archimedischangeordnet folgt abelsch} %\cite{pickert55} %bzw Priess Crampe
%Eine archimedische Gruppe $\left(G, +\right)$ ist eine abelsche Gruppe.
%\end{satz}
%\beweis{Falls $G$ ein kleinstes positives Element $z$ besitzt, so ist die von $z$ erzeugte Untergruppe $\langle z \rangle$ eine abelsche Gruppe, da jede aus einem Element erzeugte Menge eine Untergruppe ist. Nach Definition der Archimedizität \ref{archim} existiert für $0 < b \in G$ eine natürliche Zahl $n$ mit: 
%\[(n-1)\cdot z \le b < n\cdot z.\]
%Die Umkehrabbildung ${\lambda_x}^{-1} $ der bijektiven und ordnungstreuen Abbildung \[\lambda_x\colon G \to G,~ y \mapsto x+ y\] existiert und es gilt:
%\[{\lambda_x}^{-1}(a) ~ x \mapsto a-y \text{ und daher: }\]
%\[ 0 = \lambda_x{\left(n-1\right)\cdot z}^{-1}\left(\left( n-1\right) \right) ~\le ~\lambda_x{\left(n-1\right)\cdot z}^{-1}\left(b\right) ~<~ \lambda_x{\left(n-1\right)\cdot z}^{-1}(n\cdot z) = z. \]
%Da $z$ nach Voraussetzung das kleinste positive Element aus $G$ ist, erhalten wir:
%\[0 = \lambda_x{\left(n-1\right)\cdot z}^{-1}\left(b\right).\]
%Somit ist $b = (n-1)\cdot z$ und daher $L = \langle z\rangle$. $G$ ist also eine zyklische Gruppe, da sie von einem Gruppenelement erzeugt wird. Jede zyklische Gruppe ist abelsch und die Behauptung ist in diesem Fall gezeigt.\\
%Nun muss noch der Fall betrachtet werden, dass $G$ kein kleinstes positives Element besitzt. Zu jedem Element $0 < x \in G$ existiert also ein $c \in G$ mit: $0 < c < x$. Das bedeutet zwischen der Null und jedem beliebigen Element von $G$ gibt es stets noch ein Element, da kein kleinstes positives Element existiert. Es sei $x = \lambda_c{\lambda_c}^{-1}(x) = c + {\lambda_c}^{-1}(x)$, woraus wegen $c < x$ folgt, dass $0 < {\lambda_c}^{-1}(x)$ ist. Wähle $d \in G$, mit $0 < d \le $ min$\lbrace c, {\lambda_c}^{-1}(x)\rbrace$, dann gibt es für alle $x \in G$ ein derartiges Element mit $0 < 2d \le c + {\lambda_c}^{-1}(x) = x.$ Daraus folgt, es gibt natürlich auch ein Element $d' \in G$ mit $0 < 3d' \le x$.\\
%Angenommen $G$ ist nicht abelsch. Dann existieren Elemente $a, b \in G$ mit $a + b < b + a$ oder $b + a < a + b$. Wir schränken uns o.B.d.A auf den ersten Fall ein. Es sei $s \in G$ bestimmt als $s + \left(a + b\right) = 0$. Wir erhalten $0 < s + b + a$ und wir wissen, dass es ein Element $d'$ aus $G$ gibt, welches die Ungleichung $0 < 3\cdot d'\le s + \left(b + a\right)$ erfüllt. Weil $G$ archimedisch ist, gibt es ganze Zahlen $n_1, n_2$ mit $n_1 \cdot d' \le a < (n_1+1)\cdot d'$, $n_2 \cdot d' \le a < (n_2+1)\cdot d'$. Wir folgern:
%\[(n_1 + n_2) \cdot d' \le b + a < n_1 + n_2) \cdot d' + 3d' \text{ und }\]
%\[(n_1 + n_2) \cdot d' \le a + b < n_1 + n_2) \cdot d' + 3d'\].
%Aus der letzten Ungleichung folgt -($n_1 + n_2)\cdot d - 3d' < s \le - (n_1 + n_2) \cdot d$ und damit ergibt sich:
%\[s + (b + a) < 3d\]
%Wir erhalten einen Widerspruch zu unserer Annahme, daher ist die Aussage gezeigt.
%}
%
%
%
%
%
%
%
%
%
\section{Der Satz von Hölder}
In diesem Abschnitt werden wir einen sehr wichtigen Satz der Theorie angeordneter Strukturen vorstellen, den \textit{Satz von Hölder}. Dieser Satz besagt, dass jede archimedisch angeordnete Gruppe bis auf Isomorphie einer Untergruppe der additiven Gruppe der reellen Zahlen mit der Ordnung $``< "$ entspricht. Zwar wird der Beweis dieser Aussage in der verwendeten Literatur Otto Hölder (1901)\cite{hoelder1901} zugeschrieben, die grundlegenden Ideen dazu lieferte jedoch bereits Bettazi in seinem Werk $\glqq$Teoria delle grandezze$\grqq$, 1890\cite[S. 578]{Lueneburg08}.\\
Wir orientieren uns hierbei an \cite{hoelder1901} und \cite{priesscrampe83}. 
\begin{satz}\label{aga} %\cite{hoelder1901}
Eine angeordnete abelsche Gruppe ist genau dann archimedisch, wenn sie zu einer mit der natürlichen Ordnung versehenen Untergruppe der additiven Gruppe der reellen Zahlen o-isomorph ist.
\end{satz}
%
%
\beweis{
$``\Leftarrow "$ Die Rückrichtung ist klar, da jede Untergruppe der additiven Gruppe der reellen Zahlen archimedisch angeordnet ist und diese Eigenschaft durch den o-Isomorphismus ebenfalls für die angeordnete Gruppe $G$ gelten muss. \\
$``\Rightarrow "$ Sei $G$ eine angeordnete abelsche Gruppe. $G$ besitzt also einen Positivbereich $P$. Nach Voraussetzung erfüllt $G$ die archimedische Eigenschaft. Sei $G \neq \lbrace 0_G\rbrace$, wobei $0_G$ das neutrale Element der Addition in $G$ ist. Andernfalls wäre $G$ isomorph zu $\lbrace 0 \rbrace \subseteq \R$, der trivialen Untergruppe der additiven Gruppe der reellen Zahlen. \\
Somit ist $P$ nicht leer und wir nehmen ein Element $\alpha \in P$ beliebig. Für jedes $g \in G$ definieren wir:
\[S_g := \lbrace \frac{m}{n} \in \Q^{+} | m, n \in \N, m\alpha \le ng\rbrace\]
Für beliebige $m, n, p \in \N$ gilt die Äquivalenz $m\alpha \le n g \Leftrightarrow m p \alpha \le n p \alpha$. Die Darstellung von $r \in \Q^{+}$ als Quotient zweier natürlicher Zahlen hängt also nicht damit zusammen, ob $ r$ in $S_g$ enthalten ist.\\
Um die Behauptung zu zeigen, genügt es, einen Monomorphismus zu finden, der $G$ auf die additive Gruppe der reellen Zahlen abbildet. Wir zeigen nun folgende Aussagen: \\
\begin{enumerate}
\item[(i)] Für alle $g\in S_g$ gilt $S_g \neq \varnothing \text{ und } S_g \neq \Q^{+}$. Für $r, s \in \Q^{+} \text{ mit } r < s \text{ und } s \in S_g$ folgt $r \in S_g.$
\item[(ii)] Sei $S_g \subseteq \Q^{+}$, wobei $S_g$ nicht leer und beschränkt. Die Abbildung $\Phi: P \rightarrow \R^{+}, g \mapsto$ sup$\lbrace S_g \rbrace$ ist somit wohldefiniert.  
\item[(iii)]Für alle $g, h \in P$ gilt $g \le h $ genau dann, wenn $ S_g \subseteq S_h$ genau dann, wenn $\Phi(g) \le \Phi(h)$.  
\item[(iv)] Sei $g, h \in P$ und $r,s \in \Q^{+}$. Sei $r \in S_g \text{ und } s \in S_h \text{ so folgt } r+s \in S_{g+h}$. \\
Sei $r \notin S_g \text{ und } s \notin S_h \text{, so folgt } r+s \notin S_{g+h}$. 
\item[(v)] Es gilt $\Phi\left( g+h\right) = \Phi\left(g\right) + \Phi\left(h\right)$ für alle $g, h \in P$. 
\item[(vi)] $\Phi$ wird fortgesetzt auf die gesamte Gruppe $G$ durch $\Phi\left(0_G\right) = 0 \text{ und } \Phi \left(-g\right) = -\Phi \left(g\right)$ für alle g $\in P$. \\ 
\end{enumerate}
Insgesamt ist $\Phi$ ein Monomorphismus abelscher Gruppen und damit ist $G$ isomorph zu einer Untergruppe der additiven Gruppe der reellen Zahlen.\\ 
Zu (i): Wegen der archimedischen Eigenschaft gibt es zu jedem Element $g \in P$ ein $n \in \N$ mit $ng > \alpha$. Nach Definition von $S_g$ gilt $\frac{1}{n} \in S_g$ und somit ist $S_g$ nicht leer.\\ Angenommen $S_g = \Q^{+}$, dann wäre $n \in S_g$ und $n\alpha \le g$ für alle $n\in\N$, was ein Widerspruch zur archimedischen Eigenschaft ist.\\
Seien $r, s \in \Q^{+} \text{ mit } r < s \text{ und } s \in S_g \text { mit } r := \frac{k}{l} \text{ und } s:= \frac{m}{n}, \text{ wobei } m, n ,k, l \in \N$. Nach Voraussetzung gilt $kn < lm $ und da $s \in S_g: m\alpha \le ng$ und $kn\alpha \le lm\alpha \le lng.$ Daraus wiederum folgt: $\frac{kn}{ln} = r \in S_g$.\\
Zu (ii): Wir haben bereits gezeigt, dass $S_g$ nicht leer ist. Angenommen $S_g$ wäre unbeschränkt, dann gäbe es für jedes $n \in \N$ ein $r \in S_g$ mit n < r. Nach (i) folgt daraus $n \in S_g$ und für alle $n \in \N$ folgt: $ n\alpha \le g$, was ein Widerspruch zur archimedischen Eigenschaft ist.\\
Zu (iii): Zunächst beweisen wir die erste Implikation. Sei  $g, h \in P$ und es gilt $g \le h$. Sei $r \in S_g$, $r:= \frac{m}{n}, m, n \in \N $, dann erhalten wir $m\alpha \le ng$. Da $g \le h$ folgt $ m\alpha \le n h$ und damit $r \in S_h$. Die zweite Implikation folgt nach Definition von $\Phi$ offensichtlich.\\
Sei $\Phi\left(g\right) \le \Phi\left(h\right)$. Angenommen $g > h$, nach der archimedischen Eigenschaft gibt es ein $ n \in \N$ mit $n\left(g-h\right) > 2\alpha$. Wähle $m \in \N$ möglichst klein, mit $m\alpha < nh$. Es gilt $\frac{m}{n} \notin S_h$ und $\frac{m}{n} \geq \Phi\left(h\right)$
Da m minimal ist, gilt die Ungleichung $\left(m-1\right)\alpha \le nh$ und wir erhalten $ \left(m+1\right)\alpha \le nh + 2\alpha < nh + n(g-h) = ng$ und $\frac{m+1}{n} \in S_g,$ also $\frac{m+1}{n} \leq$ sup$\left(S_g\right) = \Phi\left(g\right)$. Insgesamt ergibt sich $\Phi\left(h\right) \le \frac{m}{n} < \frac{m + 1}{n} \le \Phi\left(g\right)$, Widerspruch.\\
Zu (iv): Sei $r \in S_g, s \in S_h \text{ mit } r := \frac{k}{l} \text{ und } s:= \frac{m}{n}, \text{ wobei } m, n ,k, l \in \N$. Dann gilt $k\alpha < lg $ und $ m\alpha \le nh$. Wir erhalten $kn\alpha \le lng \text{ und } lm\alpha \le lng.$ Somit liegt $r+1 = \frac{kn+lm}{ln}$ in der Menge $S_{g+h}.$ Die zweite Aussage folgt analog indem in Obigem $``\le "$ durch $``> "$ ersetzt wird.\\
Zu (v): Als erstes zeigen wir, dass $\Phi\left(g + h\right)$ eine obere Schranke von $S_{g+h}$ ist. Angenommen es gibt ein $r\in S_{g+h}$ mit $r > \Phi\left(g + h\right)$. Wähle $\epsilon = r - \Phi\left(g\right) - \Phi\left(h\right)$ und wähle s, t $\in \Q^{+}$ mit $\Phi(g) < s < \Phi(g) + \frac{\epsilon}{2} \text{ und } \Phi\left(h\right) < t < \Phi\left(h\right) +\frac{\epsilon}{2}$. Insgesamt folgt $s + t < \Phi\left(g\right) + \Phi\left(g\right) + \epsilon = r$. Da $s \notin S_g$ und $t \notin S_h$ gilt $s+t \notin S_{g+h}$ nach (i). Wir erhalten $s+ t \geq \Phi\left(g+h\right) \geq r$ und der Widerspruch $ r \le s+t < r$ zeigt, dass ein derartiges $r$ nicht existieren kann.\\
Es bleibt zu zeigen, dass $\Phi\left(g + h\right)$ die kleinste obere Schranke von $S_{g+h}$ ist. Angenommen es gäbe eine kleinere obere Schranke $ o \in \R^{+}$ und sei $\epsilon = \Phi(g) + \Phi(h) - o$. Nach Definition der Abbildung gibt es ein $r\in S_g$ mit $r < \Phi(g) - \frac{\epsilon}{2}$ und $ s \in S_h$ mit $s < \Phi(h) - \frac{\epsilon}{2}.$ Nach (iv) ist $r+s$ in $S_{g+h} $ und daher $ r +s \le o$. Widerspruch, da $r+s >\Phi(g) + \Phi(h) - \epsilon = o$.\\
Zu (vi): Falls $g, h > 0_{G}$ wurde die Äquivalenz $g\le h \Leftrightarrow \Phi(g) + \Phi(h)$ bereits gezeigt. Die Aussage ist offensichtlich, wenn eines der beiden Elemente $g$ oder $h$ gleich Null ist. Sei $g < 0_G \text{ und } h > 0_G$, dann folgt die Behauptung nach Definition. Die Aussage bleibt für $g, h < 0_G$ zu zeigen: 
\[g \le h \Leftrightarrow -g \geq -h \Leftrightarrow \Phi\left(-g\right) \geq \Phi\left(-h\right)n \Leftrightarrow \Phi\left(g\right) \le \Phi\left(h\right).\]
Wir zeigen nun $\Phi(g+h) = \Phi\left(g\right) +\Phi\left(h\right)$ für beliebige $g,h \in G$. Es genügt dies für $g, h < 0_G$ zu beweisen. Hier kann auf das bereits Bewiesene zurückgegriffen werden: 
\[\Phi\left(g+h\right) = −\Phi\left((−g)+(−h)\right) = (−\Phi(−g))+(−\Phi(−h)) = \Phi(g)+\Phi(h).\]
Zunächst betrachten wir den Fall $g\geq −h$. Dann ist $g+h \geq 0_G$, und nach der bereits gezeigten Aussage folgern wir:\\
$\Phi(g+h)+ \Phi(−h) = \Phi(g) \Leftrightarrow \Phi(g+h)−\Phi(h) = \Phi(g) \Leftrightarrow \Phi(g+h) = \Phi(g)+ \Phi(h)$.\\
Setzen wir nun $g < −h$ voraus. Dann ist $−g−h > 0$, also $\Phi(g)+\Phi(−g−h) = \Phi(−h)$, was äquivalent zu $\Phi(g)−\Phi(g+h) = −\Phi(h)$ und zu $\Phi(g)+ \Phi(h) = \Phi(g+ h)$ ist.
}
%Beweis über archimedische Eigenschaft, o-Isomorphie einer einelementigen Gruppe zu der Gruppe der ganzen Zahlen (\cite{priesscrampe69} S. 8), Kommutativität von G, Dedekindschen Schnitt und Homomorphismus (\cite{fuchs66} S. 75) } 
%
%TODO : Nun stellt sich die Frage, wann 
\begin{satz}\label{homomorphismus nach R} %\cite{fuchs66}
Sei $\left(G, +\right)$ eine Untergruppe der natürlich geordneten additiven Gruppe der reellen Zahlen und $\phi: G \rightarrow \R$ ein injektiver o-Homomorphismus. Dann gibt es eine positive reelle Zahl $r$ mit $\phi(g) = r\cdot g$ für alle $g \in G$. 
\end{satz}
\beweis{Nach Voraussetzung ist $\phi$ ein injektiver o-Homomorphismus und damit sind mit $0 < g_1$, $ g_2\in G $ auch $\phi(g_1)$ und $\phi(a_2)$ positiv. Angenommen es gilt, dass $\frac{\phi(g_1)}{\phi(g_2)}\neq \frac{g_1}{g_2}$, so gibt es eine rationale Zahl $\frac{m}{n}$ mit $m, n \in \N$, die zwischen $\frac{\phi(g_1)}{\phi(g_2)}$ und $ \frac{g_1}{g_2}$ liegt. Wir nehmen ohne Beschränkung der Allgemeinheit an, dass $\frac{\phi(g_1)}{\phi(g_2)} < \frac{m}{n} < \frac{g_1}{g_2}$ ist. Weiterhin gehen wir davon aus, dass $n \cdot g_1 > m\cdot g_2$ ist. Nach der archimedischen Eigenschaft der Gruppe $G$ stehen die Bilder $\phi(n\cdot g_1)$ und $\phi(m\cdot g_2)$ in umgekehrter Größenbeziehung zueinander. Dies steht jedoch im Widerspruch zur Ordnungstreue der Abbildung $\phi$. 
Folglich ist $\frac{\phi(g_1)}{\phi(g_2)}\neq \frac{g_1}{g_2}$. Wir erhalten somit auch für alle positiven Elemente $g \in G$, dass die Gleichung $\frac{\phi(g_1)}{g_1} = \frac{\phi(g)}{g}$ erfüllt ist. \\
Für die negativen Gruppenelemente $g \in G$, mit $g < 0$  und daher $-g >0$ erhalten wir aufgrund der Homomorphismuseigenschaften $\frac{\phi(g)}{g} = \frac{(-1)\cdot\phi(g)}{(-1)\cdot g} = \frac{\phi(-g)}{-g} = \frac{\phi(g_1)}{g_1}$. Mit der positiven Konstanten $r := \frac{m}{n}$ ist die Aussage $\phi(g) = r \cdot g$ für alle $g \in G$ gezeigt. 
}
Die Grundaussage dieses Satzes bewies erstmals Hion 1954 in seinem russischsprachigen Werk $\glqq$Archimedisch geordnete Ringe$\grqq$. Er setzte jedoch einen o-Homomorphismus zwischen zwei Untergruppen der additiven angeordneten Gruppe der reellen Zahlen voraus, ebenso wie Fuchs und Prieß-Crampe, die den Satz in ihre Arbeiten mitaufnahmen.  Der Satz \ref{homomorphismus nach R} impliziert weiterhin die o-Isomorphie zwischen der Gruppe der ordnungserhaltenden Automorphismen der archimedischen Gruppe und der multiplikativen Gruppe der positiven reellen Zahlen. \cite{priesscrampe83}
%
%
%
%
%
%TODO ABELSCH MACHEN!!!
\section{Die angeordnete Menge konvexer Untergruppen}
In diesem Abschnitt geht es um konvexe Untergruppen einer angeordneten abelschen Gruppe. Wir benötigen einige Eigenschaften dieser Menge an speziellen Untergruppen für den Nachweis des Inversen im verallgemeinerten Potenzreihenkörper. Untergruppen teilweise geordneter Gruppen besitzen eine durch die teilweise Gruppenordnung induzierte teilweise Ordnung. Wir bezeichnen die Untergruppen als angeordnet, falls die ursprüngliche teilweise Ordnung ebenso eine Anordnung war.\\
Sei $\left(G, +\right)$ eine Gruppe, $U$ eine Untergruppe und $g \in G$. 
Wir untersuchen nun die bezüglich der Inklusion linear angeordnete Menge $\Sigma$ konvexer Untergruppen von $G$. Wir orientieren unsere Ausführungen an \cite[S. 81 - 83]{fuchs66}  und \cite[S. 3]{priesscrampe83}.

\begin{defn}\label{konvexUGR} %\cite{priesscrampe69}
Eine Untergruppe $U$ einer angeordneten abelschen Gruppe $G$ nennen wir \textit{konvex}, wenn aus $a \in U$, $x \in G$, mit $0 < |x| < |a|$ folgt $x \in U$.\\
\end{defn}
$``\Sigma "$ bezeichne nun die \textit{Menge konvexer Untergruppen} einer angeordneten abelschen Gruppe $\left(G, +\right)$. 
\begin{defn}\label{Sprung} %\cite{fuchs66}
Sei $C, D \in \Sigma$. Wenn $D \subset C$ und $\Sigma$ keine weitere Untergruppe zwischen $C$ und $D$ enthält, nennen wir das Paar $C,~D$ \textit{Sprung} in $\Sigma$ und bezeichnen es mit $D \prec C$.
\end{defn}
%
% 
%
%
 %evtl. auch \cite{priesscrampe83}  Nach \cite[S. 81 - 83]{fuchs66}
\begin{satz}\label{EigenschaftenKonvexeUgr}
Die Menge der konvexen Untergruppen $\Sigma$ besitzt folgende Eigenschaften:
\begin{enumerate}
\item[S1:] Die Vereinigung und der Durchschnitt beliebig vieler Untergruppen aus $\Sigma$ liegen wieder in $\Sigma$.
\item[S2:] Ist $C \in\Sigma$ und $g \in G$, so ist $-g+C+g\in \Sigma$
\item[S3:] Sei $D \prec C$ in $\Sigma$, so ist $D$ normal in $C$ und $C/D$ ist isomorph zu einer Untergruppe der reellen Zahlen.
\end{enumerate}
\end{satz}
\beweis{
Zu S1: Seien $C, D \in \Sigma$ konvexe Untergruppen der angeordneten Gruppe $G$ und sei $c\in C, c\notin D$. Wir nehmen ohne Beschränkung der Allgemeinheit an, $c$ ist bezüglich der Anordnung von $G$ größer als das neutrale Element $0_G$. Da $c$ nicht in $D$ liegt, kann es kein Element $d \in D$ geben, sodass $0_G < c < d$, da in diesem Fall $c$ in $D$ liegen würde nach der konvexen Eigenschaft. Dies ist ein Widerspruch zur Voraussetzung und daher gilt $D \subseteq C$. Damit folgt unmittelbar, dass sowohl der Schnitt konvexer Untergruppen wieder angeordnet und konvex ist, als auch die Vereinigung.\\
Zu S2: Für $g \in C$ ist offensichtlich $g^{-1}Cg = C \in \Sigma$. Falls $g \notin C$, so ist $g^{-1}Cg$ eine Untergruppe von $G$, denn für alle $c_1, c_2 \in C, g \in G$ ist $g^{-1}c_1g \cdot g^{-1}c_2g = g^{-1}c_1c_2g \in g^{-1}Cg$ und ${g^{-1}c_1g}^{-1} = g^{-1}{c_1}^{-1}g \in g^{-1}C_1g$. Die Anordnung von $G$ überträgt sich auf $g^{-1}C_1g$ und die Untergruppe ist konvex, da $C$ nach Voraussetzung und $G$ als triviale Untergruppe konvex ist.\\
Zu S3: Nach Voraussetzung gilt $D \prec C$ und offensichtlich erfüllt jedes Element $g \in G$ die Bedingung $g^{-1}D_1g \prec  g^{-1}C_1g$. Weiterhin erhalten wir im Fall $g\in C$, dass  $g^{-1}C_1g = C$, und da $D \subset C$ ist $g^{-1}D_1g = D$. Infolgedessen ist $D$ normal in $C$ und die Faktorgruppe $C/D$ enthält, da in $\Sigma$ keine Untergruppe zwischen $C$ und $D$ existiert, dementsprechend nur die trivialen konvexen Untergruppen. In $C/D$ ist für jedes $c \in C/D$ die Menge $\lbrace g \in C/D: \exists_{m,n \in \Z} m\cdot a \leq g \le n\cdot a\rbrace$. Damit ist $C/D$ archimedisch und nach Satz von Hölder \ref{aga} isomorph zu einer Untergruppe der additiven Gruppe der reellen Zahlen.}
%
%
%
%
%
%
%
%
%
\section{Einblick in die Bewertungstheorie}
Im Nachfolgenden betrachten wir eine angeordnete abelsche Gruppe $\left(G,+\right)$ und eine angeordnete Menge $\Theta$ mit $0$ als kleinstem Element. Die Ausführungen sind orientiert an dem Kapitel $\glqq$Archimedische Klassen, Bewertungen und Bedingungen für die Anordnungsfähigkeit von Gruppen$\grqq$ in \cite[S. 9 - 11]{priesscrampe83}.
%
%
\begin{defn} %\cite{priesscrampe83}\label{bew}
Sei $G$ eine angeordnete abelsche Gruppe. Die surjektive Funktion $v\colon G \rightarrow \Theta$ wird als \textit{Bewertung} bezeichnet, wenn die folgenden Eigenschaften erfüllt sind:
%
\begin{enumerate}
\item[B1:] $v{(a)} = 0 \Leftrightarrow a = 0$ für alle $a\in G$,
\item[B2:]  $v{(a)} = -v{(a)} \text{  } \text{ für alle } a \in G $,
\item[B3:] $ v{(a+ b)} \le$ max$\{(v{(a)}, v{(b)}\}$ für alle $ a, b \in G$ .
\end{enumerate}
%
\end{defn}
Die Gleichheit in der Bedingung [B3] gilt dann, wenn $v{(a)} \ne v{(b)} $ ist.\\
 Zwei Bewertungen $\upsilon, \text{ } \upsilon' $ auf $G$ mit den Wertemengen $\Theta , \Theta' $ sind äquivalent, wenn es eine ordnungstreue bijektive Abbildung $\sigma \colon \Theta \text{ } \rightarrow \text{ } \Theta' $ gibt, so dass $ \sigma \circ \upsilon = \upsilon  $ ist.\\
Sei $\left(G, +\right)$ eine angeordnete Gruppe. Wir bezeichnen mit $[a]$ die archimedische Klasse, in der das Element $a\in G$ liegt. Die Gesamtheit der archimedischen Klassen von $G$ nennen wir $[G]$. Die Abbildung 
 \\$G \to [G] \colon a \mapsto [a]$ definieren wir als \textit{natürliche Bewertung}. %\cite{priesscrampe83}
\begin{defn}  \label{bewKoerper} %vorher nach Priess Crampe - Problem multiplikativ jetzt nach wikipedia
Sei $K$ ein Körper, ($G, +$) eine angeordnete abelsche Gruppe und $\overline{G}  = G \cup \lbrace\infty\rbrace $. Eine Abbildung $v\colon K \to \overline{G} $ wird als \textit{Bewertung eines Körpers} bezeichnet, wenn sie folgende Bedingungen erfüllt:
%TODO: Quelle checken
\begin{enumerate}
\item[B1':] $v(a) = \infty$ genau dann, wenn $a = 0$ ist,
\item[B2':] $v(ab) = v\left(a\right)+v\left(b\right) $ für alle $ a, b \in K$,
\item[B3':] $v\left(a+b\right) \ge $ min$\lbrace v(a),v\left(b \right)\rbrace \text{ für alle }  a, b \in K. $
\end{enumerate}
\end{defn}
%Ein Beispiel für eine Bewertung ist die Polordnung meromorpher Funktionen in einem festen Punkt, wie im Hauptteil \ref*{LaurentreiheBewertung} noch erörtert wird. 
Man bezeichnet $v\colon K \to \overline{G} $ als \textit{diskrete Bewertung}, falls ${G} = \Z$ ist.
%
\begin{defn} %nach Priess Crampe S39
Der Unterring $A$ eines Körpers $K$ wird als \textit{Bewertungsring} bezeichnet, wenn für jedes $a\in K$ gilt $a \in A$ oder $a^{-1}\in A$.
\end{defn}
\begin{bem}
Ist $\left(K, v\right)$ ein bewerteter Körper, dann ist $A = \lbrace a \in K\colon v(a) \geq 0_K\rbrace$ ein Bewertungsring.
\end{bem}
\begin{defn}   %\cite{hulek12}
Ein Integritätsring $R$ heißt \textit{diskreter Bewertungsring}, falls es auf dem Quotientenkörper \textup{Quot}$(R)$ von $R$ eine Bewertung $v: \textup{Quot}(R) \rightarrow \Z \cup \lbrace\infty\rbrace$ gibt und 
%\begin{enumerate}
%\item[D1: ]$v(ab) = v(a)+ v(b)$, 
%\item[D2: ] $v(a+b) \ge$ min$\lbrace v(a), v(b)\rbrace$,
%sodass $R$ der Bewertungsring von $v$ ist. Das bedeutet:
%\end{enumerate}
\[R = \lbrace x \in \textup{Quot}(R): v(a) \ge 0 \rbrace \cup \lbrace 0 \rbrace\]
ist der Bewertungsring von $v$ ist.
\end{defn}
Eine weitere, äquivalente, Definition eines diskreten Bewertungsrings findet man in \cite[S. 126]{neukirch92}.
\begin{defn} \label{bewertungsring}%\cite{neukirch92}
Ein \textit{diskreter Bewertungsring} ist ein Hauptidealring mit einem einzigen maximalen Ideal $\mathfrak{p}$.
\end{defn} 

%
\chapter{Potenzreihenkörper}\label{chap3}
%
Die aus der Analysis bekannten Potenzreihen stellen ein bekanntes und wichtiges Werkzeug dar. In mathematischen Gebieten, wie der Kombinatorik, Automaten- und Kontrolltheorie ermöglichen sie sowohl eine kompakte Darstellung von Summenformeln, als auch deren Auffindung. Potenzreihen können ebenso über den Weg der Algebra definiert werden, durch die Folge ihrer Koeffizienten. Die algebraische Sichtweise zieht den neuen Aspekt mit sich, dass grundsätzlich auf Konvergenzbetrachtungen verzichtet wird und dadurch auf beliebigen Körpern und Ringen gearbeitet werden kann. \\
Diese sogenannten formalen Potenzreihen in einer Unbekannte $z$, auf den natürlichen Zahlen, deren Koeffizienten in einem beliebigen Körper K liegen, bilden einen Ring $K[[z]]$. Aufbauend darauf stellen wir einen Zusammenhang zu den, in der Funktionentheorie häufig verwendeten, Laurentreihen her. Der Ring formaler Potenzreihen ist ein Integritätsring, woraus folgt, dass dieser in einen kleinsten Körper eingebettet werden kann. Dieser Quotientenkörper von $K[[z]]$ entspricht genau dem Körper, den die Laurentreihen $K((z))$ formen. \\
Potenzreihen bilden somit algebraische Strukturen, deren Beschaffenheit von dem Träger der Reihen abhängt. Daher stellt sich die Frage, ob die formalen Potenzreihen weiter verallgemeinert werden können und welche Voraussetzungen der Träger erfüllen muss, damit diese allgemeinen formalen Potenzreihen einen Körper ergeben. %Das Kapitel endet mit dem Beweis der zentralen Aussage, dass die Menge der formalen Potenzreihen auf einer angeordneten Gruppe über einem beliebigen Körper, unter der Voraussetzung eines wohlgeordneten Trägers, ein Körper ist.

%Bevor wir mit der allgemeinen Untersuchung von Potenzreihenkörpern beginnen, wird in diesem Kapitel ein wichtiges Beispiel von Ringen eingeführt. Zunächst wird die Menge der formalen Potenzreihen definiert und nachgewiesen, dass es sich bezüglich komponentweiser Addition und Faltung um einen Ring handelt. Anschließend beschäftigen wir uns mit dem Körper der Laurentreihen K((z)), der dem Quotientenkörper des Ringes der formalen Potenzreihen entspricht. Die genauere Analyse der Eigenschaften des Trägers der Elemente des Körpers der Laurentreihen zeigt, dass dieser auch über einer angeordneten Gruppe definiert sein kann und trotzdem durch Einbettung des Ringes ein Körper entsteht. Der dadurch entstandene Körper wird als \textit{allgemeiner Potenzreihenkörper} bezeichnet. \\ 
\section{Der Ring der formalen Potenzreihen}\label{potenzreihenring}

%
% kurze einführung in die potenzreihen mit Defintion einer formalen Potenzreihe und Eingliederung in Ring und quotientenkörper.
Wir betrachten im Folgenden die Menge der formalen Potenzreihen $K[[z]]$ über einem beliebigen Körper $K$. Dabei repräsentiert $z$ keine Variable, die für eine Zahl steht, sondern eine Unbestimmte.  
\begin{defn}
Eine \textit{formale Potenzreihe} über $K$ ist eine Abbildung $\N_0 \rightarrow K$, $n \mapsto a_n$.
\end{defn}
\begin{bem}
Wir werden formale Potenzreihen im Folgenden immer als Ausdruck der Form
\begin{equation}\label{eq: formalepotenzreihe}
\sum_{n=0}^\infty a_n z^n = a_0 + a_1z + a_2z^1 + a_2z^2 + ...
\end{equation}
schreiben, mit $a_n \in K,~ \forall n \in \N_0$.
\end{bem}

Wir bezeichnen die Menge der formalen Potenzreihen in $z$ auf $\N_0$ über $K$ mit \[K [[z]] = \lbrace \sum_{n=0}^\infty a_n z^n \vert a_n\in K \rbrace \]. 

\subsection{Addition und Multiplikation formaler Potenzreihen} \label{Rechnen}
%Im Folgenden werden Addition und Multiplikation in $K[[z]]$ definiert. Mit diesen Verknüpfungen wird $K[[z]]$ zu dem Ring der formalen Potenzreihen. \\
Formale Potenzreihen werden komponentenweise addiert.
%
%
%
%
\begin{defn}\label{AdditionMultiplikationPotenzreihen}
%
Seien $f = \sum_{n=0}^\infty a_n z^n$ und $g = \sum_{n=0}^\infty b_n z^n$ zwei formale Potenzreihen über $K$. Wir definieren ihre \textit{Summe} $f+g$ folgendermaßen:
\begin{eqnarray*}
+ \colon K [[z]] \times K [[z]] \to K[[z]]:&&\left( \sum_{n=0}^\infty a_n z^n \right) + \left( \sum_{n=0}^\infty b_n z^n \right) \\
&=& \sum_{n=0}^{\infty} (a_n + b_n) z^n 
\end{eqnarray*}
%
%
% 
%
%
Die Multiplikation zweier formaler Potenzreihen $f,g$ erfolgt durch die sogenannte Faltung:
\begin{eqnarray*}
\cdot\colon  K [[z]] \times K [[z]] \to K[[z]]:&& \left( \sum_{j=0}^\infty a_j z^j \right)\cdot \left( \sum_{k=0}^\infty b_k z^k \right) \\
&=& \sum_{n=0}^\infty \left(\sum_{j+k=n} a_j b_k\right) z^n \\
&=& \sum_{n= 0}^\infty \left(a_0b_n + a_1b_{n-1} + a_2b_{n-2} + ... + a_nb_0 \right)z_n
\end{eqnarray*}
\end{defn}
%
\vspace{0.8cm}
%
%
%
% 
%
%Satz mit obigen Verknüpfungen ist das Potenzreihenring
\begin{satz}\label{RingFormalerPR}
Die Menge $\left(K\lbrack\lbrack z\rbrack\rbrack, +, \cdot\right)$ ist mit obigen Verknüpfungen ein kommutativer Ring.
\end{satz}
\beweis{Wir weisen die Ringaxiome, wie in \ref{Ring} definiert, nach.\\
Die Assoziativität und Kommutativität der Menge $\left(K\lbrack\lbrack z\rbrack\rbrack, +\right)$ lässt sich leicht nachprüfen.
Sei $ f = \sum_{n=0}^\infty  a_n z^n$. Das \textit{neutrale Element der Addition} $0_K$ ist die Nullreihe $ g(z) := \sum_{n=0}^\infty  b_n z^n$, wobei $b_n= 0 \text{ für alle n } \in \N_0 $. Denn wir erhalten als Summe von $g$ und $f$: 
 \begin{eqnarray*}
 f + g&=& \sum_{n=0}^\infty a_nz^n + \sum_{n=0}^\infty b_nz^n \\
 &=& \sum_{n=0}^\infty \left(a_n+b_n\right)z^n \\
 &=& \sum_{n=0}^\infty a_nz^n.
 \end{eqnarray*}
Wir bezeichnen $ -f = \sum_{n=0}^\infty  (-a_n)z^n$ als das \textit{Inverse der Addition}, denn es gilt
 \begin{eqnarray*}
 f + (-f) &=& \sum_{n=0}^\infty  a_nz^n + (\sum_{n=0}^\infty  (-a_n)z^n)\\
 &=& \sum_{n=0}^{\infty}(a_n-a_n)z^n \\
 &=& 0_K.
 \end{eqnarray*}
 \\
$\left(K\lbrack\lbrack z\rbrack\rbrack, +\right)$ ist daher eine abelsche Gruppe. 
Die Assoziativität der Multiplikation und die Distributivgesetze rechnen wir nach. \\
Seien $f, g, h \in K\lbrack\lbrack z\rbrack\rbrack$, mit $f = \sum_{n=0}^\infty a_n z^n$, $g = \sum_{n=0}^\infty b_n z^n$ und $h = \sum_{n=0}^\infty c_n z^n$.
\begin{eqnarray*}
f \cdot \left( g\cdot h\right)& =& f \cdot \sum_{n=0}^\infty \left(\sum_{j+k=n} b_j c_k\right) z^n \\
&=& \sum_{n=0}^\infty \left(\sum_{l+j+k=n} a_l b_jc_k\right) z^n \\
&=& \sum_{n=0}^\infty \left(\sum_{j+k=n} a_j b_k\right) z^n \cdot h \\
&=& \left(f \cdot g\right) \cdot h.
\end{eqnarray*} 
Das \textit{neutrale Element der Multiplikation} ist die Einsreihe $1_K$. Darunter verstehen wir diejenige Reihe, bei der nur der konstante Koeffizient $a_0 = 1$ und alle anderen gleich $0$ sind: 
 \begin{eqnarray*} 
 g &=& \sum_{n=0}^\infty  a_nz^n,
 \end{eqnarray*}
wobei 
 \[a_0 = 1 \text{ und } a_n = 0 \text{ für alle n } \in \N\].\\ 
Damit folgt: $ \sum_{j=0}^\infty a_jz^j \cdot \sum_{n=0}^\infty b_nz^n = \sum_{n=0}^\infty \sum_{j+k=n} \left(a_j\cdot b_k\right)z^n = \sum_{n=0}^\infty b_nz^n. $\\ 
Die Multiplikation ist kommutativ, denn die Addition und Multiplikation in dem Körper $K$ sind kommutativ. Es genügt somit ein Distributivgesetz nachzuweisen. Es gilt
\begin{eqnarray*}
f\cdot \left(g + h\right) &=& f \cdot \sum_{n=0}^\infty \left(b_n + c_n\right) z^n\\
&=& \sum_{n=0}^\infty \left(\sum_{j+k=n} a_j \left(b_k +c_k\right)\right) z^n\\
&\stackrel{\mathrm{(*)}}=& \sum_{n=0}^\infty \left(\sum_{j+k=n} a_j b_k +\sum_{j+k=n} a_j c_k\right) z^n\\
&=& \sum_{n=0}^\infty \sum_{j+k=n} a_j b_k z^n +\sum_{n=0}^\infty \sum_{j+k=n} a_j c_k z^n\\
&=& f\cdot g + f\cdot h,
\end{eqnarray*}
wobei (*) aufgrund der Distributivität in $K$ folgt.\\
 }
%
%
%
%
%
%
\begin{satz} 
Sei $f, g \in K\lbrack\lbrack z\rbrack\rbrack$, mit $f = \sum_{n=0}^\infty a_n z^n$ und $g = \sum_{n=0}^\infty b_n z^n$. Wir bezeichnen $g$ als die \textit{Inverse Potenzreihe} von $f$, wenn für
\begin{eqnarray*}
fg &=& \left( \sum_{j=0}^\infty a_j z^j \right) \left( \sum_{k=0}^\infty b_k z^k \right)\\  
&\stackrel{\mathrm{def}}=& \sum_{n=0}^\infty\sum_{j+k=n} (a_j b_k) z^n\\ 
\end{eqnarray*}
%
und 
\begin{eqnarray*}
\sum_{n= 0}^{\infty} c_nz^n \text{, mit } c_n = \sum_{j+k=n} (a_j b_k),
\end{eqnarray*}
gilt, dass $c_0 =1$ und $c_n = 0$ für alle $n\in\N$ ist.
%Sei $c_n = \sum_{j+k=n} (a_j b_k)$. Damit das Produkt der Potenzreihen dem neutralen Element der Multiplikation entspricht, müssen alle Koeffizienten mit Indizes größer Null den Wert $0$ annehmen, während $c_0 =1$ gilt. 
\end{satz}
\beweis{Sei $h =\sum_{n=0}^\infty\sum_{j+k=n} (a_j b_k) z^n$. Wir können beliebig viele Koeffizienten aus dieser Summe herausziehen nach Definition der Addition:
\begin{eqnarray*}
h &=& \sum_{n=0}^\infty\sum_{j+k=n} (a_j b_k) z^n \\
&=& a_0b_0z^0 + \sum_{n=1}^\infty\sum_{j+k=n} (a_j b_k) z^n \\
&=& a_0b_0 + a_1b_1z^1 + \sum_{n=2}^\infty\sum_{j+k=n} (a_j b_k) z^n\\
&=& ...
\end{eqnarray*} 
Nach Definition der formalen Potenzreihe stellt $z$ eine Unbestimmte dar. Man sieht leicht, dass die Summe nur den Wert $1_K$ annimmt, falls $a_0b_0= 1_K$ erfüllt.}
%
%
%
Wir zeigen zunächst, dass zu einer formale Potenzreihe $f = \sum_{j=0}^\infty a_j z^j$ genau dann die inverse Potenzreihe existiert, wenn $a_0 \neq 0 $.
%
%
%
\begin{satz}\label{potenzreihenringEinheit}
Sei $K[[z]] $ der \textit{Ring der formalen Potenzreihen}. Dann ist eine formale Potenzreihe $f = \sum\limits_{n=0}^{\infty}a_nz^n $ genau dann eine Einheit, wenn $a_0 \neq 0$ ist.\\ 
\end{satz}
\beweis{$\grqq\Leftarrow\grqq$
Sei $f = \sum\limits_{n=0}^{\infty}a_nz^n $ und es gelte $a_0 \neq 0$. Wir wollen zeigen, dass das Produkt der formalen Potenzreihen $f, g$ mit $g = \sum_{k=0}^{\infty}b_kz^k$ den Wert $1_K$ annimmt und somit $f$ eine Einheit ist. Wir müssen nun eine entsprechende Potenzreihe $g$ finden, sodass $f\cdot g= \sum_{j=0}^{\infty}a_jz^j \sum_{k=0}^{\infty}b_kz^k = \sum_{n=0}^{\infty}\sum_{j+k=n}\left(a_jb_k\right)z^n\stackrel{\mathrm{!}}=1$ ist. \\
Wir beweisen die Rückrichtung mithilfe des Prinzips der Induktion: \\ 
Für $b_0$ muss die Gleichung $a_0b_0= 1$ erfüllt sein. Da $a_0$ ungleich null ist besitzt die Gleichung eine eindeutige Lösung, nämlich $ b_0 = a_0^{-1}.$ \\
Angenommen es existiert ein $ b_k$ mit $ k < n$, sodass alle $a_j b_k$, für $1\le m < n$, gleich $0$ sind. Für den n-ten Koeffizienten ergibt sich $0 =  a_0b_n + a_1b_{n-1} + ... + a_{n-1}b_1 + a_nb_0$. Bis auf $b_n$ sind alle Werte festgelegt. Da $ a_0$ ungleich $0$ ist, ist die Lösung für $ b_n $ eindeutig. \\
$\grqq\Rightarrow\grqq $ Es gilt $\sum_{j=0}^{\infty}a_jz^j \sum_{k=0}^{\infty}b_kz^k = 1. $ \\
Nach Voraussetzung folgt $\sum_{j+k=n}a_jb_k = 0 $ für $n > 0$.  Daraus erhalten wir unmittelbar $ a_0b_0 = 1 $. Somit muss $a_0$ ungleich $0$ sein.}
%
%
Wir haben gezeigt, dass die Einheiten des Potenzreihenrings genau die Elemente sind deren konstanter Term ungleich $0$ ist. In diesem Fall können wir die inverse Potenzreihe  konstruieren.\\
\begin{satz}\label{inverse Potenzreihe}
Sei $f = \sum\limits_{n=0}^{\infty}a_nz^n $ und $a_0 \neq 0$. Die inverse Potenzreihe
$g = \sum_{k=0}^\infty b_k z^k$ ist rekursiv definiert durch
\begin{equation*}
b_0 = \frac{1}{a_0} ~~~~~ \text{ und }~~~~~ b_n = -\frac{1}{a_0}\sum\limits_{k = 1}^{n} a_k b_{n-k}	 ~~~~~~\forall n\in\N.
\end{equation*}
\end{satz}
\beweis{Wie im Beweis \ref{potenzreihenringEinheit} verwendet, gilt $a_0b_0 = 1$, woraus $b_0 = \frac{1}{a_0}$ folgt. Für die restlichen Koeffizientenwerte muss dementsprechend
\begin{equation*}
\sum_{j+k=n} (a_j b_k) = \sum_{k=0}^{n} a_nb_{n-k} = 0
\end{equation*}
für alle $n \in \N$ gelten. Es lässt sich leicht nachrechnen, dass das Produkt aus $f$ und der gewählten Potenzreihe $g$ diese Bedingung erfüllt und damit $fg = 1$ gilt. %TODO: evtlf,g ausschreiben und schöner formulieren?

}
%
\begin{bsp} %\cite{taraz12}
Es sei $q \in \R$ beliebig und $A = \sum_{}^{}a_n z^n$ mit $a_n = q^n$ gleich der geometrischen Reihe. Wir bestimmen die inverse Potenzreihe $B = \sum_{}^{} b_n z^n$. Dazu wenden wir die Formel aus~\ref{inverse Potenzreihe} an:
\begin{center}
\begin{description}
\item $b_0 = \frac{1}{a_0} = 1$,
\item $b_1 = -a_1b_0 = -q$,
\item $b_2 = -\left(a_1b_1 + a_2b_0\right) = -\left(-q^2 + q^2\right) = 0$,
\item ...
\item $b_n = -\left(a_1b_{n-1} + a_2b_{n-2} + ... + a_{n-1}b_1 + a_nb_0\right) = -\left(-q^{n-1}(-q) + q^n\right) = 0$.
\end{description}
\end{center}
für alle $n \ge 3$ folgt induktiv, dass ebenso $b_n = 0$ gilt. Die inverse Potenzreihe zu $A(z)$ ist $B(z) := b_0 + b_1z = 1 - qz$. 
%Konvergenz betrachten wir ja nicht!!! Daraus können wir schließen:
%\[\forall z \in \R \text{ mit } |z| < |\frac{1}{q}|: A(z) = \sum_{}^{}a_n z^n = \frac{1}{1-qz}\] 
\end{bsp}
%
\subsection{Eigenschaften des Potenzreihenrings}
In diesem Abschnitt zeigen wir, dass der Potenzreihenring auch ein Integritätsring ist. Im Körper $\C$ betrachten wir den Zusammenhang zwischen $K[[z]]$ und dem Ring der konvergenten Potenzreihen. \\
Wir beweisen weiterhin, dass $K[[z]]$ ein Integritätsring und damit nullteilerfrei ist. Wir wissen also, dass der Ring $K[[z]]$ in einen kleinsten Körper, den Quotientenkörper, eingebettet werden kann
%evtl noch möglich zu zeigen dass c[[z]] ein nullteilerfreier Ring ist stellt sich nur die Frage ob das iwie nötig ist...
%

\begin{satz}\label{intring}
Der Ring $K[[z]]$ ist ein Integritätsring.
\end{satz}
%
\beweis{ Es ist zu zeigen, dass der Ring nullteilerfrei ist. \\
Seien $ f = \sum_{n=0}^\infty  a_n z^n \text{ und } g = \sum_{n=0}^\infty  b_n z^n$ mit 
\begin{eqnarray*}
f \cdot g &=&  \sum a_nz^n \cdot \sum b_nz^n \\
&=& 0.
\end{eqnarray*} 
Nach Definition der Multiplikation gilt $\sum_{j+k=n}a_jb_k = 0$, für alle $n \in \N_0$.\\
Sei nun o.B.d.A. $\sum a_nz^n \neq 0$. Wir zeigen, dass die Potenzreihe $\sum b_nz^n$ gleich null ist. Es soll also kein Index $n$ existieren, für den $b_n \neq 0$ ist. Wir folgern aus $b_0, b_1,... ,b_{n-1} = 0$ induktiv, dass $b_n=0$ für alle $n \in \N$ ist.\\ 
Sei $j$ der erste Index, sodass $a_j \neq 0$ gilt. 
\begin{eqnarray*}
\sum_{j+k=j} a_jb_k = \sum_{j+0=j} a_jb_0 \stackrel{\mathrm{Vor.}}= 0. % da k+l=k
\end{eqnarray*} 
Da $a_j \neq 0$ ist, muss $b_0=0$ gelten. \\
Seien jetzt $ b_0,..., b_{n-1}= 0$. Mit $\sum_{j+k=n+k} a_jb_{n-j} = a_j b_n= 0$. Es folgt daher auch $b_n= 0$.
}

%TODO: nochmal checken ob das stimmt, sowohl oberes als auch Konvergenzdefinition
Da Konvergenzbetrachtungen nur im Körper der reellen und komplexen Zahlen Sinn machen, beschränken wir uns in folgendem Satz auf $\C$.  
%
%
\begin{defn}\label{konvergenz}
Eine Potenzreihe $f = \sum_{n= 0}^{\infty}a_nz^n \in \C[[z]]$ heißt \textit{konvergent}, wenn es ein $z_0\in \C$ mit $z_0 \neq 0$ gibt, sodass $\sum_{n=0}^{\infty}a_n{z_0}^n$ als Reihe in $\C$ konvergiert. \\
Das heißt die Folge $s_n = \sum_{k=0}^{n}a_k{z_0}^k$ der Partialsummen ist konvergent und man schreibt für den Limes $s = \lim_{n \to \infty}s_n$:
\begin{align}
s= \sum_{n=0}^{\infty}a_n{z_0}^n
\end{align}
Auf der Menge $D$ der Punkte $z_0 \in \C$ für die $\sum_{n=0}^{\infty}a_n{z_0}^n$ konvergiert, wird somit eine Abbildung $z_0 \mapsto \sum_{n=0}^{\infty}a_n{z_0}^n $ definiert. Wir nennen $D$ den \textit{Konvergenzbereich}.
\end{defn}


%
%
%
\begin{bem}\label{konvergentUnterring}
Sei $\C\lbrace z \rbrace$ die Menge der konvergenten Potenzreihen über dem Körper der komplexen Zahlen $\C$. $\C\lbrace z \rbrace$ ist ein Unterring des Rings der formalen Potenzreihen $\C[[z]]$. 
\end{bem}
\beweis{Wir haben bereits in \ref{intring} gezeigt, dass $\C[[z]$ ein Integritätsring ist. Nun bleibt für $\C\lbrace z \rbrace$ noch zu beweisen, dass die Summe und das Produkt zweier konvergenter Potenzreihen wieder konvergent ist. \\
Betrachte zwei konvergente Potenzreihen mit den Konvergenzradien $r_1$ und $r_2$. Innerhalb des min$\lbrace r_1, r_2\rbrace $ konvergieren beide Potenzreihen und somit auch die Summe der beiden Potenzreihen. Das Produkt besitzt denselben Konvergenzradius, da beide Reihen im Radius min$\lbrace r_1, r_2\rbrace $ absolut konvergieren und nach dem großen Umordnungssatz konvergiert auch das Cauchyprodukt gegen den gleichen Wert.} 
%
%
%
%
%
%
Im nächsten Teil können wir zeigen, dass der Quotientenkörper des Ringes der formalen Potenzreihen dem Körper der formalen Laurentreihen entspricht, auf den wir später näher eingehen werden. Anschließend definieren wir eine entsprechende Bewertung auf dem Körper der formalen Laurentreihen.
%
%
%
%
\section{Der Körper der formalen Laurentreihen}
%
Eine Erweiterung des Begriffs einer formalen Potenzreihe führt zu der formalen Laurentreihe. Diese unterscheidet sich bezüglich ihres Anfangsindex $n_0 \in \Z$ von den formalen Potenzreihen. Wir bezeichnen mit $K((z))$ die Menge aller Abbildungen $f$ von $\Z$ in einen $K$, für die es ein Element $x \in \Z$ gibt, mit $f(y) = 0$ für alle $y < x $. \newline 
Laurentreihen spielen eine wichtige Rolle in der Funktionentheorie, da sie komplexe Funktionen beschreiben, welche auf einem Kreisring holomorph sind. In dieser Arbeit wird jedoch auf Konvergenzbetrachtungen verzichtet und nur formale Laurentreihen, also Laurentreihen in einer Unbestimmten z behandelt. % Quelle:  [H74] HENRICI, Peter: Applied and computational complex analysis, Volume 1, WileyInterscience publication, New York 1974.
Wir orientieren uns dabei an \cite[S. 563 - 572]{Lueneburg08}.
%
\begin{defn}
Eine \textit{formale Laurentreihe} über dem Körper $K$ ist eine Abbildung $\Z \rightarrow K$, $n \mapsto a_n$.
\end{defn}
%
%
\begin{bem}
Wir werden Laurentreihen im Folgenden meist als Reihe der Form
\begin{equation*}
\sum_{n= - k}^{\infty}a_nz^n mit k \in \Z, n \ge -k, \text{ und }a_n \in\textit{K}\text{ für alle } n\in\N 
\end{equation*} 
Dabei bezeichnet $\sum_{n=1}^{k}a_{-n}z^{-n}$ den Hauptteil, $\sum_{n=0}^{\infty}a_nz^n$ den Nebenteil der Laurentreihe. 
\end{bem}
%
Wir bezeichnen die Menge der formalen Laurentreihen in $z$ auf $\Z$ über $K$ mit
\begin{equation*} %TODO: konsistent, passt das?
K((z)) = \lbrace \sum_{n =-m}^{\infty} a_nz^n \vert a_n \in K, m \in \Z\rbrace
\end{equation*} 
%
%
%
%früher: als die Menge der Abbildungen $f$ von $\Z$ in den kommutativen Körper $K$, für die es ein $a \in \Z$ gibt mit $f(i) = 0$ für alle $i < a$. 
%
%
%
Im Unterschied zu der funktionentheoretischen Verwendung der Laurentreihen betrachten wir nur Laurentreihen mit endlich vielen negativen Summanden. Diese Beschränkung ist notwendig, da andernfalls die Multiplikation nicht definiert werden kann.
%
%
%
% 
\begin{defn}\label{traeger}
Der \textit{Träger} einer Laurentreihe $f = \sum_{n =m}^{\infty} a_nz^n \in K((z))$ ist folgendermaßen definiert: supp$(f) := \lbrace n \in \Z | a_n \neq 0 \rbrace$ . 
\end{defn}
%
%
%
%
\begin{bem}
Unter einem Träger einer Laurentreihe versteht man den Definitionsbereich der Funktion, die durch die Laurentreihe dargestellt wird.
\end{bem}
%
%
%
%
Die Addition und Mulitplikation formaler Laurentreihen erfolgt analog zur Addition und Mulitplikation formaler Potenzreihen.
\begin{defn}
Zwei Laurentreihen $f, g \in K((z))$, mit $f = \sum_{n=-k}^\infty a_n z^n$ und $g= \sum_{n=-m}^\infty b_n z^n$, werden addiert, indem man ihre entsprechenden Koeffizienten addiert: 
%
\begin{eqnarray*}
&+& \colon \sum_{n=-k}^\infty a_n z^n  +  \sum_{n=-m}^\infty b_n z^n = \sum_{n = min(-k, -m)}^{\infty}(a_n + b_n) z^n.
\end{eqnarray*}
%
%Eine derartige Darstellung existiert, da $\text{K}((z)) $ als Quotientenkörper von K[[z]] definiert ist, wie in \ref{quot} gezeigt wird. \\
%
Die Multiplikation erfolgt durch Faltung der Laurentreihen.
\begin{eqnarray*}
\cdot \colon \sum_{n=-k}^{\infty} a_n z^n  \cdot  \sum_{n=-m}^{\infty} b_n z^n = \sum_{n = -m-k}^{\infty}\sum_{i+j=n}^{}\left(a_i \cdot b_j\right) z^n. 
\end{eqnarray*}
%

\end{defn}
%
\begin{bem}
Wie bereits erwähnt, besitzen formale Laurentreihen nur endliche viele Terme mit negativen Exponenten, das bedeutet der Hauptteil besteht aus nur endlich vielen Summanden. Diese Bedingung ist unverzichtbar zur Wohldefiniertheit der Multiplikation. 
\end{bem}
%
%
%
%
%
\begin{satz}
Die Menge $\left(K((z)), + \cdot\right)$ ist mit obigen Verknüpfungen ein Ring.
\end{satz}
\beweis{Es genügt der Nachweis der Ringaxiome. Der Beweis verläuft analog zu \ref{RingFormalerPR}.
}
%
\begin{satz}\label{Laurentreihenkoerper} %\cite{Lueneburg08}
$\left(K((z)), + \cdot\right)$ ist mit der definierten Addition und Multiplikation ein Körper. 
\end{satz}
\beweis{ Sei $0 \neq f \in K((z))$. 
%Dann gibt es ein $i \in \Z$, sodass $f = \sum_{n= i}^{\infty}a_nz^n$ für alle $n \in \Z $ mit $n \geq i$ ungleich Null ist und für alle $n < i$ gleich Null ist. 
Um zu zeigen, dass $K((z))$ ein Körper ist, muss zu jedem Element $f$ von $K((z))$ ein Inverses $g$ existieren. Wir definieren $g \in K((z))$ rekursiv und zeigen, dass die so definierte Laurentreihe invers zu $f$ ist. \\
Setze $g(n) = 0 \text{ für alle } n < -i \text{ und } g(-i) = \frac{1}{f(i)}.$ Sei $w \in \N$ und $g(-i),..., g(-i+w-1)$ bereits definiert.  Wir wählen $g(-i+w) = - \frac{1}{f(i)} \sum_{m= -i}^{-i+w-1} g(m)f(w-m)$. Nach Definition der Multiplikation in $K((z))$ erhalten wir somit: 
\begin{eqnarray*}
(gf)(w) &=& \sum_{n = i-i}^{\infty}\sum_{k+l=n}^{}\left(a_k \cdot b_l\right) z^n  \\
&=& \sum_{n = -i}^{-i + w}g(n)f(w-n).
\end{eqnarray*}  
Im Fall $w < 0 $ ist die Summe $f(w-n)$ für $ -i \leq n \leq -i+w $ leer. Für $w= 0 $ folgt \\$gf(0) = g(-i)f(i)= 1$. Es bleibt der Fall $w > 0 $ zu  berücksichtigen: 
\[
(gf)(w)= \sum_{n = -i}^{-i + w - 1}g(n)f(w-n) + g(-i+w)f(i) = 0.
\]
Also ist $gf = 1_K$ und da $K((z))$ ein kommutativer Ring ist, folgt $fg = 1-k$, womit $K((z))$ ein Körper ist.
}
%
%
Mithilfe von \ref{quotkoerper} zeigen wir nun, dass der Körper der formalen Laurentreihen dem Quotientenkörper des Ringes der formalen Potenzreihen entspricht. 
%
\begin{satz}\label{quot}
Es gilt $K((z)) =$\textup{Quot}$(K[[z]])$.
\end{satz}
\beweis{Nach Konstruktion von $K((z)) $ ist klar, dass $K[[z]]\subseteq K((z)) $. 
%Genauer Beweis nach schotten:
Betrachte die Abbildung:\\
\[\Phi: K((z)) \rightarrow \text{ Quot}\left(K[[z]]\right)\]
\[\sum_{n=m}^{\infty}a_nz^n  \mapsto 
\begin{cases}
\lbrack{z^{-m}\sum_{n=m}^{\infty}a_nz^n}, ~{z^{-m}}\rbrack & \text{, falls } m < 0 \\
\lbrack\sum_{n=m}^{\infty}a_nz^n\rbrack & \text{, falls } m\geq 0
\end{cases}\]
Wir weisen nach, dass $\Phi$ ein Körperisomorphismus ist. Wir beschränken unsere Abbildung, um Fallunterscheidungen zu vermeiden auf $z^k :=1$ für alle $k\leq 0$. Sei $m\leq l$:\\
\vspace{0.8cm}
\[\Phi \left( \sum_{n=m} a_nz^n \right)+ \left( \sum_{n=l} b_nz^n \right)\]

\[= \lbrack{z^{-m}\sum_{n=m}^{\infty}\left(a_n + b_n\right) z^n}, ~{z^{-m}}\rbrack =\] \[\lbrack{z^{-m}\sum_{n=m}^{\infty}a_n z^n}, ~{z^{-m}}\rbrack+ \lbrack{z^{-m}\sum_{n=l}^{\infty}a_n z^n},~{z^{-m}}\rbrack = \]
\[\lbrack{z^{-m}\sum_{n=m}^{\infty}a_n z^n},~{z^{-m}}\rbrack+ \lbrack{z^{-l}\sum_{n=l}^{\infty}a_n z^n},~{z^{-l}}\rbrack = \] \[\Phi \left(\sum_{n=m} a_nz^n\right) + \Phi\left(\sum_{n=l} b_nz^n \right)\]
Damit ist $\Phi$ bezüglich der Addition ein Homomorphismus. Nun zur Multiplikation:
\[\Phi \left( \sum_{n=m} a_nz^n \right)\cdot \left( \sum_{n=l} b_nz^n \right)\]
\[=\lbrack{z^{-m}\left(\sum_{n=m}^{\infty}a_nz^n\right) \left(\sum_{n=l}^{\infty}b_nz^n\right)},~{z^{-m}}\rbrack =\] \[\lbrack{\left(z^{-m}\sum_{n=m}^{\infty}a_n z^n\right)\left(z^{-l}\sum_{n=l}^{\infty}b_n z^n\right)},~{z^{-m}z^{-l}}\rbrack\]
\[=\lbrack{z^{-m}\sum_{n=m}^{\infty}a_n z^n},~{z^{-m}}\rbrack \cdot \lbrack{z^{-l}\sum_{n=l}^{\infty}a_n z^n},~{z^{-l}}\rbrack \]\[=\Phi \left(\sum_{n=m}^{\infty} a_nz^n\right) \cdot \Phi\left(\sum_{n=l}^{\infty} b_nz^n \right)\]
Der Homomorphismus ist bijektiv, denn das Bild jedes Elements \\$q := \lbrack{\sum_{n=m} a_nz^n},~{\sum_{n=l} b_nz^n}\rbrack$ liegt in $Quot\left(K[[z]]\right)$, für $\sum_{n=l} b_nz^n \neq 0$. Der Kern der Abbildung ist das neutrale Einselement und da $\Phi$ ein Homomorphismus ist, gilt die Injektivität.\\
Jede Laurentreihe $ f = \sum_{n = l}^{\infty} b_nz^n $ kann deswegen in die Gestalt $z^l\sum_{n = l}^{\infty} c_nz^n $ für $c_0 \neq 0$ gebracht werden. Nach \ref{potenzreihenringEinheit} kann diese Reihe invertiert werden und wir erhalten:
\[\lbrack{\left(\sum_{n = m}^{\infty} a_nz^n\right)\left( \sum_{n = 0}^{\infty} c_nz^n\right)^{-1}},~{z^l}\rbrack\]
\[ = \Phi\left(z^{-l}\left(\sum_{n = m}^{\infty} a_nz^n\right)\left(\sum_{n = 0}^{\infty} c_nz^n\right)^{-1}\right)\]
%
Da somit jede Laurentreihe $ f = \sum_{n\in \Z} a_nz^n $ die Gestalt $\lbrack{g},~{z^-m}\rbrack$ für ein $m \in \N$ hat, mit $g\in K[[z]]$ und $m\in\N$, ist $K((z))$ der Quotientenkörper (siehe \ref{quotkoerper}) von $K[[z]]$.
}
% Genauer Beweis hierzu: http://www.mathematik.uni-muenchen.de/~schotten/FT/loesungsskizzen/blatt-2-lsg.pdf
%
%
Für $K((z))$ gilt, dass der Körper nur Reihen mit Hauptteilen aus endlichen vielen Summanden enthält. $K((z))$ entspricht, wie in \ref{quot} gezeigt, dem Quotientenkörper des Ringes der formalen Potenzreihen.  
Jede Potenzreihe $\sum_{n=0}^{\infty} a_nz^n \text{ mit } a_0\neq0$ ist invertierbar in $K[[z]]$ (siehe \ref{potenzreihenringEinheit}). In jedem Quotient $\frac{\sum_{n=0}^{\infty}  a_nz^n}{\sum_{m=0}^\infty b_mz^m}$ kann aufgrund dieser Eigenschaft alles, bis auf eine Potenz von $z$, aus dem Nenner gekürzt werden. Da  $\sum_{n \ge k}^{\infty} a_nz^n =z^k \sum_{n \ge 0} a_{n+k}z^n$ ist, enthält $K((z))$ nur Reihen, deren Hauptteil nur endlich viele negative Summanden hat. \\
%
Die formalen Laurentreihen bilden einen Oberring der Potenzreihen und stellen als Körper eine Körpererweiterung von $K$ um das transzendente Element $z$ dar.  %\footnote{http://www.mathematik.uni-muenchen.de/~schotten/MIA/Muster/4_4.pdf}
%
%
\begin{satz}
Der Quotientenkörper von  $\C\langle z\rangle$ ist isomorph zum Körper der konvergenten Laurentreihen $\C_L \langle z \rangle$.
\end{satz}
\beweis{ Wie in Beweis \ref{inverse Potenzreihe} konstruieren wir das formale Inverse zu einer formalen Potenzreihe $f = \sum_{n= 0}^{\infty} a_nz^n \in \C\langle z\rangle $ mit $a_0 \neq 0$. Wir müssen zeigen, dass das Inverse konvergiert. Nach Voraussetzung konvergiert $f$ und wir können ohne Beschränkung der Allgemeinheit annehmen, dass die Koeffizientenfolge $ (a_n)_{n \in \N}$ in $f$ beschränkt ist, also $|a_n| \le a$, für alle $n \in  \N$ mit $a \in \C$. \\
Betrachte $f(z_0) = \sum_{n = -k}^{\infty} a_n {z_0}^n $ eine konvergente Laurentreihe mit $|z_0| > 0$. Da $f(z_0)$ konvergiert, ist die Folge ${(a_n|z_0|)}_{n\in\N} $ beschränkt und für die Potenzreihe gilt: \\
\[\overline{f}(\omega) := \sum_{n=0}^{\infty}a_n{z_0}^n {\frac{z}{z_0}}^n = \sum_{n=0}^{\infty}a_nz^n =  f(z)\text{ mit }\omega:= \frac{z}{z_0}\]
Wir nehmen an, dass die Schranke $a$ der Koeffizientenfolge $a_n$ größer 1 ist und es sei ohne Einschränkung $a_0 = 1$. Wir betrachten die Koeffizientenfolge $b_n$ des Inversen wie in \ref{inverse Potenzreihe}. Es gilt:
\[
b_n = - \sum_{k=1}^{n} a_k b_{n-k}. 
\]
Indem wir zeigen, dass |$b_n$| beschränkt ist durch ein Vielfaches von $a^n$ geben wir eine positive untere Schranke des Konvergenzradius an und zeigen somit, das Inverse konvergiert. Wir beweisen durch Induktion, es existiert ein $C > 1$ mit $C \in \R $ sodass:  
\[
|b_n| \le (aC)^n
\]
Nach Konstruktion des Inversen \ref{inverse Potenzreihe} ist die Ungleichung für $b_0$ erfüllt. Gelte die Abschätzung für $b_n$. Wähle $ C:= \frac{a}{a-1}$. Dann erhalten wir: \\
\[|b_{n+1}| = |- \sum_{k=1}^{n+1} a_k b_{n+1-k}| \le \sum_{k=1}^{n+1} |a_k| |b_{n+1-k}| \le a\sum_{k = 1}^{n+1} \left(Ca\right)^{n+1-k} \le a~C^n~\sum_{k=1}^{n+1}a^k\]
\[ \le a~C^n~\frac{a^{n+1}}{a-1} \le \left(a~C\right)^{n+1}.\]
Wie in \ref{quot} zeigt man nun, dass es einen Isomorphismus zwischen dem Quotientenkörper der konvergenten Potenzreihen und dem Körper der konvergenten Laurentreihen gibt. Des weiteren ist noch zu zeigen, dass die Summe sowie das Produkt zweier konvergenter Laurentreihen wieder konvergent ist. Dies wurde bereits in \ref{konvergentUnterring} für Potenzreihen gezeigt. Die Summe zweier konvergenter Laurentreihen $f,g $ ist ebenso konvergent, es genügt, den Nebenteil betrachten. Um die Konvergenz des Produktes zweier konvergenter Laurentreihen zu beweisen, multipliziere man diese so mit den Potenzen von z, dass man eine konvergente Potenzreihe erhält und geht wie in \ref{konvergentUnterring} vor.
Nun definieren wir wie in \ref{quot} die Abbildung 
\[\Phi:\C_L\langle z \rangle \rightarrow Quot(\C \langle z \rangle ) \]
\[\sum_{n= -m}^{\infty} a_n z^n \mapsto  \begin{cases}
  \lbrack \left(z^{m} \sum_{n= -m}^{\infty}a_n z^n,~ {z^m} \right)\rbrack  & \text{wenn }m < 0,\\
  \lbrack \sum_{n= -m}^{\infty} a_n z^n \rbrack & \text{wenn } m > 0.
\end{cases}.\]
Die Abbildung $\Phi$ ist, wie in \ref{quot} bereits gezeigt, ein Isomorphismus und die Behauptung ist bewiesen. 
}
%
%
%
Nun versuchen wir auf dem Körper der Laurentreihen eine Bewertung finden.
Dazu betrachten wir zunächst den Träger \ref{traeger} der Laurentreihe supp$(f) := \lbrace n \in \Z | a_n \neq 0 \rbrace$. Nach \ref{bewKoerper} suchen wir einen surjektiven Gruppenhomomorphismus (nach B3' \ref{bewKoerper}). Betrachte min$\lbrace\left(\text{supp}(f)\right)\rbrace$, eindeutig bestimmt durch den kleinsten Index $n_0$ der Laurentreihe, ab dem der Koeffizient $a_{n_0} \neq 0 $ ist. Die Menge all dieser Elemente bildet eine angeordnete abelsche Gruppe $\Psi $ und es gibt einen Isomorphismus von $\psi: \text{ }\Psi \rightarrow \Z$. \\
%
%
%
\begin{satz} \label{LaurentreiheBewertung}
Die Abbildung  $v\colon K((z))\rightarrow \Z\cup \lbrace \infty \rbrace$ definiert durch $v(f)= min\left(\text{supp}(f)\right)$ ist eine diskrete Bewertung.
\end{satz}
%
\beweis{Klar: Die Abbildung ist surjektiv, da es zu jeder ganzen Zahl eine Laurentreihe mit diesem Startwert gibt und damit $v(f) = $min$\lbrace($supp$(f)\rbrace)\}$. Nach \ref{bewKoerper} sind noch (B1'-B3') nachzuweisen mit der angeordneten abelschen Gruppe ($\Z \cup \lbrace \infty\rbrace$, +) als Bildmenge. 
\begin{enumerate}
\item [zu B1']: Klar nach Definition.%$"\Leftarrow"$ Sei f = $0_K = \sum_{n=0}^{\infty} a_nz^n$, mit $a_n = 0 \forall n \in \N$. Es gilt v(f) = 0 genau dann wenn $n_0 = 0$, wenn $ f(z)=\sum_{n = n_0}{\infty}a_n z^n $. Angenommen f $\neq 0_K = \sum_{n=0}^{max}$. Nach Voraussetzung muss gelten $a_{n_0} = a_0 \neq 0$. 
\item[zu B2']: Sei $f(z)=\sum_{n = n_0}^{\infty}a_n z^n \text{ und } g(z)=\sum_{m = m_0}^{\infty}b_m z^m$, mit $a_{n_0} \neq 0$ und $b_{m_0} \neq 0$. Dann ist v(f) =$ n_0$ und v(g) = $m_0$. Damit gilt: v(f) + v(g) = $n_0 + m_0$. \\
\[v(fg) = v( \sum_{n \in \Z}\sum_{n= m+k}a_mb_kz^n) \stackrel{\mathrm{!}}= v(f) + v(g)\], wobei $a_m = 0 \text{ für } m < n_0   \text{ und } b_k = 0$ für $k < m_0$. Betrachte $ n < n_0 + m_0 $. Da $ n = m+k $ folgt m < $n_0$ oder k < $m_0$. Nach Voraussetzung folgt entweder $a_m = 0$, oder $b_k = 0$ und somit ist auch das Produkt $a_mb_k = 0. $ Weiterhin gilt nach Voraussetzung  $a_{n_0} \neq 0$ und $b_{m_0} \neq 0.$ Sei n = $n_0$ + $m_0$. Das Produkt $ a_{n_0}b_{m_0}$ ist ungleich Null und daher folgt: v(fg) = $n_0+m_0 = v(f) + v(g). $
\item[zu B3']: Für  $v(f+g)$ gilt, wenn $f, g$ wie oben definiert: \\
\[v\left(f+g\right) = \]
\[v\left( \sum_{n = \text{min}\lbrace n_0,m_0 \rbrace}^{\infty}(a_n + b_n) z^n\right) = \text{ min }\lbrace n_0,m_0 \rbrace \leqslant  \text{ max } \lbrace n_0, m_0\rbrace \stackrel{\mathrm{def}}= \text{ max }\lbrace v(f), v(g)\rbrace.\]
\end{enumerate}
} 
%
%\begin{satz}
%Der Körper der formalen Laurentreihen über $\C$ $\C((z)) = \lbrace \sum_{n = n_0}^{\infty}a_nz^n: n_0 \in \Z, a_n \in \C, a_{n_0} \neq 0\rbrace\ $ist vollständig bezüglich der in \ref{LaurentreiheBewertung} definierten diskreten Bewertung. 
%\end{satz}
%\beweis{Betrachte die Cauchy-Folge ${(\sum_{n= n_i}^{\infty}a_{n,i}z^n)}_{i\in\N}$. Die Menge der Startindizies $n_i$ ist nach unten beschränkt, und kann nicht beliebig klein werden. Damit ist der Träger  }%TODO: siehe forum matheraum: beweisen mit cauchyfolge
%Der Träger einer formalen Laurentreihe $f(z)=sum{n = n_0}{\infty}a_n z^n$ konzentriert sich auf die Menge $ \mathtt{T}  := supp(f) = \lbrace n_0, n_0 + 1, n_0 +2, ..., deg(f)\rbrace \subseteq \Z $.
Wie wir in \ref{quot} gezeigt haben, ist der Körper der Laurentreihen eine Obermenge des Rings der Potenzreihen $K[[z]]$. Nachdem wir auf $K((z))$ bereits eine Bewertung definiert haben, weisen wir nach, dass es sich auch bei $K[[z]]$ um einen diskreten Bewertungsring \ref{bewring} handelt und wir darauf eine Bewertung definieren können. 
%Wie in \ref{bewring} gezeigt, ist $K[[z]]$ ein diskreter Bewertungsring und der kleinste vorkommende Exponent eines Monoms liefert die Bewertung einer Potenzreihe. Der Quotientenkörper eines diskreten Bewertungsrings besitzt ebenso eine Bewertung \ref{quotbewring}. 

%
\begin{satz}\label{bewring}
$K[[z]] $ ist ein diskreter Bewertungsring. %\footnote{https://www.mathematik.uni-osnabrueck.de/fileadmin/mathematik/downloads/2012AlgKurven.pdf}
\end{satz}
%
\beweis{Wie im vorherigen Satz gezeigt, existiert auf dem Körper der Laurentreihen eine diskrete Bewertung. Wie wir in \ref{quot} bewiesen haben, ist der Quotientenkörper von $K[[z]]$ dieser Körper der Laurentreihen. Nach Definition des diskreten Bewertungsring \ref{bewring} gilt der Satz. 
Nach \ref{potenzreihenringEinheit} folgt, $K[[z]]$ besitzt genau ein maximales Ideal nämlich $\mathfrak{m} = (z)$. Für eine Potenzreihe $P$, mit $P \notin K[[z]]^* $ gilt $a_0 \neq 0$. Somit lässt sich jede derartige Potenzreihe schreiben als $P=T \widetilde{P}$, wobei  $\widetilde{P}$ die umindizierte Potenzreihe bezeichnet.\\
Die Nullteilerfreiheit folgt wie in \ref{intring} ausführlicher gezeigt, denn: 
Für die Produktreihe $FG$, wobei $F,G \in K[[z]]$ und $F, G$ von Null verschieden, gilt, dass ab den Indizes i, j gilt $a_i,~ b_j\neq 0$ und somit $c_n := a_ib_j \neq 0$. Der Hauptidealring $K[[z]]$ ist noethersch, denn jedes Ideal ist erzeugt von $z^j$, wobei j der kleinste Index ist, ab dem die Koeffizienten $c_n$ der Potenzreihen ungleich 0 in dem Ideal sind. Für das maximale Ideal muss nämlich gelten, dass es von einem Element erzeugt wird, für das gilt $a_0 = 0$.
Andernfalls wäre die entsprechende Potenzreihe eine Einheit und würde somit ganz $K[[z]]$ erzeugen. } 
 %
 %
Damit folgt, dass $K[[z]]$ isomorph zu einem, wie in Punkt \ref{chap2} beschriebenen Bewertungsring $ A:= {0} \cup \{x \in K * | v(x) \geqslant 0\}$ ist.  \\
Wie in obigem Beweis \ref{bewring} gezeigt, gilt: $ (z) \subset (z^2) \subset (z)^3 \subset (z)^4 \subset ... $. \\


\begin{lemma}\label{quotbewring}
Ist $R$ ein diskreter Bewertungsring, so ist Quot($R$) ein diskret bewerteter Körper mit der Bewertung $v(a/b)=v(a)-v(b)$. 
\end{lemma}



\section{Der verallgemeinerte Potenzreihenkörper}
%
Die Grundlagen zur Konstruktion eines Körpers über sehr allgemein definierten formalen Potenzreihen untersuchte Hahn 1907 in seiner Arbeit \glqq Über nichtarchimedische Größensysteme\grqq. Er stellte im Rahmen seines Beweises des Hahnschen Einbettungssatzes \ref{HahnscheEinbettungssatz} zuallererst die sogenannten Hahnschen Potenzreihen zunächst als Gruppen \ref{Hahn-Gruppe} vor. Die so definierten Potenzreihen erlaubten nicht nur Exponenten der Unbestimmten aus der Menge der ganzen Zahlen, sondern aus beliebigen wohlgeordneten Untergruppen der Wertegruppe. Hahn formulierte als einer der ersten Mathematiker Potenzreihen mit verallgemeinerten Exponenten, wie:\\
\[ f = 1 + z^{log 2} + z^{log 3} + z^{log 4} + ... \]
\[g = \frac{1}{2}z^{\frac{1}{2}} + \frac{3}{4}z^\frac{3}{4} + \frac{7}{8}z^\frac{7}{8} + ... + z + \frac{3}{2}z^\frac{3}{2} + ... + 2z^2 + ...\] 
Hahn konnte im Laufe des Beweises des Hahnschen Einbettungssatzes zeigen, dass die nach ihm benannten Potenzreihen nicht nur eine Gruppe, wie ursprünglich angenommen, bilden.  Während seiner Beschäftigung mit Hilberts siebzehntem Problem untersuchte er die Hahnschen Potenzreihen hinsichtlich ihrer Körpereigenschaften. Neben Hahn beschäftigten sich auch die beiden Mathematiker Neumann und Mal'cev mit den von Hahn konstruierten Reihen und deren Einbettung in einen Körper. \\
Bevor wir uns der Konstruktion der formalen Potenzreihen zuwenden, wird die Rolle des Trägers der Potenzreihen bei der Konstruktion von Potenzreihenkörpern genauer erörtert. 

\subsection{Potenzreihenstrukturen mit Träger über den ganzen Zahlen}\label{traegerGanz}
Wir kennen bereits die in \ref{potenzreihenring} definierten Potenzreihen sowie ihre Verallgemeinerung, die Laurentreihen. Bisher haben wir uns nur mit Potenzreihen beschäftigt, deren Elemente auf einer Teilmenge der ganzen Zahlen indiziert werden. Die Exponenten der Unbestimmten der induzierten algebraischen Struktur $\left(K[[z]], K((z))\right) $ gehören in beiden Fällen ebenso einer Teilmenge der ganzen Zahlen an. \\
Betrachten wir den Körper der formalen Laurentreihen über dem Körper $K$. Wie in \ref{traeger} definiert, ist $\mathtt{T}$ eine Teilmenge der ganzen Zahlen und da für jede Teilmenge ein Minimum existiert, ist $\mathtt{T}$ wohlgeordnet nach \ref{wohlgeordn}. Die Wohlordnung des Trägers ist eine Voraussetzung zur Definition der Multiplikation im Körper $K((z))$. Nun stellt sich die Frage, ob es noch allgemeinere Gruppen, als die Menge der natürlichen, oder ganzen Zahlen gibt, auf denen ein wohlgeordneter Träger von Potenzreihen definiert werden kann.
%
%
\subsection{Formale Potenzreihen auf angeordneten abelschen Gruppen}
Wir haben im vorherigen Gliederungspunkt festgestellt, dass die bisher betrachteten Potenzreihen immer auf Teilmengen der natürlichen Zahlen $\N$ (Potenzreihenring $K[[z]]$) oder den ganzen Zahlen $\Z$ (Laurentreihenkörper $K((z))$) konstruiert waren. In \ref{traegerGanz} wurde gezeigt, dass den Mengen, auf denen wir Potenzreihenringe definieren können, eine bestimmte, unverzichtbare Eigenschaft innewohnt: die Wohlordnung. Im Folgenden betrachten wir bestimmte Arten von Mengen, nämlich die bereits in dem vorherigen Kapitel \ref{chap2} vorgestellten angeordneten abelschen Gruppen. Die nachfolgenden Ausführungen orientieren sich an \cite[S. 194 - 199]{fuchs66}, \cite[S. 601 - 655]{hahn07} und \cite[S. 49 - 64]{priesscrampe83}.\\\\
Sei $\Gamma$ eine total geordnete abelsche Gruppe. Wir beschäftigen uns in dieser Arbeit ausschließlich mit der additiv geschriebenen Gruppe $\Gamma$. Die Äquivalenzklassen, die durch die archimedische Gleichheit \ref{archimedischeKlassen} entstehen, bilden eine total geordnete Menge, wir bezeichnen sie mit $\Pi$. Durch jedes positive Element $a \in \Gamma$, respektive seine Äquivalenzklasse $[a]= \pi$ werden zwei konvexe Untergruppen $L_\pi,~ U_\pi$  definiert. \\
Für einen Körper $K$ und $\left(\Gamma, + \right)$ eine angeordnete, abelsche Gruppe, bezeichnen wir mit H $\left(\Gamma, K \right)$ die Menge aller Funktionen von $\Gamma$ nach $K$. 
Wir nennen eine derartige Funktion $F\colon\Gamma \rightarrow K$, die in der Anordnung von $\Gamma$ einen wohlgeordneten Träger besitzt eine formale Potenzreihe auf $\Gamma$ über $K$. \\
Die Addition derartiger Funktionen $F, G$ ist definiert durch: 
\[\left(F + G\right)(x) = F(x) + G(x) \text{ für alle } x \in \Gamma.\] 
Sei $\lambda \in K$ dann ist 
\[ \left(\lambda G\right)(x) = \lambda F(x) \text{ für } x \in \Gamma,~ \lambda \in K \text{ und }F \in K[[\Gamma]].\]
%
Dementsprechend definieren wir die Gesamtheit dieser Elemente. Sei $\left( \Gamma, + \right)$ eine angeordnete, abelsche Gruppe und $K$ ein Körper.
\begin{defn}\label{formaleSumme}
Sei $F := \sum_{x \in \Gamma}^{}\Phi_x z^x$ mit den Koeffizienten $\Phi_x \in K$. Man nennt $F$ eine \textit{formale Potenzreihe} auf $\Gamma$ über K. Die Gesamtheit dieser formalen Potenzreihen wird im Folgenden mit $K[[z^{\Gamma}]]$ bezeichnet.
\end{defn}

Die Definition von $F$ verlangt, dass die Koeffizienten $\Phi_x \in K$ liegen und der Träger \\supp(F) = $\lbrack x \in \Gamma | \Phi_x \neq 0\rbrack $ wohlgeordnet ist bezüglich der Anordnung von $\Gamma$. \\
Die Exponenten x der Unbestimmten z sind ebenfalls Element der angeordneten Gruppe $\Gamma$. Die Potenzreihen werden aufsummiert über einer Untermenge $U$ bestehend aus Elementen $x$ aus $\Gamma$. Die Anordnung von $\Gamma$ überträgt sich auf die Untermenge $U$ nach \ref{agG}. Der Träger einer formalen Potenzreihe, besitzt als wohlgeordnete Teilmenge von $\Gamma$, bezüglich der Anordnung von $\Gamma$ ein kleinstes Element. Alternativ lässt sich die Reihe 
\[F = \Phi_{x_1}z^{x_1} + \Phi_{x_2}z^{x_2} + ... + \Phi_{x_p}z^{x_p} + ..., \] 
als Summation über den Ordinalzahlen $p$ bis zu einem fixierten $a \in \Gamma$ und Exponenten $x_1 \le ...\le x_p ...$ die bezüglich der Anordnung $\grqq \le \grqq$ von $\Gamma$ monoton steigend geordnet sind, schreiben. \cite{carruth48}\\
Eine formale Potenzreihe bezeichnet eine Funktion $\Phi: G \mapsto K$, die in der Anordnung von $\Gamma$ einen wohlgeordneten Träger supp$(\Phi) = \lbrace g \in G | \Phi(g) \neq 0 \rbrace$ besitzt. 
Wir bezeichnen die Menge aller formalen Potenzreihen auf $\Gamma$ über $K$ folgendermaßen: 
\[K[[z^{\Gamma}]] = \lbrace F :=  \sum_{x \in \Gamma}^{}a_x z^x | \text{supp(F)} \text{ ist wohlgeordnet.}\rbrace\]
%
%
\subsection{Definition der Addition und Multiplikation in $K[[z^{\Gamma}]]$}
Seien $F, G$ zwei formale Potenzreihen auf $\Gamma$, mit $F = \sum_{a \in \Gamma}^{} \Phi_a z^a \text{ und } G = \sum_{a \in \Gamma}^{} \Psi_a z^a$ wobei $\Phi_a, \Psi_a \in K)$ und supp($F$), supp($G$) wohlgeordnet. \\
Die Potenzreihen $F, G$ sind genau dann gleich wenn:
\begin{enumerate}
\item[1.]$\grqq a \in$ supp(F), $a \notin$ supp(G)$\grqq$, impliziert dass $\Phi_a = 0$,
\item[2.]$\grqq a \notin$ supp(F), $a \in$ supp(G)$\grqq$, impliziert dass $\Psi_a = 0$,
\item[3.]$\grqq a \in$ supp(F), $a \in$ supp(G)$\grqq$, impliziert dass $\Phi_a = \Psi_a$.
\end{enumerate}
Man kann leicht nachprüfen, dass es sich um eine Äquivalenzrelation handelt. \cite{carruth48}\\
Die Summe 
\[F + G := \sum_{a \in \Gamma}^{} \Phi_a z^a + \sum_{a \in \Gamma}^{} \Psi_a z^a = \sum_{a \in \Gamma}^{}\left(\Phi_a + \Psi_a\right)z^a \]
ist gegeben durch die Addition der Koeffizientenfolgen. \\
Der Träger der Summe $F + G$ ist wohlgeordnet, da offensichtlich supp$(F+ G)\subseteq $ supp(F) $\cup$ supp($G$)ist. Nach dem Lemma \ref{wohlgeordnvereinigung} ist die Vereinigung zweier wohlgeordneter Mengen wieder wohlgeordnet und jede Teilmenge einer wohlgeordneten Menge wiederum wohlgeordnet, nach Definition der Wohlordnung. Nach Voraussetzung sind supp$(F) =\lbrace g \in \Gamma| \Phi_{g} \neq 0 \rbrace \text{ und } $supp(G)=$\lbrace g \in \Gamma| \Psi_{g} \neq 0 \rbrace$ wohlgeordnet. Das kleinste Element von supp($F+G$) existiert und $\text{min}\left( \text{supp}\left(F+ G\right)\right) = \text{min}\lbrace \text{min}\left( \text{supp}(F)\right), \text{min}\left( \text{supp}(G)\right) \rbrace $. Somit ist $\Phi + \Psi$ eine formale Potenzreihe. \\
%
Für jedes Element $\lambda \in K$ ist das Produkt mit einem Körperelement definiert durch: \[\lambda F = \lambda (\sum_{x \in \Gamma}^{}a_x z^x) = \sum_{x \in \Gamma}^{}\lambda a_x z^x \in K[[z^{\Gamma}]].\]  
Bevor wir die Multiplikation der Koeffizienten zweier formaler Potenzreihen betrachten, schauen wir uns zunächst an, was bei der multiplikativen Verknüpfung mit den Exponenten der Variablen $z$ geschieht. Sei $F,~ G \in K[[z^{\Gamma}]]$ mit $ F = \sum_{g \in \Gamma}^{} \Phi_g z^g \text{ und } G = \sum_{h \in \Gamma}^{} \Psi_h z^h $. \\
Alternativ lassen sich unsere Reihen folgendermaßen schreiben: 
\[ F\cdot G = \left(\Phi_{g_1}z^{g_1} + \Phi_{g_2}z^{g_2} + ... + \Phi_{g_p}z^{g_p} + ...\right) \cdot \left( \Psi_{h_1}z^{h_1} + \Psi_{h_2}z^{h_2} + ... + \Psi_{h_p}z^{h_p} + ...\right)\]
Betrachten wir zunächst das Produkt einzelner Monome: $\Phi_gz^g  \Psi_hz^h = \Phi_g  \Psi_h z^{g+h} $. Für die Variable z gilt nach den Potenzgesetzen $z^{g_1} \cdot z^{h_1} = z^{g_1 + h_1}$. Die distributive Fortsetzung führt zur Definition der Multiplikation in $K[[z^{\Gamma}]]$. Wir erhalten das Produkt als Summe über der Summe des Produkts der einzelnen Koeffizienten: 
\[ \text{Sei } F = \sum_{a \in \Gamma}^{} \Phi_a z^a \text{ und } G = \sum_{a \in \Gamma}^{} \Psi_a z^a \text{ wobei } \Phi_a, \Psi_a \in K. \]
\begin{equation}\label{eq: multPotenzreihenkoerper}  
\cdot: H:= F \cdot G = \sum_{a \in \Gamma}^{}\sum_{a_1 + a_2 = a}^{}\Phi_{a_1} \Psi_{a_2}z^a.  
\end{equation}                         
Um eine Aussage treffen zu können, ob dieses Produkt wohldefiniert ist, müssen wir zeigen, dass der Träger des Produkts wohlgeordnet ist. Beachte, dass supp($F\cdot G$) $\subseteq$ supp$(F)+ $supp$(G)$. \\ 
Die Summe $\sum_{a_1 + a_2 = a}^{}\Phi_{a_1} \Psi_{a_2}$ kann reduziert werden auf $a_1\in$ supp(F) und $a_2\in$ supp(G), da ansonsten der Summand null ist. \\
Nach Voraussetzung sind sowohl supp(F), als auch supp$(G)$ wohlgeordnet. Die so entstandene Menge von Elementen aus $\Gamma$ ist also selbst wohlgeordnet. Nach dem Lemma von B.H. Neumann \ref{LemmaNeumann} ist damit supp$(F)$ + supp$(G)$ wohlgeordnet. Es ist bekannt, dass supp($F\cdot G) \subseteq$ supp$(F)$ + supp$(G)$ und jede Teilmenge einer wohlgeordneten Menge ist wieder wohlgeordnet nach der Definition der Wohlordnung (\ref{wohlgeordn}). Da jede nichtleere Teilmenge ein kleinstes Element besitzt, ist diese Teilmenge selbst wohlgeordnet. Wir erhalten die Wohlordnung des Trägers der Produktreihe supp($F\cdot G)$.
%Angenommen sie wäre es nicht, dann enthielte die Menge der Elemente $a_1 + a_2$ eine Teilmenge ohne kleinstes Element und damit ließen sich unendlich viele Paare ${a_1}_i$ und ${a_2}_i$ finden, sodass:
%\[{a_1}_1 + {a_2}_1 > {a_1}_2 + {a_2}_2 > ... > {a_1}_i + {a_2}_i > ...\]
%Da alle ${a_1}_i$ einer wohlgeordneten Menge angehören existiert ein kleines Element, etwa ${{a_1}_i}_1$. Für alle $ {a_1}_i (i > i_1)$ gibt es wieder ein kleinstes Element ${{a_1}_i}_2$ und so weiter. Wir erhalten:
%\[{{a_1}_i}_1 + {{a_2}_i}_1 > {{a_1}_i}_2 + {{a_2}_i}_2 > ... > {{a_1}_i}_n +{{a_2}_i}_n > ...\]
%während \[{{a_1}_i}_1 \le {{a_1}_i}_2 \le ... \le {{a_1}_i}_n  \le ....\]
%Daraus würde folgen:
%$ {{a_2}_i}_1 > {{a_2}_i}_2 > ... > {{a_2}_i}_n  > ...$.
%Dies ist ein Widerspruch, da nach Voraussetzung ${{a_2}_i}_n$ einer wohlgeordneten Menge angehört und somit ein kleinstes Element besitzen muss.\\
Wir behaupten weiter, dass die Summe $\sum_{a_1 + a_2 = a}^{}\Phi_{a_1} \Psi_{a_2} z^a$ (siehe \ref{eq: multPotenzreihenkoerper}) endlich ist. Es gibt also nur eine endliche Anzahl von Paaren $a_1, a_2$, sodass $a_1 + a_2$ ein vorgegebenes Element a $\in \Gamma$ ergibt. Die Folgerung aus dem Lemma von Neumann \ref{FolgerungNeumann} führt direkt zu dieser Aussage.  \\
%Angenommen es gäbe unendliche viele $a_1$ für welche $\Phi_{a_1} \neq 0 \text{ und } \Psi_{a - a_1} \neq 0$. Dann würden unendlich viele Paare ${a_1}_n, {a_2}_n$ existieren, für die gilt: 
%\[{a_1}_1 + {a_2}_1 = {a_1}_2 + {a_2}_2 = ... = {a_1}_i + {a_2}_i = ...\]
%Da ${{a_1}_i}$ einer wohlgeordneten Menge angehören nehmen wir im Folgenden immer an, es sei:
%\[{a_1}_1 < {{a_1}_2} < ... < {{a_1}_i} < ...\],
%Daraus würde wiederum folgen:\\
%\[{a_2}_1 > {{a_2}_2} > ... > {{a_2}_i} > ....\]
%Wir erhalten einen Widerspruch zur Wohlordnung der Menge der alle ${a_2}_n$ angehören.\\
Wir haben gezeigt, dass eine Darstellung von $a$ als Summe von $a_1+ a_2$ nur auf endlich viele Arten möglich ist, wenn Träger der beiden zu multiplizierenden Potenzreihen wohlgeordnet sind. Der Koeffizient $\Lambda$ von $z^a$ sei dann gegeben durch:\\
\[\Lambda_a = \Phi_{{a_1}_1}\Psi_{{a_2}_1} + \Phi_{{a_1}_2}\Psi_{{a_2}_2} + ... + \Phi_{{a_1}_n}\Psi_{{a_2}_n}\]
$\Lambda_a$ sei null, wenn es für $a$ keine Darstellung als Summe der Elemente der Träger von F und G gibt.\\
Das Produkt zweier formaler Potenzreihen auf $\Gamma$ über K ist somit wohldefiniert; der Träger der erhaltenen formalen Potenzreihe wohlgeordnet und der entstandene Koeffizient $\Lambda$ liegt, als endliche Summe des Produkts zweier Körperelemente $\Phi$ und $\Psi$, ebenfalls im Körper $K$. \\
Damit ist $K\lbrack\lbrack z^{\Gamma}\rbrack\rbrack$ bezüglich der definierten Addition und Multiplikation abgeschlossen. \cite[Seite 601ff]{hahn07}, \cite[S. 210- 213]{neumann49}.
%
\begin{bsp}
Bei der Reihe $F := z^{\frac{-1}{p}}+  z^{\frac{-1}{p^2}} + z^{\frac{-1}{p^3}} ...$ handelt es sich um eine formale Potenzreihe über einem beliebigen Körper, da der Träger $\lbrace \frac{-1}{p}, \frac{-1}{p^2}, \frac{-1}{p^3}, ... \rbrace$ wohlgeordnet ist.
\end{bsp}
%
\subsection{Der Ring der formalen Potenzreihen} 
Auf der Menge der formalen Potenzreihen auf der additiven, angeordneten abelschen $\Gamma$ über dem Körper $K$, $K[[z^\Gamma]]$ haben wir nun die Addition und Multiplikation definiert. In der verwendeten Literatur (\cite{priesscrampe83}, \cite{fuchs66}) findet sich die multiplikative Schreibweise der angeordneten Gruppe $\Gamma$, da diese eine noch allgemeinere Definition der Multiplikation zum Beispiel mithilfe von Faktorsystemen ermöglicht. Auf Basis dieser Definition konnte B.H. Neumann 1949 Schiefkörper von formalen Potenzreihen konstruieren. Diese lieferten wichtige Beispiele zur Einordnung der projektiven Ebenen.
Im Fall einer additiv geschriebenen, abelschen, angeordneten Gruppe erhalten wir den direkten Bezug zu dem anfangs beschriebenen Laurentreihenkörper und dem darin eingebetteten Potenzreihenring. Diese entstehen für den Fall, dass es sich bei der angeordneten, abelschen Gruppe um $\Z$ respektive $\N$ handelt. 
Wir zeigen zunächst, dass $K[[z^\Gamma]]$ ein Ring über K ist.
%Sei supp(F)supp(G) :=$ \lbrace a_1, a_2 \in G | \Phi_{a_1} \Psi_{a_2} \neq 0\rbrace.$ Nach der Definition des Trägers folgt, dass die Menge angeordnet ist, denn als Untermenge von $\Gamma$ überträgt sich die Anordnung. Wir müssen zeigen, dass jede Teilmenge von supp(FG) ein kleinstes Element besitzt nach \ref{wohlgeordn}. Nach Definition der Multiplikation in unserem Körper K gilt: \\ 
%
%
\begin{satz}
$K[[z^\Gamma]]$ ist ein Ring über K. 
\end{satz}
\beweis{Es gilt $K[[z^{\Gamma}]]$ ist eine abelsche Gruppe bezüglich der Addition.
\begin{itemize}
\item \textit{Assoziativität:} Für alle $F, G, H \in K[[z^{\Gamma}]]
$ gilt nach Definition der Addition: 
\[F+\left(G+H\right) = \sum_{a \in \Gamma}^{} \Phi_a z^a + \left( \sum_{a \in \Gamma}^{} \Psi_a z^a + \sum_{a \in \Gamma}^{} \Lambda_a z^a \right) = \sum_{a \in \Gamma}^{} \Phi_a z^a + \sum_{a \in \Gamma}^{} \left(\Psi_a + \Lambda_a\right) z^a \]
\[= \sum_{a \in \Gamma}^{} \left(\Phi_a + \Psi_a + \Lambda_a\right) z^a = \sum_{a \in \Gamma}^{} \left(\Phi_a + \Psi_a\right) z^a + \sum_{a \in G}^{} \Lambda_a z^a.\]
\item \textit{Neutrales Element der Addition:} Bezeichne $e$ das neutrale Element der Addition $e := \sum_{a \in \Gamma}^{} \Phi_a z^a$, wobei ähnlich wie in \ref{Rechnen} gilt $\Phi_a = 0$ für alle $a \in \Gamma$. Der Träger von $e$ ist die leere Menge, welche nach Definition wohlgeordnet ist.
\item \textit{Inverses Element der Addition:} Zu jedem Gruppenelement $F$ gibt es ein inverses Element der Addition $-F := \sum_{a \in \Gamma}^{} -\Phi_a z^a$, wobei supp$(F)$ = supp$(-F)$, mit $F+ F^{-1} = e$.
\item \textit{Kommutativität:} $K[[z^{\Gamma}]]$ ist abelsch, da $K$ ein Körper und nach Definition der Addition gilt: 
\[F+ G = \sum_{a \in \Gamma}^{} \Phi_a z^a + \sum_{a \in \Gamma}^{} \Psi_a z^a = \sum_{a \in \Gamma}^{}\left(\Phi_a + \Psi_a\right) z^a \stackrel{\mathrm{\Phi, \Psi \in K}}=\]
\[ \sum_{a \in \Gamma}^{}\left(\Psi_a + \Phi_a\right) z^a = \sum_{a \in \Gamma}^{} \Psi_a z^a + \sum_{a \in \Gamma}^{} \Phi_a z^a = G + F.\]
\end{itemize}
Die oben definierte Multiplikation in $K[[z^\Gamma]]$ ist kommutativ, wie sich leicht sehen lässt, da für $F, G \in$ K[[$z^\Gamma$]] mit $F := \sum_{a_1 \in \Gamma}^{} {\Phi_a}_1 z^{a_1}$ und $G := \sum_{a_2 \in \Gamma}^{} {\Phi_a}_2 z^{a_2}$ gilt:
\[FG = \sum_{a \in \Gamma}^{}\sum_{a_1 + a_2 = a}^{}\Phi_{a_1} \Psi_{a_2}z^a = \sum_{a \in \Gamma}^{}\sum_{a_2 + a_1 = a}^{}\Phi_{a_2} \Psi_{a_1}z^a = GF\]
Die Gleichheit folgt unmittelbar aus der Kommutativität von $\Gamma$ und der Kommutativität der Multiplikation im Körper $K$.\\
Des weiteren können wir die Assoziativität der Multiplikation nachweisen. Seien F, G, H $\in$ K[[$z^\Gamma$]] mit:
\[F := \sum_{a_1 \in \Gamma}^{} {\Phi_a}_1 z^{a_1}\]
\[G := \sum_{a_2 \in \Gamma}^{} {\Psi_a}_2 z^{a_2}\]
\[H := \sum_{a_3 \in \Gamma}^{} {\Phi_a}_3 z^{a_3} \]
Zur Bildung des Produkts $(F\cdot G)\cdot H$ beziehungsweise $F\cdot(G\cdot H)$ gilt die Instruktion des Index $a$ des gesuchten Koeffizienten $\Omega$ auf sämtliche Weisen als Summe $a_1 + a_2 + a_3$, beispielsweise: 
\[{a_1}_1 + {a_2}_1 + {a_3}_1 = {a_1}_2 + {a_2}_2 + {a_3}_2 = ... =  {a_1}_n + {a_2}_n + {a_3}_n.\]
Dann hat der Koeffizient $\Omega$ die Form: 
\[\Phi_{{a_1}_1} \Psi_{{a_2}_1} \Lambda_{{a_3}_1} + \Phi_{{a_1}_2} \Psi_{{a_2}_2} \Lambda_{{a_3}_2} + ... + \Phi_{{a_1}_n} \Psi_{{a_2}_n} \Lambda_{{a_3}_n}.\]
Falls keine Darstellung von $a$ als Summe der Elemente der Träger der Potenzreihen $F, G, H$ existiert, ist $\Omega$ gleich null.\\
In $K[[z^\Gamma]]$ gelten die Distributivgesetze, denn:
\begin{enumerate}
\item[(i)]\[F\cdot(G + H) := \sum_{a \in \Gamma}^{} {\Phi_a} z^{a} \left( \sum_{a \in \Gamma}^{} {\Psi_a} z^{a} +  \sum_{a \in \Gamma}^{} {\Lambda_a} z^{a} \right) = \sum_{a \in \Gamma}^{} {\Phi_a} z^{a} \left(\sum_{a \in \Gamma}^{} \left({\Psi_a}+ \Lambda_a\right) z^{a}\right) = \]
\[\sum_{a \in \Gamma}^{}\sum_{a_1 + a_2 = a}^{}\Phi_{a_1} {\left(\Psi + \Lambda\right)}_{a_2}z^a = \sum_{a \in \Gamma}^{}\sum_{a_1 + a_2 = a}^{}\Phi_{a_1} \left(\Psi_{a_2} + \Lambda_{a_2}\right) z^a=\]
\[\sum_{a \in \Gamma}^{}\sum_{a_1 + a_2 = a}^{}\Phi_{a_1} \Psi_{a_2}z^a + \sum_{a \in \Gamma}^{}\sum_{a_1 + a_2 = a}^{}\Phi_{a_1} \Lambda_{a_2}z^a = FG + FH\]
\item[(ii)] \[(F + G) \cdot H = \left(\sum_{a \in \Gamma}^{} {\Phi_a} z^{a}  \sum_{a \in \Gamma}^{} {\Psi_a} z^{a}\right) \cdot \sum_{a \in \Gamma}^{} {\Lambda_a} z^{a} = \left(\sum_{a \in \Gamma}^{} \left({\Phi_a}+ \Psi_a\right) z^{a}\right)\cdot \sum_{a \in \Gamma}^{} {\Lambda_a} z^{a} =\]
\[ \sum_{a \in \Gamma}^{} {\Lambda_a} z^{a} \cdot \left(\sum_{a \in \Gamma}^{} \left({\Phi_a}+ \Psi_a\right) z^{a}\right) =   F\cdot H + G\cdot H\]
\end{enumerate}  
In den Beweis der Distributivgesetze und die Gültigkeit der Gleichheit fließen die im Körper $K$ gültige Kommutativität der Multiplikation, die Potenzgesetze, beziehungsweise die Kommutativität der angeordneten abelschen Gruppe $\Gamma$ mit ein. So gilt für alle a, b $\in \Gamma$, dass: 
\[z^a \cdot z^b = z^{a+b} = z^{b+a} = z^b \cdot z^a.\] 
}
%
\subsection{Die Konstruktion des Inversen in $K\lbrack\lbrack z^{\Gamma}\rbrack\rbrack$ }
Die Menge $K[[z^\Gamma]]$ stellt bezüglich der definierten Addition und Multiplikation einen kommutativen Ring dar. Wir gehen bei der Konstruktion und dem Beweis des Inversen ähnlich wie \cite[S. 196- 198]{fuchs66} und \cite[S. 210- 213]{neumann49} vor.
Wir zeigen zunächst, dass die unendliche Summe des Produkts aus einem beliebigen Körperelement mit einem Element des formalen Potenzreihenrings, mit positivem Träger wohldefiniert ist und wieder in $K[[z^\Gamma]]$ liegt. Dieses Element spielt eine wichtige Rolle zur Konstruktion eines Inversen. Wir definieren die folgende Reihe:
\[\overline{F} = \sum_{n=0}^{\infty}\lambda_n\cdot F^n, \text{ mit } F \in K[[z^{\Gamma}]], \text{ wobei } \text{min}\left(\text{supp}\left(F\right)\right) > 0, \lambda_n \in K^*. \]
Wieso $\lambda$ eine Einheit und der kleinste Exponent der Unbestimmten $z$, für das der zugehörige Koeffizient ungleich null ist, positiv sein muss, klären wir im Folgenden. Die Potenzreihe $\overline{F}$ kann umgeschrieben werden:
\[\sum_{n=0}^{\infty}\left(\lambda_n\cdot F\right)^n = \lambda_0 F^0 + \sum_{n=1}^{\infty}\left(\lambda_n\cdot F\right)^n = \lambda_0 + \sum_{n=1}^{\infty}\left(\lambda_n\cdot F\right)^n.\] 
Diese Darstellung erinnert für $\lambda_0 = e$, wobei $e$ das neutrale Element der angeordneten abelschen Gruppe $\Gamma$ ist, an die geometrische Reihe. Jedes Element $G \in K[[z^\Gamma]]$ besitzt eine äquivalente Darstellung durch das Vielfache der Summe des neutralen Gruppenelements und einer formalen Potenzreihe deren Träger positiv ist. Denn für jede formale Potenzreihe $G \in K\lbrack\lbrack z^\Gamma\rbrack\rbrack$ mit $G:= \sum_{\gamma \in \Gamma}^{\infty}\Psi_\gamma z^\gamma$ ist supp($G$) wohlgeordnet und es existiert damit ein kleinstes Element $g\in G$ mit $g=$min(supp(G)). Da $\Gamma$ eine angeordnete abelsche Gruppe ist und damit für jedes Element ein additives Inverses existiert, lässt sich $G$ folgendermaßen schreiben
\[G= z^g ~\sum_{\gamma \in \Gamma} \Psi_{\gamma} z^{\gamma-g}\]
\[= z^g \left( e \Psi_g + ~\sum_{\gamma \in \Gamma\setminus \lbrace g \rbrace} \Psi_{\gamma} z^{\gamma-g}\right) = z^g ~ \Psi_g \left(e + ~\sum_{\gamma \in \Gamma\setminus \lbrace g \rbrace} \frac{\Psi_{\gamma}}{{\Psi_g}^{-1}} z^{\gamma-g}\right)\]
, wobei supp($~\sum_{\gamma \in \Gamma\setminus \lbrace g \rbrace} \frac{\Psi_{\gamma}}{{\Psi_g}^{-1}} z^{\gamma-g}) \ge 0$ ist.
Das Inverse zu $b_g$ existiert, da der Koeffizient im Körper $K$ liegt und da für $g$ als Minimum des Trägers von $G$ gelten muss, $\Psi_g \neq 0$. 
Nach dieser Argumentation ist klar ersichtlich, dass sich jede beliebige formale Potenzreihe $G$ aus $K\lbrack\lbrack z^\Gamma\rbrack\rbrack$ für $\lambda \in K, ~g \in\Gamma,~F \in K[[z^{\Gamma}]], \text{ wobei } \text{min}\left(\text{supp}\left(F\right)\right) > 0$ darstellen lässt.
\[G = \lambda\cdot z^g\cdot \left(e + F\right)\] 

Wir assozieren jetzt mit jedem Element $F := \sum_{a \in \Gamma} \Phi_a z^a$ des formalen Potenzreihenrings ein Symbol $e+ F$ und betrachten die Menge $\Upsilon$ aller $\mathfrak{F} = e + F$ mit $e$ als neutrales Element von $\Gamma$, $F \in K[[z^\Gamma]]$ und $\text{min}\left(\text{supp}\left(F\right)\right) > 0$. Diese Menge $\Upsilon$ ist eine Gruppe bezüglich der folgenden Verknüpfung \[\left(e+F\right)\left(e+G\right) = e + \left(F+G+FG\right)\]
, wobei die Operationen zwischen den Ringelementen der Addition und Multiplikation in $K[[z^\Gamma]]$ entsprechen. Die Abgeschlossenheit bezüglich der Multiplikation ist aus deren Definition klar ersichtlich. Das neutrale Element von $\Upsilon$ ist $e$. Mithilfe der geometrischen Reihe konstruieren wir das Inverse zu jedem Gruppenelement $\mathfrak{F}$ und zeigen in \ref{unendlicheSummeinPotenzreihenring}, dass dieses wohldefiniert ist und ein Element des Potenzreihenrings. Nach Definition der geometrischen Reihe gilt:
\[\frac{1}{e - F} = \sum_{n=0}^{\infty}F^n = e + \sum_{n=1}^{\infty}F^n\] oder in äquivalenter Darstellung:
\[\frac{1}{e + F} = = e + \sum_{n=1}^{\infty}(-F)^n.\]
Man sieht sofort, dass für jedes Gruppenelement $e+ F$ die Reihe $e + \sum_{n=1}^{\infty}(-F)^n$ invers ist. Wir müssen allerdings noch zeigen, dass $\sum_{n=1}^{\infty}(-F)^n$ im formalen Potenzreihenring $K\lbrack\lbrack z^\Gamma\rbrack\rbrack$ liegt.
%
%
\begin{lemma}\label{unendlicheSummeinPotenzreihenring}
Sei $F\in K\lbrack\lbrack z^\Gamma\rbrack\rbrack$ mit $\text{min}\left(\text{supp}\left(F\right)\right) > 0$, dann liegt für beliebige Körperelemente $\lambda_n$ die unendliche Reihe
\[\Xi = \sum_{n=1}^{\infty}\lambda_nF^n\] in $K\lbrack\lbrack z^\Gamma\rbrack\rbrack$ und ist wohldefiniert.
\end{lemma}
\beweis{Das Element liegt im Potenzreihenring, wenn der die Vereinigung der Träger supp($F^n$) wohlgeordnet ist in der Anordnung von G und es für jedes Element des Trägers $a \in $ supp($F$)nur endliche viele ganze Zahlen n gibt, sodass der $a$-te mit n potenzierte Koeffizient ungleich null ist. \\
Die erste Bedingung gilt als erfüllt wenn es keine unendlich abfallende Folge gibt
\begin{equation}\label{eq: folgefürinvers}
[u_1 = {a_1}_1 +{a_1}_2 + ...+ {{a_1}_n}_1 > u_2 = {a_2}_1 +{a_2}_2 + ...+ {{a_2}_n}_2 > ... > u_i = {a_i}_1 +{a_i}_2 + ...+ {{a_i}_n}_i, 
\end{equation} 
$\text{mit } \left( {a_i}_k \neq 0 \right)$.
Wir bezeichnen mit ${a_i}^* =$max$\left( {a_i}_k\right)$ das Maximum über allen k. Offensichtlich entspricht die von $u_i$ erzeugte konvexe Untergruppe, wir bezeichnen sie mit $\langle u_i\rangle$, der von dem größten Element $max_{k}\left({a_i}_k\right)$ konvexen Untergruppe. Die anderen Summanden von $u_i$ sind nach Definition des Maximums kleiner als dieses und liegen aufgrund der Konvexität der Untergruppe in dieser. Also gilt die Gleichheit$\langle u_i\rangle = \langle\text{max}\left({a_i}_k\right)\rangle$. Da der Träger supp($F^n$) wohlgeordnet ist, gibt es unter den erzeugten Untergruppen aller Elemente des Trägers eine kleinste Untergruppe U von $\Gamma$. Die konstruierte Folge \ref{eq: folgefürinvers} ist so gewählt, dass die kleinste Untergruppe möglichst klein ist. Damit bleibt die Ordnung der Folgenglieder auch für die davon erzeugten konvexen Untergruppen erhalten:
\[\langle u_1 \rangle \supseteq \langle u_2 \rangle \supseteq ... \supseteq \langle u_i \rangle \supseteq ...\]
Wir nehmen ohne Beschränkung der Allgemeinheit an, dass die von $u_i$, wobei $i$ natürliche Zahlen sind, erzeugten konvexen Untergruppen die gesamte Untergruppe erzeugt. Wir wählen nun aus jeder Folge von ${{a_i}_k}_i$ ein ${a_i}^*$, sodass die von diesem Element erzeugte konvexe Untergruppe den von $u_i$ erzeugten konvexen Untergruppen, eventuell unter weglassen endlich vieler Elemente der Folge, entspricht. Diese ist, wie oben ohne Beschränkung angenommen, ganz $U$. Da unsere Elemente $a$ aus dem Träger der Potenzreihe stammen, kann es zwar mehrere geben, die ganz $U$ erzeugen, allerdings aufgrund der Wohlordnung des Trägers nur ein kleinstes, nennen wir es $a^*$. Die von $a^*$, ${a_1}^*$ und $u_1$ erzeugten konvexen Untergruppen sind nach Annahme gleich. Für die erzeugenden Element gilt jedoch, da $a^*$ das kleinste erzeugende Element ist und $u_1$ den Summanden ${a_1}^*$ enthält, die folgende Ungleichung bezüglich der Anordnung von $G$: 
\[a^* \le {a_1}^* \le u_1\]
Aufgrund der Eigenschaften von konvexen Untergruppe einer angeordneten Gruppe existiert ein $p \in \N$ sodass $u_1 \le pa^*$ und da $u_1 > u_2 > ... > u_i > ...$ gilt $u_i \le pa^*$ für alle $i \in \N$. Wir wählen $p$ kleinstmöglich.
Jedes Element unserer anfangs gewählten Folge $u_i$ kann in einer der folgenden Formen geschrieben werden, wobei $v_i$ Summen aus Elementen von ${a_i}_k$ sind:
\begin{multicols}{2}
\item $u_i = {a_i}^*$
\item $u_i = v_i+{a_i}^*$
\end{multicols}
Da $\Gamma$ abelsch ist folgt aus diesen beiden Fällen ebenso: $u_i = {a_i}^* + v_i$. Die Elemente ${a_i}^*$ sind Elemente des Trägers und da dieser wohlgeordnet ist, gibt es keine unendlich abnehmende Folge von ${a_i}^*$ und damit existieren nur endlich viele $u_i$ der ersten Form. Da nach Voraussetzung $u_i$ eine unendlich abnehmende Folge ist muss eine unendlich abnehmende Folge $v_i$ existieren: ${v_i}_1 > {v_i}_2 > ... > {v_i}_j > ...$. Diese Folge hat die selbe Form wie \ref{eq: folgefürinvers} und die von $v_i$ erzeugte konvexe Untergruppe entspricht der minimalen von $u_i$ erzeugten konvexen Untergruppe, da $v_i \le u_i$. Wir können also wieder eine natürliche Zahl $q$ finden, sodass $v_i \le q~a^*$ für alle $i \in \N$. Diese natürliche Zahl $q$ ist kleiner gleich dem vorher gewählten $p$, da $v_i \le u_i$ und $u_i \le p~a^*$. Daraus folgt also, dass eine Folge $v_i$ aus $u_i$ konstruiert werden kann, was ein Widerspruch zur Wahl unserer Folge und der Minimaleigenschaft darstellt. Die Vereinigung der Träger supp($F^n$) muss somit wohlgeordnet sein. \\
Der erste Teil des Lemmas ist bewiesen. Wir nehmen nun an es existieren für jedes festgehaltene Element der angeordneten abelschen Gruppe $\Gamma$ existieren unendlich viele ganze Zahlen $n \in \N$,sodass \[a = {a_i}_1 +{a_i}_2 + ... + {a_i}_n,~ i \in \N,~\text{ mit } n_1 < n_2 < ... < n_i < ...\text{ und } {a_i}_k \in \text{supp}(F)\]
da die Vereinigung der Träger supp($F^n$) wohlgeordnet ist existiert ein kleinstes Element $a$ der oben definierten Form. Da supp($F^n$) wohlgeordnet ist, enthält die Folge $\left({a_i}_1\right)_{i\in \N}$ nach \ref{unendlicheFolgeEigenschaften} eine nichtabnehmende unendliche Teilfolge, die wir gleich indizieren:
\[{a_1}_1 \le {a_i}_1 \le ... \le {a_i}_1 \le ...\] und wir nehmen deshalb an, dass $\left({a_i}_1\right)_{i\in \N}$ nicht wachsend ist. Damit muss die durch $\left(a_i\right)' = -\left({a_i}_1\right) + a = {a_i}_2 + ... + {{a_i}_n}_i,~ i \in \N$ bestimmte Folge nicht wachsend und aufgrund der Wohlordnung der Vereinigung der Träger somit konstant sein. Es gibt also ein $j \in \N$ mit $\left(a_{j+m}\right)' = \left(a_j\right)' = a'$ für alle $m \in \N$. Damit liegt $a'$ in der Vereinigung der Träger supp($F^n$), für unendlich viele $n\in \N$. Weiterhin gilt $a' < a$, da $a' = -\left({a_i}_1\right) + a$ und ${a_i}_1 >e$. Dies ist ein Widerspruch zur Wahl von $a$. Damit existieren nur endlich viele ganze Zahlen $n$ für die $\left(\Phi^n\right)_n \neq 0$ ist.
}
Das Lemma liefert uns die gewünschte Aussage, $ \sum_{n=1}^{\infty}(F)^n$ liegt im formalen Potenzreihenring $K\lbrack\lbrack z^\Gamma\rbrack\rbrack$. Also enthält der Potenzreihenring ebenso $\overline{F} := \sum_{n=1}^{\infty}(-F)^n $, weil die negative formale Potenzreihe $\left(-F\right)$ die Voraussetzungen des Lemmas für $\lambda_n = 1$ erfüllt, $\text{min}\left(\text{supp}\left(-F\right)\right) > 0.$ 
Wir wissen also, dass für ein Element $\mathfrak{F} := e + F$, mit $F \in K\lbrack\lbrack z^\Gamma\rbrack\rbrack$ der Gruppe $\Upsilon$ ein Element $\overline{\mathfrak{F}} := e +\sum_{n=1}^{\infty}(-F)^n $ in $\Upsilon$ existiert, sodass das Produkt der beiden Gruppenelemente die folgende Form hat:
\[\mathfrak{F} \overline{\mathfrak{F}} \]
\[= \left(e  + F\right) \left(e + \overline{F}\right)\]
\[ = e + \left(F + \overline{F} + F\cdot\overline{F}\right) \]
\[= e + \left(\sum_{a \in \Gamma}\Phi_a z^a + \sum_{n=1}^{\infty}(-F)^n + \left(\sum_{a \in \Gamma}\Phi_a z^a\right)\cdot \left(\sum_{n=1}^{\infty}(-F)^n\right)\right)\]
\[ = e + \left( \sum_{n=2}^{\infty}(-F)^n + \left( - \sum_{n=2}^{\infty}(-F)^{n}\right)\right)\]
\[= e + e = e\]
\[ = e + \left(\sum_{n=1}^{\infty}(-F)^n + \sum_{a \in \Gamma}\Phi_a z^a + \left(\sum_{n=1}^{\infty}(-F)^n\right)\cdot \left(\sum_{a \in \Gamma}\Phi_a z^a\right)\right)\]
\[ =  \left(e  + \overline{F}\right) \left(e + F\right)\]
\[ = \mathfrak{\overline{F}}\mathfrak{F}\]
Die Menge $\Upsilon$ stellt also tatsächlich eine Gruppe dar und wir finden zu jedem Element ein Inverses. Mithilfe dieser Erkenntnisse sind wir nun in der Lage den zentralen Satz der Ausarbeitung zu beweisen.
\newpage
\begin{satz}
Die formalen Potenzreihen auf einer angeordneten Gruppe $\Gamma$ über einem Körper $K$ bilden einen Körper $K\lbrack\lbrack z^\Gamma\rbrack\rbrack$.
\end{satz}
\beweis{
Wie oben gezeigt, kann jedes Element $G\neq 0$ des Ringes $K\lbrack\lbrack z^\Gamma\rbrack\rbrack$ in der Form $G = \lambda\cdot z^g\cdot \left(e + F\right)$ geschrieben werden, mit $\lambda \in K^*, g \in\Gamma,~F \in K[[z^{\Gamma}]], \text{ wobei } \text{min}\left(\text{supp}\left(F\right)\right) > 0.$ Wir bezeichnen mit $e + \overline{F}$ das Inverse von $e + F$ in der Gruppe $\Upsilon$. Mit selbiger Argumentation wie oben wissen wir $H := \left(e + \overline{F}\right)z^{-g}\lambda^{-1} \in K[[z^{\Gamma}]]$. Da $\lambda\in K^*$ liegt, handelt es sich um eine Einheit und es existiert ein Inverses, welches wir mit $\lambda^{-1}$ bezeichnen. Da $g$ ein Element unserer angeordneten, abelschen, additiv geschriebenen Gruppe $\Gamma$ ist, gibt es auch zu $g$ ein inverses Element, das wir $-g$ nennen. Nach den Potenzgesetzen und den definierten Rechenoperationen in dem Potenzreihenring $K[[z^{\Gamma}]]$ ergibt sich, dass 
\[G\cdot H \]
\[= \left(\lambda\cdot z^g\left(e + F\right)\right) \cdot \left( \left(e + \overline{F}\right)z^{-g}\lambda^{-1}\right)\]
\[ = \left(\lambda\cdot z^g z^{-g}\lambda^{-1}\right) = \left(\lambda  z^{g-g}\lambda^{-1}\right)= \left(\lambda\lambda^{-1}\right)\]
\[= e \]
\[= \left(e + \overline{F}\right)z^{-g}\lambda^{-1}\cdot \lambda z^g \left(e + F\right)\]
\[= H\cdot G\]
Offensichtlich erweist sich e als Einselement von $K[[z^{\Gamma}]]$}
Potenzreihen auf einer angeordneten abelschen Gruppe mit wohlgeordnetem Träger formen über einem Körper K somit den Körper der formalen Potenzreihen $K\lbrack\lbrack z^\Gamma\rbrack\rbrack$. \\
Die Grundsteine dieser Theorie wurden von Hans Hahn 1907 in seinem Beweis, dass formale Potenzreihen auf einer angeordneten abelschen Gruppe über $\R$ einen Körper bilden gelegt. Neumann verallgemeinerte Hahns Ergebnisse und zeigte, dass formale Potenzreihen auf einer multiplikativen Gruppe in der nicht-kommutativen Sichtweise einen Schiefkörper formen. Im Laufe der Jahre konnte die Theorie der formalen Potenzreihen, als Verallgemeinerung der Laurentreihen und Pusieuxreihen, immer weiter ausgebaut werden. \\
Hier schließt sich der Kreis zu dem, zu Beginn des Kapitels, betrachteten Potenzreihenring $K[[z]]$ und dem Laurentreihenkörper $K((z))$. Die Menge der ganzen Zahlen ist eine angeordnete abelsche additive Gruppe. Über einem beliebigen Körper K wissen wir nun, dass der Laurentreihenkörper mit $\Gamma = \Z$ ein Beispiel für einen formalen Potenzreihenkörper darstellt.
%Wir haben nun gezeigt, dass sich jedes Element aus $K[[z^{\Gamma}]]$ mithilfe der Elemente $e + F$ der Gruppe $\Upsilon$ darstellen lässt. Da zu jedem Gruppenelement ein Inverses existiert
%Wir zeigen nun, dass die Division im Körper der formalen Potenzreihen auf $\Gamma$ über K wohldefiniert und eindeutig ausführbar, außer durch das neutrale Element der Addition $\epsilon$, ist. Das bedeutet, wenn $F, G \in K[[z^\Gamma$]] und F$\neq \epsilon$, dann gibt es ein Element H so dass $FH = G$ erfüllt und für jedes Element $G'$ des Potenzreihenkörpers gilt $G' = G$.\\
%\centerline{Sei F = $\Phi_{a_0}z^{a_0} + \Phi_{a_1}z^{a_1} + ... + \Phi_{a_n}z^{a_n} + ...$, wobei $a_0$ = min(supp(F)).}\\
%\centerline{und G = $\Psi_{b_0}z^{b_0} + \Psi_{b_1}z^{b_1} + ... + \Psi_{b_n}z^{b_n} + ...$, wobei $b_0$ = min(supp(G)),}
%zwei Elemente des Potenzreihenkörpers und die Träger der beiden Elemente somit wohlgeordnet. Gesucht wird H $\in K[[z^\Gamma]]$ derart, dass G = F$\cdot$ H gilt. Dazu setzen wir $H_1 := \frac{\Psi_{b_0}}{\Phi_{a_0}} z^{b_0 - a_0}$. Wir bilden $G_1$ = G - $H_1$F = ${\Psi_{}{{b_0}^1}}^1  z^{{b_0}^1} +{\Psi_{{b_1}^1}}^1 z^{{b_1}^1} + {\Psi_{{b_2}^1}}^1 z^{{b_2}^1}+ ... + {\Psi_{{b_n}^1}}^1 z^{{b_n}^1}+ +...$.\\
%Ist $G_1$ gleich Null, so ist $H_1$ die gesuchte Potenzreihe H. Andernfalls folgt ${\Psi_{b_0}^1}^1$ ist ungleich null und $ z^{{b_0}^1} <  z^{{b_0}}$. Wir setzen nun: \\
%\vspace{0.8cm}
%\centerline{$H_2 := H_1 + \frac{{\Psi_{b_0}^1}^1}{\Phi_{a_0}} z^{{b_0}^1 - a_0}$}\\
%und bilden: \\
%\vspace{0.8cm}
%\centerline{$G_2 = G_1 - H_2F = {\Psi_{}{{b_0}^2}}^2  z^{{b_0}^2} +{\Psi_{{b_1}^1}}^2 z^{{b_1}^2} + {\Psi_{{b_2}^1}}^2 z^{{b_2}^2}+ ... + {\Psi_{{b_n}^2}}^2 z^{{b_n}^1}+ +...$}.\\
%
%Ist $G_2$ gleich null, so ist $G_2$ die gesuchte Potenzreihe H.  Andernfalls folgt ${\Psi_{{b_0}^2}}^2$ ist ungleich null und $ z^{{b_0}^2} <  z^{{b_0}^1}$. Dieses Verfahren lässt sich fortführen. Entweder man erhält ab einem endlichen Index k zu einer Z Potenzreihe: \\
%\vspace{0.8cm}
%\centerline{$H_k = \frac{\Psi_{b_0}}{\Phi_{a_0}} z^{b_0 - a_0} + \frac{{\Psi_{b_0}^1}^1}{\Phi_{a_0}} z^{{b_0}^1 - a_0} + ... + \frac{{\Psi_{b_0}^{n-1}}^{n-1}}{\Phi_{a_0}} z^{{b_0}^{(n-1)} - a_0}$},\\
%und wir erhalten $G_n = G - H_nF = 0$.
%In diesem Fall ist $H_n$ die gesuchte Potenzreihe, sonst H und damit $G - H_nF$ für alle endlichen Indizes von null verschieden und es muss eine Potenzreihe $H_\omega$ existieren, die folgendes Aussehen hat:\\
%\vspace{0.8cm}
%\centerline{$H_\omega = \frac{\Psi_{b_0}}{\Phi_{a_0}} z^{b_0 - a_0} + \frac{{\Psi_{b_0}^1}^1}{\Phi_{a_0}} z^{{b_0}^1 - a_0} + ... + \frac{{\Psi_{b_0}^{n}}^{n}}{\Phi_{a_0}} z^{{b_0}^{n} - a_0} + ...$}.
%Damit erhalten wir für $G_\omega = G - H_\omega F = {\Psi_{}{{b_0}^\omega}}^\omega  z^{{b_0}^\omega} +{\Psi_{{b_1}^\omega}}^\omega z^{{b_1}^1} + {\Psi_{{b_2}^\omega}}^\omega z^{{b_2}^\omega}+ ... + {\Psi_{{b_n}^\omega}}^\omega z^{{b_n}^\omega}+ +...$.
%Wieder gilt, wenn $G_\omega$ = 0 ist, so ist $H_\omega$ die gesuchte Potenzreihe. Andernfalls lässt sich das Verfahren wieder fortführen, wie bereits angewendet.\\
%Allgemein ergibt sich folgende Formalisierung: Sei $\pi \in \Gamma$ und:\\
%\vspace{0.8cm}
%\centerline{$H_\omega = \frac{\Psi_{b_0}}{\Phi_{a_0}} z^{b_0 - a_0} + \frac{{\Psi_{{b_0}^1}}^1}{\Phi_{a_0}} z^{{b_0}^1 - a_0} + ... + \frac{{\Psi_{{b_0}^{\alpha}}}^{\alpha}}{\Phi_{a_0}} z^{{b_0}^{\alpha} - a_0} + ...$}  \\
%die Summe über alle Elemente der angeordneten Gruppe $\Gamma$ die < $\pi$ sind und \\
%\vspace{0.8cm}
%\centerline{$z^{b_0} > z^{{b_0}^1} > ... > z^{{b_0}^\alpha} > ...$}
%Sei p ein weiteres Element aus $\Gamma$, für das gilt: p < $\pi$. Für alle p < $\pi$ wissen wir, dass $G_p$ folgende Form hat:\\
%\vspace{0.8cm}
%\centerline{$G_p = G - H_p F =  {\Psi_{{b_0}^p}}^p  z^{{b_0}^p} +{\Psi_{{b_1}^p}}^p z^{{b_1}^p} + {\Psi_{{b_2}^p}}^p z^{{b_2}^p}+ ... + {\Psi_{{b_n}^p}}^p z^{{b_n}^p}+ +...$},\\
%wobei $\Psi_{{b_0}^p}^p \neq 0$ und der Träger der Summe wohlgeordnet ist. Wir stellen nun $G_\pi$ analog dar:\\
%\vspace{0.8cm}
%\centerline{$G_\pi = G - H_\pi F =  {\Psi_{{b_0}^\pi}}^\pi  z^{{b_0}^\pi} +{\Psi_{{b_1}^\pi}}^\pi z^{{b_1}^\pi} + {\Psi_{{b_2}^\pi}}^\pi z^{{b_2}^\pi}+ ... + {\Psi_{{b_n}^\pi}}^\pi z^{{b_n}^\pi}+ +...$}.
%Im Fall $G_\pi$ = 0, ist $H_\pi$ die gesuchte Zahl. Andernfalls können wir annehmen  $\Psi_{{b_0}^\pi}^\pi$ ist ungleich Null und wir beweisen, dass für jedes p < $\pi$ die Anordnung für den Wert der Potenz der Variablen erhalten bleibt, also: $z^{{b_0}^\pi} < z^{{b_0}^p} $.
%Wir setzen $H_\pi = H_p + H_{p*}$ und wählen $ H_{p*}$ folgendermaßen:\\
%\vspace{0.8cm}
%\centerline{ $ H_{p*}= \frac{{\Psi_{b_0}^p}^p}{\Phi_{a_0}} z^{{b_0}^p - a_0} + \frac{{\Psi_{b_0}^{p+1}}^{p+1}}{\Phi_{a_0}} z^{{b_0}^{p+1} - a_0} + ... + \frac{{\Psi_{b_0}^{\alpha}}^{\alpha}}{\Phi_{a_0}} z^{{b_0}^{\alpha} - a_0} + ... $ }.\\
%Wir erhalten $G_\pi = G - \left(G_p + G_{p*}\right) = G_p - H_{p*}F.$ Der Summand des höchsten Ranges stimmt in $G_p$ und $H_p*F$ überein, nämlich $\Psi_{{b_0}^p}^p  z^{{b_0}^p}$ und damit folgt, dass $z^{{b_0}^\pi} < z^{{b_0}^p} $ wie zu zeigen war. \\
%Wieder gilt, wenn $G - H_\pi F = 0$, so ist $H_\pi$ die gesuchte Potenzreihe, ansonsten wird der Prozess fortgesetzt. Dieses Verfahren muss terminieren. Ansonsten hätte der absteigend wohlgeordnete Träger der Potenzreihe von $G - H_\pi F$, der aus den Indizes ${b_0}^\pi$ gebildet wird, eine größere Mächtigkeit als die angeordnete Gruppe $\Gamma$. \\
%Es bleibt noch zu zeigen, dass nicht nur eine Potenzreihe bestimmt werden kann, derart dass FH = G gilt, sondern dass diese Potenzreihe auch eindeutig ist. Nach Definition der Multiplikation wird ein Produkt nur dann null, wenn einer der Faktoren der Nullreihe entspricht. 
%Es sei $F \neq 0, H\neq 0$, und sei $\Phi_{b_0}z^{b_0}$ das höchste Glied von F mit nicht verschwindendem Koeffizienten, also max(supp(F)) = $b_0$ und $\Psi_{b_0'}z^{b_0'}$ das höchste Glied von F mit nicht verschwindendem Koeffizienten, also max(supp(F)) = $b_0'$. Dann enthält G den Summand $\Phi_{b_0}\Psi_{b_0'}z^{b_0  + b_0'}$ und ist daher ebenfalls ungleich null. Sei also FH = G und FH' = G, dann gilt F(H-H') = 0 und da F aber nicht die Nullreihe ist, muss H = H' sein und damit H eindeutig.


%\begin{bsp}
%Sei $\Gamma = \N$ und $\le$ die natürliche Ordnung, dann ist $K[[z^\Gamma]] \cong K[[z]]$ wie in \ref{potenzreihenring} beschrieben.
%\end{bsp}


%\subsection{Pusieux Reihen}
%\subsection{Algebraische Abgeschlossenheit}
%The power series field K{tr
%} is algebraically closed if
%the coefficient field K is algebraically closed and if the ordered abelian
%group Y of exponents is a root group.
%PROOF. In the power series field S = K{tT
%} we introduce a valuation
%V by setting V(x) =ai if aait
%al is the first nonvanishing term in
%the power series (1) for x. In this valuation, Y is the value group and
%K the field of residue classes. Furthermore, J 5 is maximal with respect
%to this valuation, in the sense that any proper extension Sf>S to
%which the valuation V has been extended must either have a larger
%value group or a larger residue class field than S.
%Suppose now that S is not algebraically closed, so that S has a
%proper finite normal extension N. Certainly 5 is (algebraically) perfect,
%so that N/S is separable. The valuation V of 5 can be extended
%to N by the usual methods, for 5 is (topologically) perfect § with respect
%to V. The ordinary Newton polygon construction shows that each element c of N has a value of the %form a/n, with a in I\ Since F
%is a root group, a/n e T, and T is thus the value group of N. On the
%other hand, the residue class field of N must be an algebraic extension
%of the algebraically closed residue class field K of S. Thus N presents
%a proper extension of S in which neither value group nor residue class
%field is extended, contrary to the maximal property of S. 

% aus: http://projecteuclid.org/download/pdf_1/euclid.bams/1183502257 Mac Sanders Lane : The universalty of power series.

%zusatz bachelorarbeit
%
%\chapter{Schluss}
Die Grundsteine für die Theorie wurden von Hans Hahn 1907 gelegt. Seine Beweise, dass formale Potenzreihen auf einer angeordneten abelschen Gruppe über $\R$ einen Körper bilden, wurde im Lauf der Jahre immer weiter ausgebaut. Neumann verallgemeinerte Hahns Ergebnisse und zeigte, dass formale Potenzreihen auf einer multiplikativen Gruppe in der nicht-kommutativen Sichtweise einen Schiefkörper formen. Im Laufe der Jahre konnte die Theorie der formalen Potenzreihen, als Verallgemeinerung der Laurentreihen und Pusieuxreihen, immer weiter ausgebaut werden. Mithilfe von Faktorsystemen weiteten Neumann und Malcev' die Multiplikation im Körper aus. Angeführt von Wolfgang Krull, erreichte die Bewertungstheorie die Vereinfachung des Beweises, dass es sich um Potenzreihenkörper handelt.  
\bibliography{literaturverzeichnis}
\bibliographystyle{alphadin}

\end{document}


