\section{Der Hahnsche Einbettungssatz}\label{HahnscheEinbettungssatz}

Wie in den vorherigen Paragraphen ausgeführt, sind archimedische Gruppen abelsch und isomorph zu Untergruppen der additiven Gruppe der reellen Zahlen. Das heißt, es gibt einen injektiven monotonen Homomorphismus $f\colon G \mapsto$ $\left(\R , +\right)$. Existiert ein weiterer solcher Homomorphismus $g\colon G \mapsto$ $\left(\R, +\right)$, dann existiert genau eine reelle Zahl $r > 0$, mit $g(x) = r\cdot f(x)$ für alle $x \in G$, siehe Satz \ref{homomorphismus nach R}.\\
Da nur eine eingeschränkte Betrachtung geordneter Strukturen möglich war, stellte sich Hahn als einer der ersten 1907 der Fragestellung, ob es sogenannte nicht-archimedische Strukturen gibt, die eine gewisse Art der Vollständigkeit besitzen, ähnlich den reellen Zahlen.\cite{hahn07} In der abstrakten Algebra, die sich mit angeordneten abelschen Strukturen beschäftigt, lieferte Hahns Ausweitung des Satzes von Bettazi/Hölder \ref{aga}, der sogenannte Hahnsche Einbettunsgssatz, eine wichtige Beschreibung für nicht-archimedische Anordnungen. \\
In dem Kapitel %TODO Referenz einfügen
geben wir einen kleinen Ausblick in die Theorie der nicht-archimedisch angeordneten Ringe und Körper. % dieser Ausblick ist aus Lüneburg S548 übernommen.
\vspace{0.8cm}
Wir betrachten eine total geordnete abelsche Gruppe $\Gamma$ mit der Addition als Verknüpfung von Gruppenelementen. Aus den in \ref{Archimedisch angeordnete Gruppen} vorgestellten Axiomen folgt, dass eine Gruppe archimedisch ist, wenn alle Elemente $a \in \Gamma$ mit $a \neq 0$ archimedisch äquivalent sind. Der Satz von Hölder besagt, jede angeordnete abelsche Gruppe ist archimedisch genau dann, wenn sie bis auf Isomorphie einer Untergruppe der additiven Gruppe der reellen Zahlen $\left(\R, +\right)$ entspricht. An die Stelle von $\left(\R, +\right)$ tritt die sogenannte Hahn-Gruppe H($\Gamma, G_\gamma$). Die Hahn-Gruppe ist ein spezieller Funktionenraum, unter dem wir das lexikographische Produkt der Gruppen $G_\gamma$ über $\Gamma$ verstehen, für das jede Gruppe $G_\gamma$ eine Untergruppe von $\left(\R +\right)$ ist. Falls wir archimedisch angeordnete Gruppen betrachten reduziert sich der Hahnsche Einbettungssatz auf den Satz \ref{aga}, denn die archimedischen Klassen $\Omega$ bestehen nur aus der einelementigen Menge und somit gilt: $\R^\Omega = \R $.\\
Hahn bewies diese Aussage 1907 in seinem Werk $\glqq$ Über nichtarchimedische Größensysteme$\grqq$ in einem 27-seitigen Beweis, der von vielen Mathematikern heutzutage als transfiniter Marathon bezeichnet wird. Conrad lieferte 1953 als erster einen einfacheren Beweis des Einbettungssatzes, ein Jahr später übertrug Clifford die Aussage auf abelsche Gruppen. \\

Sei $\Gamma$ eine angeordnete Menge und für jedes Element der Menge sei $\left(G_\gamma, +\right)$ eine angeordnete Gruppe. $\prod_{\gamma \in \Gamma} G_\gamma$ bezeichnet das vollständige direkte Produkt der Gruppen $G_\gamma$. Wir definieren, ähnlich wie in \ref{traeger}, den Träger eines Elements $f$ des direkten Produkts als supp$\left(f\right) = \lbrace y \in \Gamma: f(y) \neq 0 \rbrace$. \cite{priesscrampe83}
\begin{defn} %\cite{priesscrampe83}
Die Menge der Elemente aus $\prod_{\gamma \in \Gamma} G_\gamma$ bezeichnen wir als das \textit{lexikographische Produkt} ${\prod_{\gamma \in \Gamma}}_{Lex} G_\gamma$ der Gruppen $G_\gamma$, wobei $\gamma \in \Gamma$, wenn der Träger wohlgeordnet ist.
\end{defn}
Das Lexikographische Produkt ${\prod_{\gamma \in \Gamma}}_{Lex} G_\gamma$ bildet eine Untergruppe von  $\prod_{\gamma \in \Gamma} G_\gamma$ und wir definieren den Positivbereich $P$ folgendermaßen:\\
\[ P = \lbrace f \in {\prod_{\gamma \in \Gamma}}_{Lex} G_\gamma: 
f\left(\text{min(supp}(f))\right) > 0 \rbrace.\]
Betrachte eine Teilmenge $A$ von $\Gamma$, sodass für alle $\alpha \in A$ und für ein $y\in\Gamma$ gilt: $\alpha < y$. Die Menge ${\prod_{\gamma \in A}}_{Lex} G_\gamma$ ist bis auf o-Isomorphie eine konvexe Untergruppe von ${\prod_{\gamma \in \Gamma}}_{Lex} G_\gamma$.\\
Wir zeigen nun, dass diese Menge ein Normalteiler von ${\prod_{\gamma \in \Gamma}}_{Lex} G_\gamma$ ist. Mithilfe von Normalteilern können Faktorgruppen gebildet werden. Zu zeigen ist:  ${\prod_{\gamma \in A}}_{Lex} G_\gamma$ ist invariant unter der Konjugation $-g+f+g = f$, wobei $f \in  {\prod_{\gamma \in A}}_{Lex} G_\gamma$ und g$\in  {\prod_{\gamma \in \Gamma}}_{Lex} G_\gamma$. Es gilt supp$\left(f\right) = $supp$\left(-g + f + g\right)$, und damit ist  ${\prod_{\gamma \in A}}_{Lex} G_\gamma$ ein Normalteiler von  ${\prod_{\gamma \in \Gamma}}_{Lex} G_\gamma$. Die daraus konstruierte Faktorgruppe  ${\prod_{\gamma \in \Gamma}}_{Lex} G_\gamma /  {\prod_{\gamma \in A}}_{Lex} G_\gamma$ ist o-isomorph zu  ${\prod_{\gamma \in \Gamma\setminus A}}_{Lex} G_\gamma$.
Hahn benötigte, um auch die Einbettbarkeit nichtarchimedisch geordneter Gruppen in einen Funktionenraum zu zeigen, eine speziellere Menge als die im Hölderschen Einbettungssatz \ref{aga} verwendete additive Gruppe der reellen Zahlen $\left(\R, +\right)$. Dazu konstruierte er die sogenannte \textit{Hahn-Gruppe}.
\begin{defn}\label{Hahn-Gruppe}  %\cite{priesscrampe83}
Das lexikographische Produkt  ${\prod_{\gamma \in \Gamma}}_{Lex} G_\gamma$, für das jede Gruppe $G_\gamma$, $\gamma \in\Gamma$ eine Untergruppe von $\left(\R, +\right)$ ist, bezeichnen wir als \textit{Hahn-Gruppe}.
\end{defn} 
Sei $\left(G, +\right)$ eine angeordnete abelsche Gruppe und $\Gamma = [G]\setminus \lbrace 0\rbrace$. Für $\gamma \in G$ bilden die konvexen Untergruppen 
$G_\gamma = \lbrace x \in G: [x] < \gamma\rbrace $ und $G^\gamma= \lbrace x \in G: [x] \le \gamma \rbrace $ einen Sprung $ G_\gamma \prec G^\gamma$ in $\Sigma$ und damit gilt für jedes Element 0 $\neq g \in G$, dass $g \in G^\gamma \setminus G_\gamma $. Die Faktorgruppe ist eine angeordnete archimedische Gruppe und damit nach dem Satz von Hölder (\ref{aga}) ordnungsisomorph zu einer Untergruppe der additiven Gruppe der reellen Zahlen.\cite{hahn07}\\\\
In folgendem Lemma betrachten wir Vektorräume über einem Körper $K$. Banachschewski zeigte die Aussage in seinem Werk für Module über Schiefkörpern \cite[Lemma 4, S. 431 - 433]{banachschewski56}, in unserem Fall genügt es, die Aussage auf Vektorräume über Körpern zu spezialisieren. 
\begin{lemma}\label{einbettungssatzLemma}
Es gibt Abbildungen $\phi: S(V) \rightarrow S(V)$, wobei $V$ einen $K$-Vektorraum bezeichnet und $S(V)$, die Menge der Untervektorräume von $V$, die jedem Element $U$ von $S(V)$ einen Untervektorraum $\phi(U)\subseteq V$ zuordnen. Diese Abbildungen erfüllen folgende Eigenschaften für alle $U, W \in S(V)$: \\
\begin{enumerate}
\item[(1)] Aus $U \subseteq W $ folgt $\phi(U) \supseteq \phi(W)$,
\item[(2)] $W \cap ~\phi(W) = 0$ und $V = W \oplus ~ \phi(W)$.
\end{enumerate} 
\end{lemma}
\beweis{Wir wählen nun einen Untervektorraum $A$ und betrachten die Menge der Untervektorräume davon, auf der wir Funktionen definieren, die die geforderten Eigenschaften erfüllen. $S(V)$ bestimmt auf jedem Untervektorraum $A$ von $V$ die Menge $S(V)_A$ aller $U \cap A$, $U \in S(V)$. Die Abbildungen auf $S(V)_A$ mit oben genannten Eigenschaften werden mit $\phi_A$ bezeichnet. Wir zeigen, dass der gewählte Untervektorraum $A$ dem Vektorraum $V$ entspricht, um das Lemma zu beweisen. \\
Dazu definieren wir zunächst eine Ordnung auf der Menge $\Phi$, bestehend aus allen Abbildungen $\phi_A$. Wählt man $A$ als die Menge des Produkts aus den Körper- und Untervektorraumelementen $A = \lbrace Ka ~|~ a\in V\rbrace$, so sieht man sofort, dass die Menge derartiger Funktionen nicht leer ist. Es gilt für $\phi_A,~ \phi_B \in \Phi $, dass $\phi_A \ge \phi_B $ ist, wenn $A \subseteq B$ und $ \phi _A(U \cap A) \subseteq  \phi_B(U \cap B)$ für jedes $U\in S(V)$. \\
Wir zeigen nun, dass jede nichtleere geordnete Teilmenge von $\Phi$ ein minimales Element besitzt, damit gilt dann der Wohlordnungssatz.\\
Sei $\Psi$ eine Kette in $\Phi$, dann bezeichne $\mathfrak{B}$ die Menge aller $B$ mit $\phi_B \in \Psi$ und $A = \bigcup B,~ B \in \mathfrak{B}$. In $S(V)_A$ ist also $\phi_A (U \cap A) = \bigcup \phi_B( U \cap B)$, wobei $B \in \mathfrak{B}$ und da $\Psi$ eine geordnete Menge ist, gilt:\\
Wenn x ein Element von $(U \cap A) \cap \phi_A(U \cap A)$ ist, dann liegt x ebenso in $(U \cap B) \cap \phi_B( U\cap B)$ mit passendem B $\in \mathfrak{B}$ und daher ist $x= 0$.\\
Die Eigenschaft (2) ist erfüllt, da für B = $(U \cap B) \oplus \phi_B( U \cap B)\subseteq (U \cap A) \oplus \phi_A(U \cap A)$ für alle B $\in \mathfrak{B}$ folgt $(U \cap A) \oplus \phi_A(U\cap A) = A$.\\ 
Für $ U \cap A  \subseteq U'\cap A$ gilt auch $U\cap B \subseteq U' \cap B$ für jedes $B \in \mathfrak{B}.$ Da $\phi_B \in \Phi$ ist, erhalten wir $\phi_B(U \cap B) \supseteq \phi_B(U' \cap B)$. Aus A = $\bigcup B$ folgt, dass $\phi_A(U \cap A) \supseteq \phi_A(U'\cap A).$ Die Abbildung $\phi_A$ erfüllt also beide Eigenschaften und gehört somit zu $\Phi$. Die Menge $\Phi$ ist fundiert geordnet. \\
Daraus folgt, dass jede nichtleere Teilmenge von $\Phi$ ein minimales Element besitzt. Sei $\Phi_A$ ein minimales Element. Wir zeigen, dass $A = V$ folgt und somit das Lemma bewiesen ist. \\
Angenommen $A \neq V$, dann existiert ein Element $c \notin  A$. Wähle $B = A \oplus Kc$, $c \in V $, wobei K der Körper ist über dem der Vektorraum V definiert ist. Für die Abbildung $\phi_B$ wählt man:
\[\phi_B( U \cap B)= \begin{cases}
  \phi_A(U\cap A) \oplus Kc,  & \text{wenn } U \cap B \subseteq A\text{, das heißt: } U \cap B = U \cap A,\\
  \phi_A(U\cap A), &  \text{ sonst.}
\end{cases}\]
Es wird gezeigt, dass die konstruierte Abbildung in der Kette $\Phi$ liegt, die, wie oben bewiesen, ein minimales Element besitzt. Nach Definition der Ordnung gilt $\phi_A \ge \phi_B$, da aus B := $A\oplus Kc$ notwendigerweise A $\subseteq$ B folgt und nach Wahl unserer Abbildung gilt:
\[\left.\phi_A(U \cap A) \subseteq \begin{cases}
  \phi_A(U\cap A) \oplus Kc,  & \text{wenn } U \cap B \subseteq A\text{, das heißt: } U \cap B = U \cap A,\\
  \phi_A(U\cap A), &  \text{ sonst.}
\end{cases}\right\}=\phi_B( U \cap B)\]
Da $A \neq B$ nach Voraussetzung erfüllt ist, wäre $\phi_A \ge \phi_B.$\\
Sei $(U' \cap A) \subseteq A$, so erhält man $(U \cap A) \subseteq ( U' \cap A)$, dass $\phi_B( U \cap B ) = \phi_A( U \cap A) \oplus Kc \supseteq \phi_A (U' \cap A) \oplus Kc = \phi_B(U' \cap B)$. Im zweiten Fall, $(U' \cap B) \nsubseteq A$, erhält man $\phi_B( U \cap B) \supseteq \phi_A( U \cap A) \supseteq \phi_A(U' \cap A) = \phi_B(U'\cap B).$ \\
Die Abbildung $\phi_B$ erfüllt im ersten Fall, wie leicht zu sehen ist, die zur Zugehörigkeit zu $\Phi$ geforderte Eigenschaft (2). Im zweiten Fall gilt offensichtlich $(U \cap B)\nsubseteq A$, das heißt $B = (U\cap B) + \phi_A( U \cap A)$, da für $ a\in A,~ 0 \neq \lambda \in K$, $a+\lambda~c$ $\in (U \cap B )$ und wegen $A \subseteq (U \cap B) + \phi_A(U\cap A)$ liegt auch c = $(\lambda^{-1} ( a + \lambda c - a)) \in (U \cap B) + \sigma_A(U\cap A)$. Weiterhin folgt aus $\phi_A (U \cap A) \subseteq A $ und da $\phi_A \in \Phi$ und somit die Eigenschaften (1), (2) erfüllt, gilt ebenso $(U \cap A) \cap \phi_A(U \cap A) = 0$. Damit erhält man $(U \cap B) \cap \phi_A(U \cap A) = 0$. Die Eigenschaft (2) ist also auch im zweiten Fall erfüllt.\\
Damit liegt $\phi_B \in \Phi$.
Nach Definition von B gilt offensichtlich $A \neq B$ und somit $\phi_A > \phi_B$. Dies ist ein Widerspruch zur Minimalität von $\phi_A$ und folglich muss gelten $A = V$. 
}
%\begin{satz}
%Jede angeordnete abelsche Gruppe ist zu einer Untergruppe eines angeordneten Vektorraumes über $\Q$ ordnungsisomorph. 
%\end{satz}
%
%\begin{satz} 
%Jeder angeordnete Vektorraum G über K(x), wobei K(x) der rationales Funktionenkörper, ist einem Unterraum des lexikographisch geordneten Funktionenraums $\Q$ o-isomorph. 
%\end{satz} Das beweisen wir auf dem Weg zum Hahnschein Einbettungssatz
Die folgende Aussage, zu deren Beweis obiges Lemma benötigt wird, stellt eine Variante des Hahnschen Einbettungssatzes dar, der anschließend formuliert wird. Die Beweisführung basiert auf der Arbeit Banachschewskis \cite[S. 431 - 433]{banachschewski56} und enthält Elemente, die Prieß-Crampe im Kapitel $\glqq$Der Hahnsche Einbettungssatz$\grqq$ \cite[Satz 2, S. 16 - 18]{priesscrampe83} verwendete.
%
%
\begin{satz}\label{satzBanachschewski}
Sei $\left(G, +\right)$ eine angeordnete teilbare abelsche Gruppe und $\Gamma = [G]\setminus \lbrace [0]\rbrace$. Es gibt einen injektiven o- Homomorphismus $\varphi: G \mapsto  {\prod_{\gamma \in \Gamma}}_{Lex} G^\Gamma/ G_\Gamma$, für den gilt $g \in G^\gamma \setminus G_\gamma$ genau dann, wenn $\gamma$ das Minimum des Trägers von $\overline{g} = \varphi(g)$ ist und es folgt $\overline{g}(\gamma) = g + G_\gamma.$
\end{satz}
\beweis{$G$ ist eine angeordnete, teilbare, abelsche Gruppe, nach \ref{angeordnetFolgtTorsionsfrei} daher torsionsfrei.  Der Struktursatz für teilbare abelsche Gruppen besagt, dass G als Vektorraum über den rationalen Zahlen betrachtet werden kann. %TODO: evtl explizit aufschreiben
Für eine konvexe Untergruppe $U$ von $G$ gilt, dass diese ebenfalls teilbar ist und, da $G$ als Vektorraum über $\Q$ gesehen werden kann, einen Untervektorraum von $G$ darstellt. Zu jedem $\lambda \in \Q$ und $x \in U$ gibt es nämlich Elemente $h, k \in \Z$ mit $hx \le \lambda x$ $\le kx$ und wegen $hx, kx \in U$ folgt $\lambda x \in U$. \\
Wie in Lemma \ref{einbettungssatzLemma} sei $S(G)$ die Menge der Untervektorräume von $G$ und $\phi: S(G) \rightarrow S(G) $ eine Abbildung, welche die geforderten Eigenschaften erfüllt. Damit gilt für alle $\gamma \in \Gamma$, dass die Gruppe $G = G^\gamma \oplus \phi(G^\gamma)$ ist. Jedes Element $g$ aus $G$ lässt sich darstellen als Summe: 
\[g = g_\gamma + {g_\gamma}^\phi \text{ mit } g_\gamma \in G^\gamma \text{ und } {g_\gamma}^\phi \in \phi(G^\gamma)\]
Die Abbildung $\varphi: G \rightarrow \prod_{\gamma \in \Gamma}G^\gamma/ G_\gamma,$ $g \mapsto \overline{g}$, mit $\overline{g}(\gamma)= g_\gamma + G^\gamma$, wobei $g_\gamma \in G^\gamma$ wie oben, ist ein Monomorphismus. Die Injektivität folgt direkt aus deren Definition. Sei $g_1 =  {g_1}_\gamma + {{g_1}_\gamma}^\phi$ und $g_2 =  {g_2}_\gamma + {{g_2}_\gamma}^\phi$. Da $\varphi(g_1) = \varphi(g_2) $ ist, folgt für $ \overline{g_1}(\gamma)=  {g_1}_\gamma + G^\gamma$ und $\overline{g_2}(\gamma)=  {g_2}_\gamma + G^\gamma$ , dass ${g_1}_\gamma = {g_2}_\gamma$. Es lässt sich leicht nachrechnen, dass es sich bei $\phi$ um einen Homomorphismus handelt.\\
Wir zeigen nun, dass für jedes Element der angeordneten, abelschen, teilbaren Gruppe $G$ der Träger supp($\overline{g}$) des Bildes unter $\varphi$ wohlgeordnet ist. So erhalten wir, dass der Wertebereich der Abbildung auf die Menge ${\prod_{\gamma \in \Gamma}}_{Lex} G^\Gamma/ G_\Gamma$, die nach Definition genau aus den Elementen besteht, deren Träger wohlgeordnet ist, eingeschränkt werden kann. Wir wählen eine nichtleere Teilmenge $T$ des Trägers supp($\overline{g}$) = $\lbrace \gamma \in \Gamma\colon g_\gamma + G^\gamma \neq G^\gamma \rbrace$. Die Vereinigung der Untergruppen $\bigcup_{\gamma \in T} G^\gamma =: G^T$ ist als Vereinigung konvexer Untergruppen nach \ref{EigenschaftenKonvexeUgr} wieder konvex und ein Untervektorraum von $G$. Damit ist $g = x +y $, $x \in G^T,~ y \in \phi(G^T)$ bezüglich der Zerlegung $G = G^T \oplus \phi(G^T)$. Da $x \in G^T$ liegt, existiert ein $\beta \in T$ sodass x$\in G^\beta$. Nach Lemma \ref{einbettungssatzLemma} (1) gilt für $\gamma \in T$ und $G^\gamma \subseteq G^T$, dass $\phi(G^T)\subseteq \phi(G^\gamma)$. Also liegt auch $y \in \phi(G^\gamma)$ für $\gamma \in T$  
und damit ist $\phi(G^\gamma)$ im Kern des Monomorphismus und da $\phi$ injektiv ist, erhält man $y_\gamma = 0$ für alle $ \gamma \in T$.\\
Es ergibt sich für $g_\gamma = x_\gamma + y_\gamma$ , $g = x_\gamma$ für $\gamma \in T$ und wir können für $x$ natürlich auch ein $\beta \in T$ wählen, dass $x = x_\beta = g_\beta \in G^\beta \setminus G_\beta$. Die lineare Ordnung der Untervektorräume, wie in \ref{einbettungssatzLemma} liefert für $\gamma \in T$ und $\gamma \le \beta$, dass $x \in G^\beta \subseteq G^\gamma$, also $x = x_\gamma = g_\gamma = x = g_\beta$. Damit ist $\beta \in T $ das kleinste Element in T und wir haben die Wohlordnung des Trägers supp($\overline{g}$) = $\lbrace \gamma \in \Gamma: g_\gamma + G^\gamma \neq G^\gamma \rbrace$ gezeigt. Die Abbildung $\varphi$ ist also ein injektiver Homomorphismus von $G$ in das lexikographische Produkt ${\prod_{\gamma \in \Gamma}}_{Lex} G^\Gamma/ G_\Gamma$.\\
Nun zeigen wir, dass für  $g \in G^\gamma \setminus G_\gamma$, $\gamma$ das Minimum des Trägers von $\overline{g} = \varphi(g)$ ist. Sei $g \in G^\gamma \setminus G_\gamma$, $g = g_\alpha$ wenn $\alpha \ge \gamma$ für alle $a \in \Gamma$ und g $\in G_\alpha$ mit $\alpha < \gamma$. Deswegen ist $\gamma$ das minimale Element für das $g_\alpha + G_\alpha \neq G_\alpha$, für $\alpha \in \Gamma$ ist und damit das Minimum des Trägers von $\overline{g}$. Wir erhalten $\overline{g(\gamma}) = g + G_\gamma$. Sei nun $\gamma$ das Minimum des Trägers von $\overline{g} = \varphi(g)$. Dann gilt für alle $\alpha \in \Gamma$ mit $\alpha < \gamma$, dass $g_\alpha \in G_\alpha$, wenn $g_\gamma \in G^\gamma \setminus G_\gamma$. Die Archimedische Klasse von $g_\gamma$ entspricht $\gamma$ und daher liegt auch das Element $g \in G^\gamma \setminus G_\gamma$.\\
Es bleibt nur noch zu zeigen, dass das Bild von positiven Elementen positiv bleibt, also die Ordnungstreue der Abbildung. Wähle $0 > g \in G$ und g $\in G^\gamma \setminus G_\gamma$. Natürlich ist $G_\gamma > g + G_\gamma = g_\gamma + G_\gamma$ und $\gamma$ das Minimum des Trägers von $\varphi(g)$, also ist $\varphi(g)$ positiv. 
}
Auf die Voraussetzung der Teilbarkeit der gewählten Gruppe $G$ kann in obigem Satz nicht verzichtet werden, denn in diesem Fall sind die Untergruppen $G^\gamma, G_\gamma$ nur noch Untergruppen der teilbaren Hülle und nicht von $A$ selbst.\\ 
Wir wählen nun eine angeordnete abelsche Gruppe $\left(A, +\right)$. Nach Satz \ref{angeordnetFolgtTorsionsfrei} ist $A$ torsionsfrei und lässt sich bis auf Isomorphie in die eindeutig bestimmte teilbare Hülle $G$ einordnen \ref{torsionsfreiHülle}, die die Anordnung von $A$ fortsetzt, was leicht mithilfe des Positivbereichs zu zeigen ist. Die Menge der archimedischen Klassen von $A$ ist bijektiv und ordnungstreu zur Menge der archimedischen Klassen der Hülle, deswegen setzen wir diese gleich.  \\
Die Gruppen $G^\gamma \setminus G_\gamma$ erfüllen die archimedische Eigenschaft und es gilt der Satz von Hölder \ref{aga}. Damit gibt es eine ordnungstreue Einbettung $\tau$ des Lexikographischen Produkts der Gruppen ${\prod_{\gamma \in \Gamma}}_{Lex} G^\Gamma/ G_\Gamma$in die Hahn-Gruppe H($\Gamma, \R$).\\
Wir können nun die zentrale Aussage des Kapitels formulieren, den \textbf{Hahnschen Einbettungssatz}:
\begin{satz} \label{hebs} %\cite{priesscrampe83}
Eine angeordnete abelsche Gruppe A lässt sich ordnungstreu in die Hahn-Gruppe H($ \Gamma $, $ \R $) einbinden, wobei $ \Gamma$ =$ [A]\setminus \{[0]\} $.
\end{satz}
Insgesamt erhalten wir also eine Reihe von Einbettungen, die zeigt, dass der Satz von Banachschewski \ref{satzBanachschewski} den Hahnschen Einbettungssatz impliziert.\\
%Bricht man die Folge der ordnungstreuen Einbettungen nach dem lexikographischen Produkt ab, so folgt die Einbettbarkeit auch von nicht-teilbaren angeordneten, abelschen Gruppen A in das lexikographische Produkt ${\prod_{\gamma \in \Gamma}}_{Lex} G^\Gamma/ G_\Gamma$.\\\\
\[A \hookrightarrow G \stackrel{\mathrm{\varphi}}\hookrightarrow {\prod_{\gamma \in \Gamma}}_{Lex} G^\Gamma/ G_\Gamma \stackrel{\mathrm{\tau}}\hookrightarrow H(\Gamma, \R)\]
%Jede angeordnete abelsche Gruppe lässt sich in eine angeordnete abelsche teilbare Gruppe G einbetten. Der Satz von Banachschewski \ref{satzBanachschewski} zeigt weiterhin, dass es einen injektiven Homomorphismus der Gruppe G mit den genannten Eigenschaften in das lexikographische Produkt ${\prod_{\gamma \in \Gamma}}_{Lex} G^\Gamma/ G_\Gamma$ gibt und 

