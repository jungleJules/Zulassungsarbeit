\chapter{Einleitung}
%
Potenzreihen finden in vielen Bereichen der Mathematik Anwendung. In dieser Arbeit beschäftigen wir uns mit formalen Potenzreihen und den algebraischen Strukturen, die auf ihnen definiert werden können. \\
Ordnung spielt  in der Geschichte der Mathematik seit jeher eine essentielle Rolle. Allerdings beschäftigten sich die ersten Mathematiker erst gegen Ende des 19. Jahrhunderts mit Ordnung in Verbindung mit algebraischen Strukturen. Beginnend mit Arbeiten von Hölder, Hahn und Hausdorff entwickelte sich die Theorie der angeordneten Strukturen. In seinem Werk $\glqq$Die Axiome der Quantität und die Lehre vom Maß$\grqq$ zeigte Hölder 1901, dass sich jede archimedisch angeordnete Gruppe in eine Untergruppe der additiven Gruppe der reellen Zahlen einbetten lässt. Hölder bediente sich dabei der von Dedekind eingeführten Schnitte in $\Q$. Diese These baute Hahn auf nichtarchimedische angeordnete Strukturen aus und konstruierte in diesem Prozess verallgemeinerte Potenzreihenkörper.    \\
Mit derartigen Körpern beschäftigen wir uns in dieser Arbeit. Nach einer kurzen Wiederholung der wichtigsten algebraischen Strukturen erfolgt ein Einblick in die Theorie der angeordneten abelschen Gruppen. Diese werden zur Konstruktion des verallgemeinerten Potenzreihenkörpers benötigt. Um auf dem verallgemeinerten Potenzreihenkörper Verknüpfungen definieren zu können, spielt die Wohlordnung bestimmter Mengen eine entscheidende Rolle. Die Archimedizität führt zur zentralen Aussage des ersten Kapitels, dem Satz von Hölder. \\
Im zweiten Teil beschreiben wir die Menge der formalen Potenzreihen mit Exponenten in den natürlichen Zahlen über einem Körper. Eine Erweiterung dieser Menge der formalen Potenzreihe führt uns zum Körper der Laurentreihen, einem Beispiel für einen Potenzreihenkörper.\\
Basierend auf Arbeiten von Fuchs und Prieß-Crampe wird die Konstruktion formaler Potenzreihen mit Exponenten in einer angeordneten abelschen Gruppe durchgeführt. Diese Potenzreihen bilden, unter bestimmten Voraussetzungen, einen verallgemeinerten Potenzreihenkörper. Abschließend gehen wir auf einige Eigenschaften des verallgemeinerten Potenzreihenkörpers ein. 




%aus bibliograhpie entfernte elemente 
%@unpublished{kaiser14,
%author = "Tobias Kaiser",
%title = "Funktionentheorie",,
%year = "2014"
%note= "Vorlesungsskript zur Veranstaltung Funktionentheorie"}
%
%@unpublished{kaiser13,
%author = "Tobias Kaiser",
%title = "Algebra und Zahlentheorie 2",,
%year = "2013"
%note= "Vorlesungsskript zur Veranstaltung Algebra und Zahlentheorie 2"}
%
%
%@ONLINE{Matheplanet,
%author = "Gockel",
%title = "Formale Laurentreihen",
%year = "2013",
%url = "http://www.matheplanet.com/default3.html?call=viewtopic.php?topic=185478&ref=https%3A%2F%2Fwww.google.de%2F"}
