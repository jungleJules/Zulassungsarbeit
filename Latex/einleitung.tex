\chapter{Einleitung}
%
Potenzreihen finden in vielen Bereichen der Mathematik Anwendung. In dieser Arbeit beschäftigen wir uns mit formalen Potenzreihen und den algebraischen Strukturen, die auf ihnen definiert werden können. Der Text ist in drei Hauptteile gegliedert. Als erstes erfolgt ein kleiner Einblick in die Theorie der angeordneten Gruppen, die später zur Konstruktion des Hahnschen Potenzreihenkörpers benötigt werden.  Im zweiten Teil werden Eigenschaften des  Potenzreihenrings über einem Körper näher beschrieben und verallgemeinert bewiesen wie der Körper der Laurentreihen entsteht. Basierend auf Arbeiten von Fuchs und Priess-Crampe wird die Konstruktion des Potenzreihenkörpers über einer angeordneten Gruppe als Träger durchgeführt.




%aus bibliograhpie entfernte elemente 
%@unpublished{kaiser14,
%author = "Tobias Kaiser",
%title = "Funktionentheorie",,
%year = "2014"
%note= "Vorlesungsskript zur Veranstaltung Funktionentheorie"}
%
%@unpublished{kaiser13,
%author = "Tobias Kaiser",
%title = "Algebra und Zahlentheorie 2",,
%year = "2013"
%note= "Vorlesungsskript zur Veranstaltung Algebra und Zahlentheorie 2"}
%
%
%@ONLINE{Matheplanet,
%author = "Gockel",
%title = "Formale Laurentreihen",
%year = "2013",
%url = "http://www.matheplanet.com/default3.html?call=viewtopic.php?topic=185478&ref=https%3A%2F%2Fwww.google.de%2F"}
