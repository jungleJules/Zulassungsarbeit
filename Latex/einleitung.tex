\chapter{Einleitung}
%
Potenzreihen finden in vielen Bereichen der Mathematik Anwendung. In dieser Arbeit beschäftigen wir uns mit formalen Potenzreihen und den algebraischen Strukturen, die auf ihnen definiert werden können. \\
Ordnung spielt seit jeher eine essentielle Rolle in der Geschichte der Mathematik, aber erst gegen Ende des 19. Jahrhunderts beschäftigte man sich mit Ordnung verbunden mit algebraischen Operationen. Zu diesem Zeitpunkt entwickelte sich die Theorie der angeordneten Strukturen, beginnend mit Arbeiten von Hölder, Hahn und Hausdorff. In seinem Werk $\glqq$Die Axiome der Quantität und die Lehre vom Maß$\grqq$ zeigte Hölder 1901, dass sich jede archimedisch angeordnete Gruppe in eine Untergruppe der additiven Gruppe der reellen Zahlen einbetten lässt. Hölder bediente sich dabei der von Dedekind eingeführten Schnitte in $\Q$. 
In der vorliegenden Ausarbeitung liegt die Aufmerksamkeit auf der Betrachtung verallgemeinerter Potenzreihenkörper.   \\
Die Arbeit ist in drei Teile gegliedert. Nach einer kurzen Wiederholung der wichtigsten algebraischen Strukturen erfolgt ein Einblick in die Theorie der angeordneten abelschen Gruppen. Diese benötigen wir später zur Konstruktion des verallgemeinerten Potenzreihenkörpers, ebenso wie die Wohlordnung. Wir zeigen später, dass diese Eigenschaft unabdingbar für die Wohldefiniertheit der Verknüpfungen im verallgemeinerten Potenzreihenkörper ist. Die Archimedizität führt zur zentralen Aussage des ersten Kapitels, dem Satz von Hölder. \\
Im zweiten Teil werden Eigenschaften des Rings der formalen Potenzreihen auf den natürlichen Zahlen über einem Körper näher beschrieben. Eine Erweiterung der Menge der formalen Potenzreihe führt uns zum Körper der formalen Potenzreihen auf einer wohlgeordneten Teilmenge der ganzen Zahlen, den Laurentreihen. Basierend auf Arbeiten von Fuchs und Prieß-Crampe wird die Konstruktion des verallgemeinerten Potenzreihenkörpers mit Exponenten in einer angeordneten abelschen Gruppe durchgeführt und gezeigt, dass es sich tatsächlich um einen Körper handelt.




%aus bibliograhpie entfernte elemente 
%@unpublished{kaiser14,
%author = "Tobias Kaiser",
%title = "Funktionentheorie",,
%year = "2014"
%note= "Vorlesungsskript zur Veranstaltung Funktionentheorie"}
%
%@unpublished{kaiser13,
%author = "Tobias Kaiser",
%title = "Algebra und Zahlentheorie 2",,
%year = "2013"
%note= "Vorlesungsskript zur Veranstaltung Algebra und Zahlentheorie 2"}
%
%
%@ONLINE{Matheplanet,
%author = "Gockel",
%title = "Formale Laurentreihen",
%year = "2013",
%url = "http://www.matheplanet.com/default3.html?call=viewtopic.php?topic=185478&ref=https%3A%2F%2Fwww.google.de%2F"}
