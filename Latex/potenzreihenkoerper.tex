\chapter{Potenzreihenkörper}\label{chap3}
%
Die aus der Analysis bekannten Potenzreihen stellen ein bekanntes und wichtiges mathematisches Werkzeug dar. In Gebieten wie der Kombinatorik, Automaten- und Kontrolltheorie ermöglichen sie sowohl eine kompakte Darstellung von Summenformeln als auch deren Auffindung. Potenzreihen können ebenso über den Weg der Algebra definiert werden anhand der Folge ihrer Koeffizienten. Die algebraische Sichtweise verzichtet grundsätzlich auf Konvergenzbetrachtungen. Dadurch kann auf beliebigen Körpern und Ringen gearbeitet werden. \\
Die sogenannten formalen Potenzreihen in einer Unbekannte $z$ mit Exponenten in den natürlichen Zahlen, deren Koeffizienten in einem beliebigen Körper K liegen, bilden einen Ring $K[[z]]$. Aufbauend darauf stellen wir einen Zusammenhang zu den, in der Funktionentheorie häufig verwendeten Laurentreihen her. Der Ring der formalen Potenzreihen ist ein Integritätsring. Daraus folgt, dass dieser in einen kleinsten Körper eingebettet werden kann. Dieser Quotientenkörper von $K[[z]]$ entspricht genau dem Körper $K((z))$, den die Laurentreihen formen. \\
Potenzreihen bilden somit algebraische Strukturen, deren Beschaffenheit von den Eigenschaften der Menge der Exponenten der Reihen abhängt. Daher stellt sich die Frage, ob die formalen Potenzreihen weiter verallgemeinert werden können und welche Voraussetzungen diese Menge erfüllen muss, damit diese allgemeinen formalen Potenzreihen einen Körper ergeben. %Das Kapitel endet mit dem Beweis der zentralen Aussage, dass die Menge der formalen Potenzreihen auf einer angeordneten Gruppe über einem beliebigen Körper, unter der Voraussetzung eines wohlgeordneten Trägers, ein Körper ist.

%Bevor wir mit der allgemeinen Untersuchung von Potenzreihenkörpern beginnen, wird in diesem Kapitel ein wichtiges Beispiel von Ringen eingeführt. Zunächst wird die Menge der formalen Potenzreihen definiert und nachgewiesen, dass es sich bezüglich komponentweiser Addition und Faltung um einen Ring handelt. Anschließend beschäftigen wir uns mit dem Körper der Laurentreihen K((z)), der dem Quotientenkörper des Ringes der formalen Potenzreihen entspricht. Die genauere Analyse der Eigenschaften des Trägers der Elemente des Körpers der Laurentreihen zeigt, dass dieser auch über einer angeordneten Gruppe definiert sein kann und trotzdem durch Einbettung des Ringes ein Körper entsteht. Der dadurch entstandene Körper wird als \textit{allgemeiner Potenzreihenkörper} bezeichnet. \\ 
\section{Formale Potenzreihen}\label{potenzreihenring}

%
% kurze einführung in die potenzreihen mit Defintion einer formalen Potenzreihe und Eingliederung in Ring und quotientenkörper.
In diesem Abschnitt beschäftigen wir uns mit den formalen Potenzreihen mit Exponenten in den natürlichen Zahlen. Wir definieren die Verknüpfungen zwischen formalen Potenzreihen und zeigen, welche algebraischen Strukturen die Menge der formalen Potenzreihen bildet.
  
\begin{defn}
Eine \textit{formale Potenzreihe} über dem Körper $K$ ist eine Abbildung $\N_0 \rightarrow K$, $n \mapsto a_n$.
\end{defn}
\newpage
\begin{bemnota}
Wir werden formale Potenzreihen im Folgenden immer als Ausdruck der Form
\begin{equation}\label{eq: formalepotenzreihe}
\sum_{n=0}^\infty a_n z^n = a_0 + a_1z^1 + a_2z^2 + a_3z^3 + ...
\end{equation}
schreiben mit $a_n \in K$ für alle $n \in \N_0$.
Anstelle von $f(z)$ schreiben wir vereinfachend $f$, da wir die Unbestimmte immer mit $z$ bezeichnen.
\end{bemnota}
%
%
%
% 
%
%
%
\begin{bsp}
Wichtige Beispiele für formale Potenzreihen aus der Analysis sind Exponential-, Sinus- und Kosinusreihe:
\begin{eqnarray*}
\exp(z) &=& \sum_{n=0}^{\infty} \frac{1}{n!} z^n
= 1 + z +\frac{1}{2!} z^2 + \frac{1}{3!} z^3 + ..., \\ \\
\sin(z) &=& \sum_{n= 0}^{\infty} \frac{\left(-1\right)^n}{\left(2n + 1\right)!} z^{2n+1} = z - \frac{1}{3!} z^3 + \frac{1}{5!} z^5 - ...,\\\\
\cos(z) &=& \sum_{n= 0}^{\infty} \frac{\left(-1\right)^n}{\left(2n\right)!} z^{2n}
= 1 - \frac{1}{2!} z^2 + \frac{1}{4!} z^4 - ....
\end{eqnarray*}
\end{bsp}
%
\begin{bem}
In formalen Potenzreihen müssen nicht alle Potenzen der Unbestimmten explizit auftreten. Die Koeffizienten der nicht auftretenden Potenzen sind $0$ und die Potenzen werden deshalb weggelassen.
\end{bem}
%
\subsection{Addition und Multiplikation formaler Potenzreihen} \label{Rechnen}
%Im Folgenden werden Addition und Multiplikation in $K[[z]]$ definiert. Mit diesen Verknüpfungen wird $K[[z]]$ zu dem Ring der formalen Potenzreihen. \\
Wir bezeichnen die Menge der formalen Potenzreihen in $z$ mit Exponenten in $\N_0$ über einem Körper $K$ mit \[K [[z]] = \lbrace \sum_{n=0}^\infty a_n z^n \vert a_n\in K \rbrace. \]
%Formale Potenzreihen werden komponentenweise addiert.
%
%
%
%
\begin{defn}\label{AdditionPotenzreihen}
%
Seien $f = \sum_{n=0}^\infty a_n z^n$ und $g = \sum_{n=0}^\infty b_n z^n$ zwei formale Potenzreihen über $K$. Wir definieren ihre \textit{Summe} $f+g$ folgendermaßen:
\begin{eqnarray*}
&&\left( \sum_{n=0}^\infty a_n z^n \right) + \left( \sum_{n=0}^\infty b_n z^n \right) = \sum_{n=0}^{\infty} (a_n + b_n) z^n 
\end{eqnarray*}
\end{defn}
%
%
% 
\begin{defn}\label{MultiplikationKPotenzreihen}
Sei $\lambda \in K$ und $f \in K\lbrack\lbrack z \rbrack\rbrack$. Das Produkt der formalen Potenzreihe $f$ mit dem Körperelement $\lambda$ ist definiert durch:
\[\lambda\cdot f = \lambda\cdot \left(\sum_{n=0}^{\infty}a_nz^n\right) = \sum_{n=0}^{\infty} \lambda a_nz^n \in K\lbrack\lbrack z\rbrack\rbrack. 
\]
\end{defn}
%
%
\begin{defn}\label{MultiplikationPotenzreihen}
Die Multiplikation zweier formaler Potenzreihen $f,g$ erfolgt durch die sogenannte Faltung:
\begin{eqnarray*}
\left( \sum_{j=0}^\infty a_j z^j \right)\cdot \left( \sum_{k=0}^\infty b_k z^k \right) &=&\sum_{n=0}^\infty \left(\sum_{j+k=n} a_j b_k\right) z^n \\
&=& \sum_{n= 0}^\infty \left(a_0b_n + a_1b_{n-1} + a_2b_{n-2} + ... + a_nb_0 \right)z_n
\end{eqnarray*}
\end{defn}
%
Für die Unbestimmte $z$ gelten die Potenzgesetze.
\begin{bem}[Potenzgesetze]\label{Potenzgesetze}
Sei $G$ eine angeordnete abelsche Gruppe und seien $g_1, g_2 \in G$, dann gilt:
\begin{itemize}
\item[(i)] $z^{g_1} \cdot z^{g_2} = z^{g_1 + g_2}$
\item[(ii)] $z^{0_G} = 1_K$
\end{itemize}
\end{bem}

%
%
%
% 
\subsection{Der Ring der formalen Potenzreihen}
%Satz mit obigen Verknüpfungen ist das Potenzreihenring
%\begin{satz}\label{Potenzreihenring K-algebra}
%Die Menge $\left(K\lbrack\lbrack z\rbrack\rbrack,+ \right)$ ist eine K-Algebra.
%\end{satz}
%
%
%
%
% 
%
%
%
\begin{satz}\label{RingFormalerPR}
Die Menge $\left(K\lbrack\lbrack z\rbrack\rbrack, +, \cdot\right)$ ist mit obigen Verknüpfungen ein kommutativer Ring.
\end{satz}
\beweis{Wir weisen die Ringaxiome, wie in \ref{Ring} definiert, nach.\\
Die Assoziativität und Kommutativität der Addition lassen sich leicht nachprüfen.\\
Das \textit{neutrale Element der Addition} $0_K$ ist die Nullreihe $0_{K}:= \sum_{n=0}^\infty  b_n z^n$, wobei $b_n= 0 \text{ ist für alle } n\in \N_0 $. \\
%Denn wir erhalten als Summe von $g$ und $f$: 
% \begin{eqnarray*}
% f + g&=& \sum_{n=0}^\infty a_nz^n + \sum_{n=0}^\infty b_nz^n \\
% &=& \sum_{n=0}^\infty \left(a_n+b_n\right)z^n \\
% &=& \sum_{n=0}^\infty a_nz^n.
% \end{eqnarray*}
Für $f = \sum_{n=0}^\infty  (a_n)z^n$ ist $ -f = \sum_{n=0}^\infty  (-a_n)z^n$ das \textit{negative Element der Addition}.
% \begin{eqnarray*}
% f + (-f) &=& \sum_{n=0}^\infty  a_nz^n + (\sum_{n=0}^\infty  (-a_n)z^n)\\
% &=& \sum_{n=0}^{\infty}(a_n-a_n)z^n \\
% &=& 0_K.
% \end{eqnarray*}
 \\
$\left(K\lbrack\lbrack z\rbrack\rbrack, +\right)$ ist daher eine abelsche Gruppe. \\
Die Assoziativität der Multiplikation rechnen wir nach.
Seien $f, g, h \in K\lbrack\lbrack z\rbrack\rbrack$ mit $f = \sum_{n=0}^\infty a_n z^n$, $g = \sum_{n=0}^\infty b_n z^n$ und $h = \sum_{n=0}^\infty c_n z^n$. Es gilt:\\
\begin{eqnarray*}
f \cdot \left( g\cdot h\right)& =& f \cdot\left( \sum_{n=0}^\infty \left(\sum_{j+k=n} b_j c_k\right) z^n \right)\\
&=& \sum_{n=0}^\infty \left(\sum_{l+j+k=n} a_l b_jc_k\right) z^n \\
&=& \left(\sum_{n=0}^\infty \left(\sum_{l+j=n} a_l b_j\right) z^n\right) \cdot h \\
&=& \left(f \cdot g\right) \cdot h.
\end{eqnarray*} 
Das \textit{neutrale Element der Multiplikation} ist die Einsreihe $1_K$. Darunter verstehen wir diejenige Reihe, bei der nur der konstante Koeffizient $a_0 = 1$ und alle anderen gleich $0$ sind.
% \begin{eqnarray*} 
% g &=& \sum_{n=0}^\infty  a_nz^n,
% \end{eqnarray*}
%wobei 
% \[a_0 = 1 \text{ und } a_n = 0 \text{ für alle n } \in \N \text{ sind. }\]\\ 
%Damit folgt: $ \sum_{n=0}^\infty a_nz^n \cdot \sum_{n=0}^\infty b_nz^n = \sum_{n=0}^\infty \sum_{j+k=n} \left(a_j\cdot b_k\right)z^n = \sum_{n=0}^\infty b_nz^n. $\\ 
Die Multiplikation ist kommutativ, denn die Addition und Multiplikation in dem Körper $K$ sind kommutativ. 
%Es genügt somit ein Distributivgesetz nachzuweisen. 
Es gilt:
\begin{eqnarray*}
f\cdot \left(g + h\right) &=& f \cdot \left( \sum_{n=0}^\infty \left(b_n + c_n\right) z^n \right)\\
&=& \sum_{n=0}^\infty \left(\sum_{j+k=n} a_j \left(b_k +c_k\right)\right) z^n\\
&\stackrel{\mathrm{(*)}}=& \sum_{n=0}^\infty \left(\sum_{j+k=n} a_j b_k +\sum_{j+k=n} a_j c_k\right) z^n\\
&=& \sum_{n=0}^\infty \sum_{j+k=n} a_j b_k z^n +\sum_{n=0}^\infty \sum_{j+k=n} a_j c_k z^n\\
&=& f\cdot g + f\cdot h,
\end{eqnarray*}
wobei ($*$) aufgrund der Distributivität in $K$ folgt. Somit gilt das Distributivgesetz.\\
 }
%
%
%
%
\begin{bem}
Der Ring $K\lbrack\lbrack z\rbrack\rbrack$ bildet mit den in \ref{AdditionPotenzreihen}, \ref{MultiplikationKPotenzreihen} und \ref{MultiplikationPotenzreihen} definierten Verknüpfungen eine $K$-Algebra.
\end{bem}
%
%
%\begin{bem} \label{InversesPotenzreihenRingAussehen}
%Sei $f, g \in K\lbrack\lbrack z\rbrack\rbrack$, mit $f = \sum_{n=0}^\infty a_n z^n$ und $g = \sum_{n=0}^\infty b_n z^n$. Wir bezeichnen $g$ als die \textit{Inverse Potenzreihe} von $f$, wenn für
%\begin{eqnarray*}
%fg &=& \left( \sum_{j=0}^\infty a_j z^j \right) \left( \sum_{k=0}^\infty b_k z^k \right)\\  
%&\stackrel{\mathrm{def}}=& \sum_{n=0}^\infty\sum_{j+k=n} (a_j b_k) z^n\\ 
%\end{eqnarray*}
%%
%und 
%\begin{eqnarray*}
%\sum_{n= 0}^{\infty} c_nz^n \text{, mit } c_n = \sum_{j+k=n} (a_j b_k),
%\end{eqnarray*}
%gilt, dass $c_0 =1$ und $c_n = 0$ für alle $n\in\N$ ist.
%Sei $c_n = \sum_{j+k=n} (a_j b_k)$. Damit das Produkt der Potenzreihen dem neutralen Element der Multiplikation entspricht, müssen alle Koeffizienten mit Indizes größer Null den Wert $0$ annehmen, während $c_0 =1$ gilt. 
%\end{bem}
%
%
%
%
%\beweis{Sei $h =\sum_{n=0}^\infty\sum_{j+k=n} (a_j b_k) z^n$. Wir können beliebig viele Koeffizienten aus dieser Summe herausziehen nach Definition der Addition:
%\begin{eqnarray*}
%h &=& \sum_{n=0}^\infty\sum_{j+k=n} (a_j b_k) z^n \\
%&=& a_0b_0z^0 + \sum_{n=1}^\infty\sum_{j+k=n} (a_j b_k) z^n \\
%&=& a_0b_0 + \left(a_0b_1 + a_1b_0\right)z^1 + \sum_{n=2}^\infty\sum_{j+k=n} (a_j b_k) z^n\\
%&=& ...
%\end{eqnarray*} 
%Nach Definition der formalen Potenzreihe stellt $z$ eine Unbestimmte dar. Man sieht leicht, dass die Summe nur den Wert $1_K$ annimmt, falls $a_0b_0= 1_K$ erfüllt.}
%
%
%
Wir zeigen zunächst, dass zu einer formale Potenzreihe $f = \sum_{n=0}^\infty a_n z^n$ genau dann die inverse Potenzreihe existiert, wenn $a_0 \neq 0 $.
%
%
%
\begin{satz}\label{potenzreihenringEinheit}
Die formale Potenzreihe $f \in K[[z]] $ mit $f = \sum\limits_{n=0}^{\infty}a_nz^n $ ist genau dann eine Einheit, wenn $a_0 \neq 0$ ist.\\ 
\end{satz}
\beweis{$``\Leftarrow "$
Sei $f \in K[[z]]$ mit $f = \sum_{n=0}^{\infty}a_nz^n $ und es gelte $a_0 \neq 0$. Wir wollen zeigen, dass es eine Potenzreihe $g = \sum_{n=0}^{\infty}b_nz^n$ in $K[[z]]$ gibt, sodass 
\[f\cdot g= \left(\sum_{n=0}^{\infty}a_nz^n\right) \left(\sum_{n=0}^{\infty}b_nz^n\right) = \sum_{n=0}^{\infty}\sum_{j+k=n}\left(a_jb_k\right)z^n\stackrel{\mathrm{!}}=1\] ist. \\
Der Beweis erfolgt durch Induktion.  \\ 
Sei $n=0$, dann ist $1 \stackrel{\mathrm{!}}= \sum_{n=0}\sum_{j+k=n}\left(a_jb_k\right)z^n = \sum_{j+k=0}\left(a_jb_k\right)z^0 = a_0b_01_K = a_0b_0$.
%Für $b_0$ muss die Gleichung $a_0b_0= 1$ erfüllt sein. 
Da $a_0$ ungleich null ist, besitzt die Gleichung $a_0b_0 = 1$ eine eindeutige Lösung, nämlich $ b_0 = a_0^{-1}.$ \\
Wir nehmen an, dass die Koeffizienten $b_0, ..., b_{n-1}$ existieren  und eindeutig bestimmt sind. Mithilfe dieser Annahme zeigen wir die Induktionsbehauptung, dass $b_n$ existiert und eindeutig definiert ist.\\
Für den n-ten Koeffizienten des Produkts ergibt sich $0 =  a_0b_n + a_1b_{n-1} + ... + a_{n-1}b_1 + a_nb_0$. Da $ a_0$ ungleich $0$ ist, existiert die Lösung für $ b_n $. Sie ist eindeutig bestimmt durch 
\[b_n = -\frac{1}{a_0} \sum_{j+k = n, ~j>0~}^{} a_jb_k.\] Damit existiert ein eindeutiges Inverses zu $f$ und $f$ ist eine Einheit. \\
$``\Rightarrow "$ Es gilt $\left(\sum_{n=0}^{\infty}a_nz^n\right) \left(\sum_{n=0}^{\infty}b_nz^n \right)= 1. $ \\
Nach Definition \ref{definitionEinheitNullteiler} folgt, dass $\sum_{j+k=n}a_jb_k = 0 $ sein muss für $n > 0$ und damit $ a_0b_0 = 1 $ gilt. Somit muss $a_0$ ungleich $0$ sein.}
%
%
Wir haben gezeigt, dass die Einheiten des Potenzreihenrings genau die Elemente sind, deren konstanter Term ungleich $0$ ist. In diesem Fall können wir die inverse Potenzreihe  konstruieren.\\
\begin{korollar}\label{inverse Potenzreihe}
Sei $f = \sum_{n=0}^{\infty}a_nz^n $ und $a_0 \neq 0$. Die Koeffizienten der inversen Potenzreihe
$g = \sum_{k=0}^\infty b_k z^k$ sind rekursiv definiert durch
\begin{equation*}
b_0 = \frac{1}{a_0} ~~~~~ \text{ und }~~~~~ b_n = -\frac{1}{a_0}\sum\limits_{j+k = n, ~j>0~} a_j b_{k}	 ~~~~~~\forall n\in \N.
\end{equation*}
\end{korollar}
\beweis{siehe Rückrichtung im Beweis \ref{potenzreihenringEinheit}.


%Wie im Beweis \ref{potenzreihenringEinheit} verwendet, gilt $a_0b_0 = 1$, woraus $b_0 = \frac{1}{a_0}$ folgt. Für die restlichen Koeffizientenwerte muss dementsprechend
%\begin{equation*}
%\sum_{j+k=n} a_j b_k = 0
%\end{equation*}
%für alle $n \in \N_0$ gelten. Es lässt sich leicht nachrechnen, dass das Produkt aus $f$ und der gewählten Potenzreihe $g$ diese Bedingung erfüllt und damit $fg = 1$ gilt. %TODO: evtlf,g ausschreiben und schöner formulieren?

}
%
\begin{bsp} %\cite{taraz12}
Es sei $K = \R$ und $0\neq q \in \R$ beliebig und $f = \sum_{n=0}^{\infty}q^n z^n \in \R[[z]]$. Wir bestimmen die inverse Potenzreihe $g = \sum_{n = 0}^{\infty} b_n z^n$. Dazu wenden wir die Formel aus~\ref{inverse Potenzreihe} an:
\begin{center}
\begin{description}
\item $b_0 = \frac{1}{q^0} = 1$,
\item $b_1 = -q^1b_0 = -q$,
\item $b_2 = -\left(q^1b_1 + q^2b_0\right) = -\left(-q^2 + q^2\right) = 0$,
\item ...
\item $b_n = -\left(q^1b_{n-1} + q^2b_{n-2} + ... + q^{n-1}b_1 + q^nb_0\right) = -\left(-q^{n-1}(-q) + q^n\right) = 0$.
\end{description}
\end{center}
Für alle $n \ge 3$ folgt induktiv, dass ebenso $b_n = 0$ gilt. Die inverse Potenzreihe zu $f$ ist $g := b_0 + b_1z = 1 - qz$. \\
Für $q = 1$ entspricht $f$ der geometrischen Reihe. Diese konvergiert bekanntlich für $|z| < 1$ und es gilt $f = \sum_{n= 0}^{\infty} z^n = \frac{1}{1-z}$.
%Konvergenz betrachten wir ja nicht!!! Daraus können wir schließen:
%\[\forall z \in \R \text{ mit } |z| < |\frac{1}{q}|: A(z) = \sum_{}^{}a_n z^n = \frac{1}{1-qz}\] 
\end{bsp}
%
%
%
%
%
%
%
%
%
%
\subsection{Eigenschaften des Potenzreihenrings}

Wir beweisen zunächst, dass $K[[z]]$ ein Integritätsring und damit nullteilerfrei ist. Daraus erhalten wir die Einbettbarkeit von $K[[z]]$ in einen kleinsten Körper, den Quotientenkörper.
Anschließend betrachten wir den Zusammenhang zwischen dem Ring der formalen Potenzreihen und der Menge der konvergenten Potenzreihen im Körper der komplexen Zahlen $\C$. 
%evtl noch möglich zu zeigen dass c[[z]] ein nullteilerfreier Ring ist stellt sich nur die Frage ob das iwie nötig ist...
%
\begin{satz}\label{intring}
Der Ring $K[[z]]$ ist ein Integritätsring.
\end{satz}
%
\beweis{ Es ist zu zeigen, dass der Ring nullteilerfrei ist. \\
Seien $ f = \sum_{n=0}^\infty  a_n z^n \text{ und } g = \sum_{n=0}^\infty  b_n z^n$ mit 
\begin{eqnarray*}
f \cdot g &=& \left( \sum_{n=0}^\infty  a_n z^n\right) \cdot\left( \sum_{n=0}^\infty  b_n z^n\right)
= 0.
\end{eqnarray*} 
Nach Definition der Multiplikation gilt $\sum_{j+k=n}a_jb_k = 0$ für alle $n \in \N_0$.\\
Sei nun o.B.d.A. $\sum_{n=0}^\infty  a_n z^n \neq 0$. Wir zeigen, dass die Potenzreihe $\sum_{n=0}^\infty  b_n z^n$ gleich null ist. Es soll also kein Index $n$ existieren, für den $b_n \neq 0$ ist. Wir beweisen mithilfe von Induktion, dass $b_n = 0$ ist für alle $n \in\N_0$. \\
%
%folgern aus $b_0, b_1,... ,b_{n-1} = 0$ induktiv, dass $b_n=0$ für alle $n \in \N_0$ ist
%
Sei $n= 0$ und sei $j$ der erste Index, sodass $a_j \neq 0$ gilt. Man hat 
\begin{eqnarray*}
\sum_{i+k=j} a_ib_k = a_0b_j + a_1b_{j-1} + ... + a_jb_0 = a_jb_0 \stackrel{\mathrm{Vor.}}= 0. % da k+l=k
\end{eqnarray*} 
Da $a_j \neq 0$ ist, muss $b_0=0$ gelten. \\
Angenommen die Behauptung gilt für alle Indizes $j\in \N$ mit $j \leq n-1$. Dann gilt $ b_0,..., b_{n-1}= 0$. Wir erhalten
\begin{eqnarray*}
\sum_{i+k=j+n} a_ib_{k} &=& a_0b_{j+n} + a_1b_{j+(n-1)} + ... +a_jb_n + a_{j+1}b_{n-1} + ... + a_{j+n}b_0\\
&=& a_jb_n + a_{j+1}b_{n-1} + ... + a_{j+n}b_0= a_jb_n \stackrel{\mathrm{Vor.}}= 0.
\end{eqnarray*}
Da $a_j\neq 0$ ist folgt $b_n= 0$. 
}
%
%
%TODO: nochmal checken ob das stimmt, sowohl oberes als auch Konvergenzdefinition
Nun werden konvergente Potenzreihen über dem Körper $\C$ betrachtet. 
%
%
\begin{defn}\label{konvergenz}
Eine Potenzreihe $f = \sum_{n= 0}^{\infty}a_nz^n \in \C[[z]]$ heißt \textit{konvergent}, wenn es ein $z_0\in \C$ mit $z_0 \neq 0$ gibt, sodass $\sum_{n=0}^{\infty}a_n{z_0}^n$ als Reihe in $\C$ konvergiert. \\
Das heißt, die Folge  der Partialsummen $\left(s_n\right)_{n\in\N_0}$, wobei $s_n:= \sum_{k=0}^{n}a_k{z_0}^k$ ist, konvergiert, und man schreibt für den Limes $s = \lim_{n \to \infty}s_n$:
\begin{align}
s= \sum_{n=0}^{\infty}a_n{z_0}^n
\end{align}
Auf der Menge $D$ der Punkte $z_0 \in \C$, für die $\sum_{n=0}^{\infty}a_n{z_0}^n$ konvergiert, wird somit eine Abbildung $z_0 \mapsto \sum_{n=0}^{\infty}a_n{z_0}^n $ definiert. Wir nennen $D$ den \textit{Konvergenzbereich}. \\
Wir bezeichnen $R = R(f) = \sup{\lbrace r\in \R_{\geq 0} ~|~ \sum_{n=0}^{\infty}a_n{r}^n \text{ konvergiert}\rbrace } \in \R_{\geq 0}\cup \lbrace\infty\rbrace$ als \textit{Konvergenzradius}. Ist $R(f) >0 $, so heißt die Reihe konvergent, ist $R(f)= 0 $, so heißt sie divergent.
Insbesondere konvergiert f auf der offenen Kugel $B(0,R(f))$ absolut.
%
% Ich glaube das brauch ich nichz mehr, weil Lemma von Abel e wegfällt und sonst brauch ich D(---) nirgends.
%
%Mit $D(0, r)$ bezeichnen wir die Menge aller Punkte in der offenen Kugel um $0$ mit Radius $r$, für die die Potenzreihe konvergiert.
\end{defn}


%
%
%
\begin{bem}\label{konvergentUnterring}
Sei $\C\lbrace z \rbrace$ die Menge der konvergenten Potenzreihen über dem Körper der komplexen Zahlen $\C$. Die Menge $\C\lbrace z \rbrace$ ist ein Unterring des Rings der formalen Potenzreihen $\C[[z]]$. 
\end{bem}
\beweis{Es reicht, für $\C\lbrace z \rbrace$ zu beweisen, dass die Summe und das Produkt zweier konvergenter Potenzreihen wieder konvergent ist. \\
Seien $f= \sum_{n=0}^{\infty} a_nz^n, g= \sum_{n=0}^{\infty} b_nz^n$ konvergente Potenzreihen. Die Summe $f+g$ ist ebenfalls eine konvergente Potenzreihe mit $R(f+g) \geq \min\lbrace R(f), R(g)\rbrace$.\\
Da beide Potenzreihen $f, g$ im Inneren ihres Konvergenzradius absolut konvergieren, kann das Cauchyprodukt bestimmt werden und das Produkt konvergiert absolut mit $R(fg) \geq \min\lbrace R(f), R(g)\rbrace$.
%Betrachte zwei konvergente Potenzreihen mit den Konvergenzradien $r_1$ und $r_2$. Innerhalb des Radius min$\lbrace r_1, r_2\rbrace $ konvergieren beide Potenzreihen und somit auch die Summe der beiden Potenzreihen. Das Produkt konvergiert innerhalb des Radius $r$ mit $r \geq\min\lbrace r_1, r_2\rbrace $, denn beide Reihen konvergieren absolut. Das Cauchyprodukt existiert, da die Menge der konvergenten Potenzreihen eine Teilmenge des Rings der formalen Potenzreihen ist, auf dem das Cauchyprodukt eindeutig definiert ist. Nach dem großen Umordnungssatz konvergiert auch das Cauchyprodukt mit Konvergenzradius $r \geq\min\lbrace r_1, r_2\rbrace $.
} 
%
%
%
\begin{satz}\label{EinheitenimKonvergentenUnterring}
In $\C \lbrace z\rbrace$ sind genau die Potenzreihen $\sum_{n=0}^{\infty} a_nz^n$ invertierbar, für die $a_0 \neq 0$ gilt.
\end{satz}

\beweis{Zunächst konstruieren wir formal das Inverse, wie wir es in Satz \ref{inverse Potenzreihe} bereits durchgeführt haben. Es bleibt zu zeigen, dass diese inverse Potenzreihe konvergiert. 
Sei eine konvergente Potenzreihe $\sum_{n= 0}^{\infty} a_nz^n \in \C\lbrace z\rbrace$ mit $a_0 \neq 0$ gegeben.
Wir können ohne Beschränkung der Allgemeinheit annehmen, dass die Koeffizientenfolge $ (a_n)_{n \in \N_0}$ beschränkt ist, also $|a_n| \le a$, für alle $n \in  \N_0$ und $a \in \R_{\geq 0}$ gilt, denn: \\
Nach Definition der Konvergenz formaler Potenzreihen gibt es ein $z_0 \in \C$ mit $z_0 \neq 0$, sodass die Reihe $f(z_0) := \sum_{n=0}^{\infty} a_n {z_0}^n $ konvergiert. 
Die Folge ${(|a_n||{z_0}^n|)}_{n\in\N_0} $ ist somit eine Nullfolge und daher beschränkt.
% Nach dem Lemma von Abel \cite[S. 46]{ebeling2013funktionentheorie}konvergiert nun die Reihe $\sum_{n= 0}^{\infty} a_n{\zeta}^n$ für $\zeta \in\C$, $|\zeta| < |z_0|$ auf $D\left(0, |z_0|\right)$. 
 Wähle $q = \frac{\zeta}{z_0}$. Wir erhalten, dass
\[
\overline{f}(q) = \sum_{n=0}^{\infty}\left(a_n{z_0}^n\right) \left({\frac{\zeta}{z_0}}\right)^n = \sum_{n=0}^{\infty}a_n\zeta^n = f(\zeta)
\]
konvergent ist. Aus dieser Gleichheit folgt, dass wir die Beschränktheit der Koeffizientenfolge $ (a_n)_{n \in \N_0}$ annehmen können. Nun bestimmen wir das Inverse ähnlich wie in \ref{inverse Potenzreihe} und zeigen, dass die Koeffizientenfolge wieder beschränkt ist, woraus die Konvergenz der inversen Potenzreihe folgt.
Wir nehmen an, dass die Schranke $a$ der Koeffizientenfolge $\left(a_n\right)_{n\in\N}$ größer 1 ist und es sei ohne Einschränkung $a_0 = 1$. Wir betrachten die Koeffizientenfolge $\left(b_n\right)_{n\in\N}$ des Inversen wie wir sie in Satz \ref{inverse Potenzreihe} konstruiert haben. Es gilt:
\[
b_n = - \sum_{j+k=n,~j>0} a_j b_{k}. 
\]
Wir zeigen, dass |$b_n$|  durch ein Vielfaches von $a^n$ beschränkt ist. Ist dies gezeigt,  können wir eine positive untere Schranke des Konvergenzradius angeben. Wir wählen $C \in R$ mit $ C> \max\lbrace\frac{a}{a-1}, 1\rbrace$. Mithilfe von Induktion beweisen wir, dass   
\[
|b_n| \le (aC)^n
\]
ist. 
Wir wissen nach Konstruktion des Inversen \ref{inverse Potenzreihe} und den oben getroffenen Annahmen, dass $b_0 =\frac{1}{a_0} = 1 < a$ ist. Damit ist die Ungleichung für $b_0$ erfüllt. \\
Die Abschätzung gelte für $b_n$. Als Abschätzung für den Koeffizienten $b_{n+1}$ erhalten wir, dass \\
\begin{eqnarray*}
|b_{n+1}| &=& |- \sum_{j+k=n+1, ~j>0} a_j b_{k}| 
\le \sum_{j+k=n+1, ~j>0} |a_j| |b_{k}| 
\le a\sum_{j+k=n+1, ~j>0} \left(Ca\right)^{k} \\
&\le& a~C^n~\sum_{k=0}^{n}a^k
\le a~C^n~\frac{a^{n+1}}{a-1} 
\le \left(a~C\right)^{n+1}
\end{eqnarray*}
 gilt.}




%
%
%
%
%
%
%
%
%
%
%
%
%
Im Verlauf des nächsten Teils können wir zeigen, dass der Quotientenkörper des Ringes der formalen Potenzreihen dem Körper der formalen Laurentreihen entspricht. Wir gehen zunächst auf formale Laurentreihen ein.
%Anschließend definieren wir eine entsprechende Bewertung auf dem Körper der formalen Laurentreihen.
%
%
%
%
\section{Formale Laurentreihen}
%
Eine Erweiterung des Begriffs einer formalen Potenzreihe führt zu der formalen Laurentreihe. Diese unterscheidet sich bezüglich ihres Anfangsindex $n_0 \in \Z$ von den formalen Potenzreihen. \\
%Wir bezeichnen mit $K((z))$ die Menge aller Abbildungen $f$ von $\Z$ in einen $K$, für die es ein Element $x \in \Z$ gibt, mit $f(y) = 0$ für alle $y < x $. \newline 
Laurentreihen spielen eine wichtige Rolle in der Funktionentheorie, da sie komplexe Funktionen beschreiben, welche auf einem Kreisring holomorph sind. In dieser Arbeit verzichten wir weitgehend auf Konvergenzbetrachtungen und behandeln formale Laurentreihen aus algebraischer Sichtweise. % Quelle:  [H74] HENRICI, Peter: Applied and computational complex analysis, Volume 1, WileyInterscience publication, New York 1974.
Wir orientieren uns dabei an \cite[S. 563 - 572]{Lueneburg08}.
%
\begin{defn}
Eine \textit{formale Laurentreihe} über dem Körper $K$ ist eine Abbildung $f \colon \Z \rightarrow K$, $n \mapsto a_n$, wobei ein $k\in \Z$ existiert, sodass $f(n) = 0$ ist für alle $n < k$.
\end{defn}
%
%
\begin{nota}
Wir werden Laurentreihen im Folgenden meist als Reihen der Form
\begin{equation*}
\sum_{n= k}^{\infty}a_nz^n \text{ mit } k \in \Z \text{ und }a_n \in\textit{K}\text{ für alle } n\in\Z 
\end{equation*} 
schreiben.
Dabei bezeichnet $\sum_{n=k}^{-1}a_{n}z^{n}$ den Hauptteil, $\sum_{n=0}^{\infty}a_nz^n$ den Nebenteil der Laurentreihe. 
\end{nota}
%
%
%
%
%früher: als die Menge der Abbildungen $f$ von $\Z$ in den kommutativen Körper $K$, für die es ein $a \in \Z$ gibt mit $f(i) = 0$ für alle $i < a$. 
%
%
%
%
%
%
% 
\begin{defn}\label{traeger}
Der \textit{Träger} einer Laurentreihe $f = \sum_{n =k}^{\infty} a_nz^n \in K((z))$ ist folgendermaßen definiert: 
\[\text{supp}(f) := \lbrace n \in \Z | a_n \neq 0 \rbrace.\] 
\end{defn}
%
%
%
%
\newpage
\begin{bem}
Unter einem Träger einer Laurentreihe versteht man den Definitionsbereich der Funktion, die durch die Laurentreihe dargestellt wird.
\end{bem}
%
%
%
%
\subsection{Addition und Multiplikation formaler Laurentreihen}\label{RechnenMitLaurentreihen}
%
%
Im Unterschied zu der funktionentheoretischen Verwendung der Laurentreihen betrachten wir nur Laurentreihen mit endlich vielen negativen Summanden. Diese Beschränkung ist notwendig, da andernfalls die Multiplikation nicht definiert werden kann.
Wir bezeichnen die Menge der formalen Laurentreihen in $z$ mit Exponenten in $\Z$ über dem Körper $K$ mit
\begin{equation*} %TODO: konsistent, passt das?
K((z)) = \lbrace \sum_{n =k}^{\infty} a_nz^n \vert a_n \in K, k \in \Z\rbrace.
\end{equation*} 
Die Addition und Multiplikation formaler Laurentreihen erfolgt analog zur Addition und Multiplikation formaler Potenzreihen.
\begin{defn}\label{AdditionLreihen}
Zwei Laurentreihen $f, g \in K((z))$ mit $f = \sum_{n=k}^\infty a_n z^n$ und $g= \sum_{n=m}^\infty b_n z^n$ werden addiert, indem man ihre entsprechenden Koeffizienten addiert: 
%
\begin{eqnarray}
\sum_{n=k}^\infty a_n z^n  +  \sum_{n=m}^\infty b_n z^n = \sum_{n = min(k, m)}^{\infty}(a_n + b_n) z^n.
\end{eqnarray}
\end{defn}
%
%Eine derartige Darstellung existiert, da $\text{K}((z)) $ als Quotientenkörper von K[[z]] definiert ist, wie in \ref{quot} gezeigt wird. \\
%
% 
%
\begin{defn}\label{MultiplikationKLreihen}
Sei $\lambda \in K$ und $f \in K\left(\left( z \right)\right)$. Das Produkt der formalen Potenzreihe $f$ mit dem Körperelement $\lambda$ ist definiert durch:
\[\lambda\cdot f = \lambda\cdot \left(\sum_{n=m}^{\infty}a_nz^n\right) = \sum_{n=m}^{\infty} \lambda\cdot  a_nz^n \in K\left(\left( z\right)\right). 
\]
\end{defn}
%
%
%
\begin{defn}\label{MultiplikationLreihen}
Die Multiplikation erfolgt durch Faltung der Laurentreihen.
\begin{eqnarray}
\label{eq: multiplikationLaurent}
\sum_{n=k}^{\infty} a_n z^n  \cdot  \sum_{n=m}^{\infty} b_n z^n = \sum_{n = m+k}^{\infty}\sum_{i+j=n}^{}\left(a_i \cdot b_j\right) z^n. 
\end{eqnarray}
%

\end{defn}
%
\begin{bem}
Die Multiplikation formaler Laurentreihen ist unter der Bedingung wohldefiniert, dass formale Laurentreihen höchstens endliche viele Terme mit negativen Exponenten besitzen. \\
Diese Forderung ist unverzichtbar, denn andernfalls wäre die Summe $\sum_{i+j=n}^{}\left(a_i \cdot b_j\right)$ in \ref{eq: multiplikationLaurent} unendlich und das Produkt somit nicht bestimmbar.
\end{bem}

\begin{satz}\label{traegerwohlgeordnet}
Sei $f \in K((z))$ eine formale Laurentreihe. Dann ist der Träger der Laurentreihe wohlgeordnet.
\end{satz}
\beweis{
Sei $f\in K((z))$ eine formale Laurentreihe. Es gilt supp$(f)$ ist eine Teilmenge der ganzen Zahlen. Nach Definition einer formalen Laurentreihe ist supp$(f)$ nach unten beschränkt. Daraus folgt mithilfe von \ref{teilmengeGanzeZahlenwohlgeordnet} die Behauptung.
%, die obige Bedingung erfüllt. Wir können $f$ nach der Definition einer formalen Laurentreihe schreiben als $f = \sum_{n= - k}^{\infty}a_nz^n = \sum_{n=1}^{k}a_{-n}z^{-n} + \sum_{n=0}^{\infty}a_nz^n = g + h$. Wie in \ref{OrdnungNundZ} gezeigt, ist die Menge der ganzen Zahlen $\Z$ total geordnet. Der Hauptteil $h$ jeder formalen Laurentreihe besteht aus endlich vielen Summanden, was die Wohlordnung von supp($h$), dem Trägers des Hauptteils, impliziert (wir erinnern an \ref{TotalGeordnetEndlichIstWOhlgeordnet}). Der Träger des Nebenteils, supp($g$), ist aufgrund der Wohlordnung der natürlichen Zahlen wohlgeordnet. \\
%Für supp$(f)$ gilt trivialerweise, dass er der Vereinigung von supp($g$) und supp($h$) entspricht. Unter Verwendung von Satz \ref{wohlgeordnvereinigung} erhalten wir die Wohlordnung des Trägers der formalen Laurentreihe. 
}
%
%
%
%
%
\subsection{Der Körper der formalen Laurentreihen}

In diesem Abschnitt beschäftigen wir uns mit der Menge der formalen Laurentreihen $K((z))$ und weisen deren Körperstruktur nach. Auf diesen Ergebnissen aufbauend stellen wir die Verbindung zwischen dem Körper der Laurentreihen und dem zuvor behandelten Ring der formalen Potenzreihen her. Bezugnehmend auf \ref{konvergentUnterring} betrachten wir aus Gründen der Vollständigkeit die Menge der konvergenten Laurentreihen. \\
Abschließend konstruieren wir mithilfe der Wohlordnung des Trägers eine Bewertung auf dem Körper der formalen Laurentreihen und stellen den bewertungstheoretischen Zusammenhang zu dem Ring der formalen Potenzreihen her.
%
%
% 
\begin{satz}
Die Menge $\left(K((z)), +, \cdot\right)$ ist mit obigen Verknüpfungen ein kommutativer Ring.
\end{satz}
\beweis{Es genügt das Nachrechnen der Ringaxiome. Das neutrale Element der Addition ist die Nullreihe $0_K := \sum_{n=k}^{\infty} a_kz^k$ mit $a_n = 0$ für alle $n \geq k$. Das neutrale Element der Multiplikation ist die Einsreihe $1_K:= \sum_{n=k}^{\infty}a_kz^k$ mit $a_0 = 1$ und $a_n = 0$ für alle $0 \neq n \geq k$. Der Beweis der verbleibenden Bedingungen verläuft ähnlich zu \ref{RingFormalerPR}.
}
%
%
%
%
%
%
%
\begin{bem}
Der Ring $K((z))$ bildet mit den in \ref{AdditionLreihen}, \ref{MultiplikationKLreihen} und \ref{MultiplikationLreihen} definierten Verknüpfungen eine $K$-Algebra.
\end{bem}
%
\begin{satz}\label{Laurentreihenkoerper} %\cite{Lueneburg08}
$\left(K((z)), +, \cdot\right)$ ist mit der definierten Addition und Multiplikation ein Körper. 
\end{satz}
\beweis{ Sei $0 \neq f \in K((z))$ und $f = \sum_{n= m}^{\infty} a_nz^n$ mit $a_m\neq 0$.
%Dann gibt es ein $i \in \Z$, sodass $f = \sum_{n= i}^{\infty}a_nz^n$ für alle $n \in \Z $ mit $n \geq i$ ungleich Null ist und für alle $n < i$ gleich Null ist. 
%Wir wissen bereits $K((z))$ ist ein kommutativer Ring. 
%Um zu zeigen, dass $K((z))$ ein Körper ist, genügt es zu beweisen, dass zu jedem Element $f$ von $K((z))$ ein Inverses $g$ existiert. 
Wir definieren das Inverse $g \in K((z))$ rekursiv und zeigen, dass die so entstandene Laurentreihe invers zu $f$ ist. Die Konstruktion von $g$ läuft ähnlich zur Konstruktion der inversen Potenzreihe in Satz \ref{inverse Potenzreihe}. \\
Setze $g = \sum_{n= -m}^{\infty} b_nz^n$ mit $b_n = 0_K$ für alle $n < -m$ und $b_{-m} = \frac{1}{a_m}.$ Sei $l \in \N_0$ und $b_{-m},..., b_{-m+l-1}$ bereits definiert.  Wir wählen 
\[b_{-m+l} = - \frac{1}{a_m} \sum_{n= -m}^{-m+l-1} b_na_{l-n}.\]
Nach Definition der Multiplikation in $K((z))$ erhalten wir für den Koeffizienten des Produkts $ba$ bei $z^l$ somit: 
\begin{eqnarray*}
{\left(ba\right)}_l &=& \sum_{k+j=l}^{}\left(a_k \cdot b_j\right)  \\
&=& \sum_{n = -m}^{-m + l} b_na_{l-n}.
\end{eqnarray*}  
Für $l= 0 $ folgt $b_0a_0 = b_{-m}a_{m}= 1$. Es bleibt der Fall $l > 0 $ zu  berücksichtigen: 
\begin{eqnarray*}
{\left(ba\right)}_l &=&
\sum_{n = -m}^{-m + l - 1}b_na_{l-n} + b_{-m+l}a_m \\
&=& 0.
\end{eqnarray*}

Also ist $gf = 1_K$ und da $K((z))$, wie bereits gezeigt, ein kommutativer Ring ist, folgt $fg = 1_K$. Wir können zu jedem Element $f\neq 0$ aus $K((z))$ folglich ein Inverses konstruieren, womit $K((z))$ ein Körper ist.
}
%
%
Mithilfe von \ref{quotkoerper} zeigen wir nun, dass der Körper der formalen Laurentreihen dem Quotientenkörper des Ringes der formalen Potenzreihen entspricht. 
%
\begin{satz}\label{quot}
Es gilt $K((z)) =~$\textup{Quot}$(K[[z]])$.
\end{satz}
%Quotientenkörper Beweis nach Kaiser:
\beweis{Wir definieren zunächst die naheliegende Abbildung $\phi \colon K[[z]] \to K((z)), f \mapsto f$. Wie leicht zu sehen ist, handelt es sich bei der Abbildung $\phi$ um einen injektiven Homomorphismus. Wir wissen bereits nach Satz \ref{Laurentreihenkoerper}, dass $K((z))$ ein Körper ist und $K[[z]]$ ist, wie in \ref{intring} gezeigt, ein Integritätsring. Nach der universellen Eigenschaft des Quotientenkörpers (\ref{UniverselleEigenschaftQuot}) gibt es genau einen injektiven Körperhomomorphismus \[\Phi\colon \textup{Quot}(K[[z]])\to K((z)), ~~\frac{\sum_{n=0}^{\infty}a_nz^n}{\sum_{m=0}^{\infty}b_mz^m}\mapsto \sum_{k=n-m}^{\infty}\frac{a_k}{b_k}z^k, 
\text{ sodass } \Phi_{|K[[z]]}= \phi \text{ ist.}\]
Wir zeigen nun die Surjektivität der Abbildung $\Phi$ und erhalten somit die Isomorphieeigenschaft. Sei $f\in K((z))$ mit $\gamma = \min(\textup{supp}(f)) < 0$. Die Laurentreihe $f= \sum_{n=\gamma}a_nz^n$ kann man nach \ref{Potenzgesetze} als $a_\gamma z^{\gamma} \cdot \left(\sum_{n=0}^{\infty} b_nz^n\right)$, wobei $b_n = \frac{a_n}{a_\gamma}$ ist, schreiben. Man sieht sofort, dass $\sum_{n=0}^{\infty} b_nz^n \in K[[z]]$ ist. Da $a_\gamma z^\gamma$ nach Voraussetzung ungleich null und ein Körperelemente ist, können wir $a_\gamma z^\gamma = \frac{1}{{\left(a_\gamma\right)}^{-1}z^{-\gamma}}$ schreiben mit ${\left(a_\gamma\right)}^{-1}z^{-\gamma} \in K[[z]]$. Dieser Quotient ist für $\gamma \in G$ eindeutig. Jede Laurentreihe lässt sich daher in eindeutiger Weise als Quotient zweier Potenzreihen darstellen. Die Abbildung $\Phi$ ist damit surjektiv und die Behauptung ist gezeigt.
}
%Es gilt $K((z)) =~$\textup{Quot}$(K[[z]])$.
%\end{satz}
%\beweis{Nach Konstruktion von $K((z)) $ ist klar, dass $K[[z]]\subseteq K((z)) $. 
%%Genauer Beweis nach schotten:
%Betrachte die Abbildung:\\
%\[\Phi: K((z)) \rightarrow \text{ Quot}\left(K[[z]]\right)\]
%\begin{eqnarray} \label{eq: isomorphismusQuotientenkörper}
%\sum_{n=m}^{\infty}a_nz^n  \mapsto 
%\begin{cases}
%\frac{{z^{-m}\sum_{n=m}^{\infty}a_nz^n}}{{z^{-m}}} & \text{, falls } m < 0 \\
%\frac{\sum_{n=m}^{\infty}a_nz^n}{1} & \text{, falls } m\geq 0
%\end{cases}
%\end{eqnarray}
%Die Abbildung ist wohldefiniert.
%% im Bezug auf $m$, dem kleinsten Index für den $a_m\neq0$ gilt.
% Denn ist $m\geq 0$, so entspricht der Hauptteil der Laurentreihe der Nullreihe und die Reihe liegt somit in $K[[z]]$. Für den Fall $m \le 0$ existiert ein eindeutiges $g \in K[[z]]\subseteq \textup{Quot}K[[z]]$ und wir können $\sum_{n=m}^{\infty}a_nz^n = z^{-m}g$ schreiben.\\
%Wir weisen nach, dass $\Phi$ ein Körperisomorphismus ist.\\ 
%Damit wir die Fallunterscheidung nach unserer Definition der Abbildung $\Phi$ nicht explizit durchführen müssen, setzen wir  $z^{-m} = 1$ falls $m \geq 0$ ist. Durch diese Vereinfachung müssen wir den unteren Fall in \ref{eq: isomorphismusQuotientenkörper} nicht gesondert formulieren. 
%Sei $m\leq l$: 
%%und $m<0$ und $l<0$:\\
%\begin{eqnarray*}
%\Phi \left(\left( \sum_{n=m} a_nz^n \right)+ \left( \sum_{n=l} b_nz^n \right)\right)
%&=& \frac{{z^{-m}\sum_{n=m}^{\infty}\left(a_n + b_n\right) z^n}}{{z^{-m}}}\\
%&=& \frac{{z^{-m}\sum_{n=m}^{\infty}a_n z^n}}{{z^{-m}}}+ \frac{{z^{-m}\sum_{n=l}^{\infty}b_n z^n}}{{z^{-m}}} \\
%&=& \frac{{z^{-m}\sum_{n=m}^{\infty}a_n z^n}}{{z^{-m}}}+ \frac{{z^{-l}\sum_{n=l}^{\infty}b_n z^n}}{{z^{-l}}} \\
%&=&\Phi \left(\sum_{n=m} a_nz^n\right) + \Phi\left(\sum_{n=l} b_nz^n \right)
%\end{eqnarray*}
%%
%%
%%Sei $m\leq l$ und $m<0$ und $l>0$:\\
%%\begin{eqnarray*}
%%\Phi \left(\left( \sum_{n=m} a_nz^n \right)+ \left( \sum_{n=l} b_nz^n \right)\right)
%%&=& \frac{{z^{-m}\sum_{n=m}^{\infty}\left(a_n + b_n\right) z^n}}{{z^{-m}}}\\
%%&=& \frac{{z^{-m}\sum_{n=m}^{\infty}a_n z^n}}{{z^{-m}}}+ \frac{{z^{-m}\sum_{n=l}^{\infty}b_n z^n}}{{z^{-m}}} \\
%%&=& \frac{{z^{-m}\sum_{n=m}^{\infty}a_n z^n}}{{z^{-m}}}+ \frac{{\sum_{n=l}^{\infty}b_n z^n}}{1} \\
%%&=&\Phi \left(\sum_{n=m} a_nz^n\right) + \Phi\left(\sum_{n=l} b_nz^n \right)
%%\end{eqnarray*}
%%%
%%Sei $m\leq l$ und $m>0$:\\
%%\begin{eqnarray*}
%%\Phi \left(\left( \sum_{n=m} a_nz^n \right)+ \left( \sum_{n=l} b_nz^n \right)\right)
%%&=& \frac{\sum_{n=m}^{\infty}\left(a_n + b_n\right) z^n}{1}\\
%%&=& \frac{{\sum_{n=m}^{\infty}a_n z^n}}{1}+ \frac{{\sum_{n=l}^{\infty}b_n z^n}}{1} \\
%%&=&\Phi \left(\sum_{n=m} a_nz^n\right) + \Phi\left(\sum_{n=l} b_nz^n \right)
%%\end{eqnarray*}
%%
%% 
%%
%%
%Damit ist $\Phi$ bezüglich der Addition ein Homomorphismus. Nun betrachten wir die Multiplikation. Aufgrund der oben getroffenen Vereinfachung gilt:
%\begin{eqnarray*}
%\Phi\left( \left( \sum_{n=m} a_nz^n \right)\cdot \left( \sum_{n=l} b_nz^n \right)\right)
%&=&\frac{z^{-m-l}\left(\sum_{n=m}^{\infty}a_nz^n\right) \left(\sum_{n=l}^{\infty}b_nz^n\right)}{z^{-m-l}} \\ 
%&=& \frac{\left(z^{-m}\sum_{n=m}^{\infty}a_n z^n\right)\left(z^{-l}\sum_{n=l}^{\infty}b_n z^n\right)}{z^{-m}z^{-l}}  \\
%&=& \frac{z^{-m}\sum_{n=m}^{\infty}a_n z^n}{z^{-m}} \cdot \frac{z^{-l}\sum_{n=l}^{\infty}b_n z^n}{z^{-l}}\\
%&=& \Phi \left(\sum_{n=m}^{\infty} a_nz^n\right) \cdot \Phi\left(\sum_{n=l}^{\infty} b_nz^n \right)
%\end{eqnarray*}
%Der Kern der Abbildung besteht aus dem Element $0_K$ und da $\Phi$ ein Homomorphismus ist, erhalten wir die Injektivität.\\
%Wir zeigen, dass der Homomorphismus auch surjektiv ist. Sei $m,l \geq 0$ und $q = \frac{{\sum_{n=m}^{\infty} a_nz^n}}{{\sum_{n=l}^{\infty} b_nz^n}} \in$ Quot$\left(K[[z]]\right)$ mit $\sum_{n=l} b_nz^n \neq 0$. \\
%Die Reihe $\sum_{n = l}^{\infty} b_nz^n $ kann deswegen in die Gestalt $z^l\sum_{n = 0}^{\infty} c_nz^n $, $c_0 \neq 0$ gebracht werden. Nach Satz \ref{potenzreihenringEinheit} kann die Reihe $\sum_{n = 0}^{\infty} c_nz^n $ invertiert werden und wir erhalten
%\begin{eqnarray*}
%q &=& \frac{{\sum_{n=m}^{\infty} a_nz^n}}{{\sum_{n=l}^{\infty} b_nz^n}} \\
%&=& \frac{{\sum_{n=m}^{\infty} a_nz^n}}{z^l\sum_{n = 0}^{\infty} c_nz^n } \\
%&=& \frac{\left(\sum_{n = m}^{\infty} a_nz^n\right)\left( \sum_{n = 0}^{\infty} c_nz^n\right)^{-1}}{{z^l}}\\
%&=& \Phi\left(z^{-l}\left(\sum_{n = m}^{\infty} a_nz^n\right)\left(\sum_{n = 0}^{\infty} c_nz^n\right)^{-1}\right).
%\end{eqnarray*}
%%
%Wir sehen sofort, dass $(z^{-l}\left(\sum_{n = m}^{\infty} a_nz^n\right)\left(\sum_{n = 0}^{\infty} c_nz^n\right)^{-1} \in K((z))$ ist. Der Homomorphismus ist folglich bijektiv und die Behauptung ist gezeigt. 
%%Da somit jede Laurentreihe $ f = \sum_{n\in \Z} a_nz^n $ die Gestalt $\lbrack{g},~{z^-m}\rbrack$ für ein $m \in \N$ hat, mit $g\in K[[z]]$ und $m\in\N$, ist $K((z))$ der Quotientenkörper (siehe \ref{quotkoerper}) von $K[[z]]$.
%}
% Genauer Beweis hierzu: http://www.mathematik.uni-muenchen.de/~schotten/FT/loesungsskizzen/blatt-2-lsg.pdf
%
%

%Für $K((z))$ gilt, dass der Körper nur Reihen mit Hauptteilen aus endlichen vielen Summanden enthält. $K((z))$ entspricht, wie in \ref{quot} gezeigt, dem Quotientenkörper des Ringes der formalen Potenzreihen.  
%Jede Potenzreihe $\sum_{n=0}^{\infty} a_nz^n \text{ mit } a_0\neq0$ ist invertierbar in $K[[z]]$ (siehe \ref{potenzreihenringEinheit}). In jedem Quotient $\frac{\sum_{n=0}^{\infty}  a_nz^n}{\sum_{m=0}^\infty b_mz^m}$ kann aufgrund dieser Eigenschaft alles, bis auf eine Potenz von $z$, aus dem Nenner gekürzt werden. Da  $\sum_{n \ge k}^{\infty} a_nz^n =z^k \sum_{n \ge 0} a_{n+k}z^n$ ist, enthält $K((z))$ nur Reihen, deren Hauptteil nur endlich viele negative Summanden hat. \\
%
Die formalen Laurentreihen bilden einen Oberring der Potenzreihen und stellen als Körper eine Körpererweiterung von $K$ um das transzendente Element $z$ dar.  %\footnote{http://www.mathematik.uni-muenchen.de/~schotten/MIA/Muster/4_4.pdf}
%
%
%
%
% 
%
%
%
Im Folgenden beschränken wir uns, im Sinne der Konvergenzbetrachtung, erneut auf den Körper $\C$. 
%
%Nach Kaiser Skript
\begin{defn}
Eine Laurentreihe $f = \sum_{n=k}^\infty a_n z^n$  konvergiert in $z_0 \in \C$, wenn ihr Haupt- und Nebenteil in $z_0$ konvergieren.
\end{defn}
%
%
%TODO: Dazu könnte man Abbildungen reinpacken, will man das?
\begin{bem}
Ist $\frac{1}{r} \in \lbrack 0,\infty\rbrack$ der Konvergenzradius von $\sum_{n=k}^{-1} a_{n}z^n$ und $R\in \lbrack 0,\infty\rbrack$ der Konvergenzradius des Nebenteils $\sum_{n=0}^{\infty} a_{n}z^n$, so konvergiert die Laurentreihe $\sum_{n=k}^\infty a_n z^n$ für alle $z$ mit $r \le |z| \le R$.
\end{bem} %nach \cite{janich99}
%
%
%
%
%
\begin{satz}
Die Menge der konvergenten Laurentreihen $\C_L\lbrace z\rbrace$ bildet einen Körper
\end{satz}
\beweis{Die Nullreihe $ 0_{\C}$ und die Einsreihe $1_{\C}$ sind trivialerweise konvergent und entsprechen dem neutralen Element der Addition beziehungsweise Multiplikation in $\C_L\lbrace z\rbrace$.\\ %evtl Null und Einsreihe nochmal ausschreiben??
Wir zeigen nun, dass zu jedem Element $f\in \C_L\lbrace z\rbrace$ mit $f\neq 0$ ein Inverses existiert und ebenso konvergiert. Da $\C_L\lbrace z\rbrace$ offensichtlich eine Teilmenge von $\C((z))$ ist, liegt $f$ in $\C((z))$ und nach \ref{Laurentreihenkoerper} existiert ein eindeutiges Inverses. Wir zeigen nun, dass das Inverse ebenso konvergiert und damit in $\C_L\lbrace z\rbrace$ liegt.\\
Wir können jede Laurentreihe $f\in \C_L\lbrace z\rbrace$ mit nichtleerem Hauptteil schreiben als $f= \sum_{n=m}^{\infty} a_nz^n = z^{m} g$ wobei $g\in K\lbrack\lbrack z\rbrack\rbrack$ ist. Da $f$ konvergent ist und der Faktor $z^{m}$ konstant ist, konvergiert $g$. Wir haben in Satz \ref{konvergentUnterring} bereits gezeigt, dass ein Inverses zu $g$ existiert und konvergiert. Da weiterhin ein inverses Element zu $z^{m}$, nämlich $z^{-m}$ existiert, können wir die zu $f$ inverse Laurentreihe definieren als $f^{-1} = \left(z^{m}g\right)^{-1} = g^{-1}z^{-m}$ und es gilt $ff^{-1}= z^m g g^{-1} z^{-m} = 1_K$. Da $g^{-1}$ nach \ref{konvergentUnterring} konvergiert und $z^{-m}$ konstant ist, konvergiert $f^{-1}$ ebenso.\\
Des weiteren bleibt zu zeigen, dass die Summe sowie das Produkt zweier konvergenter Laurentreihen wieder konvergent ist. Da der Hauptteil nach Definition einer Laurentreihe endlich ist, reicht es den Nebenteil zu betrachten. Die Konvergenz der Summe und des Produkts formaler Potenzreihen wurde bereits in \ref{konvergentUnterring} gezeigt. Die Summe zweier konvergenter Laurentreihen $f,g $ ist damit ebenso konvergent. \\
Um die Konvergenz des Produktes zweier konvergenter Laurentreihen zu beweisen, multipliziere man diese so mit den Potenzen von z, dass man eine konvergente Potenzreihe erhält. Sei $f,h \in \C_L\lbrace z\rbrace$ mit $f=\sum_{n=m}^{\infty}a_nz^n$ und $h= \sum_{n=l}^{\infty}b_nz^n$, dann gilt:
\begin{eqnarray*}
\left(\sum_{n=m}^{\infty}a_nz^n\right)\cdot \left(\sum_{n=l}^{\infty}b_nz^n\right) &=& \left(z^m\sum_{n=0}^{\infty}c_nz^n \right)\cdot \left(z^l\sum_{n=0}^{\infty}d_nz^n \right)\\
& =& z^mz^l \sum_{n=0}^{\infty}c_nz^n \sum_{n=0}^{\infty}d_nz^n
\end{eqnarray*} 
Wir erhalten ein Produkt konvergenter Potenzreihen und gehen wie in \ref{konvergentUnterring} vor. Das Produkt konvergenter Laurentreihen konvergiert also innerhalb eines Radius nur nicht im Punkt $0$.\\

}
%
%
%
%
%
\begin{satz}
Der Quotientenkörper von  $\C\lbrace z\rbrace$ ist isomorph zum Körper der konvergenten Laurentreihen $\C_L \lbrace z \rbrace$.
\end{satz}
\beweis{Wir haben in \ref{EinheitenimKonvergentenUnterring} bereits gezeigt, dass das Inverse zu einer Potenzreihe aus $\C\lbrace z\rbrace$ wieder konvergiert und in $\C\lbrace z\rbrace$ liegt.\\
Wie in Satz \ref{quot} zeigt man nun, dass es einen Isomorphismus zwischen dem Quotientenkörper der konvergenten Potenzreihen und dem Körper der konvergenten Laurentreihen gibt. \\
Nun gehen wir wie in \ref{quot} vor und erhalten die Isomorphie der beiden Körper. Damit ist die Behauptung bewiesen.}
%
%
%
Nun versuchen wir auf dem Körper der Laurentreihen eine Bewertung finden.
Dazu betrachten wir zunächst den Träger \ref{traeger} der Laurentreihe supp$(f)$. 
% Betrachte min$\lbrace\left(\text{supp}(f)\right)\rbrace$, eindeutig bestimmt durch den kleinsten Index $n_0$ der Laurentreihe, ab dem der Koeffizient $a_{n_0} \neq 0 $ ist. Die Menge all dieser Elemente bildet eine angeordnete abelsche Gruppe $\Psi $ und es gibt einen Isomorphismus von $\psi: \text{ }\Psi \rightarrow \Z$. \\
%%
%
%
\begin{satz} \label{LaurentreiheBewertung}
Die Abbildung  $v\colon K((z))\rightarrow \Z\cup \lbrace \infty \rbrace$ definiert durch $v(f)= \min\left(\textup{supp}(f)\right)$ ist eine diskrete Bewertung, wobei wir $\min\left(\varnothing\right)= \infty$ setzen.
\end{satz}
%
\beweis{
%TODO DIe Surjektivität bracuh ich doch??
Die Abbildung ist surjektiv, da es zu jeder ganzen Zahl eine Laurentreihe mit diesem Startwert gibt. Zu jedem $n \in \Z \cup \lbrace\infty\rbrace$ existiert also eine formale Laurentreihe $f \in K((z))$, sodass $v(f) = \min(\textup{supp}(f)) = n$ ist. 
Nach Definition \ref{bewKoerper} sind für den Körper $K((z))$ und die angeordnete abelsche Gruppe $\Z$ noch die Axiome B1'-B3' nachzuweisen. 
\begin{description}
\item[\normalfont{zu B1'}]: Klar nach Definition.%$"\Leftarrow"$ Sei f = $0_K = \sum_{n=0}^{\infty} a_nz^n$, mit $a_n = 0 \forall n \in \N$. Es gilt v(f) = 0 genau dann wenn $n_0 = 0$, wenn $ f(z)=\sum_{n = n_0}{\infty}a_n z^n $. Angenommen f $\neq 0_K = \sum_{n=0}^{max}$. Nach Voraussetzung muss gelten $a_{n_0} = a_0 \neq 0$. 
\item[\normalfont{zu B2'}]: Sei $f=\sum_{n = n_0}^{\infty}a_n z^n \text{ und } g=\sum_{m = m_0}^{\infty}b_m z^m$ mit $a_{n_0} \neq 0$ und $b_{m_0} \neq 0$. Dann ist $v(f) = n_0$ und $v(g) = m_0$. Damit gilt, dass $v(f) + v(g) = n_0 + m_0$ entspricht. \\
Wir wollen zeigen, dass das Bild von $fg$ unter der Abbildung
\[v(fg) = v( \sum_{n =n_0+m_0}\sum_{j+k=n}a_jb_kz^n) \stackrel{\mathrm{!}}= v(f) + v(g),\] ist, wobei $a_j = 0 \text{ für } j < n_0   \text{ und } b_k = 0$ für $k < m_0$ gilt. 
Wir erhalten nach Definition des Produkts zweier Laurentreihen sofort $v(fg) = n_0+m_0 = v(f) + v(g)$.
%Wir betrachten $ n < n_0 + m_0 $. 
%Da $ n = m+k $ folgt m < $n_0$ oder k < $m_0$. 
%Nach Voraussetzung folgt entweder $a_m = 0$, oder $b_k = 0$ und somit ist auch das Produkt $a_mb_k = 0. $ Weiterhin gilt nach Voraussetzung  $a_{n_0} \neq 0$ und $b_{m_0} \neq 0.$ Sei $n = n_0$ + $m_0$. Das Produkt $ a_{n_0}b_{m_0}$ ist ungleich Null und daher erhalten wir, dass $v(fg) = n_0+m_0 = v(f) + v(g)$ ist.
\item[\normalfont{zu B3'}]: Wenn $f, g$ wie oben definiert sind, erhalten wir für $v(f+g)$ die folgende Gleichung: 
\begin{eqnarray*}
v\left(f+g\right) &=&  v\left( \sum_{n = \text{min}\lbrace n_0,m_0 \rbrace}^{\infty}(a_n + b_n) z^n\right) \\
&\geq& \min\lbrace n_0,m_0 \rbrace \\ 
&\stackrel{\mathrm{def}}=& \min\lbrace v(f), v(g)\rbrace
\end{eqnarray*}
\end{description}
} 
%
%\begin{satz}
%Der Körper der formalen Laurentreihen über $\C$ $\C((z)) = \lbrace \sum_{n = n_0}^{\infty}a_nz^n: n_0 \in \Z, a_n \in \C, a_{n_0} \neq 0\rbrace\ $ist vollständig bezüglich der in \ref{LaurentreiheBewertung} definierten diskreten Bewertung. 
%\end{satz}
%\beweis{Betrachte die Cauchy-Folge ${(\sum_{n= n_i}^{\infty}a_{n,i}z^n)}_{i\in\N}$. Die Menge der Startindizies $n_i$ ist nach unten beschränkt, und kann nicht beliebig klein werden. Damit ist der Träger  }%TODO: siehe forum matheraum: beweisen mit cauchyfolge
%Der Träger einer formalen Laurentreihe $f(z)=sum{n = n_0}{\infty}a_n z^n$ konzentriert sich auf die Menge $ \mathtt{T}  := supp(f) = \lbrace n_0, n_0 + 1, n_0 +2, ..., deg(f)\rbrace \subseteq \Z $.
Wie wir in \ref{quot} gezeigt haben, ist der Körper der Laurentreihen eine Obermenge des Rings der Potenzreihen $K[[z]]$. Nachdem wir auf $K((z))$ bereits eine Bewertung definiert haben, weisen wir nach, dass es sich bei $K[[z]]$ um einen diskreten Bewertungsring \ref{bewring} handelt. 
%Wie in \ref{bewring} gezeigt, ist $K[[z]]$ ein diskreter Bewertungsring und der kleinste vorkommende Exponent eines Monoms liefert die Bewertung einer Potenzreihe. Der Quotientenkörper eines diskreten Bewertungsrings besitzt ebenso eine Bewertung \ref{quotbewring}. 

%
\begin{satz}\label{bewring}
$K[[z]] $ ist ein diskreter Bewertungsring. %\footnote{https://www.mathematik.uni-osnabrueck.de/fileadmin/mathematik/downloads/2012AlgKurven.pdf}
\end{satz}
\beweis{
%
Der Ring der formalen Potenzreihen $K[[z]]$ ist der Bewertungsring der Bewertung
$v\colon K((z))\rightarrow \Z\cup \lbrace \infty \rbrace$. Denn nach Definition des Bewertungsrings besteht dieser aus den Elementen des Körpers $K$ deren Bewertung größer gleich dem Nullelement der Gruppe ist. Die Elemente in $K((z))$ mit $v(a) \geq 0_\Z$ sind genau die Elemente aus $K[[z]]\subseteq K((z))$. Da $K((z)) = \textup{Quot}(K[[z]])$ nach \ref{quot} ist, gilt die Behauptung.
%
%


%Wie im vorherigen Satz gezeigt, existiert auf dem Körper der Laurentreihen eine diskrete Bewertung. Wie wir in \ref{quot} bewiesen haben, ist der Quotientenkörper von $K[[z]]$ dieser Körper der Laurentreihen. 
%Nach \ref{potenzreihenringEinheit} folgt, $K[[z]]$ besitzt genau ein maximales Ideal nämlich $\mathfrak{m} = (z)$. Für eine Potenzreihe $f$, die in $K[[z]]$ keine Einheit ist, nimmt der konstante Term den Wert $0$ an und wir erhalten, dass $a_0 = 0$ ist. Somit lässt sich jede derartige Potenzreihe schreiben als $f=z \widetilde{f}$, wobei  $\widetilde{f}$ die umindizierte Potenzreihe bezeichnet.\\
%Die Nullteilerfreiheit folgt wie in \ref{intring} ausführlicher gezeigt. Sei $0 \neq f,g \in K[[z]]$ und wir definieren der Einfachheit halber $c_n := a_ib_j \neq 0$.  
%Für die Produktreihe $fg$ erhalten wir, ab den Indizes i, j, dass $a_i,~ b_j\neq 0$ ist. Der Hauptidealring $K[[z]]$ ist noethersch %definieren?
%, denn jedes Ideal ist erzeugt von $z^j$, wobei $j$ der kleinste Index ist, ab dem die Koeffizienten $c_n$ der Potenzreihen ungleich 0 in dem Ideal sind. Für das maximale Ideal muss gelten, dass es von einem Element erzeugt wird, für das gilt $a_0 = 0$.
%Andernfalls wäre die entsprechende Potenzreihe eine Einheit und würde somit ganz $K[[z]]$ erzeugen.
%Nach Definition des diskreten Bewertungsring \ref{bewertungsring} gilt die Behauptung. 
 } 
 %
 %
%Damit folgt, dass $K[[z]]$ isomorph zu einem, wie in Punkt \ref{chap2} beschriebenen Bewertungsring $ A= \lbrace 0\rbrace \cup \{x \in K * | v(x) \geqslant 0\}$ ist.  \\
%Wie in obigem Beweis \ref{bewring} gezeigt, gilt: $ (z) \subset (z^2) \subset (z)^3 \subset (z)^4 \subset ... $. \\


%\begin{lemma}\label{quotbewring}
%Ist $R$ ein diskreter Bewertungsring, so ist Quot($R$) ein diskret bewerteter Körper mit der Bewertung $v(a/b)=v(a)-v(b)$. 
%\end{lemma}
%
% 
%
%
%
%
%
\section{Der verallgemeinerte Potenzreihenkörper}
%
Die Grundlagen zur Konstruktion eines Körpers über sehr allgemein definierten formalen Potenzreihen untersuchte Hahn 1907 in seiner Arbeit \glqq Über nichtarchimedische Größensysteme\grqq. 
%Er stellte im Rahmen seines Beweises des Hahnschen Einbettungssatzes die sogenannten Hahnschen Potenzreihen zunächst als Gruppen vor. Die so definierten Potenzreihen erlaubten nicht nur Exponenten der Unbestimmten aus der Menge der ganzen Zahlen, sondern aus beliebigen wohlgeordneten Untergruppen der Wertegruppe.
Hahn formulierte als einer der ersten Mathematiker Potenzreihen mit verallgemeinerten Exponenten aus einer Teilmenge von $\R$, wie beispielsweise:\\
\[ f = 1 + z^{\log 2} + z^{\log 3} + z^{\log 4} + ... \]
\[g = \frac{1}{2}z^{\frac{1}{2}} + \frac{3}{4}z^\frac{3}{4} + \frac{7}{8}z^\frac{7}{8} + ... + z + \frac{3}{2}z^\frac{3}{2} + ... + 2z^2 + ...\] 
Die einzige, unverzichtbare Bedingung, die Hahn an diese verallgemeinerten Potenzreihen stellte, war die Wohlordnung des Trägers der verallgemeinerten Potenzreihe.
Im Laufe des Beweises des Hahnschen Einbettungssatzes konnte er zeigen, dass die nach ihm benannten Potenzreihen nicht nur eine Gruppe, wie ursprünglich angenommen, bilden, sondern einen Körper, einen verallgemeinerten Potenzreihenkörper. \\ 
%Während seiner Beschäftigung mit Hilberts siebzehntem Problem untersuchte er die Hahnschen Potenzreihen hinsichtlich ihrer Körpereigenschaften. 
Neben Hahn beschäftigten sich auch die beiden Mathematiker Neumann und Mal'cev mit den von Hahn konstruierten Reihen und deren Einbettung in einen Körper. \\
Wir fokussieren unsere Betrachtungen auf verallgemeinerte Potenzreihen mit Exponenten in angeordneten abelschen Gruppen und die algebraischen Strukturen, die wir auf ihnen definieren können. 

%Bevor wir uns der Konstruktion der formalen verallgemeinerten Potenzreihen zuwenden, wird die Rolle des Trägers der Potenzreihen bei der Konstruktion von Potenzreihenkörpern genauer erörtert. 
%
%
%
%
%
%\subsection{Potenzreihenstrukturen mit Träger über den ganzen Zahlen}\label{traegerGanz}
%Wir kennen bereits die in \ref{potenzreihenring} definierten Potenzreihen sowie ihre Verallgemeinerung, die Laurentreihen. Bisher haben wir uns nur mit Potenzreihen beschäftigt, deren Elemente auf einer Teilmenge der ganzen Zahlen indiziert werden. Die Exponenten der Unbestimmten der induzierten algebraischen Struktur $\left(K[[z]], K((z))\right) $ gehören in beiden Fällen ebenso einer Teilmenge der ganzen Zahlen an. \\
%Betrachten wir den Körper der formalen Laurentreihen über dem Körper $K$. Die Wohlordnung des Trägers ist eine Voraussetzung zur Definition der Multiplikation im Körper $K((z))$.  Nun stellt sich die Frage, ob es noch allgemeinere Gruppen, als die Menge der natürlichen, oder ganzen Zahlen gibt, auf denen ein wohlgeordneter Träger von Potenzreihen definiert werden kann.
%
%
%
%
%
%
%
% 
\subsection{Formale Potenzreihen auf angeordneten abelschen Gruppen}
%
% 
%
%
%
%
Wir haben bisher ausschließlich formale Reihen betrachtet, deren Exponenten Elemente einer wohlgeordneten Teilmenge der ganzen Zahlen waren. In Satz \ref{traegerwohlgeordnet} haben wir gezeigt, dass die Wohldefiniertheit der Multiplikation mit der Wohlordnung des Trägers der Laurentreihe zusammenhängt. \\
Im Folgenden betrachten wir nicht mehr nur Teilmengen der ganzen Zahlen, sondern die bereits in dem vorherigen Kapitel \ref{chap2} vorgestellten angeordneten abelschen Gruppen. In diesem Abschnitt zeigen wir, dass Potenzreihen mit Exponenten in einer angeordneten abelschen Gruppe, unter der Voraussetzung der Wohlordnung des Trägers, addiert und multipliziert werden können.\\  
%Wir haben im vorherigen Gliederungspunkt festgestellt, dass die bisher betrachteten Potenzreihen immer auf Teilmengen der natürlichen Zahlen $\N$ (Potenzreihenring $K[[z]]$) oder den ganzen Zahlen $\Z$ (Laurentreihenkörper $K((z))$) konstruiert waren. In \ref{traegerGanz} wurde gezeigt, dass den Mengen, auf denen wir Potenzreihenringe definieren können, eine bestimmte, unverzichtbare Eigenschaft innewohnt: die Wohlordnung.
  Die nachfolgenden Ausführungen orientieren sich an \cite[S. 194 - 199]{fuchs66}, \cite[S. 601 - 655]{hahn07}, \cite[S. 6 - 20]{ucsnay63} und \cite[S. 49 - 64]{priesscrampe83}.\\\\
Wir bezeichnen im Folgenden mit $K$ einen Körper und mit $G$ eine angeordnete abelsche Gruppe in additiver Schreibweise.
%
%
%
% Die Äquivalenzklassen, die durch die archimedische Gleichheit \ref{archimedischeKlassen} entstehen, bilden eine total geordnete Menge, wir bezeichnen sie mit $\Pi$. Durch jedes positive Element $a \in \Gamma$, respektive seine Äquivalenzklasse $[a]= \pi$ werden zwei konvexe Untergruppen $L_\pi,~ U_\pi$  definiert. \\
%
%
%
\begin{defn}\label{funktionsDefinitionFormalePotenzreihe}
Eine \textit{formale Potenzreihe über $K$ mit Exponenten in $G$} ist eine Abbildung $f\colon G \to K, ~g \mapsto a_g$, wobei $\textup{supp}(f) := f^{-1} (K^*)$ wohlgeordnet ist.
\end{defn}
%ein $h \in G$ existiert, sodass $f(g) = 0$ für alle $g < h$.
%
%
%



%Für einen Körper $K$ und $\left(\Gamma, + \right)$ eine angeordnete, abelsche Gruppe, bezeichnen wir mit H $\left(\Gamma, K \right)$ die Menge aller Funktionen von $\Gamma$ nach $K$. 


\begin{defn}\label{TraegerFormalePotenzreihe}
Der \textit{Träger} einer formalen Potenzreihe $f$ mit Exponenten in $G$ über $K$ ist folgendermaßen definiert: 
\[ \text{supp}(f) := \lbrace g \in G \vert ~ a_g \neq 0\rbrace\]
\end{defn}
%
%
%
Wir stellen die in \ref{funktionsDefinitionFormalePotenzreihe} definierte formale Potenzreihe im Folgenden meist nicht mehr in Funktionsschreibweise, sondern als Reihe dar. Aus diesem Grund präsentieren wir diese gebräuchlichere Definition einer formalen Potenzreihe mit Exponenten in einer angeordneten abelschen Gruppe.
\begin{defn}
Die Reihe 
\[f = \sum_{g \in G}^{}a_g z^g, \text{ mit } a_g \in K, \text{ deren Träger supp}(f) \text{ wohlgeordnet ist}, \]
wird als \textit{formale Potenzreihe mit Exponenten in einer angeordneten abelschen Gruppe} bezeichnet.
\end{defn}
%
%
% 
%
%
%
\begin{nota}
Wenn wir ab jetzt von formalen Potenzreihen mit Exponenten in angeordneten abelschen Gruppen über einem Körper sprechen, bezeichnen wir diese vereinfachend als \textit{formale Potenzreihen}.
\end{nota}
%
%
%
%
%
%
%Wir nennen eine derartige Funktion $F\colon\Gamma \rightarrow K$, die in der Anordnung von $\Gamma$ einen wohlgeordneten Träger besitzt eine formale Potenzreihe auf $\Gamma$ über $K$. \\
%Die Addition derartiger Funktionen $F, G$ ist definiert durch: 
%\[\left(F + G\right)(x) = F(x) + G(x) \text{ für alle } x \in \Gamma.\] 
%Sei $\lambda \in K$ dann ist 
%\[ \left(\lambda G\right)(x) = \lambda F(x) \text{ für } x \in \Gamma,~ \lambda \in K \text{ und }F \in K[[\Gamma]].\]
%%
%Dementsprechend definieren wir die Gesamtheit dieser Elemente. Sei $\left( \Gamma, + \right)$ eine angeordnete, abelsche Gruppe und $K$ ein Körper.
%\begin{defn}\label{formaleSumme}
%Sei $F := \sum_{x \in \Gamma}^{}\Phi_x z^x$ mit den Koeffizienten $\Phi_x \in K$. Man nennt $F$ eine \textit{formale Potenzreihe} auf $\Gamma$ über K. Die Gesamtheit dieser formalen Potenzreihen wird im Folgenden mit $K[[z^{\Gamma}]]$ bezeichnet.
%\end{defn}
%
%Die Definition von $F$ verlangt, dass die Koeffizienten $\Phi_x \in K$ liegen und der Träger \\supp(F) = $\lbrack x \in \Gamma | \Phi_x \neq 0\rbrack $ wohlgeordnet ist bezüglich der Anordnung von $\Gamma$. \\
%Die Exponenten x der Unbestimmten z sind ebenfalls Element der angeordneten Gruppe $\Gamma$. Die Potenzreihen werden aufsummiert über einer Untermenge $U$ bestehend aus Elementen $x$ aus $\Gamma$. Die Anordnung von $\Gamma$ überträgt sich auf die Untermenge $U$ nach \ref{agG}. Der Träger einer formalen Potenzreihe, besitzt als wohlgeordnete Teilmenge von $\Gamma$, bezüglich der Anordnung von $\Gamma$ ein kleinstes Element. Alternativ lässt sich die Reihe 
%\[F = \Phi_{x_1}z^{x_1} + \Phi_{x_2}z^{x_2} + ... + \Phi_{x_p}z^{x_p} + ..., \] 
%als Summation über den Ordinalzahlen $p$ bis zu einem fixierten $a \in \Gamma$ und Exponenten $x_1 \le ...\le x_p ...$ die bezüglich der Anordnung $\grqq \le \grqq$ von $\Gamma$ monoton steigend geordnet sind, schreiben. \cite{carruth48}\\
%Eine formale Potenzreihe bezeichnet eine Funktion $\Phi: G \mapsto K$, die in der Anordnung von $\Gamma$ einen wohlgeordneten Träger supp$(\Phi) = \lbrace g \in G | \Phi(g) \neq 0 \rbrace$ besitzt. 
%
%
%
%
%
%
Wir bezeichnen die Menge aller formalen Potenzreihen mit 
\[K\left(\left(z^{G}\right)\right) = \lbrace f := \sum_{g \in G}^{}a_g z^g |~ a_g \in K, \text{  supp}(f) \text{ ist wohlgeordnet}\rbrace.\]
%
%
\subsection{Addition und Multiplikation in $K((z^{G}))$}
%
%
%
%
%
%
In diesem Abschnitt definieren wir die Verknüpfungen auf $K((z^{G}))$. Wir gehen dabei ähnlich wie in Teil \ref{RechnenMitLaurentreihen} vor. Wir müssen berücksichtigen, dass die Verknüpfungen nur dann wohldefiniert sind, wenn die Wohlordnung des Trägers erhalten bleibt.
%
%
%
%
%Die Potenzreihen $F, G$ sind genau dann gleich wenn:
%\begin{enumerate}
%\item[1.]$\grqq a \in$ supp(F), $a \notin$ supp(G)$\grqq$, impliziert dass $\Phi_a = 0$,
%\item[2.]$\grqq a \notin$ supp(F), $a \in$ supp(G)$\grqq$, impliziert dass $\Psi_a = 0$,
%\item[3.]$\grqq a \in$ supp(F), $a \in$ supp(G)$\grqq$, impliziert dass $\Phi_a = \Psi_a$.
%\end{enumerate}
%Man kann leicht nachprüfen, dass es sich um eine Äquivalenzrelation handelt. \cite{carruth48}\\%
%
%
%
%
%
\begin{defn}\label{AdditionFormalerPotenzreihen}
Seien $f, h \in K\left(\left(z^G\right)\right)$ mit $f = \sum_{g \in G}^{} a_g z^g \text{ und } h = \sum_{g \in G}^{} b_g z^g$.\\
Die Summe zweier formaler Potenzreihen $f, g$ ist definiert durch die Addition der Koeffizientenfolgen:
\begin{eqnarray*}
f + h &=& \sum_{g \in G}^{} a_g z^g + \sum_{g \in G}^{} b_g z^g
= \sum_{g \in G}^{}\left(a_g + b_g\right)z^g. 
\end{eqnarray*}
\end{defn}
%
%
%
%
%
\begin{satz}\label{wohldefiniertheitAddition}
Die in \ref{AdditionFormalerPotenzreihen} definierte Addition zweier formaler Potenzreihen $f, g \in K\left(\left(z^{G}\right)\right)$ ist wohldefiniert und die Summe $f +h$ ist wieder eine formale Potenzreihe.
\end{satz}
\beweis{ Wir zeigen, dass der Träger der Summe $f + h$ wohlgeordnet ist.
Es gilt offensichtlich supp$(f+ h)\subseteq $ supp$(f)~\cup$ supp$(h)$. Nach Lemma \ref{wohlgeordnvereinigung} ist die Vereinigung zweier wohlgeordneter Mengen wieder wohlgeordnet. Die Definition der Wohlordnung besagt, dass jede Teilmenge einer wohlgeordneten Menge wiederum wohlgeordnet ist. \\
Nach Voraussetzung sind $ \text{supp}(f) = \lbrace g \in G \vert a_g \neq 0\rbrace \text{ und } \text{supp}(h) = \lbrace g \in G \vert b_g \neq 0\rbrace\ $  wohlgeordnet. 
Daraus folgt die Behauptung.
%Das kleinste Element von supp$(f+h)$ existiert und es gilt 
%\[\min\left( \text{supp}\left(f+ h\right)\right) = \text{min}\lbrace \text{min}\left( \text{supp}(f)\right), \text{min}\left( \text{supp}(h)\right) \rbrace.\]
%Somit ist $f + g$ eine formale Potenzreihe. 
}
%
%
%
\begin{defn}\label{MultiplikationMitKörperelement}
Sei $\lambda \in K$ und $f \in K\left(\left(z^{G}\right)\right)$. Das Produkt der formalen Potenzreihe $f$ mit dem Körperelement $\lambda$ ist definiert durch: 
\[\lambda\cdot f = \lambda \cdot \left(\sum_{g \in G}^{} a_g z^g\right) = \sum_{g \in G}^{} \lambda a_g z^g \in K\left(\left(z^{G}\right)\right).\]
\end{defn}
  
Bevor wir die Multiplikation zweier formaler Potenzreihen definieren können, benötigen wir etwas Vorarbeit.\\
Sei $f,~ h \in K\left(\left(z^{G}\right)\right)$ mit $ f = \sum_{g_1 \in G}^{} a_{g_1} z^{g_1} \text{ und } h = \sum_{g_2 \in G}^{} b_{g_2} z^{g_2} $. 
%
%
%
%
%
%
Wir betrachten zunächst die multiplikative Verknüpfung einzelner Monome. 
\begin{defn}
Seien $a_{g_1} z^{g_1}, b_{g_2} z^{g_2} \in K\left(\left(z^G\right)\right)$. Das Produkt der Monome ist nach den Potenzgesetzen (\ref{Potenzgesetze}) folgendermaßen definiert: 
\[a_{g_1} z^{g_1} \cdot b_{g_2}z^{g_2} = a_{g_1} b_{g_2} z^{g_1 + g_2}.\]
\end{defn}
%
%
\begin{bem}
Wir haben die obige Definition der Multiplikation zweier Monome bereits, ohne explizite Nennung, für die Definition der Multiplikation formaler Laurentreihen und Potenzreihen verwendet. 
\end{bem}
%
%Betrachten wir zunächst das Produkt einzelner Monome: $\Phi_gz^g  \Psi_hz^h = \Phi_g  \Psi_h z^{g+h} $. Für die Unbestimmte $z$ gilt nach den Potenzgesetzen $z^{g_1} \cdot z^{g_1} = z^{g_1 + h_1}$. 
%
%
%
%
%
%
Die distributive Fortsetzung der Multiplikation von Monomen führt zur Definition der Multiplikation in $K\left(\left(z^{G}\right)\right)$. Wir erhalten das Produkt als Summe über der Summe des Produkts der einzelnen Koeffizienten.
%
%
%
\begin{defn}\label{MultiplikationformalePotenzreihen}
Seien $f, h \in K\left(\left(z^G\right)\right)$ mit $f = \sum_{g_1 \in G}^{} a_{g_1} z^{g_1} \text{ und } h = \sum_{g_2 \in G}^{} b_{g_2} z^{g_2}$. Die Multiplikation zweier formaler Potenzreihen $f, h$ erfolgt durch Faltung: 
\begin{eqnarray}\label{eq: multPotenzreihenkoerper}  
 f\cdot h &=& \sum_{g_1 \in G}^{} a_{g_1} z^{g_1} \cdot \sum_{g_2 \in G}^{} b_{g_2} z^{g_2} 
= \sum_{g \in G}^{}\sum_{g_1 + g_2 = g}^{}a_{g_1} b_{g_2}z^g.  
\end{eqnarray} 
\end{defn}
%
%
%
%\begin{bemnota}
%Das Symbol für die Multiplikation wird im Folgenden oft weggelassen. Anstelle von $f\cdot h$ schreiben wir $fh$.
%\end{bemnota}
%
%
%\[ \text{Sei } F = \sum_{a \in \Gamma}^{} \Phi_a z^a \text{ und } G = \sum_{a \in \Gamma}^{} \Psi_a z^a \text{ wobei } \Phi_a, \Psi_a \in K. \]
%
%
%
%
\begin{satz}\label{wohldefiniertMultiplikation}
Die in \ref{eq: multPotenzreihenkoerper} definierte Multiplikation zweier formaler Potenzreihen $f, h \in K\left(\left(z^G\right)\right)$ ist wohldefiniert. Das Produkt $fh$ ist ebenso eine formale Potenzreihe in $K\left(\left(z^G\right)\right)$.
\end{satz}
%
%
\beweis{Seien $f, h \in K\left(\left(z^G\right)\right)$. Wir untergliedern den Beweis und zeigen zunächst, dass für alle $g\in G$ die Summe $\sum_{g_1 + g_2 = g}^{}a_{g_1} b_{g_2}$ endlich ist.\\
%
%
%Der Träger der Summe $\sum_{g_1 + g_2 = g}^{}a_{g_1} b_{g_2}$ entspricht der Summe der Elemente der Träger der Summanden supp$(f)$ und supp$(h)$. Die Summe existiert und ist ungleich der leeren Menge, wenn mindestens ein Träger der Summanden ungleich der leeren Menge ist.
%%
%
%Die Summe existiert, wenn die Träger supp$(f)$ und supp$(h)$ nicht leer sind. Andernfalls erhalten wir die leere Summe. 
%Der Träger der Summe $\sum_{g_1 + g_2 = g}^{}a_{g_1} b_{g_2}$ kann reduziert werden auf die Elemente $g = g_1 +g_2$ für die gilt, dass $g_1\in$ supp$(f)$ und $g_2\in$ supp$(h)$ ist. Andernfalls wäre $a_{g_1}$ oder $b_{g_2}$ gleich null und das Produkt somit ebenso null. Da $G$ eine angeordnete abelsche Gruppe ist und $g_1, g_2 \in G$ sind, ist aufgrund der Abgeschlossenheit der Addition $g = g_1 + g_2 \in G$. Die Existenz der Summe ist nachgewiesen. Die Summanden der Summe $\sum_{g_1 + g_2 = g}^{}a_{g_1} b_{g_2}$ sind ungleich null für alle $g_1\in$ supp$(f)$ und $g_2\in$ supp$(h)$.\\
Die Summe $\sum_{g_1 + g_2 = g}^{}a_{g_1} b_{g_2}$ ist endlich, wenn es nur endlich viele Darstellungen für $g \in G$ in der Form $g = g_1+g_2$ gibt, wobei $g_1 \in$ supp $(f)$ und  $g_2 \in $ supp$(h)$. Nach Voraussetzung sind sowohl supp$(f)$ als auch supp$(h)$ wohlgeordnet. Die Gruppe $G$ ist angeordnet abelsch und wir können die Folgerung aus dem Lemma von Neumann (\ref{FolgerungNeumann}) anwenden. Die Aussage ist somit bewiesen.\\
%
%
%
%
Nun bleibt zu zeigen, dass der Träger des Produkts supp$(fh)$ wohlgeordnet ist.
Nach obiger Argumentation ist leicht zu sehen, dass supp($fh$) in der Summe der Träger, supp$(f)$ + supp$(h)$, enthalten ist. \\ 
%Die Summe $\sum_{a_1 + a_2 = a}^{}\Phi_{a_1} \Psi_{a_2}$ kann reduziert werden auf $a_1\in$ supp(F) und $a_2\in$ supp(G), da ansonsten der Summand null ist. \\
Die Mengen supp$(f)$ und supp$(h)$ sind als Teilmengen der angeordneten Gruppe $G$ angeordnet. Die Summe supp$(f)$ + supp$(h)$ ist aufgrund der Abgeschlossenheit der Addition in $G$ ebenso eine angeordnete Teilmenge von $G$. 
Nach Voraussetzung sind supp$(f)$ und supp$(h)$ wohlgeordnet. Die Summe supp($f$) + supp$(h)$ ist nach dem Lemma von Neumann \ref{LemmaNeumann} wohlgeordnet. Jede angeordnete Teilmenge einer wohlgeordneten Menge ist wohlgeordnet laut der Definition einer Wohlordnung. Daher gilt, dass supp$(fh)$ als angeordnete Teilmenge einer wohlgeordneten Menge (siehe Bemerkung \ref{Teilmengewohlgeordnet}) wohlgeordnet ist. \\
Somit ist die Multiplikation zweier formaler Potenzreihen $f,h \in K\left(\left(z^G\right)\right)$ wohldefiniert und das Produkt $fh$ liegt in $K\left(\left(z^G\right)\right)$.



%Nach dem Lemma von B.H. Neumann \ref{LemmaNeumann} ist damit supp$(F)$ + supp$(G)$ wohlgeordnet. Es ist bekannt, dass supp($F\cdot G) \subseteq$ supp$(F)$ + supp$(G)$ und jede Teilmenge einer wohlgeordneten Menge ist wieder wohlgeordnet nach der Definition der Wohlordnung (\ref{wohlgeordn}). Da jede nichtleere Teilmenge ein kleinstes Element besitzt, ist diese Teilmenge selbst wohlgeordnet. Wir erhalten die Wohlordnung des Trägers der Produktreihe supp($F\cdot G)$.
%
%
%
%
%                        
% % % % % %Angenommen sie wäre es nicht, dann enthielte die Menge der Elemente $a_1 + a_2$ % % % % % %eine Teilmenge ohne kleinstes Element und damit ließen sich unendlich viele % % % % % % %Paare ${a_1}_i$ und ${a_2}_i$ finden, sodass:
% % % % %\[{a_1}_1 + {a_2}_1 > {a_1}_2 + {a_2}_2 > ... > {a_1}_i + {a_2}_i > ...\]
% % % % % %Da alle ${a_1}_i$ einer wohlgeordneten Menge angehören existiert ein kleines   % % % % %Element, etwa ${{a_1}_i}_1$. Für alle $ {a_1}_i (i > i_1)$ gibt es wieder ein %       kleinstes Element ${{a_1}_i}_2$ und so weiter. Wir erhalten:
%       \[{{a_1}_i}_1 + {{a_2}_i}_1 > {{a_1}_i}_2 + {{a_2}_i}_2 > ... > {{a_1}_i}_n    	% % % % % +{{a_2}_i}_n > ...\]
%       während \[{{a_1}_i}_1 \le {{a_1}_i}_2 \le ... \le {{a_1}_i}_n  \le ....\]
%       Daraus würde folgen:
%       $ {{a_2}_i}_1 > {{a_2}_i}_2 > ... > {{a_2}_i}_n  > ...$.
%       Dies ist ein Widerspruch, da nach Voraussetzung ${{a_2}_i}_n$ einer wohlgeordneten %       Menge angehört und somit ein kleinstes Element besitzen muss.\\
%
%
%
%Angenommen es gäbe unendliche viele $a_1$ für welche $\Phi_{a_1} \neq 0 \text{ und } \Psi_{a - a_1} \neq 0$. Dann würden unendlich viele Paare ${a_1}_n, {a_2}_n$ existieren, für die gilt: 
%\[{a_1}_1 + {a_2}_1 = {a_1}_2 + {a_2}_2 = ... = {a_1}_i + {a_2}_i = ...\]
%Da ${{a_1}_i}$ einer wohlgeordneten Menge angehören nehmen wir im Folgenden immer an, es sei:
%\[{a_1}_1 < {{a_1}_2} < ... < {{a_1}_i} < ...\],
%Daraus würde wiederum folgen:\\
%\[{a_2}_1 > {{a_2}_2} > ... > {{a_2}_i} > ....\]
%Wir erhalten einen Widerspruch zur Wohlordnung der Menge der alle ${a_2}_n$ angehören.\\
%
%
%
%
%
%
%
%Wir haben gezeigt, dass eine Darstellung von $a$ als Summe von $a_1+ a_2$ nur auf endlich viele Arten möglich ist, wenn Träger der beiden zu multiplizierenden Potenzreihen wohlgeordnet sind. 
%TODO. stimmt das mit endliche Teilmenge und so schon??
%\begin{bem}
%Wir haben oben gezeigt,dass die durch Multiplikation entstandene Summe $\sum_{g_1 + g_2 = g}^{}a_{g_1} b_{g_2}$ aus \ref{eq: multPotenzreihenkoerper} endlich ist. Deswegen können wir den Koeffizient $\gamma_g$ der Unbestimmten $z^g$ in dem Produkt aus \ref{eq: multPotenzreihenkoerper} für ein festes $g\in G$ explizit darstellen:
%\[\gamma_g = a_{{g_1}_1}b_{{g_2}_1} + a_{{g_1}_2}b_{{g_2}_2} + ... + a_{{g_1}_n}b_{{g_2}_n}\]
%%Die Variable $n$ ist Element einer endlichen Teilmenge der natürlichen Zahlen $\N$.
%\end{bem}
%$\Lambda_a$ sei null, wenn es für $a$ keine Darstellung als Summe der Elemente der Träger von F und G gibt.\\ %
%
%
%
%Das Produkt zweier formaler Potenzreihen auf $\Gamma$ über K ist somit wohldefiniert; der Träger der erhaltenen formalen Potenzreihe wohlgeordnet und der entstandene Koeffizient $\Lambda$ liegt, als endliche Summe des Produkts zweier Körperelemente $\Phi$ und $\Psi$, ebenfalls im Körper $K$. \\
%
%
}

Insgesamt erhalten wir, dass $K\left(\left(z^{G}\right)\right)$ bezüglich der definierten Addition und Multiplikation abgeschlossen ist (\cite[Seite 601ff]{hahn07}, \cite[S. 210- 213]{neumann49}).
%
% 
%
\begin{bsp}
Sei $G = \Q$ eine angeordnete abelsche Gruppe. Bei der Reihe $f := z^{\frac{-1}{p}}+  z^{\frac{-1}{p^2}} + z^{\frac{-1}{p^3}} ...$ mit $p\in \Z\setminus\lbrace0\rbrace$ handelt es sich um eine formale Potenzreihe über einem beliebigen Körper, da der Träger $\lbrace \frac{-1}{p}, \frac{-1}{p^2}, \frac{-1}{p^3}, ... \rbrace$ mit $\min\left(\textup{supp}(f)\right) \geq -1$ wohlgeordnet ist.
\end{bsp}
%
%
%
%
%
%
%
%
\subsection{Der verallgemeinerte Ring der formalen Potenzreihen} 
Wir haben bisher die Addition und Multiplikation auf $K\left(\left(z^{G}\right)\right)$ definiert. In diesem Abschnitt zeigen wir, dass $K\left(\left(z^{G}\right)\right)$ ein kommutativer Ring über $K$ ist.\\
In der verwendeten Hauptliteratur (\cite{priesscrampe83}, \cite{fuchs66}) findet sich die multiplikative Schreibweise der angeordneten Gruppe $G$. Die multiplikative Schreibweise ermöglicht eine noch allgemeinere Definition der Multiplikation mithilfe von Faktorsystemen. Auf Basis dieser Definition konnte B.H. Neumann 1949 Schiefkörper von formalen Potenzreihen konstruieren.  \\
Im Fall einer additiv geschriebenen angeordneten abelschen Gruppe erhalten wir den direkten Bezug zu dem beschriebenen Laurentreihenkörper und dem darin eingebetteten Potenzreihenring. Dieser entsteht, wenn es sich bei der angeordneten abelschen Gruppe um $\Z$ handelt. 

%Sei supp(F)supp(G) :=$ \lbrace a_1, a_2 \in G | \Phi_{a_1} \Psi_{a_2} \neq 0\rbrace.$ Nach der Definition des Trägers folgt, dass die Menge angeordnet ist, denn als Untermenge von $\Gamma$ überträgt sich die Anordnung. Wir müssen zeigen, dass jede Teilmenge von supp(FG) ein kleinstes Element besitzt nach \ref{wohlgeordn}. Nach Definition der Multiplikation in unserem Körper K gilt: \\ 
%
%
\begin{satz}\label{verallgemeinerterPotenzreihenring}
Die Menge $\left(K\left(\left(z^{G}\right)\right), +, \cdot\right)$ ist ein kommutativer Ring über $K$. 
\end{satz}
\beweis{Seien im Folgenden f, h, k $\in K\left(\left(z^{G}\right)\right)$ mit
\begin{eqnarray*}
f &=& \sum_{g \in G}^{} a_g z^g, ~~~~~~~ h = \sum_{g \in G}^{} b_g z^g, ~~~~~~~ k = \sum_{g \in G}^{} c_g z^g.
\end{eqnarray*}
Die Menge $K\left(\left(z^{G}\right)\right)$ ist eine abelsche Gruppe bezüglich der Addition.
\begin{itemize}
\item \textit{Assoziativität:} Für alle $f, h, k \in K\left(\left(z^{G}\right)\right)
$ gilt nach Definition der Addition: 
\begin{eqnarray*}
 f+\left(h+k\right) &= & \sum_{g \in G}^{} a_g z^g + \left( \sum_{g \in G}^{} b_g z^g + \sum_{g \in G}^{}c_g z^g \right) \\
& =& \sum_{g \in G}^{} a_g z^g + \sum_{g \in G}^{} \left(b_g + c_g\right) z^g \\
&=& \sum_{g \in G}^{} \left( a_g + b_g + c_g\right) z^g \\
& =& \sum_{g \in G}^{} \left(a_g + b_g\right) z^g + \sum_{g \in G}^{} c_g z^g\\
&=& \left(f + h\right)+k.
\end{eqnarray*}

\item \textit{Neutrales Element der Addition:} Bezeichne $0_K$ das neutrale Element der Addition $0_K = \sum_{g \in G}^{} a_g z^g$, wobei ähnlich wie in \ref{Rechnen} gilt $a_g = 0$ für alle $g \in G$. Der Träger von $0_K$ ist die leere Menge, welche nach Definition wohlgeordnet ist.
\item \textit{Negatives Element der Addition:} Zu jedem Gruppenelement $f$ gibt es ein negatives Element der Addition $-f = \sum_{g \in G}^{} -a_g z^g$ mit $f+ \left(-f\right) = 0_K$, wobei supp$(f)$ = supp$(-f)$.
\item \textit{Kommutativität:} 
Die Addition ist kommutativ, denn es gilt 
\begin{eqnarray*}
f+ h &=& \sum_{g \in G}^{} a_gz^g + \sum_{g \in G}^{} b_gz^g 
= \sum_{g \in G}^{} \left(a_g  + b_g\right) z^g \\
&\stackrel{\mathrm{(*)}}=& \sum_{g \in G}^{} \left(b_g  + a_g\right) z^g 
= \sum_{g \in G}^{} b_gz^g +  \sum_{g \in G}^{} a_gz^g \\
&=& h + f.
\end{eqnarray*}
Die Gleichheit in ($*$) gilt, da $K$ ein Körper ist und $a_g, b_g \in K$ sind.  
\end{itemize}
%
%
%
%
%
%
Die Menge $K\left(\left(z^{G}\right)\right)$ ist ein kommutatives Monoid bezüglich der oben definierten Multiplikation.
\begin{itemize}
\item \textit{Assoziativität:}
\begin{eqnarray*}
 f\cdot \left(h \cdot k\right) &= & \sum_{g \in G}^{} a_g z^g \cdot \left( \sum_{g \in G}^{} b_g z^g \cdot \sum_{g \in G}^{}c_g z^g \right) \\
&=& \sum_{g \in G}^{} a_g z^g \cdot \left( \sum_{g \in G}^{}\sum_{g_1 + g_2 = g}^{} \left(b_{g_1} \cdot c_{g_2}\right) z^g \right)\\
&=&  \sum_{g \in G}^{}\sum_{g_0 + g_1 + g_2  = g}^{} a_{g_0} b_{g_1} c_{g_2} z^g \\
& =& \left(\sum_{g \in G}^{}\sum_{g_0 + g_1 = g}^{} a_{g_0} b_{g_1}z^g\right) \sum_{g \in G}^{}c_g z^g   \\
&= &  \left(\sum_{g \in G}^{} a_g z^g \cdot \sum_{g \in G}^{} b_g z^g \right)\cdot \sum_{g \in G}^{}c_g z^g \\
&=& \left(fh\right) k .
\end{eqnarray*}
%
%
%
%
%
%Zur Bildung des Produkts $(F\cdot G)\cdot H$ beziehungsweise $F\cdot(G\cdot H)$ gilt die Instruktion des Index $a$ des gesuchten Koeffizienten $\Omega$ auf sämtliche Weisen als Summe $a_1 + a_2 + a_3$, beispielsweise: 
%\[{a_1}_1 + {a_2}_1 + {a_3}_1 = {a_1}_2 + {a_2}_2 + {a_3}_2 = ... =  {a_1}_n + {a_2}_n + {a_3}_n.\]
%Dann hat der Koeffizient $\Omega$ die Form: 
%\[\Phi_{{a_1}_1} \Psi_{{a_2}_1} \Lambda_{{a_3}_1} + \Phi_{{a_1}_2} \Psi_{{a_2}_2} \Lambda_{{a_3}_2} + ... + \Phi_{{a_1}_n} \Psi_{{a_2}_n} \Lambda_{{a_3}_n}.\]
%Falls keine Darstellung von $a$ als Summe der Elemente der Träger der Potenzreihen $F, G, H$ existiert, ist $\Omega$ gleich null.\\
%
%
%
%
\item \textit{Kommutativität:} Seien $f,h \in K\left(\left(z^{G}\right)\right)$. Es gilt 
\[fh = \sum_{g \in G}^{}\sum_{g_1 + g_2 = g}^{}a_{g_1} b_{g_2}z^g = \sum_{g \in G}^{}\sum_{g_2 + g_1 = g}^{}b_{g_2} a_{g_1}z^g = hf.\]
Die Gleichheit folgt unmittelbar aus der Kommutativität von $G$ und der Kommutativität der Multiplikation im Körper $K$.\\
\end{itemize}

Es reicht ein Distributivgesetz nachzuweisen, da die Multiplikation in $K$ kommutativ ist. Es gilt
\begin{eqnarray*}
f \cdot(h + k) &=& \sum_{g \in G}^{} a_g z^g  \left(\sum_{g \in G}^{} b_g z^g + \sum_{g \in G}^{} c_g z^g\right) \\
&=& \sum_{g \in G}^{} a_g z^g  \left(\sum_{g \in G}^{} \left(b_g + c_g\right) z^g \right)\\
&=& \sum_{g \in G}^{} \sum_{g_1 + g_2 = g}^{} a_{g_1}{\left(b_{g_2} + c_{g_2}\right)} z^g \\
&=& \sum_{g \in G}^{} \sum_{g_1 + g_2= g}^{} a_{g_1}b_{g_2} + a_{g_1}c_{g_2} z^g \\
&=& \sum_{g \in G}^{} \sum_{g_1 + g_2= g}^{} a_{g_1}b_{g_2}z^g + \sum_{g \in G}^{} \sum_{g_1 + g_2= g}^{} a_{g_1}c_{g_2} z^g \\
&=& fh + fk.
\end{eqnarray*}
%
%
%
% 
%
%\item[(ii)] \[(F + G) \cdot H = \left(\sum_{a \in \Gamma}^{} {\Phi_a} z^{a}  \sum_{a \in \Gamma}^{} {\Psi_a} z^{a}\right) \cdot \sum_{a \in \Gamma}^{} {\Lambda_a} z^{a} = \left(\sum_{a \in \Gamma}^{} \left({\Phi_a}+ \Psi_a\right) z^{a}\right)\cdot \sum_{a \in \Gamma}^{} {\Lambda_a} z^{a} =\]
%\[ \sum_{a \in \Gamma}^{} {\Lambda_a} z^{a} \cdot \left(\sum_{a \in \Gamma}^{} \left({\Phi_a}+ \Psi_a\right) z^{a}\right) =   F\cdot H + G\cdot H\]
 
In den Beweis der Distributivgesetze fließen die im Körper $K$ gültige Distributivität, die Rechengesetze für die Unbestimmte \ref{Potenzgesetze}
%\ref{AdditionFormalerPotenzreihen}[Potenzgesetze} 
, beziehungsweise die Kommutativität der angeordneten abelschen Gruppe $G$ mit ein.  
}
%
\subsection{Das Inverse in $K\left(\left(z^{G}\right)\right)$ }
%
%
%
%
%
Im vorherigen Abschnitt haben wir gezeigt, dass $K\left(\left(z^{G}\right)\right)$ bezüglich der definierten Addition und Multiplikation einen kommutativen Ring bildet. \\
Wir benötigen etwas Vorarbeit, bevor wir zeigen können, dass zu jeder formalen Potenzreihe, die ungleich Null ist, ein Inverses existiert und die Menge  $K\left(\left(z^{G}\right)\right)$ ein Körper ist.\\
In diesem Abschnitt orientieren wir uns an \cite[S. 196- 198]{fuchs66} und \cite[S. 210- 213]{neumann49}.
% 
%
%
%
%
%
\begin{satz}\label{JedePotenzreihesoDarstellbar}
Jede formale Potenzreihe $0 \neq f \in K\left(\left(z^{G}\right)\right)$ lässt sich in eindeutiger Weise als  \\
\[f = \lambda\cdot z^g\cdot \left(1_K + h\right)\text{ mit } \lambda \in K^*, ~g \in G,~h \in K\left(\left(z^{G}\right)\right) \text{ und } \textup{min}\left(\textup{supp}\left(h\right)\right) > 0 \] darstellen, wobei $g= \min(\textup{supp}(f))$ ist.
\end{satz}
%
%
\beweis{
Sei  $f \in K\left(\left(z^{G}\right)\right)$ mit $f = \sum_{g \in G}^{}a_g z^g$. Nach Voraussetzung ist supp$(f)$ wohlgeordnet. Nach Definition der Wohlordnung existiert ein kleinstes Element $\gamma\in G$ in supp$(f)$ mit $\gamma=$ min(supp$(f))$. Da $G$ eine angeordnete abelsche Gruppe ist, existiert für jedes Element ein additives Inverses. Es gilt:
\begin{eqnarray*}
f&=& z^\gamma ~\sum_{g \in G} a_{g} z^{g-\gamma}\\
&=& z^\gamma \left(a_\gamma 1_K  + ~\sum_{g \in G \setminus \lbrace \gamma \rbrace} a_{g} z^{g -\gamma}\right) \\
&=& z^\gamma ~ a_\gamma \left(1_K + ~\sum_{g \in G \setminus \lbrace \gamma \rbrace} \frac{a_{g}}{{a_\gamma}^{-1}}z^{g-\gamma}\right) \\
&=& z^\gamma ~ a_\gamma \left(1_K + ~\sum_{g \in G \setminus \lbrace \gamma \rbrace} b_{g-\gamma} z^{g-\gamma}\right),
\end{eqnarray*}
%
%
mit $b_{g-\gamma} = \frac{a_{g}}{{a_\gamma}^{-1}}$. \\
%Die Gleichheit in (*) gilt, denn zu $a_\gamma$, mit $a_\gamma \in K$ und $a_\gamma \neq 0$ für $\gamma \in$supp$(f)$, existiert das Inverse. \\
Es gilt weiterhin $g-\gamma > 0$ für alle $g\in \textup{supp}(f)\setminus \lbrace \gamma\rbrace$, da $\gamma = \min(\textup{supp}(f))$.
 %und damit ist $g > \gamma$ für alle $g\in G$. 
 Wir erhalten die Positivität des Trägers der entstandenen formalen Potenzreihe $h = \sum_{g \in G \setminus \lbrace \gamma \rbrace} b_{g-\gamma} z^{g-\gamma}$. Die Darstellung ist eindeutig. Denn sowohl $\gamma = \min(\textup{supp}(f))$ als auch $a_\gamma$ sind nach Definition der formalen Potenzreihe $f$ eindeutig bestimmt. Angenommen es gäbe ein $h' \in K\left(\left(z^G\right)\right)$ mit $h'\neq h$, dann würde gelten:
\begin{eqnarray*}
&& f = a_\gamma z^\gamma\left(1_K + h\right) = a_\gamma z^\gamma\left(1_K + h'\right) \\
&\Leftrightarrow& \left(1_K + h\right) = \left(1_K + h'\right) \\
&\Leftrightarrow& h = h'.
\end{eqnarray*}
Dies ist ein Widerspruch zur Voraussetzung und wir erhalten die Eindeutigkeit der Darstellung.
}
%
% 
%
%
%
\begin{lemma}\label{VereinigungWohlgeordnet}
Sei $G$ eine angeordnete abelsche Gruppe mit Positivbereich $P$ und sei $U\subseteq P$ wohlgeordnet. Dann ist $\mathfrak{U} := \bigcup_{n\in\N} nU$ ebenfalls wohlgeordnet.
\end{lemma}
%
%
%
\beweis{Angenommen $\mathfrak{U}$ sei nicht wohlgeordnet. Dann existiert eine streng monoton fallende Folge $\left(u_n\right)_{n\in\N}$ in $\mathfrak{U}$ mit $u_n\in \mathfrak{U}$ für $n\in\N$, die wir folgendermaßen schreiben:
\begin{equation}\label{eq: folgefürinvers}
u_1 = {g_1}_1 +{g_1}_2 + ...+ {g_1}_{n_1} > u_2 = {g_2}_1 +{g_2}_2 + ...+ {g_2}_{n_2} > ... > u_i = {g_i}_1 +{g_i}_2 + ...+ {g_i}_{n_i} > ..., 
\end{equation} 
$\text{wobei } {g_i}_k \in U\text{ ist.}$
%
%
%
% 
%
%
%Konvexe Untergruppe eines Folgenelements wird von maximum der Summanden erzeugt:
Da $U \subseteq G$ und $G$ eine angeordnete Gruppe ist, können die ${g_i}_k \in G$ angeordnet werden. \\
%
Wir können ein maximales Element ${g_i}^* := \max_{k}\left( {g_i}_k\right)$ für jedes $i\in\N$ bestimmen. Da $U$ wohlgeordnet ist, gibt es unter den ${g_i}^*$ ein kleinstes Element $g^*$. \\
Nach Voraussetzung ist $g > 0$ für jedes $g\in U$. 
Wir bezeichnen mit $\langle u_i\rangle$ die von $u_i$ erzeugte konvexe Untergruppe. Nach Definition einer konvexen Untergruppe wissen wir, dass in der angeordneten abelschen Gruppe $G$
\[ \langle u_i\rangle = \langle \max_k\left( {g_i}_k\right)\rangle \]
gilt. \\
%
%
%
Wir erhalten also, dass $\langle u_i\rangle = \langle {g_i}^*\rangle$ für $i=1, 2,...$ ist und $\langle g^*\rangle = \min_{i\in \N}\left(\langle u_i\rangle\right)$. 
Die Folge $\left(u_i\right)_{i\in\N }$ ist streng monoton fallend und wir wissen, dass es unter den Elementen ${g_i}_k$ aufgrund der Wohlordnung von $U$ ein kleinstes gibt. Daher gibt es ein $n\in \N$, sodass $\langle u_n\rangle
= \langle u_{n+1}\rangle = \langle u_{n+m}\rangle = \langle g^*\rangle$ für alle $m\in \N$ ist. \\
Das bedeutet also wir können jeder streng monoton fallenden Folge $\left({u'}_i\right)_{i\in\N}$ in $\mathfrak{U}$ ein Element ${g^*}'$ aus $U$ zuordnen, sodass ab einem bestimmten Index $i_0$, die von ${g^*}'$ erzeugte konvexe Untergruppe der von ${u'}_i$ erzeugten konvexen Untergruppe entspricht.
Da $U$ nach Voraussetzung wohlgeordnet ist, gibt es unter den ${g^*}'$ ein kleinstes Element. \\
Wir nehmen an für unsere Folge $\left(u_n\right)_{n\in\N}$ ist $g^*$ das kleinste Element unter den ${g^*}'$.\\
Die Folge $\left(u_i\right)_{i\in\N }$ ist streng monoton fallend und offensichtlich gilt wegen der Konvexitätseigenschaft, dass 
\[\langle u_1 \rangle \supseteq \langle u_2 \rangle \supseteq ... \supseteq \langle u_i \rangle \supseteq ....\]

ist.
%obda weil es gibt in jeder untermenge kleinste Untergruppe wieso sollte es die nicht schon von Anfang an geben.
Wir können somit ohne Beschränkung der Allgemeinheit annehmen, dass
\[\langle u_i \rangle = \langle g^*\rangle ~~~~~~~~~\text{ für }\left(i = 1, 2,...\right)\]

gilt.
%
Es können mehrere $\gamma$ aus $U$ existieren, sodass $\langle g^*\rangle=\langle\gamma\rangle$ gilt. Unter diesen gibt es eines, nennen wir es $\mathfrak{g}$, das in der Anordnung von $G$ das kleinste ist. Wir erhalten für dieses $\mathfrak{g}$, dass $\mathfrak{g}\leq g^*\leq u_1$ ist, wobei $\langle \mathfrak{g}\rangle = \langle g^*\rangle = \langle u_1 \rangle$.\\
%
%
%Da $\langle u_1\rangle$ die kleinste erzeugte konvexe Untergruppe bezüglich $\mathfrak{U}$ ist, besitzt $\langle u_1\rangle$ in $\mathfrak{U}$ nur die trivialen konvexen Untergruppen 
Aufgrund der Eigenschaft von archimedischen Untergruppen einer angeordneten Gruppe existiert ein $p \in \N$, sodass $u_1 \le p\mathfrak{g}$ und da $u_1 > u_2 > ... > u_i > ...$ gilt 
\[u_i \le p\mathfrak{g}\text{ für alle }i \in \N.\] 
%
%
Wir wählen diese natürliche Zahl $p$ so klein wie möglich. \\
%
%
Jeder streng monoton fallenden Folge $\left({u'}_i\right)_{i\in\N}$ mit $\langle \mathfrak{g}\rangle = \langle u_i \rangle$ %u'_k ??????
wird auf diese Weise eine natürliche Zahl $p'$ zugeordnet. Da die natürlichen Zahlen wohlgeordnet sind, existiert eine unter den dadurch auftretenden natürlichen Zahlen kleinste Zahl $\overline{p}$. Wir nehmen an, dass die streng monoton fallende Folge $\left(u_i\right)_{i\in\N}$ so gewählt ist, dass wir ihr diese kleinste Zahl $\overline{p}$ zuordnen können.
%
%
%
%
Aufgrund der Kommutativität der Gruppe $G$ kann die Darstellung eines Folgenelements $u_i$ auf die beiden folgenden Fälle eingeschränkt werden
%
\[
(*)~~~ u_i = {g_i}^* ~~~~~~~~~~~~~~~~~~~~~~~   (**)~~~    u_i = v_i+{g_i}^*,
\]
%
%
%
wobei $v_i \in \mathfrak{U}$ gilt. 
Die Elemente ${g_i}^*$ sind Elemente von $U$. Nach Definition der Wohlordnung gibt es keine streng monoton fallende Folge unter den ${g_i}^*$. Aus diesem Grund und da die Folge ${\left(u_i\right)}_{n\in \N}$ nach Annahme streng monoton fallend ist, existieren nur endlich viele $u_i$ der ersten Form. \\
%
%
%
%

%
%
%
Folglich muss eine streng monoton fallende Teilfolge $\left(u_{\phi(i)}\right)_{i\in\N}$ von $\left(u_i\right)_{i\in\N}$ existieren, sodass alle Elemente wie in $(**)$ dargestellt werden können. Wiederum gilt nach Definition der Wohlordnung, dass es keine streng monoton fallende Folge unter den ${g_{\phi(i)}}^*$  gibt. Deswegen muss $\left({v_{\phi(i)}}\right)$ eine streng monoton fallende Teilfolge $\left({{v'}_{i}}\right) $ enthalten.

Diese Folge hat die selbe Form wie \ref{eq: folgefürinvers}. Wir erhalten mit der gleichen Argumentation, dass $\langle {v'}_i\rangle = \langle g^*\rangle$ ist.
%
%
%
%
%
Wir wissen, dass $v_i = u_i - {g_i}^*$ und daher ${v}_i \le u_i$ ist. Weiterhin gilt $u_i < \overline{p}\mathfrak{g}$ und wir können eine natürliche Zahl $q = \overline{p}-1$ finden, sodass $v_i \le q\mathfrak{g}$ für alle $i \in \N$.\\
Dies ist ein Widerspruch zu unserer Wahl der Folge ${\left(u_i\right)}_{i\in\N}$. Folglich ist $\mathfrak{U}$ wohlgeordnet.
%
%
%
%
%

%
%
%
%

}
\begin{lemma}\label{unendlicheSummeinPotenzreihenring}
%
%
Sei $\sum_{g \in G}^{}a_g z^g =: f \in K\left(\left(z^{G}\right)\right)$ mit $\min\left(\textup{supp}\left(f\right)\right) > 0$. Die unendliche Reihe
%
%
\begin{eqnarray*}
h &=& \sum_{n=1}^{\infty}\lambda_nf^n \\
&=& \sum_{n=1}^{\infty}\lambda_n{\left(\sum_{g \in G}^{}a_g z^g\right)}^n
\end{eqnarray*}
%
%
ist für beliebige Körperelemente $\lambda_n$ wohldefiniert und liegt in $K\left(\left(z^{G}\right)\right)$.
\end{lemma}
% 
% 
%
%
%
% 
%
\beweis{Wir zeigen als erstes die Existenz der Reihe.
Für $\sum_{g \in G}^{}a_g z^g =: f \in K\left(\left(z^{G}\right)\right)$ mit $\min\left(\textup{supp}\left(f\right)\right) > 0$ kann die unendliche Summe  $\sum_{n=1}^{\infty}\lambda_n{\left(\sum_{g \in G}^{}a_g z^g\right)}^n$ für $\lambda_n \in K$ definiert werden, wenn ${\left(f^n\right)}_{\gamma} := {\left(\sum_{g \in G}^{}a_g z^g\right)}^n$, wobei $a_g=0$ für alle $g<\gamma$ ist, nur für endlich viele $n\in \N$ ungleich null ist.\\ %oder ist das g\neq \gamma??? also is es die summe über g = \gamma oder g\geq \gamma???
%
%
%
Sei $U := \text{supp}(f)$ und $\gamma$ ein Element von unendlich vielen der Mengen $nU$ mit $n\in\N$. Wir wissen aus Lemma \ref{VereinigungWohlgeordnet}, dass $\mathfrak{U} = \bigcup_{n\in\N} nU$ wohlgeordnet ist. \\
%
%
%
Angenommen es gibt ein Element $\gamma\in G$, sodass ${\left(f^n\right)}_\gamma$ für unendlich viele $n\in\N$ ungleich null ist. 
Wir wählen als $\gamma$ das kleinste der Elemente, die in unendlich vielen der Mengen $nU$, $n\in\N$ liegen. 
Es gibt also unendlich viele Darstellungen von $\gamma$ als Summe von Elementen aus $U$. 
Wir ordnen diese Darstellungen nach wachsender Länge $\gamma = \gamma_{i1} +\gamma_{i2}+\gamma_{i3}+... +\gamma_{in_i}$ mit $\gamma_{ij} \in U$ und $n_1 <n_2 <...$. 
%
Die Folge ${\left(\gamma_{i1}\right)}_{i\in\N}}$ enthält eine streng monoton wachsende Teilfolge, da $U$ wohlgeordnet ist. \\
%
%
Wir nehmen an, dass ${\left(\gamma_{i1}\right)}_{i\in\N}}$ streng monoton wachsend ist. Die durch ${\gamma'}_i = -{\gamma_{i1}} +\gamma$ bestimmte Folge ist also fallend und weil $\mathfrak{U}$ wohlgeordnet ist und keine streng monoton fallende Folge existiert, ist ${\left({\gamma'}_i\right)}_{i\in\N}$ ab einem bestimmten Index konstant. Es gibt also ein $j\in\N$ mit ${\gamma'}_{j+m} = {\gamma'}_j = \gamma'$ für alle $m\in\N$. Das Element $\gamma'$ liegt somit in $\mathfrak{U}$ und daher in unendlich vielen der Mengen $nU$, $n\in\N$.\\
%
%
Nach Definition von $\gamma'$ und weil $\gamma_{i1} > 0_G$ ist, wissen wir, dass $\gamma' <\gamma$ ist. Dies ist ein Widerspruch zur minimalen Wahl von $\gamma$. 
Es gilt also nur für endlich viele $n\in\N$, dass ${\left(f^n\right)}_{\gamma}\neq 0$ ist.\\
Die unendliche Summe ist somit wohldefiniert und ihr Träger ist eine Teilmenge von $\mathfrak{U}$ und nach Lemma \ref{VereinigungWohlgeordnet} wohlgeordnet. Also ist \[\sum_{n=1}^{\infty}\lambda_n{\left(\sum_{g \in G}^{}a_g z^g\right)}^n \in K\left(\left(z^G\right)\right).\]
}
%
%
%
%
%
%
%
%
%
%
%
%
%
%
%
%
%
%
%
%
%
%
%
%
%
%
% 
%
%
%
%
%
%
%
%
%TODO: AB HIER URSPRÜNGLICHER LEMMA BEWEIS_EINFACH KOMMENATR ENTFERNEN WENN ICH WIEDER HABEN WILL
%\begin{lemma}\label{unendlicheSummeinPotenzreihenring}
%Sei $\sum_{g \in G}^{}a_g z^g = f \in K\left(\left(z^{G}\right)\right)$ mit $\min\left(\textup{supp}\left(f\right)\right) > 0$. Die unendliche Reihe
%\begin{eqnarray*}
%h &=& \sum_{n=1}^{\infty}\lambda_nf^n \\
%&=& \sum_{n=1}^{\infty}\lambda_n{\left(\sum_{g \in G}^{}a_g z^g\right)}^n
%\end{eqnarray*}
%ist für beliebige Körperelemente $\lambda_n$ wohldefiniert und liegt in $K\left(\left(z^{G}\right)\right)$.
%\end{lemma}
%\beweis{
%%
%Wir unterteilen den Beweis in zwei Blöcke. Die unendliche Reihe $h$, wie im Lemma definiert, liegt genau dann in $K\left(\left(z^{G}\right)\right)$, wenn ihr Träger wohlgeordnet ist. Der Träger supp$(h)$ ist eine Teilmenge der Vereinigung der Träger supp$(f^n)$. Es genügt nachzuweisen, dass 
%\begin{itemize}
%\item[1. ] die Vereinigung der Träger, $\bigcup_{n\in\N}$supp$(f^n)$, wohlgeordnet ist,
%\item[2. ] für ein festes $\gamma\in G$ gibt es nur endliche viele $n \in \N$, sodass ${\left(f^n\right)}_\gamma \neq 0$ ist. Wir schreiben ${\left(f^n\right)}_\gamma = \left({\sum_{g \in G} \sum_{g_1+ ... + g_n =g} a_{g_1}a_{g_2}... a_{g_n}z^g}\right)_\gamma =  \sum_{g_1+ ... + g_n =\gamma} a_{g_1}a_{g_2}... a_{g_n}z^\gamma $
%\end{itemize} 
%%
%%
%%
%%
%%
%%
%zu 1.: Die Bedingung ist erfüllt, wenn es keine streng monoton fallende Folge
%
%
%\begin{equation}\label{eq: folgefürinvers}
%u_1 = g_{11} +g_{12} + ...+ g_{1{n_1}} > u_2 = g_{21} +g_{22} + ...+ g_{2n_2} > ... > u_i = g_{i1} +g_{i2} + ...+ g_{in_i} > ..., 
%\end{equation} 
%$\text{mit } g_{ik} \in \text{supp}(f)\text{, gibt.}$
%%
%%Widerspruchsannahme
%Angenommen es gäbe eine derartige Folge in $\bigcup_{n\in\N}$ supp$(f^n)$.
%%
%%
%% 
%%
%%
%%Konvexe Untergruppe eines Folgenelements wird von maximum der Summanden erzeugt:
%Da $G$ insbesondere eine angeordnete Gruppe ist, können die $g_ik \in G$ angeordnet werden. Wir können ein maximales Element, $\max\left( g_{ik}\right)$ für jedes $i\in\N$ bestimmen. Nach Definition einer konvexen Untergruppe und da $\min\left(\bigcup_{n \in \N} \text{supp}\left(f^n\right)\right) > 0$ ist, gilt in $G$ offensichtlich, dass
%\[ \langle u_i\rangle = \langle \max_k\left( g_{ik}\right)\rangle \]
%besteht. \\
%%
%%
%%
%%Da Träger supp(F) wohlgeordnet -> gibt es kleinste erzeugt Untergruppe der Elemente des Trägers.
%Der Träger der formalen Potenzreihe $f$ ist nach Voraussetzung wohlgeordnet. Betrachte die Menge der von $a \in \text{supp}(f)$ erzeugten Untergruppe $\langle a \rangle$. Aufgrund der Wohlordnung von supp$(f)$ existiert eine kleinste Untergruppe $U$ von $G$.
%Wir wählen die Untergruppe $U$ möglichst klein.\\
%%
%%
%%
%%
%Offensichtlich gilt wegen der Konvexitätseigenschaft, 
%\[\langle u_1 \rangle \supseteq \langle u_2 \rangle \supseteq ... \supseteq \langle u_i \rangle \supseteq ....\]
%%
%%obda weil es gibt in jeder untermenge kleinste Untergruppe wieso sollte es die nicht schon von Anfang an geben.
%Wir nehmen ohne Beschränkung der Allgemeinheit an, dass
%\[\langle u_i \rangle = U ~~~~~~~~~\text{ für }\left(i = 1, 2,...\right)\]
%
%ist.
%%
%%% Offensichtlich entspricht die von $u_i$ erzeugte konvexe Untergruppe, wir bezeichnen sie mit $\langle u_i\rangle$, der von dem größten Element $max_{k}\left({a_i}_k\right)$ konvexen Untergruppe. Die anderen Summanden von $u_i$ sind nach Definition des Maximums kleiner als dieses und liegen aufgrund der Konvexität der Untergruppe in dieser. Also gilt die Gleichheit $\langle u_i\rangle = \langle\text{max}\left({a_i}_k\right)\rangle$. 
%%Da der Träger supp($F^n$) wohlgeordnet ist, gibt es unter den erzeugten Untergruppen aller Elemente des Trägers eine kleinste Untergruppe U von $\Gamma$. Die konstruierte Folge \ref{eq: folgefürinvers} ist so gewählt, dass die kleinste Untergruppe möglichst klein ist. Damit bleibt die Ordnung der Folgenglieder auch für die davon erzeugten konvexen Untergruppen erhalten:
%%\[\langle u_1 \rangle \supseteq \langle u_2 \rangle \supseteq ... \supseteq \langle u_i \rangle \supseteq ...\]
%%
%%
%%
%%
%%
%%
%%
%%
%%Wir nehmen ohne Beschränkung der Allgemeinheit an, dass die von $u_i$, wobei $i$ natürliche Zahlen sind, erzeugten konvexen Untergruppen die gesamte Untergruppe erzeugt. 
%%Wir wählen nun aus jeder Folge von ${{a_i}_k}_i$ ein ${a_i}^*$, sodass die von diesem Element erzeugte konvexe Untergruppe den von $u_i$ erzeugten konvexen Untergruppen, eventuell unter weglassen endlich vieler Elemente der Folge, entspricht. Diese ist, wie oben ohne Beschränkung angenommen, ganz $U$. 
%%
%%
%%
%%
%%
%%Es können mehrere $g\in$supp$(f)$ existieren , kann es mehrere geben, die ganz $U$ erzeugen, allerdings aufgrund der Wohlordnung des Trägers nur ein kleinstes, nennen wir es $a^*$. Die von $a^*$, ${a_1}^*$ und $u_1$ erzeugten konvexen Untergruppen sind nach Annahme gleich. Für die erzeugenden Element gilt jedoch, da $a^*$ das kleinste erzeugende Element ist und $u_1$ den Summanden ${a_1}^*$ enthält, die folgende Ungleichung bezüglich der Anordnung von $G$: 
%%\[a^* \le {a_1}^* \le u_1\]
%%
%%
%%
%%
%%
%%
%Wir wählen nun aus jeder Folge von ${g_{ik}}_i$ ein ${g_i}^*$, sodass $\langle {g_{ik}}_i\rangle = \langle u_i \rangle =  U$ gilt.
%Es können mehrere $g\in$ supp$(f)$ existieren, die die Gleichheit $\langle g\rangle = U$ erfüllen. Aufgrund der Wohlordnung des Trägers supp$(f)$ gibt es unter diesen Elementen, bezüglich der Anordnung von $G$, ein kleinstes, wir bezeichnen es mit $g^*$. 
%Es gilt $g^* \leq {g_1}^* \leq u_1$, wobei $\langle g^*\rangle = \langle {g_1}^* \rangle = \langle u_1\rangle = U$. Die erste Ungleichung erhalten wir aufgrund der Minimalität von $g^*$. Die zweite Ungleichung folgt aus der Definition von $u_1$. \\
%%
%%
%%
%%
%%Konvexitätseigenschaft
%%
%Aufgrund der Eigenschaften von konvexen Untergruppen einer angeordneten Gruppe existiert ein $p \in \N$, sodass $u_1 \le pg^*$ und da $u_1 > u_2 > ... > u_i > ...$ gilt 
%\[u_i \le pg^*\text{ für alle }i \in \N.\] 
%
%%
%%
%Aufgrund der Kommutativität der Gruppe $G$ kann die Darstellung der Folge $u_i$ auf die beiden folgenden Fälle eingeschränkt werden
%%
%%
%%Jedes Element unserer anfangs gewählten Folge $u_i$ kann in einer der folgenden Formen geschrieben werden, wobei $v_i$ Summen aus Elementen von ${a_i}_k$ sind:
%%
%%
%% 
%%
%\[
%u_i = {g_i}^* ~~~~~~~~~~~~~~~~~~~~~~~       u_i = v_i+{g_i}^*,
%\]
%wobei $v_i$ bestimmte Summen von $g_{ik}$ bezeichnen. 
%Die Elemente ${g_i}^*$ sind Elemente des Trägers. Nach Definition der Wohlordnung gibt es keine streng monoton fallende Folge unter den ${g_i}^*$. Aus diesem Grund und da die Folge ${\left(u_i\right)}_{n\in \N}$ nach Annahme streng monoton fallend ist, existieren nur endlich viele $u_i$ der ersten Form. \\
%Folglich muss eine streng monoton fallende Folge $v_i$ existieren: ${v_i}_1 > {v_i}_2 > ... > {v_i}_j > ...$. Diese Folge hat die selbe Form wie \ref{eq: folgefürinvers}. Wir erhalten mit der gleichen Argumentation, dass $\langle v_i\rangle = U$ ist. Wir wissen, dass $v_i \le u_i$ ist, da $u_i = v_i+{g_i}^*$ gilt. Wir können also wieder eine natürliche Zahl $q = p-1$ finden, sodass $v_i \le qa^*$ für alle $i \in \N$. \\ 
%Wir haben aus $u_i$ eine Folge konstruiert, die ebenso $\langle v_i\rangle = U$ und $v_i \leq \left(p-1\right)g^*$ erfüllt.
%Dies widerspricht jedoch der Minimalitätseigenschaft von $U$. 
%%
%%
%%Daraus folgt also, dass eine Folge $v_i$ aus $u_i$ konstruiert werden kann, was ein Widerspruch zur Wahl unserer Folge und der Minimaleigenschaft darstellt. 
%Die Vereinigung der Träger supp($F^n$) muss somit wohlgeordnet sein. \\
%%
%%
%%
%%
%Nach Definition \ref{wohlgeordn} und Bemerkung \ref{Teilmengewohlgeordnet} ist supp$(h)$, als Teilmenge der wohlgeordneten Menge $\bigcup_{n\in\N}$supp$(f^n)$, wohlgeordnet.\\
%%
%%
%%
%%
%%
%%
%zu 2.: Wir nehmen nun an, es existieren für jedes festgehaltene Element der angeordneten abelschen Gruppe $G$ unendlich viele ganze Zahlen $n \in \N$, sodass \[g = g_{i1} +g_{i2} + ... + g_{in_i},~ i \in \N,~\text{ mit } n_1 < n_2 < ... < n_i < ...\text{ und } {g_i}_k \in \text{supp}(f).\]
%Die Vereinigung der Träger supp($f^n$) ist wohlgeordnet. Somit existiert ein kleinstes Element $g$ der oben definierten Form. Nach Lemma \ref{unendlicheFolgeEigenschaften} enthält die Folge $\left({g_i}_1\right)_{i\in \N}$ eine streng monoton steigende Teilfolge, die wir gleich indizieren:
%\[g_{11} \le ... \le g_{i1} \le ... . \] 
%Die Folge $g_{i2} + ... + g_{in_i}$ ist mit $i \in \N$ konstant. Damit muss die durch $\left(g_i\right)'  = g_{i2} + ... + g_{in_i},~ i \in \N$ bestimmte Folge nicht wachsend und aufgrund der Wohlordnung der Vereinigung der Träger somit konstant. Es gibt also ein $j \in \N$ mit $\left(g_{j+m}\right)' = \left(g_j\right)' = g'$ für alle $m \in \N$. Damit liegt $g'$ in der Vereinigung der Träger supp($F^n$), für unendlich viele $n\in \N$. Weiterhin gilt $g' < g$, da $g' = -\left({g_i}_1\right) + g$ und ${g_i}_1 >1_G$. Dies ist ein Widerspruch zur Wahl von $g$. Damit existieren nur endlich viele ganze Zahlen $n$ für die der Koeffizient der formalen Potenzreihe $\left(a^n\right)_g \neq 0$ ist.
%}
%
%
%
%
%
%
%
%
%
%
%
%
%
%
%
%
%
%
%Wir zeigen zunächst, dass die unendliche Summe des Produkts aus einem beliebigen Körperelement mit einem Element des formalen Potenzreihenrings, mit positivem Träger wohldefiniert ist und wieder in $K\left(\left(z^{G}\right)\right)$ liegt. Dieses Element spielt eine wichtige Rolle zur Konstruktion eines Inversen. 
%
%
%
%
%
%
\begin{bem}
Die in Lemma \ref{unendlicheSummeinPotenzreihenring} definierte Reihe $\sum_{n=1}^{\infty}\lambda_n{\left(\sum_{g \in G}^{\infty}a_g z^g\right)}^n$ mit $\lambda_n \in K$ und $f:= \sum_{g \in G}^{\infty}a_g z^g$ $ \in K\left(\left(z^{G}\right)\right)$ mit $\min(\textup{supp}(f)) > 0$ kann folgendermaßen umgeschrieben werden:
\begin{eqnarray*}
\sum_{n=0}^{\infty}\lambda_n\cdot \left(f\right)^n &=& \lambda_0 f^0 + \sum_{n=1}^{\infty}\lambda_n\cdot f^n \\
&=& \lambda_0 + \sum_{n=1}^{\infty}\lambda_n\cdot f^n.
\end{eqnarray*}
%
%
%
%Diese Darstellung erinnert für $\lambda_0 = 1_K$ an die geometrische Reihe. 
%
%
%
\end{bem}
%
%
%
%
%
%
%
%Wir definieren die folgende Reihe:
%\[\overline{F} = \sum_{n=0}^{\infty}\lambda_n\cdot F^n, \text{ mit } F \in K\left(\left(z^{G}\right)\right), \text{ wobei } \text{min}\left(\text{supp}\left(F\right)\right) > 0, \lambda_n \in K^*. \]
%Wieso $\lambda$ eine Einheit und der kleinste Exponent der Unbestimmten $z$, für das der zugehörige Koeffizient ungleich null ist, positiv sein muss, klären wir im Folgenden. Die Potenzreihe $\overline{F}$ kann umgeschrieben werden:
%\[\sum_{n=0}^{\infty}\left(\lambda_n\cdot F\right)^n = \lambda_0 F^0 + \sum_{n=1}^{\infty}\left(\lambda_n\cdot F\right)^n = \lambda_0 + \sum_{n=1}^{\infty}\left(\lambda_n\cdot F\right)^n.\] 
 
\begin{defn}
Sei $\Upsilon :=\lbrace 1_K + f~|~ f \in K\left(\left(z^{G}\right)\right) \text{ und } \min(\text{supp}(f)) >0\rbrace$. Für $f', h' \in \Upsilon$ gilt:
\begin{eqnarray*}
 \left(f'h'\right):= \left(1_K + f\right) \cdot \left(1_K + h\right) = 1_K + \left(f+h+fh\right).
\end{eqnarray*} 
\end{defn}
%
%
%
%
%
%Wir assozieren jetzt mit jedem Element $F := \sum_{a \in \Gamma} \Phi_a z^a$ des formalen Potenzreihenrings ein Symbol $e+ F$ und betrachten die Menge $\Upsilon$ aller $\mathfrak{F} = e + F$ mit $e$ als neutrales Element von $\Gamma$, $F \in K\left(\left(z^{G}\right)\right)$ und $\text{min}\left(\text{supp}\left(F\right)\right) > 0$. Diese Menge $\Upsilon$ ist eine Gruppe bezüglich der folgenden Verknüpfung \[\left(e+F\right)\left(e+G\right) = e + \left(F+G+FG\right)\]
%, wobei die Operationen zwischen den Ringelementen der Addition und Multiplikation in $K\left(\left(z^{G}\right)\right)$ entsprechen. Die Abgeschlossenheit bezüglich der Multiplikation ist aus deren Definition klar ersichtlich. Das neutrale Element von $\Upsilon$ ist $e$. Mithilfe der geometrischen Reihe konstruieren wir das Inverse zu jedem Gruppenelement $\mathfrak{F}$ und zeigen in \ref{unendlicheSummeinPotenzreihenring}, dass dieses wohldefiniert ist und ein Element des Potenzreihenrings. Nach Definition der geometrischen Reihe gilt:
%\[\frac{1}{e - F} = \sum_{n=0}^{\infty}F^n = e + \sum_{n=1}^{\infty}F^n\] oder in äquivalenter Darstellung:
%\[\frac{1}{e + F} = = e + \sum_{n=1}^{\infty}(-F)^n.\]
%Man sieht sofort, dass für jedes Gruppenelement $e+ F$ die Reihe $e + \sum_{n=1}^{\infty}(-F)^n$ invers ist. Wir müssen allerdings noch zeigen, dass $\sum_{n=1}^{\infty}(-F)^n$ im formalen Potenzreihenring $K\left(\left(z^{G}\right)\right)$ liegt.
%%
%

%
%
%
%
%
%
%
%
%
%
%
%
%Das Lemma liefert uns die gewünschte Aussage, $ \sum_{n=1}^{\infty}(f)^n$ liegt im formalen Potenzreihenring $K\left(\left(z^{G}\right)\right)$. \\
%Die formale Potenzreihe $\left(-f\right)$ erfüllt die Voraussetzungen des Lemmas, denn es gilt $\text{min}\left(\text{supp}\left(-f\right)\right) > 0.$ 
%Somit enthält der Potenzreihenring ebenso $\overline{f} = \sum_{n=1}^{\infty}(-f)^n $. 
%
%
%
%
%
\begin{lemma}
Sei $1_K + f \in \Upsilon$ mit $f \in K\left(\left(z^{G}\right)\right)$ und $\min(\textup{supp}(f)) >0$. Das Element 
\[\sum_{n=0}^{\infty} \left(-f\right)^n = 1_K +\sum_{n=1}^{\infty}(-f)^n\]
liegt in $\Upsilon$ und ist invers zu $1_K + f$.

\end{lemma}
\beweis{Sei $\mathfrak{f} := 1_K + f \in \Upsilon$ mit $f \in K\left(\left(z^{G}\right)\right)$ und min(supp$(f)) >0$. Die Menge $K\left(\left(z^{G}\right)\right)$ ist nach Satz \ref{verallgemeinerterPotenzreihenring} ein kommutativer Ring. Es gilt somit $(-f) \in K\left(\left(z^{G}\right)\right)$. Der Träger der formalen Potenzreihe $(-f)$ entspricht dem Träger von $f$ und es gilt min(supp$(-f)) >0$. Nach Lemma \ref{unendlicheSummeinPotenzreihenring} gilt, dass $\overline{f} := \sum_{n=1}^{\infty}(-f)^n \in K\left(\left(z^{G}\right)\right)$ ist. Der Träger von $\overline{f}$ ist in der Vereinigung der Träger supp$((-f)^n)$ enthalten. Der Träger supp$(f^n)$ ist positiv für jedes $n\in \N$. Die Vereinigung positiver Mengen ist ebenfalls positiv und es folgt, dass der Träger der Reihe $\overline{f}$ positiv ist.
Wir erhalten also, dass $\overline{\mathfrak{f}} := 1_K +\sum_{n=1}^{\infty}(-f)^n $ in $\Upsilon$ liegt. Die Verknüpfung der beiden Gruppenelemente $\mathfrak{f}$ und $ \overline{\mathfrak{f}}$ hat folgende Form:
%
%
%
%
\begin{eqnarray*}
\mathfrak{f}~ \overline{\mathfrak{f}} &=& \left(1_K + f\right) \left(1_K + \overline{f}\right) \\
&=& 1_K + \left(f + \overline{f} + f \cdot\overline{f}\right) \\
&=& 1_K + \left(\sum_{g \in G} a_g z^g + \sum_{n=1}^{\infty}(-f)^n + \left(\sum_{g \in G} a_g z^g\right)\cdot \left(\sum_{n=1}^{\infty}(-f)^n\right)\right) \\
&=& 1_K + \left(\sum_{g \in G}a_gz^g + \left(-f\right)^1 + \sum_{n=2}^{\infty}\left(-f\right)^n + \left(-\left(-f\right)\cdot \sum_{n=1}^{\infty}(-f)^n\right)\right)\\
&=& 1_K + \left( \sum_{n=2}^{\infty}(-f)^n + \left( - \sum_{n=2}^{\infty}(-f)^{n}\right)\right)\\
&=& 1_K + 0_K = 1_K. 
\end{eqnarray*}
}
%
%
%
%%
%
%\[ = e + \left(\sum_{n=1}^{\infty}(-F)^n + \sum_{a \in \Gamma}\Phi_a z^a + \left(\sum_{n=1}^{\infty}(-F)^n\right)\cdot \left(\sum_{a \in \Gamma}\Phi_a z^a\right)\right)\]
%\[ =  \left(e  + \overline{F}\right) \left(e + F\right)\]
%
%
%
%
%
%TODO: Wirklich draussenlassen?
%\begin{bem}
%Sei $1_K + f \in \Upsilon$ mit $f \in K((z^G))$ und $\min\textup{supp}(f) > 0$, dann ist $1_K +f$ invertierbar mit dem Inversen $\sum_{n=0}^{\infty}\left(-f\right)^n$.
%\end{bem}
% 
%
%
%
%und $\overline{1_K +\sum_{n=1}^{\infty}(-f)^n}$ das Inverse. Der Träger supp$(f)$ ist größer Null und somit existiert das Inverse $1_K +\sum_{n=1}^{\infty}(-f)^n$ zu $1_K + f$.  Es gelten folgende Äquivalenzen
%\begin{eqnarray*}
%1_K &=& \left(1_K + f\right)\left( 1_K + \sum_{n=1}^{\infty}(-f)^n \right) \Leftrightarrow\\
%\frac{1_K}{1_K + f} &=& 1_K + \sum_{n=1}^{\infty}(-f)^n \Leftrightarrow \\
%\frac{1_K}{1_K + f} &=& \sum_{n=0}^{\infty}(-f)^n \Leftrightarrow\\
%\frac{1_K}{1_K - f} &=& \sum_{n=0}^{\infty}(f)^n
%\end{eqnarray*}
%Diese Form der Darstellung zeigt deutlich die Verbindung des inversen Elements mit der geometrischen Reihe.
%
%
% 
%
%
%
Mithilfe dieser Erkenntnis und Lemma \ref{JedePotenzreihesoDarstellbar} sind wir nun in der Lage, den zentralen Satz der Ausarbeitung zu beweisen.
%
\begin{satz}
$\left(K\left(\left(z^{G}\right)\right), +, \cdot\right)$ ist mit der definierten Addition und Multiplikation ein Körper.
\end{satz}
\beweis{
Wir wissen bereits, dass $K\left(\left(z^{G}\right)\right)$ nach \ref{verallgemeinerterPotenzreihenring} ein kommutativer Ring ist. Es genügt zu zeigen, dass zu jedem $0 \neq f \in K\left(\left(z^{G}\right)\right)$ ein Inverses existiert. Wie in Lemma \ref{JedePotenzreihesoDarstellbar} gezeigt, kann jedes Element $f\neq 0$ des Ringes $K\left(\left(z^{G}\right)\right)$ in der Form $f = \lambda\cdot z^g\cdot \left(1_K + h\right)$ geschrieben werden mit $\lambda \in K^*, g \in G,~h \in K\left(\left(z^{G}\right)\right), \text{ wobei } \text{min}\left(\text{supp}\left(h\right)\right) > 0$ ist. Wir bezeichnen mit $1_K + \overline{h}$ das Inverse von $1_K + h$ in der Gruppe $\Upsilon$. Da $\lambda\in K^*$ ist, handelt es sich um eine Einheit und es existiert ein eindeutiges Inverses, welches wir mit $\lambda^{-1}$ bezeichnen. Da $g$ ein Element der angeordneten, abelschen, additiv geschriebenen Gruppe $G$ ist, gibt es auch zu $g$ ein negatives Element, das wir $-g$ nennen. Mit selbiger Argumentation wie in Lemma \ref{JedePotenzreihesoDarstellbar} wissen wir, dass das Element $k := \left(1_K + \overline{h}\right)z^{-g}\lambda^{-1} $ in $ K\left(\left(z^{G}\right)\right)$ liegt. Nach den Rechengesetzen für die Unbestimmte \ref{Potenzgesetze} und den definierten Rechenoperationen in dem Potenzreihenring $K\left(\left(z^{G}\right)\right)$ erhalten wir:
%
%
%
%
% 
%
%
\begin{eqnarray*}
f\cdot k 
&=& \left(\lambda\cdot z^g\left(1_K + h\right)\right) \cdot \left( \left(1_K + \overline{f}\right)z^{-g}\lambda^{-1}\right) \\
&=& \left(\lambda\cdot z^g z^{-g}\lambda^{-1}\right) = \left(\lambda  z^{g-g}\lambda^{-1}\right)= \left(\lambda\lambda^{-1}\right)\\
&=& 1_K.\\
%&=& \left(1_K + \overline{f}\right)z^{-g}\lambda^{-1}\cdot \lambda z^g \left(1_K + f\right)\\
%&=& k\cdot f
\end{eqnarray*}
}

%
%
%
%
\begin{bsp}
Sei $G$ eine archimedisch angeordnete Gruppe. Nach dem Satz von Hölder \ref{aga} lässt sich $G$ in die Gruppe der additiven reellen Zahlen einbetten.
Somit lässt sich $K\left(\left(z^G\right)\right)$ in $K\left(\left(z^\R\right)\right)$ einbetten.
\end{bsp}
%
%
% 
%
%
%
%
%Potenzreihen auf einer angeordneten abelschen Gruppe $G$ mit wohlgeordnetem Träger formen über einem Körper $K$ somit den Körper der formalen Potenzreihen $K\left(\left(z^{G}\right)\right)$. \\
%Die Grundsteine dieser Theorie wurden von Hans Hahn 1907 in seinem Beweis, dass formale Potenzreihen auf einer angeordneten abelschen Gruppe über $\R$ einen Körper bilden, gelegt. Neumann verallgemeinerte Hahns Ergebnisse und zeigte, dass formale Potenzreihen auf einer multiplikativen Gruppe in der nicht-kommutativen Sichtweise einen Schiefkörper formen. Im Laufe der Jahre konnte die Theorie der formalen Potenzreihen, als Verallgemeinerung der Laurentreihen und Pusieuxreihen, immer weiter ausgebaut werden. \\

%
%
%
% 
\begin{bsp}
Es gilt $K\left(\left(z^\Z\right)\right) = K\left(\left(z\right)\right)$
\end{bsp}
%
%
%
%
Wir können nun die bewertungstheoretischen Aussagen, die wir in dem Körper der formalen Laurentreihen bewiesen haben, auf den verallgemeinerten Potenzreihenkörper ausweiten.
%
\begin{satz}
Die Abbildung $v\colon K\left(\left(z^G\right)\right) \to G\cup \lbrace\infty\rbrace,~ f \mapsto \min(\textup{supp}(f))$ mit $\min(\varnothing)=\infty$  ist eine Bewertung auf $K\left(\left(z^G\right)\right)$.
\end{satz}
\beweis{Die Abbildung ist surjektiv, denn zu jedem Gruppenelement existiert eine formale Potenzreihe $f \in K\left(\left(z^G\right)\right)$ mit diesem Startwert, sodass $v(f) = \min\left(\textup{supp}(f)\right)$ gilt. Wir weisen nun die Eigenschaften B1'-B3' aus Definition \ref{bewKoerper} nach.\\
\begin{description}
\item[\normalfont{zu B1':}] Klar nach Definition.
\item[\normalfont{zu B2':}] Sei $f=\sum_{g\in G}a_g z^g \text{ und } h=\sum_{g\in G}b_g z^g$ mit $g_1 = \min(\textup{supp}(f))$ und $g_2 = \min(\textup{supp}(h))$. Dann ist $v(f) = g_1$ und $v(h) = g_2$. Damit gilt, dass $v(f) + v(h) = g_1 + g_2$ entspricht. \\
Wir wollen zeigen, dass das Bild von $fh$ unter der Abbildung
\[v(fh) = v( \sum_{g \in G}\sum_{k+j=g}a_kb_j z^g) \stackrel{\mathrm{!}}= v(f) + v(h),\] ist, wobei $a_k = 0 \text{ für } k < g_1   \text{ und } b_j = 0$ für $j < g_2$ ist. \\
Wir betrachten zunächst $ g < g_1 + g_2 $. 
%Da $ n = m+k $ folgt m < $n_0$ oder k < $m_0$. 
Nach Voraussetzung ist entweder $a_k = 0$, oder $b_j = 0$ und wir erhalten $a_kb_j = 0. $ Weiterhin gilt nach Voraussetzung $a_{g_1} \neq 0$ und $b_{g_2} \neq 0.$ Damit ist $v(fh)= \min(\textup{supp}(fh))\geq g_1+g_2$\\
Sei nun $g = g_1 + g_2$. Die Summe $\sum_{k + j = g_1 + g_2} a_kb_jz^{g_1+g_2} = a_{g_1}b_{g_2}z^{g_1+g_2}$ ist ungleich Null und daher erhalten wir, dass $v(fh) = g_1+g_2 = v(f) + v(h)$ ist.
\item[\normalfont{zu B3'}]: Wenn $f, h$ wie oben definiert sind, erhalten wir für $v(f+h)$ die folgende Gleichung: 
\begin{eqnarray*}
v\left(f+h\right) &=&  v\left( \sum_{g = \text{min}\lbrace g_1,g_2 \rbrace}^{\infty}(a_g + b_g) z^g\right) \\
&\geq& \min\lbrace g_1, g_2 \rbrace \\ 
&\stackrel{\mathrm{def}}=& \min\lbrace v(f), v(h)\rbrace
\end{eqnarray*}
\end{description}
}
%
%
%
%
%
%
%
\begin{satz}
Die Menge $K\lbrack\lbrack z^G\rbrack\rbrack := \lbrace f \in K\left(\left(z^G\right)\right)~ \vert~ \textup{supp}(f) \geq 0\rbrace$ ist ein Bewertungsring von $v$.
\end{satz}
\beweis{Dies folgt unmittelbar aus der Definition von $K\lbrack\lbrack z^G\rbrack\rbrack$ und der Definition eines Bewertungsrings \ref{BewertungsringVonBewertung}.}
%
%
%
%
%
%
%
%
\begin{bsp}
Es gilt $K\lbrack\lbrack z^\Z\rbrack\rbrack = K\lbrack\lbrack z\rbrack\rbrack$.
\end{bsp}
%
%
Hier schließt sich der Kreis zu dem, zu Beginn des Kapitels betrachteten Potenzreihenring $K[[z]]$ und dem Laurentreihenkörper $K((z))$. Die Menge der ganzen Zahlen ist eine angeordnete abelsche additive Gruppe. Über einem beliebigen Körper K wissen wir nun, dass der Laurentreihenkörper mit $G = \Z$ ein Beispiel für einen verallgemeinerten Potenzreihenkörper darstellt.\\

%
%
%
%
%
%
%
%Wir haben nun gezeigt, dass sich jedes Element aus $K[[z^{\Gamma}]]$ mithilfe der Elemente $e + F$ der Gruppe $\Upsilon$ darstellen lässt. Da zu jedem Gruppenelement ein Inverses existiert
%Wir zeigen nun, dass die Division im Körper der formalen Potenzreihen auf $\Gamma$ über K wohldefiniert und eindeutig ausführbar, außer durch das neutrale Element der Addition $\epsilon$, ist. Das bedeutet, wenn $F, G \in K[[z^\Gamma$]] und F$\neq \epsilon$, dann gibt es ein Element H so dass $FH = G$ erfüllt und für jedes Element $G'$ des Potenzreihenkörpers gilt $G' = G$.\\
%\centerline{Sei F = $\Phi_{a_0}z^{a_0} + \Phi_{a_1}z^{a_1} + ... + \Phi_{a_n}z^{a_n} + ...$, wobei $a_0$ = min(supp(F)).}\\
%\centerline{und G = $\Psi_{b_0}z^{b_0} + \Psi_{b_1}z^{b_1} + ... + \Psi_{b_n}z^{b_n} + ...$, wobei $b_0$ = min(supp(G)),}
%zwei Elemente des Potenzreihenkörpers und die Träger der beiden Elemente somit wohlgeordnet. Gesucht wird H $\in K[[z^\Gamma]]$ derart, dass G = F$\cdot$ H gilt. Dazu setzen wir $H_1 := \frac{\Psi_{b_0}}{\Phi_{a_0}} z^{b_0 - a_0}$. Wir bilden $G_1$ = G - $H_1$F = ${\Psi_{}{{b_0}^1}}^1  z^{{b_0}^1} +{\Psi_{{b_1}^1}}^1 z^{{b_1}^1} + {\Psi_{{b_2}^1}}^1 z^{{b_2}^1}+ ... + {\Psi_{{b_n}^1}}^1 z^{{b_n}^1}+ +...$.\\
%Ist $G_1$ gleich Null, so ist $H_1$ die gesuchte Potenzreihe H. Andernfalls folgt ${\Psi_{b_0}^1}^1$ ist ungleich null und $ z^{{b_0}^1} <  z^{{b_0}}$. Wir setzen nun: \\
%\vspace{0.8cm}
%\centerline{$H_2 := H_1 + \frac{{\Psi_{b_0}^1}^1}{\Phi_{a_0}} z^{{b_0}^1 - a_0}$}\\
%und bilden: \\
%\vspace{0.8cm}
%\centerline{$G_2 = G_1 - H_2F = {\Psi_{}{{b_0}^2}}^2  z^{{b_0}^2} +{\Psi_{{b_1}^1}}^2 z^{{b_1}^2} + {\Psi_{{b_2}^1}}^2 z^{{b_2}^2}+ ... + {\Psi_{{b_n}^2}}^2 z^{{b_n}^1}+ +...$}.\\
%
%Ist $G_2$ gleich null, so ist $G_2$ die gesuchte Potenzreihe H.  Andernfalls folgt ${\Psi_{{b_0}^2}}^2$ ist ungleich null und $ z^{{b_0}^2} <  z^{{b_0}^1}$. Dieses Verfahren lässt sich fortführen. Entweder man erhält ab einem endlichen Index k zu einer Z Potenzreihe: \\
%\vspace{0.8cm}
%\centerline{$H_k = \frac{\Psi_{b_0}}{\Phi_{a_0}} z^{b_0 - a_0} + \frac{{\Psi_{b_0}^1}^1}{\Phi_{a_0}} z^{{b_0}^1 - a_0} + ... + \frac{{\Psi_{b_0}^{n-1}}^{n-1}}{\Phi_{a_0}} z^{{b_0}^{(n-1)} - a_0}$},\\
%und wir erhalten $G_n = G - H_nF = 0$.
%In diesem Fall ist $H_n$ die gesuchte Potenzreihe, sonst H und damit $G - H_nF$ für alle endlichen Indizes von null verschieden und es muss eine Potenzreihe $H_\omega$ existieren, die folgendes Aussehen hat:\\
%\vspace{0.8cm}
%\centerline{$H_\omega = \frac{\Psi_{b_0}}{\Phi_{a_0}} z^{b_0 - a_0} + \frac{{\Psi_{b_0}^1}^1}{\Phi_{a_0}} z^{{b_0}^1 - a_0} + ... + \frac{{\Psi_{b_0}^{n}}^{n}}{\Phi_{a_0}} z^{{b_0}^{n} - a_0} + ...$}.
%Damit erhalten wir für $G_\omega = G - H_\omega F = {\Psi_{}{{b_0}^\omega}}^\omega  z^{{b_0}^\omega} +{\Psi_{{b_1}^\omega}}^\omega z^{{b_1}^1} + {\Psi_{{b_2}^\omega}}^\omega z^{{b_2}^\omega}+ ... + {\Psi_{{b_n}^\omega}}^\omega z^{{b_n}^\omega}+ +...$.
%Wieder gilt, wenn $G_\omega$ = 0 ist, so ist $H_\omega$ die gesuchte Potenzreihe. Andernfalls lässt sich das Verfahren wieder fortführen, wie bereits angewendet.\\
%Allgemein ergibt sich folgende Formalisierung: Sei $\pi \in \Gamma$ und:\\
%\vspace{0.8cm}
%\centerline{$H_\omega = \frac{\Psi_{b_0}}{\Phi_{a_0}} z^{b_0 - a_0} + \frac{{\Psi_{{b_0}^1}}^1}{\Phi_{a_0}} z^{{b_0}^1 - a_0} + ... + \frac{{\Psi_{{b_0}^{\alpha}}}^{\alpha}}{\Phi_{a_0}} z^{{b_0}^{\alpha} - a_0} + ...$}  \\
%die Summe über alle Elemente der angeordneten Gruppe $\Gamma$ die < $\pi$ sind und \\
%\vspace{0.8cm}
%\centerline{$z^{b_0} > z^{{b_0}^1} > ... > z^{{b_0}^\alpha} > ...$}
%Sei p ein weiteres Element aus $\Gamma$, für das gilt: p < $\pi$. Für alle p < $\pi$ wissen wir, dass $G_p$ folgende Form hat:\\
%\vspace{0.8cm}
%\centerline{$G_p = G - H_p F =  {\Psi_{{b_0}^p}}^p  z^{{b_0}^p} +{\Psi_{{b_1}^p}}^p z^{{b_1}^p} + {\Psi_{{b_2}^p}}^p z^{{b_2}^p}+ ... + {\Psi_{{b_n}^p}}^p z^{{b_n}^p}+ +...$},\\
%wobei $\Psi_{{b_0}^p}^p \neq 0$ und der Träger der Summe wohlgeordnet ist. Wir stellen nun $G_\pi$ analog dar:\\
%\vspace{0.8cm}
%\centerline{$G_\pi = G - H_\pi F =  {\Psi_{{b_0}^\pi}}^\pi  z^{{b_0}^\pi} +{\Psi_{{b_1}^\pi}}^\pi z^{{b_1}^\pi} + {\Psi_{{b_2}^\pi}}^\pi z^{{b_2}^\pi}+ ... + {\Psi_{{b_n}^\pi}}^\pi z^{{b_n}^\pi}+ +...$}.
%Im Fall $G_\pi$ = 0, ist $H_\pi$ die gesuchte Zahl. Andernfalls können wir annehmen  $\Psi_{{b_0}^\pi}^\pi$ ist ungleich Null und wir beweisen, dass für jedes p < $\pi$ die Anordnung für den Wert der Potenz der Variablen erhalten bleibt, also: $z^{{b_0}^\pi} < z^{{b_0}^p} $.
%Wir setzen $H_\pi = H_p + H_{p*}$ und wählen $ H_{p*}$ folgendermaßen:\\
%\vspace{0.8cm}
%\centerline{ $ H_{p*}= \frac{{\Psi_{b_0}^p}^p}{\Phi_{a_0}} z^{{b_0}^p - a_0} + \frac{{\Psi_{b_0}^{p+1}}^{p+1}}{\Phi_{a_0}} z^{{b_0}^{p+1} - a_0} + ... + \frac{{\Psi_{b_0}^{\alpha}}^{\alpha}}{\Phi_{a_0}} z^{{b_0}^{\alpha} - a_0} + ... $ }.\\
%Wir erhalten $G_\pi = G - \left(G_p + G_{p*}\right) = G_p - H_{p*}F.$ Der Summand des höchsten Ranges stimmt in $G_p$ und $H_p*F$ überein, nämlich $\Psi_{{b_0}^p}^p  z^{{b_0}^p}$ und damit folgt, dass $z^{{b_0}^\pi} < z^{{b_0}^p} $ wie zu zeigen war. \\
%Wieder gilt, wenn $G - H_\pi F = 0$, so ist $H_\pi$ die gesuchte Potenzreihe, ansonsten wird der Prozess fortgesetzt. Dieses Verfahren muss terminieren. Ansonsten hätte der absteigend wohlgeordnete Träger der Potenzreihe von $G - H_\pi F$, der aus den Indizes ${b_0}^\pi$ gebildet wird, eine größere Mächtigkeit als die angeordnete Gruppe $\Gamma$. \\
%Es bleibt noch zu zeigen, dass nicht nur eine Potenzreihe bestimmt werden kann, derart dass FH = G gilt, sondern dass diese Potenzreihe auch eindeutig ist. Nach Definition der Multiplikation wird ein Produkt nur dann null, wenn einer der Faktoren der Nullreihe entspricht. 
%Es sei $F \neq 0, H\neq 0$, und sei $\Phi_{b_0}z^{b_0}$ das höchste Glied von F mit nicht verschwindendem Koeffizienten, also max(supp(F)) = $b_0$ und $\Psi_{b_0'}z^{b_0'}$ das höchste Glied von F mit nicht verschwindendem Koeffizienten, also max(supp(F)) = $b_0'$. Dann enthält G den Summand $\Phi_{b_0}\Psi_{b_0'}z^{b_0  + b_0'}$ und ist daher ebenfalls ungleich null. Sei also FH = G und FH' = G, dann gilt F(H-H') = 0 und da F aber nicht die Nullreihe ist, muss H = H' sein und damit H eindeutig.


%\begin{bsp}
%Sei $\Gamma = \N$ und $\le$ die natürliche Ordnung, dann ist $K[[z^\Gamma]] \cong K[[z]]$ wie in \ref{potenzreihenring} beschrieben.
%\end{bsp}


%\subsection{Pusieux Reihen}
%\subsection{Algebraische Abgeschlossenheit}
%The power series field K{tr
%} is algebraically closed if
%the coefficient field K is algebraically closed and if the ordered abelian
%group Y of exponents is a root group.
%PROOF. In the power series field S = K{tT
%} we introduce a valuation
%V by setting V(x) =ai if aait
%al is the first nonvanishing term in
%the power series (1) for x. In this valuation, Y is the value group and
%K the field of residue classes. Furthermore, J 5 is maximal with respect
%to this valuation, in the sense that any proper extension Sf>S to
%which the valuation V has been extended must either have a larger
%value group or a larger residue class field than S.
%Suppose now that S is not algebraically closed, so that S has a
%proper finite normal extension N. Certainly 5 is (algebraically) perfect,
%so that N/S is separable. The valuation V of 5 can be extended
%to N by the usual methods, for 5 is (topologically) perfect § with respect
%to V. The ordinary Newton polygon construction shows that each element c of N has a value of the %form a/n, with a in I\ Since F
%is a root group, a/n e T, and T is thus the value group of N. On the
%other hand, the residue class field of N must be an algebraic extension
%of the algebraically closed residue class field K of S. Thus N presents
%a proper extension of S in which neither value group nor residue class
%field is extended, contrary to the maximal property of S. 

% aus: http://projecteuclid.org/download/pdf_1/euclid.bams/1183502257 Mac Sanders Lane : The universalty of power series.

%zusatz bachelorarbeit
%