\chapter{Mathematische Grundlagen}
In diesem Kapitel werden die für die Ausarbeitung benötigten theoretischen Grundlagen zusammengestellt. Wir betrachten zunächst zentrale algebraische Strukturen und fokussieren uns im weiteren Verlauf auf Gruppen und Ordnungen die in ihnen definiert werden können.
%Erster Teil: Definition Gruppe, Ring, Körper (Algebraische Strukturen)
%Zweiter Teil: Bewertung (Einblick in die Bewertungstheorie) 
%Dritter Teil: Anordnung Wohlordnung Definitionen.(Ordnungsbegriff) viell. erst auf teilweise Ordnung -> Anordnung -> Wohlordnung
\section{Algebraische Strukturen}
Wir beginnen mit der Definition der elementaren algebraischen Strukturen Gruppe, Ring, Körper und Quotientenkörper. Die Vertrautheit mit grundlegenden Begriffen über Mengen und Abbildungen, sowie den wichtigen Zahlenmengen $\N, ~\Z,~ \Q,~ \R$ und $\C$ wird vorausgesetzt. Die folgenden Ausführungen sind orientiert an \cite{rainer08} und \cite{fischer08}.
%
\begin{defn}\label{Gruppe}
Eine nichtleere Menge $G$ mit der Verknüpfung $\circ \colon G \times G \rightarrow G, \left( a, b\right) \mapsto a \circ b$ heißt \textit{Gruppe}, wenn die folgenden Bedingungen erfüllt sind:
\begin{enumerate}
\item[G1: ] (Assoziativgesetz) $a\circ \left(b\circ c\right) = \left(a\circ b\right) \circ c$ für alle $a, b, c \in G$.
\item[G2: ] (Neutrales Element) Es gibt ein eindeutig bestimmtes Element $1_G \in G$ mit $1_G \circ a  = a \circ 1_G = a$ für alle $a \in G$.
\item[G3: ] (Inverses Element) Zu jedem $a \in G$ gibt es ein Element $a^{-1}$ in $G$ mit $a^{-1} \circ a = a \circ a^{-1} = 1_G$. \\
\item[] Die Gruppe heißt \textit{abelsch}, falls folgendes gilt: 
\item[G4: ] (Kommutativgesetz) $a \circ b = b \circ a$ für alle $a, b \in G$.  
\end{enumerate} 
\end{defn}
%
%
\begin{bem}
Wenn $\left(G, \circ\right)$ eine multiplikativ geschriebene Gruppe ist, so wird das Inverse eines Elements $a\in G$ mit $a^{-1}$ bezeichnet. \\
Wenn nichts anderes gesagt ist, verwenden wir in abelschen Gruppen die Verknüpfung $+$. Wir nennen das neutrale Element $0_G$ und $-a$ das Inverse zu $a\in G$. 
\end{bem}
%
\begin{bsp}
$\left( \Z, +\right), \left(\Q, +\right)$ und $\left(\R, +\right)$ sind abelsche Gruppen. 
\end{bsp}
%
\begin{defn}\label{Untergruppe}
Sei $\left(G, \circ\right)$ eine Gruppe mit neutralem Element $1_G$. Die Teilmenge $U\subseteq G$ heißt \textit{Untergruppe}, wenn gilt:
\begin{enumerate}
\item[U1: ] $1_G \in U$.
\item[U2: ] $a, b \in U \Rightarrow a\circ b \in U$.
\item[U3: ] $a \in U \Rightarrow a^{-1} \in U$.
\end{enumerate} 
\end{defn}


\begin{defn}\label{Ring} %\cite{fischer08} nach Skript Funktionentheorie Kaiser
Sei $R$ eine nichtleere Menge und seien $+ : R \times R \to R \text{ und } \cdot: R \times R \to R $ zwei Verknüpfungen auf $R$. Das Tripel $\left(R, +, \cdot\right)$ bezeichnen wir als \textit{Ring mit Eins}, wenn gilt:
%
\begin{enumerate}
\item[R1: ] $(R, +)$ ist eine abelsche Gruppe (deren neutrales Element mit $0_R$ bezeichnet wird).
\item[R2: ] Die Multiplikation $\cdot$ ist assoziativ: Für $a, b,c \in R$ gilt $a\cdot \left(b \cdot c\right) = \left(a \cdot b \right) \cdot c$ (das neutrale Element wird mit $1_R$ bezeichnet).%Blöd weil ich Monoid brauche: $\left(R, \cdot\right)$ ist ein Monoid,
\item[R3: ] (Distributivgesetze) Für alle $a,b,c \in$ R gilt\\
\[a \cdot(b +c) = a \cdot b + a \cdot c \text{ und }
(a+b) \cdot c = a \cdot c + b \cdot c. \] 
\end{enumerate}
\end{defn}
%Das neutrale Element bezüglich der Addition wird mit \textbf{$0_R$}, das neutrale Element bezüglich der Multiplikation mit \textbf{$1_R$} bezeichnet.

\begin{bem} %evtl doch keine extra Bemerkung
Ist die Multiplikation kommutativ, so heißt $\left(R, +, \cdot\right)$ \textit{kommutativer Ring mit Eins}. Anstelle von $\left(R, +, \cdot\right)$ sprechen wir vereinfachend von dem Ring $R$.
\end{bem}
%
%evtl Definition von Nullteiler, Einheit, Integritätsbereich? alles nach Erinnerung Kaiser
\begin{defn}
Sei $\left(R, +, \cdot\right)$ ein kommutativer Ring mit Eins und sei $a \in R$.
\begin{enumerate}
\item Ein Element $a \in R$ heißt \textit{Nullteiler}, falls $a \neq 0$ und ein $0 \neq b \in R$ existiert mit $ab =0$.
\item Ein Element $a \in R$ heißt \textit{Einheit}, falls ein $b \in R$ existiert mit $ab = 1$. 
\end{enumerate}
\end{defn}
%
%
%
\begin{bem}
Falls $a\in R$ eine Einheit ist, existiert das Inverse und es ist eindeutig bestimmt. Wir bezeichnen das Inverse mit $a^{-1}$.
\end{bem}

\begin{defn} \label{Integritätsbereich}
Ein kommutativer Ring $R$ mit Eins heißt \textit{Integritätsbereich}, falls es in $R$ keine Nullteiler gibt.
\end{defn}
%
%
%
%
%
%
% 
%
%
% 
\begin{defn}
Sei $K$ eine nichtleere Menge mit mindestens zwei Elementen und seien $+$ und $\cdot$ zwei Verknüpfungen auf $K$. Genau dann ist $\left(K, ~+,~ \cdot\right)$ ein \textit{Körper}, wenn folgende Gesetze gelten:
\begin{enumerate}
\item[(a)] Addition
\begin{enumerate}
\item[(i)] (Assoziativgesetz) Für alle $a, b, c \in K$ ist $\left(a + b \right) + c = a + \left(b +c\right)$.
\item[(ii)] (Kommutativgesetz) Für alle $a, b \in K$ ist $a+b = b + a$.
\item[(iii)] (Nullelement) Es gibt genau ein Element $0_K \in K$ mit $0_K + a = a + 0_K = a$ für alle $a \in K$.
\item[(iv)] (Negatives Element) Zu jedem $a \in K$ gibt es (genau) ein $-a \in K$ mit $\left(-a\right)+ a = a + \left(-a\right)= 0$.
\end{enumerate} 
\item[(b)] Multiplikation:
\begin{enumerate}
\item[(i)] (Assoziativgesetz) Für alle $a, b, c \in K$ ist $\left(ab\right)c = a\left(bc\right)$.
\item[(ii)] (Kommutativgesetz) Für alle $a, b \in K$ ist $ab = ba$.
\item[(iii)] (Einselement) Es gibt genau ein Element $1_K \in K$ mit $1_K a = a 1_K = a$ für alle $a \in K$.
\item[(iv)] (Inverses Element) Zu jedem $a \in K$ mit $a\neq 0 $ gibt es genau ein $a^{-1} \in K$ mit $a^{-1} a = a a^{-1} = 1$.
\end{enumerate}
\item[(c)] Distributivgesetz:
\begin{itemize}
\item Für alle $a, b, c \in K$ ist $a\left(b +c\right) = ab + bc$.
\end{itemize}
\end{enumerate}
\end{defn}


%Quotientenkörper Definition
%In seinem großen Werk $\glqq$Von Zahlen und Größen - Dritthalbtausend Jahre Praxis und Theorie$\grqq$ beschreibt Lüneburg die lange Entwicklungsgeschichte der Mathematik und ihre Ursprünge. Darin definiert er auch den Quotientenkörper \cite[S. 557]{Lueneburg08}.

\begin{satz}\label{quotkoerper} %\cite{Lueneburg08} statt RxR* mach ich RxR\0 weil R ist nicht definiert.
Es sei $R$ ein Integritätsbereich bestehend aus wenigstens zwei Elementen. Wir definieren auf $R\times R\setminus\lbrace 0\rbrace$ eine Äquivalenzrelation $\sim$. Sei $\left(r, u\right), \left(s, v\right) \in R\times R\setminus\lbrace 0\rbrace$. Es gilt, dass $\left(r,u\right) \sim \left(s, v\right)$ ist genau dann, wenn $rv = su$ ist. Man sieht sofort, dass die Relation reflexiv und symmetrisch ist. Sei $\left(r,u\right) \sim \left(s, v\right)$ und $ \left(s, v\right) \sim \left(t, w\right) $, also $rv = su$ und $sw = tv$. Wir können nun schreiben: \\
\[rwv = rvw = suw = swu = tvu = tuv.\]
Nach Voraussetzung ist $R$ ein Integritätsbereich. Da $v$ ungleich null ist, folgt $rw = tu$, so dass $\left(r,u \right) \sim \left(t, w\right)$ ist. Auf $R\times R\setminus \lbrace 0\rbrace$ definieren wir eine Addition und eine Multiplikation durch 
\[\left(r,u\right)+ \left(s,v\right) = \left(rv + su, uv\right),\]
\[\left(r,u\right)\left(s,v\right) = \left(rs, uv\right)\text{ definiert.}\] 
Wir setzen \textup{Quot}$(R) = \left(R\times R\setminus\lbrace 0\rbrace , \sim\right)$ und bezeichnen die Äquivalenzklasse von $(r,u) $ mit $\frac{r}{u}$. In diesem Fall gilt: 
\[\frac{r}{u} + \frac{s}{v} = \frac{rv + su}{uv}\]
\[\frac{r}{u}\frac{s}{v} = \frac{rs}{uv}\]
\textup{Quot}$\left(R, +, \cdot\right)$ ist ein Körper, wir nennen ihn den \textit{Quotientenkörper} von $R$.
\end{satz}
\beweis{Wir prüfen die Wohldefiniertheit der Verknüpfungen. \\Seien $\left(r_1, u_1 \right), \left(r_2, u_2\right), \left(s, v\right) \in R\times R\setminus\lbrace 0\rbrace$ und sei $\left(r_1, u_1 \right) \sim \left(r_2, u_2\right)$. Nach Definition der Äquivalenzrelation ist $r_1u_2 = r_2u_1$. Daher erhalten wir
\begin{equation*}
\left(r_1v + su_1\right)u_2v \\= r_1u_2v^2 + su_1u_2v \\
=r_2u_1v^2 + su_1u_2v \\
=\left( r_2v + su_2\right)u_1v \\
\end{equation*}
also 
\[\left(r_1v + su_1, u_1v\right) \sim \left(r_2v + su_2, u_2v\right)\]
Die Wohldefiniertheit der Multiplikation folgt analog. Man sieht
\begin{equation*}
\left(r_2s, s_2v\right) \sim \left(r_1s, u_1v\right),
\end{equation*}
denn 
\begin{equation*}
\left(r_2s\right)\left(u_1v\right) = r_1u_2sv = \left(r_1s\right)\left(u_2v\right).
\end{equation*}
Es genügt, die Körperaxiome für (Quot(R), +, $\cdot$) nachzurechnen. 
%Der Quotientenkörper ist nach Definition der kleinste Körper in den ein Integritätsring \textit{R} eingebettet werden kann. %\textit{Quot(R)} enthält alle Elemente der Form $\frac{a}{b}, \text{ mit} a,b \in R, b \neq 0. $ Es gibt einen injektiven Ringhomomorphismus $\phi : R \rightarrow Quot(R), \phi(a) = frac{a}{1}$.
}
%
%
% 
%
%
\begin{bem}
Der Quotientenkörper ist bis auf Isomorphie der kleinste Körper in den $R$ als Unterring eingebettet werden kann.
\end{bem}
%
%
%
%
%
%
%
\begin{bsp}
Ist D $\subseteq \C$ ein Gebiet, so ist der Ring $ \mathcal{O} (D) $ der in $D$ holomorphen Funktionen ein Integritätsring. Man nennt
$ M \left(D\right) := $Quot$\left( \mathcal{O} \left( D \right)\right) = \lbrace \frac{f}{g}: f,g \in \mathcal{O} (D), ~g \neq 0\rbrace$ den Körper der meromorphen Funktionen. Der Nenner kann unendlich viele Nullstellen besitzen, diese liegen allerdings isoliert. 
\end{bsp}
%

%
%
\section{Ordnung in algebraischen Strukturen}
\subsection{Anordnung}
\begin{defn}\label{defgs} 
Eine Menge $A$ heißt \textit{teilweise geordnet}, wenn es eine Relation $``\leq " $ auf $A$ gibt die folgende Eigenschaften für alle $ a,b,c \in A$  erfüllt.
%
\begin{enumerate}
\item[T1:] (Reflexivität)  $a \leq  a$,
\item[T2:] (Antisymmetrie)  Aus $a \leq  b$ und $b~ \leq a$ folgt $a = b$,
\item[T3:] (Transitivät) Aus $a \leq b$ und $b \leq c$ folgt $a \leq c$.
\end{enumerate}
%
Die Relation $``\leq "$ bezeichnet eine teilweise Ordnung auf $A$.
\end{defn}
Die oben definierte Ordnungsrelation wird als Anordnung beziehungsweise totale Ordnung bezeichnet, wenn neben T1-T3 die anschließende Bedingung erfüllt ist:
%
\begin{enumerate}
\item[T4:] Für alle $a, b \in A$ ist entweder $a < b$, oder $a = b$, oder $a > b$. Dabei gilt $a < b$ genau dann, wenn $a \leq b$ und $a\neq b$ ist. 
\end{enumerate}
%
%
%
%
%
%
%
\begin{defn}\label{ordnungsisomorph}
Seien $A$ und $A'$ teilweise geordnete Mengen. Eine Abbildung $\phi \colon A \rightarrow A', a \mapsto a'$ wird \textit{Ordnungsisomorphismus} von $A$ nach $A'$ genannt, falls folgende Anforderungen erfüllt sind:
\begin{enumerate}
\item (Ordnungstreue) Wenn $a \leq b$ gilt, so folgt $\phi(a) \leq \phi(b)$ für alle $a, b  \in A$.
\item (Bijektivität) Für jedes $a' \in A'$ existiert genau ein $a \in A$, mit $a' = \phi(a)$.
\end{enumerate}
$A$ und $A'$ werden in diesem Fall \textit{ordnungsisomorph}, kurz \textit{o-isomorph} genannt.
%Ist $G$ mit dem Positivbereich $P$ eine angeordnete Gruppe,
%so ist $G$ auch mit dem Positivbereich $\left(−P\right)$ eine angeordnete Gruppe. Die Abbildung
%$a \mapsto −a$ ist ein \textit{ordnungserhaltender Isomorphismus} zwischen diesen
%beiden Gruppen. Man nennt einen solchen Isomorphismus einen \textit{Ordnungsisomorphismus}.
%Man nennt Gruppen \textit{ordnungsisomorph (o-isomorph)}, wenn es zwischen ihnen einen Ordnungsisomorphismus gibt.
\end{defn}
%
\begin{defn}\label{twgG} % \cite{fuchs66}
Eine \textit{teilweise geordnete Gruppe} bezeichnet eine Menge $G$ mit folgenden Eigenschaften: 
%
\begin{enumerate}
\item[G1:] $G$ ist eine Gruppe bezüglich der Multiplikation,
\item[G2:] eine teilweise geordnete Menge bezüglich einer Relation $``\leq " $, wie in \ref{defgs}, 
\item[G3:] das Monotoniegesetz ist erfüllt: Für $a, b \in  G$ gilt: Aus $a \leq b$ folgt $ca \leq  cb$ und \\ $ac \leq bc$ für alle $c \in G$.
\end{enumerate}
% 
\end{defn}
%
%
\begin{defn}\label{agG}
Eine Gruppe wird als \textit{angeordnete Gruppe} bezeichnet, wenn ihre Ordnung total ist.
\end{defn}
%
%
%
\begin{bsp}\label{beispielUntergruppeAngeordnet}
Eine Untergruppe $U$ einer angeordneten Gruppe $G$ ist bezüglich der selben Relation wie $G$ angeordnet.
\end{bsp}
%
\begin{bsp}\label{OrdnungNundZ}
Wir betrachten die natürliche Ordnung auf den natürlichen, ganzen und reellen Zahlen.
\begin{enumerate}
\item Die Menge der natürlichen Zahlen $\N$ ist total geordnet bezüglich der Relation $``\leq "$. Es gilt für $a, b \in \N$, dass
\[ a \leq b \text{ ist genau dann, wenn } b-a \in \N
\]
für alle $a,b \in \N$.
\item Die Gruppe der ganzen Zahlen $\Z$ ist total geordnet bezüglich der Relation $``\leq "$. Es gilt für $a, b \in \Z$, dass
\[ a\leq b \text{ ist, genau dann, wenn } b -a \in \N \text{ oder } a = b \text{ gilt.}
\]
\item Die Gruppe der reellen Zahlen $\R$ ist total geordnet bezüglich der Relation $``\leq "$. Es gilt für alle $a, b \in \R$, dass
\[a \leq b \text{ ist, genau dann, wenn } 0 < b -a \text{ oder } a = b \text{ gilt.} 
\]
\end{enumerate}
\end{bsp}
%
% 
%
%
%ab jetzt nach Priess crampe
\begin{bem} \label{angeordnetFolgtTorsionsfrei} %\cite{priesscrampe83}
Jede angeordnete Gruppe ist torsionsfrei. 
\end{bem}
%
\beweis{
Dies folgt unmittelbar aus obiger Definition einer angeordneten Gruppe. Denn angenommen die angeordnete Gruppe wäre nicht torsionsfrei, so würde sich für die Elemente der Torsionsgruppe ein Widerspruch mit dem Monotoniegesetz G3 ergeben. %(formal siehe kleiner Block)
}
%
\begin{bem}\label{afG} %nach Fuchs S. 25
%
Genügt eine Teilmenge $P = \lbrace x \in G | x \geq 0\rbrace$ einer Gruppe $G$ den Bedingungen P1- P3, so nennt man $\left(G,\circ\right)$ \textit{anordnungsfähig}. Wir nennen $P$ den \textit{Positivbereich} von $G$.
%
\begin{enumerate}
\item[P1:] $\lbrace0\rbrace \cup P\cup -P = G$, $P \cap -P = \varnothing$,
\item[P2:] $P \circ P \subseteq P$,
\item[P3:] $x+P+(-x) \subseteq P$ für jedes $x \in G$.
\end{enumerate}

\end{bem}
%
%
%
\begin{bsp}
Ist $G$ mit dem Positivbereich $P$ eine angeordnete Gruppe, so ist $G$ auch mit dem Positivbereich $(-P)$ eine angeordnete Gruppe. Die Abbildung $a \mapsto -a$ ist ein Ordnungsisomorphismus.
\end{bsp}
%

\subsection{Wohlordnung}
Nun beschäftigen wir uns mit dem Begriff der Wohlordnung. Diese wird später bei der näheren Betrachtung des Trägers einer Potenzreihe eine wichtige Rolle spielen. Wir definieren wohlgeordnete Mengen wie in \cite[S. 16]{fuchs66}.
\begin{defn} \label{wohlgeordn} %\cite{fuchs66} 
Eine angeordnete Menge $W$ nennt man \textit{wohlgeordnet}, wenn jede nichtleere Teilmenge $V$ von $W$ ein kleinstes Element enthält. Es existiert also ein Element $ u \in V, \text{ mit } u \le v $ für alle $ v \in V.$ 
\end{defn}
%
Der Wohlordnungssatz, ein von Ernst Zermelo bewiesenes Prinzip der Mengenlehre, besagt, dass auf jeder Menge eine Wohlordnung existiert. Dieses Theorem, so stellte sich nach erfolglosen Widerlegungsversuchen zahlreicher Mathematiker heraus, ist äquivalent zum Auswahlaxiom und dem Lemma von Zorn. \\
Beispielsweise ist die natürliche Anordnung der natürlichen Zahlen $\N$ eine Wohlordnung. Die Menge $\Z$ ist mit der natürlichen Anordnung $``\leq "$ total geordnet, jedoch nicht wohlgeordnet, da die negativen Elemente von $\Z$ nicht nach unten beschränkt sind und somit $\Z$ kein kleinstes Element enthält. Nach der Konstruktion der ganzen Zahlen auf Basis der natürlichen Zahlen mittels einer Äquivalenzrelation auf $\N \times \N$  überträgt sich das Wohlordnungsprinzip von $\N \text{ auf } \Z$.
\begin{bem} %\cite{rainer08}
Ist $W \subseteq \Z$ eine nach unten beschränkte Teilmenge, so hat $W$ ein eindeutig bestimmtes kleinstes Element. 
\end{bem} 
%Der Beweis wird nicht benötigt. \beweis{ Es gilt: $ \Z$ ist ein kommutativer nullteilerfreier Ring mit Einselement und somit ein Integritätsbereich. Die Rechenoperationen sind wohldefiniert, wie leicht zu zeigen ist. Sei $M \subseteq \Z \text{ eine nach unten beschränkte Teilmenge von } \Z. $ Da M nach unten beschränkt ist gibt es ein $ a \in M \text{ sodass für alle } m \in M: a \le m.$ Noch zu zeigen ist, dass a eindeutig bestimmt ist. Dies folgt da $\le$ eine totale Ordnung auf $\Z$ definiert. Angenommen es gibt ein Element a' $\in$ M mit $a' \neq a \text{ und } \forall m \in M: a' \le m.$ Dann folgt nach Voraussetzung $a\le a' \text{ und } a'\le a$, und nach Definition einer totalen Ordnung \ref{twgG} [T2] $a' = a$, damit Widerspruch zur Voraussetzung. \\
%}

%

%\begin{bsp}
%Betrachte auf $\Z$ die Ordnung: $ a\prec b \Leftrightarrow (|a| \le|b| \vee |a| = |b|, a > 0).$ \newline %\footnote{http://www.mathematik.tu-dortmund.de/lsviii/new/media/veranstaltungen/wise1011/mathinf1/SkriptRek.pdf}. 
%Daher gilt in $\Z$: $ 0 \prec 1 \prec -1 \prec 2 \prec -2 ...$. \newline
%Das kleinste Element von $\Z$ in dieser Ordnung entspricht dem Element mit dem kleinsten Index. 
%\end{bsp}
%
%
\begin{bem}\label{Teilmengewohlgeordnet}
Jede Teilmenge einer wohlgeordneten Menge ist wohlgeordnet.
\end{bem}
\begin{bsp}
Betrachte die Relation $``\preceq " $ auf $\Z$. Es gilt,  
\[a \preceq b \text{ genau dann, wenn } |a| \le |b| \text{ oder }  |a| = |b| \text{ und } a \leq  b\]
ist.
Die Relation $``\preceq "$ ist eine Wohlordnung auf $\Z$ und wir erhalten 0$  \preceq -1 \preceq 1 \preceq -2 \preceq 2 \preceq 3 \preceq -3 .... $ 
\end{bsp}
%

%
 
\chapter{Angeordnete abelsche Gruppen}\label{chap2} %TODO: Wichtig hier nur alles was abelsch ist rein!!!
%Teil 1: Definition angeordnete abelsche Gruppe und Beispiele (Natürlichen Zahlen, usw.)
%Teil 2: Zusammenhang angeordnete abelsche Gruppe mit Wohlordnung
%Teil 3: Satz von Hölder
In diesem Kapitel fassen wir jene Begriffe und Bezeichnungen zusammen, die zur Betrachtung des Potenzreihenkörpers benötigt werden. Nach einer Einführung in die Theorie angeordneter abelscher Gruppen beschäftigen wir uns mit deren Wohlordnung, eine Eigenschaft, die für die Konstruktion des Potenzreihenkörpers unabdingbar ist. Mithilfe der Archimedizität führen wir eine spezielle Art der Anordnung von Gruppen ein. Die Familie der konvexen Untergruppen führt uns zu Aussagen über die Anordnungsfähigkeit von Gruppen. Daran schließt die zentrale Aussage des Kapitels an: der Satz von Hölder. Der Satz besagt, dass archimedisch angeordnete Gruppen in die additive Gruppe des $\R$ eingebettet werden können. \\
%TODO: vielleicht zu ausufernd? Die Theorie der angeordneten Strukturen, in unserem Fall ausschließlich Gruppen, liefert wichtige Erkenntnisse zur späteren Konstruktion des Körpers von formalen Potenzreihen. Die Funktionen, die durch Potenzreihen dargestellt werden, sind nicht mehr nur auf den natürlichen Zahlen, sondern jeder angeordneten abelschen Gruppe definierbar, wobei auf die Wohlordnung nicht verzichtet werden kann. 
Die nachfolgenden Ausführungen sind angelehnt an \cite[S. 21 - 28]{fuchs66} und \cite[S. 1 -  4]{priesscrampe83}.
%
%

% Kaiser Antwort abwarten-> raus
%\begin{defn} %\cite{priesscrampe83}
%Eine abelsche Gruppe $\left(G, +\right))$ heißt \textit{teilbar}, wenn zu jedem $a\in G$ und $n \in \N$ ein Element $b \in$ $G$ mit $nb = a$ existiert.
%\end{defn}
%\begin{satz}\label{torsionsfreiHülle} %\cite{priesscrampe83}
%Eine torsionsfreie, abelsche Gruppe $\left(G, +\right)$ ist bis auf Isomorphie in genau einer minimalen teilbaren, abelschen Gruppe $\left((\overline{G}, +\right)$ enthalten. $\overline{G}$ heißt die teilbare Hülle von $G$. 
%\end{satz}
%\beweis{
%Wir betrachten $\overline{G} $, bestehend aus der Menge der Paare $\left(x, n\right)$ mit $x\in G$, $n\in \N$ wobei $\left(x, n\right) = \left(y, m\right)$ für $mx = ny$ gelte. Die Addition über $\overline{G}$ wird definiert durch $\left(x, n\right) + \left(y, m\right) = \left(mx + ny, mn\right)$. Mit dieser Verknüpfung ist $\overline{G}$ eine abelsche Gruppe, denn $\left(x, n\right) + \left(y, m\right) = \left(mx + ny, mn\right) = \left(ny + mx, nm\right)= \left(y, m\right) + \left(x, n\right)$, da $G$ nach Voraussetzung abelsch und $\N$ ein kommutativer Halbring ist. \\
%Die Gruppe $\overline{G}$ ist torsionsfrei, da jedes Element, bis auf das neutrale, unendliche Ordnung hat, nach Konstruktion von $\overline{G}$. Durch die Abbildung  $G \rightarrow \overline{G}, a \mapsto (a, 1)$ ist eine Einbettung von $G$ in $\overline{G}$ gegeben, die jedem Element aus $G$ ein Element in $\overline{G}$ zuordnet.\\
%Wir konstruieren eine minimal teilbare Gruppe einer teilbaren Obergruppe $G^*$ von $G$. Als minimal teilbare Untergruppe von $G^*$, die $G$ enthält nach Konstruktion, wählen wir $\Q G = \lbrace q x: q\in \Q, x\in G\rbrace$. Durch die Abbildung $(a, n) \mapsto \frac{1}{n} \cdot a$ ist ein Isomorphismus von $\overline{G}$ auf $\Q\cdot G$ definiert.}
%%

%
\begin{defn} %
Eine angeordnete abelsche Gruppe ist ein Tripel $\left(G, +, \leq\right)$, wobei $\left(G, +\right)$ eine additiv geschriebene, abelsche Gruppe ist, die bezüglich $``\leq "$ total geordnet ist.
\end{defn}
%
%
%
%
\begin{nota}
Wir verwenden im Hauptteil \ref{chap3} für eine angeordnete abelsche Gruppe $\left(G, +, \leq\right)$ meist vereinfachend die Bezeichnung $G$.
\end{nota}

\begin{bem}\label{angeordnetAbelsch} %\cite{Lueneburg08} 
Ist eine abelsche Gruppe $G$ mit dem Positivbereich $P$ angeordnet, so definieren wir $a \leq b$ genau dann, wenn $b - a \in P$ ist für $a, b \in G$.
%aus: http://wwwmath.uni-muenster.de/users/ischebeck/algebra.pdf 
\end{bem}
%
%
%
%
%
\begin{bsp}
\begin{itemize}
\item[]
\item Die Menge der ganzen Zahlen $\left(\Z, +, \leq\right)$ ist eine angeordnete abelsche Gruppe.
\item Die Menge der reellen Zahlen $\left(\R, +, \leq\right)$ ist eine angeordnete abelsche Gruppe.
\end{itemize} 
\end{bsp}
                                                                                                                     
%

\section{Wohlordnung in angeordneten abelschen Gruppen}
Die folgenden Aussagen orientieren sich an der Arbeit $\glqq$A residue theorem for Malcev--Neumann series$\grqq$ von Guoce Xin.
\begin{satz}\label{wohlgeordnetabnehmendeFolge} 
Sei $`` \le "$ eine totale Ordnung auf der Menge $W$. Dann ist $W$ genau dann wohlgeordnet, wenn es keine streng monoton fallende Folge in $W$ gibt.
\end{satz}
\beweis{ $``\Rightarrow "$ Sei W wohlgeordnet. Angenommen es gibt eine streng monoton fallende Folge von Elementen in W, nämlich $a_1 > a_2 > a_3 > ...$. Damit erhalten wir eine Teilmenge $\left( a_i\right)_{i\in \N}$, die kein kleinstes Element besitzt. Dies ist ein Widerspruch zur Wohlordnung von $W$.\\
$``\Leftarrow " $ Es gibt keine streng monoton fallende Folge in $W$. Angenommen $W$ ist nicht wohlgeordnet. Dann gibt es eine Teilmenge $A$ von $W$, die kein kleinstes Element enthält. Für ein beliebiges Element $a_1 \in A$ finden wir ein $a_2 \in A$ mit $a_2 < a_1$. Dieses Verfahren lässt sich endlos fortsetzen und wir erhalten eine streng monoton fallende Folge $a_1 > a_2 > a_3 > ...$, Widerspruch.}
%Quelle: http://arxiv.org/pdf/math/0405133v1.pdf
%
%
%
%
%
%
\begin{bsp}\label{TotalGeordnetEndlichIstWOhlgeordnet}
Total geordnete endliche Mengen sind wohlgeordnet.
\end{bsp}
%
%
%
%
%
%
%
%
Betrachte nun die Menge aller wohlgeordneten Teilmengen $W_A$ einer total geordneten Menge $A$, die nicht zwangsläufig wohlgeordnet ist. 
\begin{lemma}\label{wohlgeordnvereinigung} %\cite{xin04}
Sei $w_{n} \in W_A$ für alle $n\in \N$, dann gilt $\bigcap_{n \in \N} w_n \in W_A$ und für $w_1, w_2 \in W_A$ gilt $w_1 \cup w_2 \in W_A$.
\end{lemma}
\beweis{Die erste Aussage ist trivial. Zum Beweis der zweiten Behauptung führen wir einen Widerspruchsbeweis. Angenommen $w_1\cup w_2$ sei nicht wohlgeordnet, dann gibt es nach \ref{wohlgeordnetabnehmendeFolge} eine streng monoton fallende Folge $a_1 > a_2 > ...$ in $w_1\cup w_2$.\\
Betrachten wir alle Elemente der Teilmenge $w_1$. Wir können diese, da $w_1$ total geordnet ist, als Folge $a_{i_1} > a_{i_2}...$ schreiben. Aufgrund der Wohlordnung von $w_1$ ist die so erhaltene fallende Folge endlich. Die selbe Argumentation wählen wir für $w_2$ und erhalten die endliche fallende Folge $a_{j_1} > a_{j_2}...$. Aber jedes Element der streng monoton fallenden Folge $a_1 > a_2 > ...$ ist in einer der beiden endlichen Folgen enthalten.  Widerspruch!}
%
%
%
%
% 
%
%
%
%
Die Menge $W_A$ ist somit unter endlicher Vereinigung und unendlichem Schnitt abgeschlossen.
\begin{lemma}\label{unendlicheFolgeEigenschaften}
Wir betrachten eine total geordnete Menge $A$. Jede Folge $\left(a_n\right)_{n \in \N}$ in $A$ erfüllt mindestens eine der drei folgenden Eigenschaften:
\begin{enumerate}
\item[(1)] $a_1, a_2, ...$ enthält eine streng monoton wachsende Teilfolge.
\item[(2)] $a_1, a_2, ...$ enthält eine konstante Teilfolge.
\item[(3)] $a_1, a_2, ...$ enthält eine streng monoton fallende Teilfolge.
\end{enumerate}
\beweis{
Angenommen die Folge $\left(a_n\right)_{n \in \N}$ erfüllt weder die Bedingung (2), noch (3). Wir wollen zeigen, dass sie eine streng monoton wachsende Teilfolge enthält.\\
Da die Folge nach Voraussetzung keine streng monoton fallende Teilfolge enthält, gibt es ein kleinstes Element $a_{i_1}$. Andernfalls ließe sich eine streng monoton fallende Teilfolge konstruieren. Die Folge bleibt unendlich, wenn wir die ersten Folgenglieder $i_1$ aus $\lbrace a_n\rbrace_{n\ge 1 }$ entfernen, da es nur endlich viele Folgenelemente nach Voraussetzung gibt, die gleich $a_{i1}$ sind. In der daraus entstandenen Folge ist jedes Element größer als $a_{i1}$. Sie enthält wiederum keine streng monoton fallende oder konstante Teilfolge. In der so entstandenen Folge ist jedes enthaltene Element echt größer als $a_{i1}$. \\
Wir wiederholen das durchgeführte Verfahren und konstruieren so die streng monoton wachsende Teilfolge $a_{i1} < a_{i2} <...$.\\
}
%
\end{lemma}
%
Bernhard Hermann Neumann ein deutsch-englisch-australischer Mathematiker bewies in seinem Werk $\glqq$On ordered division rings$\grqq$ \cite[S. 206]{neumann49} die beiden folgenden wichtigen Lemmata, deren volle Bedeutung sich im Hauptteil \ref{eq: multPotenzreihenkoerper} erschließen wird. 
\begin{lemma}\label{wohlgeordnetwennkeineabfallendeFolge} %\cite{neumann49}
Die Menge $W$ ist genau dann wohlgeordnet, wenn jede Folge $\left(w_n\right)_{n\in\N}$ von Elementen aus W eine monoton steigende Teilfolge  $w_{\tau(1)} \le w_{\tau(2)} \le ...$ enthält.
\end{lemma}
\beweis{$``\Rightarrow "$ Sei die total geordnete Menge $W$ wohlgeordnet. Dann gilt nach Lemma \ref{unendlicheFolgeEigenschaften} und mit Satz \ref{wohlgeordnetabnehmendeFolge}, dass eine Folge $\left(w_n\right)_{n\in\N}$ von Elementen aus $W$ entweder eine streng monoton steigende oder konstante Teilfolge enthält. Jede Folge aus $W$ besitzt daher eine monoton steigende Teilfolge.\\
$``\Leftarrow "$ Jede Folge von Elementen aus $W$ enthält eine monoton steigende Teilfolge. Angenommen $W$ sei nicht wohlgeordnet. Nach Satz \ref{wohlgeordnetabnehmendeFolge} existiert somit eine streng monoton fallende Folge in $W$. Nach Voraussetzung muss jede Folge eine monoton steigende Teilfolge enthalten. Dies führt zu einem Widerspruch, da eine streng monoton fallende Folge keine monoton steigende Teilfolge enthalten kann.}
Wir bezeichnen die Menge der wohlgeordneten Teilmengen einer angeordneten abelschen Gruppe $G$ mit $W_G$.
%
%

%Auskommentiert, weil gleiche Aussage wie von Neumann Lemma
%\begin{satz}\label{produktInWohlordnung}
%Sei $G$ eine angeordnete abelsche Gruppe. Seien $w_1, w_2 \in W_S$ dann gilt $w_1+w_2 \in W_G$. 
%\end{satz}
%\beweis{
%Angenommen $w_1+w_2$ liegt nicht in der Menge der wohlgeordneten Teilmengen $W_G$. Es gibt also eine streng monoton fallende Folge $a_1+b_1 > a_2+b_2 > ...$, wobei $a_i \in w_1, b_i \in w_2, \forall i \in \N$. Da $w_1$ wohlgeordnet ist, enthält die streng monoton fallende Folge $\left( a_n\right)_{n\ge1} $ nach Definition der Wohlordnung keine streng monoton fallende Folge. Nach \ref{unendlicheFolgeEigenschaften} gibt es eine streng monoton steigende oder konstante Teilfolge $a_{i1} \le a_{i2} \le ...$. Da aber gilt $a_{i1}+b_{i1} > a_{i2}+b_{i2} > ...$ erhalten wir eine  streng monoton fallende Folge $b_{i1} > b_{i2} > ...$ in $w_2$. Dies widerspricht der Tatsache, dass $w_2$ wohlgeordnet ist.}
%%die drei Lemmas aus: http://arxiv.org/pdf/math/0405133v1.pdf


\begin{lemma}[Lemma von B.H. Neumann] \label{LemmaNeumann} %\cite{neumann49}
Sei $G$ eine angeordnete Gruppe und $V, W \subseteq G$ wohlgeordnet, dann ist U = V + W ebenso wohlgeordnet. 
\end{lemma}
\beweis{
Sei \[u_1 = v_1 + w_1,~ u_2 = v_2+w_2 ~...\text{, mit } v_r \in V, w_r\in W \text{ und } r\in\N \] eine beliebige Folge von Elementen aus $U$. Es gibt eine monton steigende Teilfolge $v_1, v_2, ...$ mit $v_{\tau(1)}\le v_{\tau(2)} \le ...$ in der Menge $V$. Die entsprechende Folge in $W$, nämlich$w_{\tau(1)}, w_{\tau(2)},...$ enthält ebenso eine monoton steigende Teilfolge $w_{\tau(\sigma(1))}\le w_{\tau(\sigma(2))} \le ...$. Daraus folgt, es gibt zu der beliebigen Folge von Elementen aus $U$ ebenso eine monoton steigende Teilfolge $u_{\tau(\sigma(1))}\le u_{\tau(\sigma(2))} \le ...$, Nach Lemma \ref{wohlgeordnetwennkeineabfallendeFolge} ist $U$ damit wohlgeordnet.}
%
%
% 
%
%
%
%
\begin{folg}\label{FolgerungNeumann} %\cite{neumann49}
Seien V, W wohlgeordnete Teilmengen einer angeordneten Gruppe $G$, dann gibt es für ein $g \in G$ nur endlich viele Paare $\left(v, w\right) \in V\times W$ mit $v + w = g$.
\end{folg}
\beweis{Angenommen es gäbe unendlich viele $v_n \in V, w_n \in W$ wobei $g = v_n + w_n$, $n \in \N$, und die Folge besitzt paarweise verschiedene Glieder. Da $V$ und $W$ wohlgeordnet sind, besitzt weder $\left(v_n\right)_{n\in\N}$ noch $\left(w_n\right)_{n\in\N}$ eine streng monoton fallende Teilfolge. Nach Satz \ref{unendlicheFolgeEigenschaften} hat $\left(v_n + w_n\right)_{n \in \N}$ eine streng monoton steigende Teilfolge. Widerspruch, da $g$ konstant bleiben muss.
}
%
%
%
%
%
%
%
%
%
%
\section{Archimedisch angeordnete abelsche Gruppen}\label{Archimedisch angeordnete Gruppen}
%
Erst seit dem Ende des 19. Jahrhunderts kristallisierte sich die hohe Bedeutung geordneter Strukturen in der Mathematik heraus. Man erkannte, dass das archimedische Axiom unverzichtbar für die nähere Untersuchung dieses Bereichs war. Es spielte unter anderem eine wichtige Rolle bei der Entwicklung der reellen Zahlen mithilfe des Dedekindschen Schnittes (1872). Genau genommen ermöglicht die archimedische Eigenschaft die Herstellung von Kommutativität und Vollständigkeit. \\ %\cite{hahn07}\\
Wir orientieren uns an dem Kapitel $\glqq$Angeordnete Gruppen$\grqq$ in \cite[S. 73 - 93]{fuchs66}, sowie an Arbeiten von Prieß-Crampe \cite{priesscrampe69}, \cite{priesscrampe83}.

\begin{defn} \label{betrag}
Der \textit{absolute Betrag} $|a|$ eines Elements a $\in  G$, wobei $G$ eine angeordnete Gruppe sei, ist definiert als $|a| = \textup{max}\lbrace a, -a \rbrace$.
\end{defn}

Wenn die angeordnete Gruppe zusätzlich abelsch ist, gilt die \textit{Dreiecksungleichung} für alle $a, b \in G$:
\[|a+ b | \le |a| + |b|, \text{ für alle } a, b \in G.\]
Die Ungleichung gilt trivialerweise wenn beide Elemente das gleiche Vorzeichen haben. Sei also $a < 0$ und $b > 0$. Dann ist $a= -|a|$.\\
Falls $|a|\le |b|$ ist, erhalten wir
\[ |a + b |= |-|a|+b| = b - |a| \le b = |b|\le|a| + |b|. \]
Falls $|a| >b$ ist, erhalten wir 
\[|a+b| = |-|a| +b| = |a| - b \le |a|\le |a| + |b|.\]  
%
% 
%
%
%
%
%
\begin{defn}\label{archim}
Eine angeordnete abelsche Gruppe $\left(G,+\right)$ heißt \textit{archimedisch}, wenn es für alle $a, b \in G$  mit $0 < a < b$ ein $n \in \N $ gibt, mit $b < na$.
\end{defn}
%
%
\begin{defn}\label{uek}
Seien $a, b \in G$, wobei $G$ eine angeordnete abelsche Gruppe ist. Das Element $a$ wird als \textit{unendlich kleiner} als $b$ bezeichnet, wenn für alle $  n \in \N $ gilt: 
\[n|a| < |b|.\]
In Zeichen schreiben wir $a \ll b$.
\end{defn}
%
\begin{defn}\label{aae}
Sei $G$ eine angeordnete abelsche Gruppe, und $|a|$ der absolute Betrag eines Elements $a$ aus $G$. Zwei Elemente $a,b \in G$ werden als \textit{archimedisch äquivalent} bezeichnet, wenn natürliche Zahlen $m$ und $n$ existieren, so dass: 
\[|a| < m|b| \text{ und } |b| < n|a|.\]
In diesem Fall schreiben wir $ a \sim b $. 
\end{defn}
%
\begin{folg}
Für jedes Paar von Elementen $a, b \in G$ gilt genau eine der anschließenden Relationen: 
\begin{multicols}{3}
\begin{enumerate}
\item[(i)] $a \ll b$
\item[(ii)] $a \sim b$
\item[(iii)] $b \ll a$, 
\end{enumerate}
\end{multicols}
%
Des Weiteren schließen wir aus Definition \ref{uek} und \ref{aae}:
\begin{enumerate}
\item[(i)] Aus $a \ll b$ folgt $-g+a+g $ $\ll$ $-g+b+g$ für alle $g \in G$;
\item[(ii)] Aus $a \ll b$ und $a \sim c$ folgt $c \ll b$;
\item[(iii)] Aus $a \ll b$ und $b \sim d$, folgt $a \ll d$;
\item[(iv)] Aus $a \ll b$ und $b \ll c$ folgt $a \ll c$;
\item[(v)] Aus $a \sim b$ und $b \sim c$ folgt $a \sim c$.
\end{enumerate}
Sind alle Elemente einer Gruppe $G\setminus\lbrace 0 \rbrace$ archimedisch äquivalent, so ist die Gruppe \textit{archimedisch angeordnet}. \\ 
Durch die archimedische Äquivalenz werden die Elemente von $G$ in disjunkte Klassen unterteilt, die angeordnet werden können. Es bezeichne $[g]$ die \textit{archimedische Klasse} in der das Element $g \in G$ liegt, $[G]$ die Gesamtheit aller archimedischen Klassen von $G$. \label{archimedischeKlassen}\\
% evtl überflüssig: Sind zwei Elemente $a,~b \in G$ nicht archimedisch äquivalent, gilt entweder für alle
%$n \in \N$ ist $n|a|<|b|$, oder für alle $n \in \N$ ist $n|b| <|a|$.
\end{folg}
%
%\begin{satz}\label{agkku}
%Eine archimedische Gruppe G enthält keine konvexen Untergruppen außer sich selbst und der trivialen.
%\end{satz}
%
%
%\beweis{
%Angenommen es gibt eine Untergruppe $\left(U, \circ\right))$ $\subseteq$ G, mit $ U \neq \varnothing und U \neq G$. Nach den Untergruppenaxiomen gilt für $a, b \in U: a\circ b \in U$ Da G archimedisch geordnet ist und alle Elemente aus U natürlich auch in G liegen, muss für $a, b \in U \text{ gelten, wenn 0 < a < b dann gibt es ein n \in \N: } b < na.$ Da U echte Untergruppe von G, gibt es ein Element $x\in G, \text{ aber } x \notin U$. Da G archimedisch geordnet gibt es ein   }
%
%
%
% 
%
%
%
%\begin{satz}\label{archimedischangeordnet folgt abelsch} %\cite{pickert55} %bzw Priess Crampe
%Eine archimedische Gruppe $\left(G, +\right)$ ist eine abelsche Gruppe.
%\end{satz}
%\beweis{Falls $G$ ein kleinstes positives Element $z$ besitzt, so ist die von $z$ erzeugte Untergruppe $\langle z \rangle$ eine abelsche Gruppe, da jede aus einem Element erzeugte Menge eine Untergruppe ist. Nach Definition der Archimedizität \ref{archim} existiert für $0 < b \in G$ eine natürliche Zahl $n$ mit: 
%\[(n-1)\cdot z \le b < n\cdot z.\]
%Die Umkehrabbildung ${\lambda_x}^{-1} $ der bijektiven und ordnungstreuen Abbildung \[\lambda_x\colon G \to G,~ y \mapsto x+ y\] existiert und es gilt:
%\[{\lambda_x}^{-1}(a) ~ x \mapsto a-y \text{ und daher: }\]
%\[ 0 = \lambda_x{\left(n-1\right)\cdot z}^{-1}\left(\left( n-1\right) \right) ~\le ~\lambda_x{\left(n-1\right)\cdot z}^{-1}\left(b\right) ~<~ \lambda_x{\left(n-1\right)\cdot z}^{-1}(n\cdot z) = z. \]
%Da $z$ nach Voraussetzung das kleinste positive Element aus $G$ ist, erhalten wir:
%\[0 = \lambda_x{\left(n-1\right)\cdot z}^{-1}\left(b\right).\]
%Somit ist $b = (n-1)\cdot z$ und daher $L = \langle z\rangle$. $G$ ist also eine zyklische Gruppe, da sie von einem Gruppenelement erzeugt wird. Jede zyklische Gruppe ist abelsch und die Behauptung ist in diesem Fall gezeigt.\\
%Nun muss noch der Fall betrachtet werden, dass $G$ kein kleinstes positives Element besitzt. Zu jedem Element $0 < x \in G$ existiert also ein $c \in G$ mit: $0 < c < x$. Das bedeutet zwischen der Null und jedem beliebigen Element von $G$ gibt es stets noch ein Element, da kein kleinstes positives Element existiert. Es sei $x = \lambda_c{\lambda_c}^{-1}(x) = c + {\lambda_c}^{-1}(x)$, woraus wegen $c < x$ folgt, dass $0 < {\lambda_c}^{-1}(x)$ ist. Wähle $d \in G$, mit $0 < d \le $ min$\lbrace c, {\lambda_c}^{-1}(x)\rbrace$, dann gibt es für alle $x \in G$ ein derartiges Element mit $0 < 2d \le c + {\lambda_c}^{-1}(x) = x.$ Daraus folgt, es gibt natürlich auch ein Element $d' \in G$ mit $0 < 3d' \le x$.\\
%Angenommen $G$ ist nicht abelsch. Dann existieren Elemente $a, b \in G$ mit $a + b < b + a$ oder $b + a < a + b$. Wir schränken uns o.B.d.A auf den ersten Fall ein. Es sei $s \in G$ bestimmt als $s + \left(a + b\right) = 0$. Wir erhalten $0 < s + b + a$ und wir wissen, dass es ein Element $d'$ aus $G$ gibt, welches die Ungleichung $0 < 3\cdot d'\le s + \left(b + a\right)$ erfüllt. Weil $G$ archimedisch ist, gibt es ganze Zahlen $n_1, n_2$ mit $n_1 \cdot d' \le a < (n_1+1)\cdot d'$, $n_2 \cdot d' \le a < (n_2+1)\cdot d'$. Wir folgern:
%\[(n_1 + n_2) \cdot d' \le b + a < n_1 + n_2) \cdot d' + 3d' \text{ und }\]
%\[(n_1 + n_2) \cdot d' \le a + b < n_1 + n_2) \cdot d' + 3d'\].
%Aus der letzten Ungleichung folgt -($n_1 + n_2)\cdot d - 3d' < s \le - (n_1 + n_2) \cdot d$ und damit ergibt sich:
%\[s + (b + a) < 3d\]
%Wir erhalten einen Widerspruch zu unserer Annahme, daher ist die Aussage gezeigt.
%}
%
%
%
%
%
%
%
%
%
\section{Der Satz von Hölder}
In diesem Abschnitt werden wir einen sehr wichtigen Satz der Theorie angeordneter Strukturen vorstellen, den \textit{Satz von Hölder}. Dieser Satz besagt, dass jede archimedisch angeordnete Gruppe bis auf Isomorphie einer Untergruppe der additiven Gruppe der reellen Zahlen mit der Ordnung $``< "$ entspricht. Zwar wird der Beweis dieser Aussage in der verwendeten Literatur Otto Hölder (1901)\cite{hoelder1901} zugeschrieben, die grundlegenden Ideen dazu lieferte jedoch bereits Bettazi in seinem Werk $\glqq$Teoria delle grandezze$\grqq$, 1890\cite[S. 578]{Lueneburg08}.\\
Wir orientieren uns hierbei an \cite{hoelder1901} und \cite{priesscrampe83}. 
\begin{satz}\label{aga} %\cite{hoelder1901}
Eine angeordnete abelsche Gruppe ist genau dann archimedisch, wenn sie zu einer mit der natürlichen Ordnung versehenen Untergruppe der additiven Gruppe der reellen Zahlen o-isomorph ist.
\end{satz}
%
%
\beweis{
$``\Leftarrow "$ Die Rückrichtung ist klar, da jede Untergruppe der additiven Gruppe der reellen Zahlen archimedisch angeordnet ist und diese Eigenschaft durch den o-Isomorphismus ebenfalls für die angeordnete Gruppe $G$ gelten muss. \\
$``\Rightarrow "$ Sei $G$ eine angeordnete abelsche Gruppe. $G$ besitzt also einen Positivbereich $P$. Nach Voraussetzung erfüllt $G$ die archimedische Eigenschaft. Sei $G \neq \lbrace 0_G\rbrace$, wobei $0_G$ das neutrale Element der Addition in $G$ ist. Andernfalls wäre $G$ isomorph zu $\lbrace 0 \rbrace \subseteq \R$, der trivialen Untergruppe der additiven Gruppe der reellen Zahlen. \\
%Jetzt den Fall, dass G nicht dicht ist (wird nur nicht explizit so bezeichnet, weil ich dann dicht wieder einführen müsste^^)
%
Angenommen es gibt ein $g \in P$ mit $0_G \leq g_1 \le g$ und für jedes $h\in G$, wobei $0_G \leq h \le g$ ist, folgt, dass $h = 0_G$ gilt. Wegen der archimedischen Eigenschaft gibt es zu jedem $h\in G$ eine ganze Zahl $n\in \Z$, sodass $ng \leq h \le \left(n+1\right)g$ gilt. Daher erhalten wir $0\leq h + \left(-n\right)g \le (n+1)g -ng = g$. Also gilt, dass $h + \left(-n\right)g = 0$ ist. Infolgedessen ist $G= \langle g\rangle$, wobei $\langle g\rangle$ die von $g$ erzeugte Untergruppe ist, o-isomorph zur Gruppe der ganzen Zahlen ist bezüglich der Zuordnung $ng \mapsto n$.\\
Im weiteren Beweis gehen wir davon aus, dass kein derartiges $h\in G$ existiert, sich also zu jedem $g\in P$ ein $h\in G$ finden lässt, sodass $0 < h < g$ gilt. Wir bezeichnen deswegen die Gruppe $G$ als \textit{dicht}.\\
%
%
%
%
%
%
%
%
%
Somit ist $P$ nicht leer und wir wählen ein Element $\alpha \in P$ beliebig. Für jedes $g \in P$ definieren wir:
\[S_g := \lbrace \frac{m}{n} \in \Q^{+} | m, n \in \N, m\alpha \le ng\rbrace\]
Für beliebige $m, n, p \in \N$ gilt die Äquivalenz $m\alpha \le n g \Leftrightarrow m p \alpha \le n p \alpha$. Die Darstellung von $r \in \Q^{+}$ als Quotient zweier natürlicher Zahlen hängt also nicht damit zusammen, ob $ r$ in $S_g$ enthalten ist.\\
Um die Behauptung zu zeigen, genügt es, einen Monomorphismus zu finden, der $G$ auf die additive Gruppe der reellen Zahlen abbildet. Wir zeigen nun folgende Aussagen: \\
\begin{enumerate}
\item[(i)] Für alle $g\in S_g$ gilt $S_g \neq \varnothing \text{ und } S_g \neq \Q^{+}$. Für $r \in \Q^{+} \text{ und } s \in S_g \text{ mit } r < s$ folgt $r \in S_g.$
\item[(ii)] Sei $S_g \subseteq \Q^{+}$, wobei $S_g$ nicht leer und beschränkt. Die Abbildung $\Phi: P \rightarrow \R^{+}, g \mapsto$ sup$\lbrace S_g \rbrace$ ist somit wohldefiniert.  
\item[(iii)]Für alle $g, h \in P$ gilt $g \le h $ genau dann, wenn $ S_g \subseteq S_h$ genau dann, wenn $\Phi(g) \le \Phi(h)$.  
\item[(iv)] Sei $g, h \in P$ und $r,s \in \Q^{+}$. Sei $r \in S_g \text{ und } s \in S_h \text{ so folgt } r+s \in S_{g+h}$. \\
Sei $r \notin S_g \text{ und } s \notin S_h \text{, so folgt } r+s \notin S_{g+h}$. 
\item[(v)] Es gilt $\Phi\left( g+h\right) = \Phi\left(g\right) + \Phi\left(h\right)$ für alle $g, h \in P$. 
\item[(vi)] $\Phi$ wird fortgesetzt auf die gesamte Gruppe $G$ durch $\Phi\left(0_G\right) = 0 \text{ und } \Phi \left(-g\right) = -\Phi \left(g\right)$ für alle g $\in P$. \\ 
\end{enumerate}
Insgesamt ist $\Phi$ ein injektiver Homomorphismus abelscher Gruppen und damit ist $G$ isomorph zu einer Untergruppe der additiven Gruppe der reellen Zahlen.\\ 
Zu (i): Wegen der archimedischen Eigenschaft gibt es zu jedem Element $g \in P$ ein $n \in \N$ mit $ng > \alpha$. Nach Definition von $S_g$ gilt $\frac{1}{n} \in S_g$ und somit ist $S_g$ nicht leer.\\ Angenommen $S_g = \Q^{+}$, dann wäre $n \in S_g$ und $n\alpha \le g$ für alle $n\in\N$, was ein Widerspruch zur archimedischen Eigenschaft ist.\\
Seien $r, s \in \Q^{+} \text{ mit } r < s \text{ und } s \in S_g \text { mit } r := \frac{k}{l} \text{ und } s:= \frac{m}{n}, \text{ wobei } m, n ,k, l \in \N$. Nach Voraussetzung gilt $kn < lm $ und da $s \in S_g: m\alpha \le ng$ und $kn\alpha \le lm\alpha \le lng.$ Daraus wiederum folgt: $\frac{kn}{ln} = r \in S_g$.\\
%
%
% 
%
Zu (ii): Wir haben bereits gezeigt, dass $S_g$ nicht leer ist. Angenommen $S_g$ wäre unbeschränkt, dann gäbe es für jedes $n \in \N$ ein $r \in S_g$ mit n < r. Nach (i) folgt daraus $n \in S_g$ und für alle $n \in \N$ folgt: $ n\alpha \le g$, was ein Widerspruch zur archimedischen Eigenschaft ist.\\
%
%
%
%
%
Zu (iii): Zunächst beweisen wir die erste Implikation. Sei  $g, h \in P$ und es gilt $g \le h$. Sei $r \in S_g$, $r:= \frac{m}{n}, m, n \in \N $, dann erhalten wir $m\alpha \le ng$. Da $g \le h$ ist, folgt $ m\alpha \le n h$ und damit $r \in S_h$. Die zweite Implikation folgt nach Definition von $\Phi$ offensichtlich.\\
Sei $\Phi\left(g\right) \le \Phi\left(h\right)$. Angenommen es sei $g > h$. Nach der archimedischen Eigenschaft gibt es ein $ n \in \N$ mit $n\left(g-h\right) > 2\alpha$. Wähle $m \in \N$ möglichst klein, mit $m\alpha < nh$. Dieses $m$ existiert, da $G$ dicht ist. Es gilt $\frac{m}{n} \notin S_h$ und $\frac{m}{n} \geq \Phi\left(h\right).$
Da m minimal ist, gilt die Ungleichung $\left(m-1\right)\alpha \le nh$ und wir erhalten $ \left(m+1\right)\alpha \le nh + 2\alpha < nh + n(g-h) = ng$ und $\frac{m+1}{n} \in S_g,$ also $\frac{m+1}{n} \leq$ sup$\left(S_g\right) = \Phi\left(g\right)$. Insgesamt ergibt sich $\Phi\left(h\right) \le \frac{m}{n} < \frac{m + 1}{n} \le \Phi\left(g\right)$, Widerspruch.\\
%
%
%
%
%
%
Zu (iv): Sei $r \in S_g, s \in S_h \text{ mit } r := \frac{k}{l} \text{ und } s:= \frac{m}{n}, \text{ wobei } m, n ,k, l \in \N$. Dann gilt $k\alpha < lg $ und $ m\alpha \le nh$. Wir erhalten $kn\alpha \le lng \text{ und } lm\alpha \le lng.$ Somit liegt $r+1 = \frac{kn+lm}{ln}$ in der Menge $S_{g+h}.$ Die zweite Aussage folgt analog indem in Obigem $``\le "$ durch $``> "$ ersetzt wird.\\
%
%
%
%
%
%
Zu (v): Als erstes zeigen wir, dass $\Phi\left(g + h\right)$ eine obere Schranke von $S_{g+h}$ ist. Angenommen es gibt ein $r\in S_{g+h}$ mit $r > \Phi\left(g + h\right)$. Wähle $\epsilon = r - \Phi\left(g\right) - \Phi\left(h\right)$ und wähle s, t $\in \Q^{+}$ mit $\Phi(g) < s < \Phi(g) + \frac{\epsilon}{2} \text{ und } \Phi\left(h\right) < t < \Phi\left(h\right) +\frac{\epsilon}{2}$. Insgesamt folgt $s + t < \Phi\left(g\right) + \Phi\left(g\right) + \epsilon = r$. Da $s \notin S_g$ und $t \notin S_h$ gilt $s+t \notin S_{g+h}$ nach (i). Wir erhalten $s+ t \geq \Phi\left(g+h\right) \geq r$ und der Widerspruch $ r \le s+t < r$ zeigt, dass ein derartiges $r$ nicht existieren kann.\\
Es bleibt zu zeigen, dass $\Phi\left(g + h\right)$ die kleinste obere Schranke von $S_{g+h}$ ist. Angenommen es gäbe eine kleinere obere Schranke $ o \in \R^{+}$ und sei $\epsilon = \Phi(g) + \Phi(h) - o$. Nach Definition der Abbildung gibt es ein $r\in S_g$ mit $r < \Phi(g) - \frac{\epsilon}{2}$ und $ s \in S_h$ mit $s < \Phi(h) - \frac{\epsilon}{2}.$ Nach (iv) ist $r+s$ in $S_{g+h} $ und daher $ r +s \le o$. Widerspruch, da $r+s >\Phi(g) + \Phi(h) - \epsilon = o$.\\
%
%
%
%
%
Zu (vi): Falls $g, h > 0_{G}$ ist haben wir bereit gezeigt, dass aus $g\le h$ folgt, dass $\Phi(g) \leq \Phi(h)$ ist. Die Aussage gilt offensichtlich, wenn eines der beiden Elemente $g$ oder $h$ gleich Null ist. Sei $g < 0_G \text{ und } h > 0_G$, dann folgt die Behauptung nach Definition. Die Behauptung bleibt für $g, h < 0_G$ zu zeigen. Wir erhalten
\[g \le h \Leftrightarrow -g \geq -h \Leftrightarrow \Phi\left(-g\right) \geq \Phi\left(-h\right)n \Leftrightarrow \Phi\left(g\right) \le \Phi\left(h\right).\]
Wir zeigen nun $\Phi(g+h) = \Phi\left(g\right) +\Phi\left(h\right)$ für beliebige $g,h \in G$. Es genügt dies für $g, h < 0_G$ zu beweisen. Hier kann auf das bereits Bewiesene zurückgegriffen werden: 
\[\Phi\left(g+h\right) = −\Phi\left((−g)+(−h)\right) = (−\Phi(−g))+(−\Phi(−h)) = \Phi(g)+\Phi(h).\]
Nun betrachten wir den Fall $g\geq −h$. Dann ist $g+h \geq 0_G$, und nach der bereits gezeigten Aussage folgern wir:\\
$\Phi(g+h)+ \Phi(−h) = \Phi(g) \Leftrightarrow \Phi(g+h)−\Phi(h) = \Phi(g) \Leftrightarrow \Phi(g+h) = \Phi(g)+ \Phi(h)$.\\
Setzen wir nun $g < −h$ voraus. Dann ist $−g−h > 0$, also $\Phi(g)+\Phi(−g−h) = \Phi(−h)$, was äquivalent zu $\Phi(g)−\Phi(g+h) = −\Phi(h)$ und zu $\Phi(g)+ \Phi(h) = \Phi(g+ h)$ ist.
\\ Die übrigen Fälle folgen analog. 
Da $\Phi$ offensichtlich eine monoton steigende Funktion ist, ist $\Phi$ ein ordnungserhaltender Isomorphismus.}
%Beweis über archimedische Eigenschaft, o-Isomorphie einer einelementigen Gruppe zu der Gruppe der ganzen Zahlen (\cite{priesscrampe69} S. 8), Kommutativität von G, Dedekindschen Schnitt und Homomorphismus (\cite{fuchs66} S. 75) } 
%
%TODO : Nun stellt sich die Frage, wann 
\begin{satz}\label{homomorphismus nach R} %\cite{fuchs66}
Sei $\left(G, +\right)$ eine Untergruppe der natürlich geordneten additiven Gruppe der reellen Zahlen und $\phi: G \rightarrow \R$ ein injektiver o-Homomorphismus. Dann gibt es eine positive reelle Zahl $r$ mit $\phi(g) = r\cdot g$ für alle $g \in G$. 
\end{satz}
\beweis{Nach Voraussetzung ist $\phi$ ein injektiver o-Homomorphismus und damit sind mit $0 < g_1$, $ g_2\in G $ auch $\phi(g_1)$ und $\phi(g_2)$ positiv. Angenommen es gilt, dass $\frac{\phi(g_1)}{\phi(g_2)}\neq \frac{g_1}{g_2}$, so gibt es eine rationale Zahl $\frac{m}{n}$ mit $m, n \in \N$, die zwischen $\frac{\phi(g_1)}{\phi(g_2)}$ und $ \frac{g_1}{g_2}$ liegt. Wir nehmen ohne Beschränkung der Allgemeinheit an, dass $\frac{\phi(g_1)}{\phi(g_2)} < \frac{m}{n} < \frac{g_1}{g_2}$ ist. Weiterhin gehen wir davon aus, dass $n \cdot g_1 > m\cdot g_2$ ist. Nach der archimedischen Eigenschaft der Gruppe $G$ stehen die Bilder $\phi(n\cdot g_1)$ und $\phi(m\cdot g_2)$ in umgekehrter Größenbeziehung zueinander. Dies steht jedoch im Widerspruch zur Ordnungstreue der Abbildung $\phi$. 
Folglich ist $\frac{\phi(g_1)}{\phi(g_2)}= \frac{g_1}{g_2}$. Wir erhalten somit auch für alle positiven Elemente $g \in G$, dass die Gleichung $\frac{\phi(g_1)}{g_1} = \frac{\phi(g)}{g}$ erfüllt ist. \\
Für die negativen Gruppenelemente $g \in G$, mit $g < 0$  und daher $-g >0$ erhalten wir aufgrund der Homomorphismuseigenschaften $\frac{\phi(g)}{g} = \frac{(-1)\cdot\phi(g)}{(-1)\cdot g} = \frac{\phi(-g)}{-g} = \frac{\phi(g_1)}{g_1}$. Mit der positiven Konstanten $r = \frac{m}{n} = \frac{\Phi(g_1)}{g_1}$ ist die Aussage $\phi(g) = r \cdot g$ für alle $g \in G$ gezeigt. 
}
Die Grundaussage dieses Satzes bewies Hion 1954 in seinem russischsprachigen Werk $\glqq$Archimedisch geordnete Ringe$\grqq$. Er setzte jedoch einen o-Homomorphismus zwischen zwei Untergruppen der additiven angeordneten Gruppe der reellen Zahlen voraus, ebenso wie Fuchs und Prieß-Crampe, die den Satz in ihre Arbeiten mitaufnahmen.  Der Satz \ref{homomorphismus nach R} impliziert weiterhin die o-Isomorphie zwischen der Gruppe der ordnungserhaltenden Automorphismen der archimedischen Gruppe und der multiplikativen Gruppe der positiven reellen Zahlen. \cite{priesscrampe83}
%
%
%
%
%
%TODO ABELSCH MACHEN!!!
\section{Die angeordnete Menge konvexer Untergruppen}
In diesem Abschnitt geht es um konvexe Untergruppen einer angeordneten abelschen Gruppe. Wir benötigen einige Eigenschaften dieser Menge an speziellen Untergruppen für den Nachweis des Inversen im verallgemeinerten Potenzreihenkörper. Untergruppen teilweise geordneter Gruppen besitzen eine durch die teilweise Gruppenordnung induzierte teilweise Ordnung. Wir bezeichnen die Untergruppen als angeordnet, falls die ursprüngliche teilweise Ordnung ebenso eine Anordnung war.\\
Sei $\left(G, +\right)$ eine Gruppe, $U$ eine Untergruppe und $g \in G$. 
Wir untersuchen nun die bezüglich der Inklusion linear angeordnete Menge $\Sigma$ konvexer Untergruppen von $G$. Wir orientieren unsere Ausführungen an \cite[S. 81 - 83]{fuchs66}  und \cite[S. 3]{priesscrampe83}.

\begin{defn}\label{konvexUGR} %\cite{priesscrampe69}
Eine Untergruppe $U$ einer angeordneten abelschen Gruppe $G$ nennen wir \textit{konvex}, wenn aus $a \in U$, $x \in G$, mit $0 < |x| < |a|$ folgt $x \in U$.\\
\end{defn}
$``\Sigma "$ bezeichne nun die \textit{Menge konvexer Untergruppen} einer angeordneten abelschen Gruppe $\left(G, +\right)$. 
\begin{defn}\label{Sprung} %\cite{fuchs66}
Sei $C, D \in \Sigma$. Wenn $D \subset C$ und $\Sigma$ keine weitere Untergruppe zwischen $C$ und $D$ enthält, nennen wir das Paar $C,~D$ \textit{Sprung} in $\Sigma$ und bezeichnen es mit $D \prec C$.
\end{defn}
%
% 
%
%
 %evtl. auch \cite{priesscrampe83}  Nach \cite[S. 81 - 83]{fuchs66}
\begin{satz}\label{EigenschaftenKonvexeUgr}
Die Menge der konvexen Untergruppen $\Sigma$ besitzt folgende Eigenschaften:
\begin{enumerate}
\item[S1:] Die Vereinigung und der Durchschnitt beliebig vieler Untergruppen aus $\Sigma$ liegen wieder in $\Sigma$.
\item[S2:] Ist $C \in\Sigma$ und $g \in G$, so ist $-g+C+g\in \Sigma$
\item[S3:] Sei $D \prec C$ in $\Sigma$, so ist $D$ normal in $C$ und $C/D$ ist isomorph zu einer Untergruppe der reellen Zahlen.
\end{enumerate}
\end{satz}
\beweis{
Zu S1: Seien $C, D \in \Sigma$ konvexe Untergruppen der angeordneten abelschen Gruppe $G$ und sei $c\in C, c\notin D$. Wir nehmen ohne Beschränkung der Allgemeinheit an, $c$ ist bezüglich der Anordnung von $G$ größer als das neutrale Element $0_G$. Da $c$ nicht in $D$ liegt, kann es kein Element $d \in D$ geben, sodass $0_G < c < d$. In diesem Fall würde $c$ nach der konvexen Eigenschaft in $D$ liegen. Dies ist ein Widerspruch zur Voraussetzung und daher gilt $D \subseteq C$. Der Schnitt $C \cap D =D$, da $D \subseteq C$. Die Untergruppe $D$ ist konvex nach Voraussetzung. Ebenso gilt $C\cup D = C$ und $C$ ist auch eine konvexe Untergruppe nach Voraussetzung. Damit folgt unmittelbar, dass sowohl der Schnitt konvexer Untergruppen wieder angeordnet und konvex ist, als auch die Vereinigung.\\
Zu S2: Jede Untergruppe einer angeordnet abelschen Gruppen ist ein Normalteiler. Die Aussage folgt also sofort aus den Eigenschaften einer angeordneten abelschen Gruppe.
%Für $g \in C$ ist offensichtlich $-g + C+ g = C \in \Sigma$. Falls $g \notin C$, so ist $-g + C+ g$ eine Untergruppe von $G$, denn für alle $c_1, c_2 \in C, g \in G$ ist $-g+ c_1+g + -g+ c_2+ g = -g + c_1+ c_2+ g \in -g + C+ g$ und $-\left(-g+ c_1+ g\right) = g + \left(- c_1\right) + -g= -g + \left(- c_1\right) +g \in -g+ C_1+ g$. Die Anordnung von $G$ überträgt sich auf $-g+ C_1+ g$ und die Untergruppe ist konvex, da $C$ nach Voraussetzung und $G$ als triviale Untergruppe konvex ist.\\
%
%
%
%
%
Zu S3: Der erste Teil der Aussage ist trivial (s. Beweis S2). In $\Sigma$ existiert keine Untergruppe zwischen $C$ und $D$. Die Faktorgruppe $C/D$ enthält dementsprechend nur die trivialen konvexen Untergruppen $0_G$ und $G$. Wir zeigen zunächst, dass $C/D$ archimedisch geordnet ist. Sei $0_G \neq g \in G$. Das Element $h\in C/D$ gehört nicht zu einer von $g \in C/D$ erzeugten konvexen Untergruppe, wenn $ng < h$ für $b = 0,   \pm1,\pm 2, ...$. Da $C/D$ nur die trivialen konvexen Untergruppen enthält, ist $g = 0_G$. Die Faktorgruppe ist archimedisch angeordnet.
%  Damit liegt $h$ nicht in $\langle g \rangle$ der von $g$ erzeugten konvexen Untergruppe. ist für jedes $c \in C/D$ die Menge $\lbrace g \in C/D: \exists_{m,n \in \Z} m\cdot a \leq g \le n\cdot a\rbrace$. Damit ist $C/D$ archimedisch und 
Der Satz von Hölder \ref{aga} liefert nun, dass $C/D$ isomorph zu einer Untergruppe der additiven Gruppe der reellen Zahlen ist.
% Nach Voraussetzung gilt $D \prec C$ und offensichtlich erfüllt jedes Element $g \in G$ die Bedingung $g^{-1}D_1g \prec  g^{-1}C_1g$. Weiterhin erhalten wir im Fall $g\in C$, dass  $g^{-1}C_1g = C$, und da $D \subset C$ ist $g^{-1}D_1g = D$. Infolgedessen ist $D$ normal in $C$ und die Faktorgruppe $C/D$ enthält, da in $\Sigma$ keine Untergruppe zwischen $C$ und $D$ existiert, dementsprechend nur die trivialen konvexen Untergruppen. In $C/D$ ist für jedes $c \in C/D$ die Menge $\lbrace g \in C/D: \exists_{m,n \in \Z} m\cdot a \leq g \le n\cdot a\rbrace$. Damit ist $C/D$ archimedisch und nach Satz von Hölder \ref{aga} isomorph zu einer Untergruppe der additiven Gruppe der reellen Zahlen.
}
%
%
%
%
%
%
%
%
%
\section{Einblick in die Bewertungstheorie}
Im Nachfolgenden betrachten wir eine angeordnete abelsche Gruppe $\left(G,+\right)$ und eine angeordnete Menge $\Theta$ mit $0$ als kleinstem Element. Die Ausführungen sind orientiert an dem Kapitel $\glqq$Archimedische Klassen, Bewertungen und Bedingungen für die Anordnungsfähigkeit von Gruppen$\grqq$ in \cite[S. 9 - 11]{priesscrampe83}.
%
%
\begin{defn} %\cite{priesscrampe83}\label{bew}
Sei $G$ eine angeordnete abelsche Gruppe. Die surjektive Funktion $v\colon G \rightarrow \Theta$ wird als \textit{Bewertung} bezeichnet, wenn die folgenden Eigenschaften erfüllt sind:
%
\begin{enumerate}
\item[B1:] $v{(a)} = 0 \Leftrightarrow a = 0$ für alle $a\in G$,
\item[B2:]  $v{(a)} = -v{(a)} \text{  } \text{ für alle } a \in G $,
\item[B3:] $ v{(a+ b)} \le$ max$\{(v{(a)}, v{(b)}\}$ für alle $ a, b \in G$ .
\end{enumerate}
%
\end{defn}
Die Gleichheit in der Bedingung [B3] gilt dann, wenn $v{(a)} \ne v{(b)} $ ist.\\
 Zwei Bewertungen $\upsilon, \text{ } \upsilon' $ auf $G$ mit den Wertemengen $\Theta , \Theta' $ sind äquivalent, wenn es eine ordnungstreue bijektive Abbildung $\sigma \colon \Theta \text{ } \rightarrow \text{ } \Theta' $ gibt, so dass $ \sigma \circ \upsilon = \upsilon  $ ist.\\
Sei $\left(G, +\right)$ eine angeordnete Gruppe. Wir bezeichnen die archimedische Klasse, in der das Element $a\in G$ liegt  mit $[a]$. Die Gesamtheit der archimedischen Klassen von $G$ nennen wir $[G]$. Die Abbildung $G \to [G] \colon a \mapsto [a]$ definieren wir als \textit{natürliche Bewertung}. %\cite{priesscrampe83}
\begin{defn}  \label{bewKoerper} %vorher nach Priess Crampe - Problem multiplikativ jetzt nach wikipedia
Sei $K$ ein Körper, ($G, +$) eine angeordnete abelsche Gruppe und $\overline{G}  = G \cup \lbrace\infty\rbrace $. Eine Abbildung $v\colon K \to \overline{G} $ wird als \textit{Bewertung eines Körpers} bezeichnet, wenn sie folgende Bedingungen erfüllt:
%TODO: Quelle checken
\begin{enumerate}
\item[B1':] $v(a) = \infty$ genau dann, wenn $a = 0$ ist,
\item[B2':] $v(ab) = v\left(a\right)+v\left(b\right) $ für alle $ a, b \in K$,
\item[B3':] $v\left(a+b\right) \ge $ min$\lbrace v(a),v\left(b \right)\rbrace \text{ für alle }  a, b \in K. $
\end{enumerate}
\end{defn}
%Ein Beispiel für eine Bewertung ist die Polordnung meromorpher Funktionen in einem festen Punkt, wie im Hauptteil \ref*{LaurentreiheBewertung} noch erörtert wird. 
Man bezeichnet $v\colon K \to \overline{G} $ als \textit{diskrete Bewertung}, falls ${G} = \Z$ ist.
%
\begin{defn} %nach Priess Crampe S39
Der Unterring $A$ eines Körpers $K$ wird als \textit{Bewertungsring} bezeichnet, wenn $a \in A$ oder $a^{-1}\in A$ für jedes $a\in K$ gilt.
\end{defn}
\begin{bem}
Ist $\left(K, v\right)$ ein bewerteter Körper, dann ist $A = \lbrace a \in K\colon v(a) \geq 0_K\rbrace$ ein Bewertungsring.
\end{bem}
\begin{defn}   %\cite{hulek12}
Ein Integritätsring $R$ heißt \textit{diskreter Bewertungsring}, falls es auf dem Quotientenkörper \textup{Quot}$(R)$ von $R$ eine Bewertung $v: \textup{Quot}(R) \rightarrow \Z \cup \lbrace\infty\rbrace$ gibt und 
%\begin{enumerate}
%\item[D1: ]$v(ab) = v(a)+ v(b)$, 
%\item[D2: ] $v(a+b) \ge$ min$\lbrace v(a), v(b)\rbrace$,
%sodass $R$ der Bewertungsring von $v$ ist. Das bedeutet:
%\end{enumerate}
\[R = \lbrace x \in \textup{Quot}(R): v(a) \ge 0 \rbrace \cup \lbrace 0 \rbrace\]
der Bewertungsring von $v$ ist.
\end{defn}
Eine weitere, äquivalente, Definition eines diskreten Bewertungsrings findet man in \cite[S. 126]{neukirch92}.
\begin{defn} \label{bewertungsring}%\cite{neukirch92}
Ein \textit{diskreter Bewertungsring} ist ein Hauptidealring mit einem einzigen maximalen Ideal $\mathfrak{p}$.
\end{defn} 
