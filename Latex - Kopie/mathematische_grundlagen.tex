\chapter{Mathematische Grundlagen}\label{chap2}
%
In diesem Kapitel werden wir die Angeordneten Gruppen, deren Eigenschaften sowie die Wohlordnung auf Gruppen einführen. Anschließend betrachten wir die konvexen Untergruppen einer angeordneten Gruppe.  
%
\section{Angeordnete Gruppen}
%
\begin{defn}\label{defgs}
Eine Menge A heißt teilweise geordnet, wenn die Relation $ \leqslant $ folgende Eigenschaften, für alle a,b,c$ \in $ A  erfüllt.
%
\begin{enumerate}
\item[T1:] \textit{Reflexivität} a $\leqslant $ a,
\item[T2:] \textit{Antisymmetrie} aus a $\leqslant $ b, b$ \leqslant $ a folgt a = b,
\item[T3:] \textit{Transitivät} aus a $ \leqslant $ b, b $\leqslant $ c folgt a $ \leqslant $ c
\end{enumerate}
%
$ \leqslant $ bezeichnet eine teilweise Ordnung auf A.
\end{defn}

Die oben definierte Ordnungsrelation wird als Anordnung beziehungsweise totale Ordnung bezeichnet, wenn neben T1-T3 anschließende Bedingung erfüllt ist:
%
\begin{enumerate}
\item[T4] Für alle a, b $ \in $ A besteht entweder a < b, oder a = b, oder a > b. (bisher nach Lsazlo Fuchs alles)
\end{enumerate}
%
%
\vspace{0.2cm}
\noindent
%
%
\begin{defn}\label{twgG}
Eine teilweise geordnete Gruppe G ist eine Menge G mit folgenden Eigenschaften: (nach Fuchs)
%
\begin{enumerate}
\item[G1:] G ist eine Gruppe bezüglich der Multiplikation,
\item[G2:] eine teilweise geordnete Menge wie in \ref{defgs} bezüglich einer Relation $ \leqslant $ 
\item[G3:] das Monotoniegesetz ist erfüllt: für a, b $\in $ G gilt aus a $ \leqslant $ b folgt ca $ \leqslant $ cb und ac $\leqslant $ bc für $\forall$ c $\in$ G.
\end{enumerate}
% 
\end{defn}
%
%
\begin{defn}\label{agG}
Eine Gruppe wird als angeordnete Gruppe bezeichnet, wenn ihre Ordnung total ist.
\end{defn}
%
%ab jetzt nach Priess crampe
\begin{satz} \label{satzaGtf} 
Eine angeordnete Gruppe ist torsionsfrei.
\end{satz}
%
\beweis{
Klar, denn angenommen die angeordnete Gruppe wäre nicht torsionsfrei würde sich für die Elemente der Torsionsgruppe ein Widerspruch mit den Monotoniegesetz G4 ergeben. (formal siehe kleiner Block)
}
%
%
\begin{satz}\label{afG}
%
Genügt eine Teilmenge P einer Gruppe G den Bedingungen P1- P3, so ist (G,$\circ $) anordnungsfähig.
%
\begin{enumerate}
\item[P1] {0} $\cup $P $\cup$ -P = G, P $\cap$ -P = $\varnothing$ 
\item[P2] P $\circ$ P $\subseteq$ P
\item[P3] P ist normal in G / % oder P vereinigt -P = G ??
\end{enumerate}
%(stimmt das?)
\end{satz}
%
%
\subsection{Archimedisch angeordnete Gruppen}
%
%
\begin{defn}\label{archim}
Eine Gruppe (G,+) heisst \textit{archimedisch}, wenn es $\forall$ a,b $\in$ G  mit 0 < a < b ein n $\in$ $\N $ gibt, mit b < na.
\end{defn}
%
%
\begin{defn}\label{uek}
Seien a,b $\in$ G, wobei G eine angeordnete Gruppe. Das Element a wird als \textit{unendlich kleiner} als b bezeichnet, wenn gilt: \\
\centerline{$|a|^n < |b|$  $\forall n \in \N $}
\end{defn}
%
\begin{defn}\label{aae}
Sei G eine angeordnete Gruppe, sei $|a|$ der absolute Betrag eines Elements a $\in$ G. Zwei Elemente a,b $\in$ G werden als \textit{archimedisch äquivalent} bezeichnet, a~b, wenn positive Zahlen m und n existieren, so dass: 
$|a| < |b|^m $ und $|b| < |a|^n$. 
\end{defn}
%
\newpage
Daraus folgt, dass für jedes Paar von Elemente a,b$\in$ G genau eine der anschließenden Relationen gilt: (nach Fuchs)
\begin{multicols}{3}
\begin{enumerate}
\item[(i)] a $\ll$ b, 
\item[(ii)] a $\sim$ b,
\item[(iii)] b $\ll$ a.
\end{enumerate}
\end{multicols}{3}
%
Des weiteren folgern wir aus Definition \ref{uek}, \ref{aae}:
\begin{enumerate}
\item[(i)] aus a $\ll$ b folgt $x^{-1}ax $ $\ll$ $x^{-1}bx$ $\forall$ x$\in$ G;
\item[(ii)] aus a $\ll$ b und a $\sim$ b folgt c $\ll$ b;
\item[(iii)] aus a $\ll$ b und b $\sim$ d, folgt a $\ll$ d;
\item[(iv)] aus a $\ll$ b und b $\ll$ c folgt a $\ll$ c;
\item[(v)] aus a $\sim$ b und b $\sim$ c folgt a $\sim$ c.
\end{enumerate}
Sind alle Elemente einer Gruppe archimedisch äquivalent, so ist die Gruppe archimedisch angeordnet. \\ Durch die archimedische Äquivalenz werden die Elemente von G in disjunkte Klassen unterteilt, die angeordnet werden können. (nach Fuchs) Bezeichne [x] die archimedische Klasse, in der das Element x $\in$ G liegt, [G] die Gesamtheit aller archimedischen Klassen von G. \\
Sind zwei Elemente a, b $\in$ G nicht archimedisch äquivalent, gilt entweder:
$\forall$ n $\in \N$ gilt: n*|a|<|b| oder $\exists$ n $\in \N $ sodass: n*|b| <|a|
%
\begin{satz}\label{agkku}
Eine archimedische Gruppe enthält keine konvexen Untergruppen außer sich selbst und der trivialen.
\end{satz}
%
%
\beweis{
??`machen!}
%
%
\begin{satz}\label{aga} (nach Hölder)
Eine angeordnete Gruppe ist genau dann archimedisch, wenn sie einer mit der natürlichen Ordnung versehenen Untergruppe der additiven Gruppe der reellen Zahlen o-isomorph ist. 
Folglich sind alle archimedisch angeordneten Gruppen kommutativ.
\end{satz}
%
%
\beweis{
"$\leftarrow$" Die Rückrichtung ist klar, da jede Untergruppe der additiven Gruppe der reellen Zahlen archimedisch ist und diese Eigenschaft durch den o-Isomorphismus ebenfalls für die angeordnete Gruppe G gelten muss. \\
"$\rightarrow$" 
Beweis über archimedische Eigenschaft, o-Isomorphie einer einelementigen Gruppe zu der Gruppe der ganzen Zahlen (siehe S8 priess crampe), kommutativität von G, Dedekindschen Schnitt und Homomorphismus (s. fuchs S75) } 
%
%
\subsection{Kette konvexer Untergruppen}
In diesem Abschnitt geht es um konvexe Untergruppen einer angeordneten Gruppe. Eine Untergruppe einer angeordneten Gruppe nennen wir \textit{konvex}, wenn aus a $\in$ U, x $\in$ G, mit 0 < |x| < |a| folgt x $\in$ U. (nach \cite{priesscrampe69})
Bezeichne $\sum$ die Kette konvexer Untergruppen einer angeordneten Gruppe G. $\sum$ besitzt folgende Eigenschaften:
\begin{enumerate}
\item[(1)] Gilt e $\in \sum$ und G $\in$ $\sum$ dann gilt, wenn $C_\lambda$ Untergruppe, dann gilt $\cup C_\lambda$ und $\cap C_\lambda$ liegen in $\sum$.
\item[(2)] Ist C $\in\sum$ und g $\in\sum$, so ist $g^{-1}Cg\in\sum$
\item[(3)] Sei D $\subset$ C und $\sum$ enthält keine Untergruppe zwischen C und D, so ist D normal in C und C/D ist isomorph zu einer Untergruppe der reellen Zahlen.
\item[(4)] Sei D $\subset$ C und $\sum$ enthält keine Untergruppe zwischen C und D, so erzeugen dei Hauptautomorphismen von C/D einen Integritätsbereich $\Phi{C/D}$ im Endomorphismenring $\Psi{C/D}$. Der durch $\Phi{C/D}$ und durch die Quadratwurzeln der Hauptautomorphismen erzeugte Körper $\Gamma{C/D}$ ist einem Unterkörper der reellen Zahlen isomorph.
\end{enumerate}
%
\begin{defn}\label{SkUgr}  (nach Malzew)
Ein System $\sum$ aus Untergruppen einer Gruppe G ist System aller konvexen Untergruppen einer Anordnung von G, genau dann wenn $\sum$ den obigen Bedingungen (1)-(4) genügt.
\end{defn}
%
%
\subsection{Bewertung einer angeordneten abelschen Gruppe}
Im Nachfolgenden betrachten wir (G,+) eine angeordnete abelsche Gruppe und eine angeordnete Menge $\Theta$ mit einem maximalen Element $\mu$. 
%
%
\begin{defn}\label{bew}
Eine \textit{Bewertung $\omega{a}$} mit $a\in$ G ist eine Funktion $\omega$: G -> $\Theta$, so dass
%
\begin{enumerate}
\item[(i)] $\omega{(a)} = \mu $  <=> a = 0,
\item[(ii)]  $\omega{(na)} = \omega{(a)} \forall n \in \N $
\item[(iii)] $ \omega{(a+ b)} \ge min{(\omega{(a)}, \omega{b})}$
\end{enumerate}
%
\end{defn}
Die Gleichheit in der Bedingung (iii) gilt dann, wenn $\omega{(a)} \ne \omega{(b)} $. Zwei Bewertungen $\upsilon, \upsilon' $ auf G mit den Wertemengen $\Gamma , \Gamma' $ sind äquivalent, wenn es eine ordnungstreue bijektive Abbildung $\sigma : \Gamma  \Gamma' $ gibt, so dass $ \sigma \circ \upsilon = \upsilon  $
Ein Beispiel für eine Bewertung ist die Polordnung meromorpher Funktionen in einem festen Punkt, wie im Hauptteil (Potenzringe/Laurentreihen) noch erörtert wird. %\referenz einfügen
% 
%
\subsection{Der Hahnsche Einbettungssatz}
%
Wie in den vorherigen Paragraphen ausgeführt sind archimedische Gruppen abelsch und Untergruppen der additiven Gruppe der reellen Zahlen. Das heißt es gibt einen injektiven monotonen Homomorphismus f: G $\mapsto$ ($\R $, +). Existiert ein weiterer solcher Homomorphismus, g: G $\mapsto$ ($\R$, +), dann existiert genau eine reelle Zahl $\lambda$ > 0, mit g(x) = r*f(x), wobei x $\in$ G
%
\begin{satz}
Jede angeordnete abelsche Gruppe ist zu einer Untergruppe eines angeordneten Vektorraumes über $\Q$ ordnungsisomorph. 
\end{satz}
%
\begin{satz} (Hahn Einbettung nach Fuchs)
Jeder angeordnete Vektorraum G über K(x), wobei K(x) der rationales Funktionenkörper, ist einem Unterraum des lexikographisch geordneten Funktionenraums W(G) o-isomorph. 
\end{satz}


\begin{satz} \label{hebs} (Hahn Einbettung nach Priess Crampe)
 (Hahnscher Einbettungssatz, Hahn 1907)
Eine angeordnete abelsche Gruppe A lässt sich ordnungstreu in die Hahn-Gruppe H($ \Gamma $, $ \R $) einbinden, wobei $ \Gamma$ = [A]\{[0]} .
\end{satz}
%
%