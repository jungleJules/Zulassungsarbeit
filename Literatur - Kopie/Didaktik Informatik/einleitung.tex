\chapter{Einleitung}
%
In einem Artikel der 53. Ausgabe der \textsc{Mathematical Gazette} (1969) beschrieben Cole and Davie (vgl. \cite{cole69}) ein kleines Spiel für 2 Spieler, das auf dem Euklidischen Algorithmus basiert. Sie nannten es anlässlich dieses Zusammenhangs das \textsc{Spiel Euklid}.\\
Wie bei fast allen Spielen stellt sich die Frage, ob es Strategien gibt, die zu einem sicheren Sieg führen. Wir werden in dieser Arbeit zeigen, dass bei bestimmten Ausgangssituationen ein Spieler durch geschickte Spielzüge immer siegt.\\
Die Seminararbeit gliedert sich in 3 Kapitel: Nach der Einleitung werden in Kapitel \ref{chap2} die Fibonacci-Zahlen und der Goldene Schnitt wiederholt. Wir werden sehen, dass sie bei diesem Spiel eine tragende Rolle spielen. In Kapitel \ref{chap3} werden zunächst kurz die Spielregeln erläutert. Danach werden wir die Gewinnstrategie des Spieles näher betrachten und erkennen, dass das Spiel bereits durch die Anfangskonstellation entschieden ist, falls beide Spieler die Strategie kennen. Abschließend geben wir noch einen kurzen Ausblick auf das Spiel 3-Euklid, welches eine Erweiterung des ursprünglichen Spiels darstellt.
